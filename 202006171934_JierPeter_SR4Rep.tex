% 四位移、四速度和四加速度

\pentry{斜坐标表示洛伦兹变换\upref{SROb},坐标变换与过渡矩阵}
%需要补充“过渡矩阵”词条
%未完成

\subsection{概念}
\subsubsection{四位置和四位移}
把时间坐标和空间坐标都看成时空的坐标,那么一个事件在时空中的位置就被称为其\textbf{四位置(4-position)}.四位置本身是一个向量,但其坐标表示取决于所选取的惯性参考系.同一个四位置在不同惯性系中的坐标,可以利用洛伦兹矩阵作为过渡矩阵来相互转化.

两个事件之间的四位置之差,称为这两个事件之间的\textbf{四位移(4-displacement)}.

\subsubsection{四速度}

假设某质点在三维空间中运动.经典物理中认为,质点轨迹上某一点的速度是一个向量,其方向与该点处轨迹相切.但速度的大小具体是多少,三维轨迹完全没有提供足够的信息.同样的轨迹完全可以是用不同的瞬时速度来走过的.

为了完全从几何角度描述该质点的速度,我们可以把视角提高到四维时空,将三维的轨迹拉升到四维,这样就有充足的信息来描述质点的速度大小了,而三维轨迹就是四维轨迹在三维空间中的投影.取四维轨迹上某一点的切向量$\vec{v}$,使其在时间轴上的投影为$1$(单位时间长度),那么$\vec{v}$在三维空间中的投影就是速度.

这个例子启发我们研究四维速度比研究三维速度更加全面,由此有了以下概念:

\begin{definition}{四速度}

一个质点的\textbf{四速度},定义为质点的四维运动轨迹的切向量$\vec{v}$,满足$\vec{v}$在瞬时自身系中的时间轴投影长度是$1$.

\end{definition}

