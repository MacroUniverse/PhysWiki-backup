% 黎曼联络
% 仿射联络|Riemannian connection|联络|挠率|度量|黎曼度量|Riemannian metric|流形|manifold

\pentry{仿射联络\upref{affcon},线性代数}

\subsection{黎曼流形}

\textbf{黎曼度量与伪黎曼度量}\upref{RiMetr}词条中包含了黎曼度量的概念,但其中使用了高度凝练的术语,对初学者并不友好,因此我们在这里顺便给出更通俗的定义.

\begin{definition}{黎曼度量}
给定一个实流形$M$,定义其上一个映射$g$,它将$M$上任意点处的两个切向量映射为一个数字.如果$g$满足对于任意向量$x, y$,都有:
\begin{enumerate}
\item \textbf{线性性}:$g(ax, by)$
\item \textbf{对称性}:$g(x, y)=g(y, x)$;
\item \textbf{正定性}:$g(x, y)\geq 0$,且仅在$x=0$时有$g(x, x)=0$.
\end{enumerate}
则称$g$是$M$上的一个\textbf{黎曼度量(Riemannian metric)}.
\end{definition}

从定义可知,黎曼度量实际上就是规定了“如何做内积”,进而得到“切向量的长度”、“切向量之间的角度”等概念.注意,黎曼度量只对同一个切空间中的向量有用,不同切空间的向量之间无法定义黎曼度量.

如果在某个切点附近给定一个图(chart),那么黎曼度量可以表达为这个图中的一个矩阵$g_{ab}$,而该切点处两个切向量$x^a, y^b$的内积就是$g_{ab}x^ay^b$.

\textbf{对称性}意味着$g_{ab}\equiv g_{ba}$.如果将$g_{ab}$写成方阵\footnote{注意,按照我们的规范表达,$g_{ab}$应为“行矩阵的行矩阵”,所以这里说的方阵实际上是指把第$a$行$b$列的元素定义为$g_{ab}$.当然,反过来把第$b$行$a$列的元素定义为$g_{ab}$也行.},那么它就是线性代数中讨论的\textbf{正定矩阵}.这就是\textbf{黎曼度量与伪黎曼度量}\upref{RiMetr}词条中“对称”与“正定”二词的含义.“截面”实际上就是指这是一个映射.


\begin{definition}{黎曼流形}
给定实流形$M$.若在$M$上处处定义一个黎曼度量$g$,且在任意图中$g$的坐标都是光滑函数,则称$(M, g)$为一个\textbf{黎曼流形(Riemannian manifold)}.
\end{definition}





\subsection{黎曼联络}

\begin{definition}{黎曼联络}
给定黎曼流形$(M, g)$,如果$M$上有一个仿射联络$\nabla$,满足以下条件:
\begin{enumerate}
\item \textbf{挠率为零(无挠)}:对于任意光滑向量场$X, Y\in\mathfrak{X}(M)$,都有$\nabla_XY-\nabla_YX={[X, Y]}$;
\item \textbf{与黎曼度量相容}:对于任意光滑向量场$X, Y, Z\in\mathfrak{X}(M)$,都有$Zg(X, Y)=g(\nabla_ZX, Y)+g(X, \nabla_ZY)$.
\end{enumerate}
则称$\nabla$是$(M, g)$上的一个\textbf{黎曼联络(Riemannian connection)}.
\end{definition}

黎曼联络中的“相容”条件,实际上就是方向导数对内积的Leibniz律.注意$g(X, Y)$是$M$上的一个光滑函数,而$Zg(X, Y)$就是对这个函数沿着$Z$的方向求方向导数的结果.我们已经知道,光滑函数的求导不依赖于图的选择,因此$Zg(X, Y)$不必表达成$\nabla_Zg(X, Y)$的形式.

同一个流形上可能有多个不同的仿射联络,但是一旦确定了黎曼度量,则也就唯一确定了一个黎曼联络.我们通过一系列命题来讨论这件事.







\begin{lemma}{}
给定一个黎曼流形$(M, g)$,如果对于任意光滑向量场$Z$,都能计算出$g(X, Z)$,那么$X$唯一确定.
\end{lemma}

\textbf{证明}:

假设存在向量场$X, Y$使得对于任意$Z$都有$g(X, Z)=g(Y, Z)$

\textbf{证毕}.






\begin{theorem}{}
给定一个黎曼流形$(M, g)$,其上有一个仿射联络$\nabla$.如果有对于任意光滑切向量场$X, Y, Z\in\mathfrak{X}(M)$,有:
\begin{enumerate}
\item $\nabla_XY-\nabla_YX-[X, Y]=T$,其中$T\in \mathfrak{X}(M)$是一确定的光滑向量场;
\item $Zg(X, Y)=g(\nabla_ZX, Y)+g(X, \nabla_ZY)$.
\end{enumerate}
则$\nabla$存在且唯一确定.
\end{theorem}

\textbf{证明}:

只需要证明,根据已知条件,我们可以用黎曼度量$g$计算出$\nabla$即可.

首先列出\textbf{两组}条件:

\begin{equation}\label{RieCon_eq1}
\nabla_XY-\nabla_YX-[X, Y]=T
\end{equation}
和
\begin{equation}\label{RieCon_eq2}
Xg(Y, Z)=g(\nabla_XY, Z)+g(Y, \nabla_XZ)
\end{equation}

\begin{equation}\label{RieCon_eq3}
Yg(Z, X)=g(\nabla_YZ, X)+g(Z, \nabla_YX)
\end{equation}

\begin{equation}\label{RieCon_eq4}
Zg(X, Y)=g(\nabla_ZX, Y)+g(X, \nabla_ZY)
\end{equation}


其中\autoref{RieCon_eq1} 是“给定挠率”条件,\autoref{RieCon_eq2} 、\autoref{RieCon_eq3} 和\autoref{RieCon_eq4} 是定理条件中的第二条,也就是“和黎曼度量相容”条件.

将\autoref{RieCon_eq1} 代入\autoref{RieCon_eq3} ,可以得到
\begin{equation}\label{RieCon_eq5}
Yg(Z, X)=g(\nabla_YZ, X)+g(Z, \nabla_XY)-g(Z, [X, Y])
\end{equation}

计算(\autoref{RieCon_eq2} $-$ \autoref{RieCon_eq4} $+$ \autoref{RieCon_eq5} )可得:
\begin{equation}
\begin{aligned}
2g(\nabla_XY, Z)=&Xg(Y, Z)+Yg(X, Z)-Zg(X, Y)\\&-g(X, [Y, Z])+g(Y, [Z, X])+g(Z, [X, Y])\\&-g(T, Z)
\end{aligned}
\end{equation}

由于
\textbf{证毕}.



