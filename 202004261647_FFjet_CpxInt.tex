% 复变函数的积分概念

\pentry{定积分\upref{DefInt}}
我们之前已经接触过了实函数的积分,那么我们如何推广到复数上呢?

\subsection{复积分的定义}

实际上,同高等数学一样,也采用“分割”、“作和”、“取极限”的步骤来定义积分.
\begin{definition}{复积分的定义}
设$C $为一条起点在$a $,终点在$b $的有向光滑曲线(或逐段光滑曲线),其方程为
\begin{equation}
z=z(t)=x(t)+\mathrm{i} y(t) \quad,(\alpha \leqslant t \leqslant \beta, a=z(\alpha), b=z(\beta))
\end{equation}
函数$ f (z) $定义在$C $上用一组点$z_{0}=a, z_{1}, z_{2}, \cdots, z_{n-1}, z_{n}=b$沿曲线从$ a $到$b $,对曲线$C$进行分割:
\begin{figure}[ht]
\centering
\includegraphics[width=10cm]{./figures/CpxInt_1.png}
\caption{复积分的定义} \label{CpxInt_fig1}
\end{figure}
设$\Delta z_{k}=z_{k}-z_{k-1}$,$\zeta_k$为弧$z_{k-1}z_k$上任意一点,作和$\displaystyle S_{n}=\sum_{k=1}^{n} f\left(\zeta_{k}\right) \Delta z_{k}$.

当分割点的数量无限增加,并且分割$C$所得各个弧段长度中的最大值$d \to 0$时,不论对$C$的分法及$\zeta_k$的取法如何,$S_n$存在极限$S$,则称$ f (z)$沿$C $(从$a$到$b$)可积,称$ S $为$ f (z)$沿$C$(从$a$到$b $)的积分,记作
\begin{equation}
S=\int_{C} f(z) \mathrm{d} z
\end{equation}
其中 $f (z)$称为\textbf{被积函数},$C$称为\textbf{积分路径},而$\displaystyle S=\lim _{n \rightarrow \infty \atop d \rightarrow 0} S_{n}$.
\end{definition}

现在来看一下用定义计算复积分的一个简单例子.
\begin{example}{}
设$C $是一条起点为$a $终点为$b $的逐段光滑曲线,试计算$\displaystyle S=\int_C\mathrm dz$.

用定义来计算该积分.由定义,将$f(z)=1$带入,得:
\begin{equation}
S_{n}=\sum_{k=1}^{n}\left(z_{k}-z_{k-1}\right)=b-a
\end{equation}
所以
\begin{equation}
\int_{C} \mathrm{d} z=\lim _{d \rightarrow 0} S_{n}=b-a
\end{equation}
\end{example}

通过这道例题,我们发现对于函数$f(z)=1$,它的积分值只依赖于积分路径$C$的起点$a$和终点$b$,而与积分路径的形状是无关的.这个性质对于更一般的$f$是否成立?在后面的柯西-古萨定理中,我们将给出答案.

有了积分定义后,最先关心的问题是:积分存在的条件,积分的性质与积分的计算.下面就来讨论这几个问题.

\subsection{复积分存在的一个条件}

为了寻求复变函数积分存在的条件,现在唯一可利用的只有定义.于是问题就归结为考察极限$\displaystyle \lim _{n \rightarrow \infty \atop d \rightarrow 0} S_{n}$的存在条件.为此,不妨将$S_n$变形后再加以考察.

设
\begin{equation}
\begin{array}{l}f(z)=u(x, y)+i v(x, y) \\ \Delta z_{k}=\Delta x_{k}+\mathrm{i} \Delta y_{k} \\ \Delta x_{k}=x_{k}-x_{k-1}, \Delta y_{k}=y_{k}-y_{k-1} \\ \zeta_{k}=\xi_{k}+\mathrm{i} \eta_{k} \\ u_{k}=u\left(\xi_{k}, \eta_{k}\right), v_{k}=v\left(\xi_{k}, \eta_{k}\right)\end{array}
\end{equation}
于是
\begin{equation}
\begin{aligned} \int_{C} f(z) \mathrm{d} z &=\lim _{n \rightarrow \infty \atop d \rightarrow 0} S_{n}=\lim _{n \rightarrow \infty \atop d \rightarrow 0} \sum_{k=1}^{n} f\left(\zeta_{k}\right) \Delta z_{k}=\lim _{n \rightarrow \infty \atop d \rightarrow 0} \sum_{k=1}^{n} f\left(\xi_{k}+\mathrm{i} \eta_{k}\right)\left(\Delta x_{k}+\mathrm{i} \Delta y_{k}\right) \\ &=\lim _{n \rightarrow \infty \atop d \rightarrow 0} \sum_{k=1}^{n}\left[u\left(\xi_{k}, \eta_{k}\right)+\mathrm{i} v\left(\xi_{k}, \eta_{k}\right)\right]\left[\Delta x_{k}+\mathrm{i} \Delta y_{k}\right] \\ &=\lim _{n \rightarrow \infty \atop d \rightarrow 0} \sum_{k=1}^{n}\left(u_{k}+\mathrm{i} v_{k}\right)\left(\Delta x_{k}+\mathrm{i} \Delta y_{k}\right) \\ &=\lim _{n \rightarrow \infty \atop d \rightarrow 0} \sum_{k=1}^{n}\left(u_{k} \Delta x_{k}-v_{k} \Delta y_{k}\right)+\mathrm{i} \lim _{n \rightarrow \infty \atop d \rightarrow 0} \sum_{k=1}^{n}\left(v_{k} \Delta x_{k}-u_{k} \Delta y_{k}\right) \end{aligned}
\end{equation}
由高等数学知道,当$u(x, y)$与$v(x, y)$在$C$上连续时,上式右端两个极限存在,且分别为$\displaystyle \int u \mathrm{d} x-v \mathrm{d} y $与$\displaystyle \int v \mathrm{d} x+u \mathrm{d} y$.

至此,获得积分$\displaystyle \int_{c} f(z) \mathrm{d} z$存在的一个条件是$u(x, y)$与$v(x, y)$均在$C$上连续.而$u(x, y)$与$v(x, y)$在$C$上连续等价于$ f (z)$在$C$上连续,所以,函数$ f (z)$在$C$上连续是积分$\displaystyle \int_{c} f(z) \mathrm{d} z$存在的一个条件.

并且通过上述计算我们还发现了如下定理:
\begin{theorem}{} 
若$ f (z) = u(x, y) + \mathrm iv(x, y)$在曲线$C$上连续,则$f (z)$沿$C$可积,并有
\begin{equation} \label{CpxInt_eq1}
\int_{C} f(z) \mathrm{d} z=\int_{C} u \mathrm{d} x-v \mathrm{d} y+\mathrm{i} \int_{C} v \mathrm{d} x+u \mathrm{d} y
\end{equation}
\end{theorem}

这个积分的重要之处有三:第一,它给出了复变函数积分存在的一个充分条件;第二,它提供了计算复变函数积分的一种方法;其三,\autoref{CpxInt_eq1}表明,研究复变函数的积分问题,可以转化为研究实变量的二元实值函数沿曲线$C $的线积分问题.

\subsection{复积分的性质与计算}

上面我们看到了将线积分和复积分联系起来的\autoref{CpxInt_eq1}.容易想到,线积分的一些性质可移到复变函数的积分上来.

若$ f (z)$与$g (z)$沿曲线$C,C^-$($C^-$表示与曲线$C$方向相反的同一条曲线)可积,则有
\begin{enumerate}
\item \begin{equation}
\int_{C} A f(z) \mathrm{d} z=A \int_{C} f(z) \mathrm{d} z \quad(A \text { 为复常数 })
\end{equation}
\item \begin{equation}
\int_{C}[f(z) \pm g(z)] \mathrm{d} z=\int_{C} f(z) \mathrm{d} z \pm \int_{C} g(z) \mathrm{d} z
\end{equation}
\item \begin{equation}
\int_{C} f(z) \mathrm{d} z=-\int_{C^{-}} f(z) \mathrm{d} z
\end{equation}
\item \begin{equation}
\int_{C} f(z) \mathrm{d} z=\int_{C_{1}} f(z) \mathrm{d} z+\int_{C_{2}} f(z) \mathrm{d} z\left(C由 C_{1} \text { 与 } C_{2} \text { 首尾相接而成 }\right)
\end{equation}
\item 设$ L$为曲线$C$的长度,若$ f (z)$沿$C$可积,且在$C$上满足$ f (z) \leqslant M $,则
\begin{equation} \label{CpxInt_eq2}
\left|\int_{c} f(z) \mathrm{d} z\right| \leqslant \int_{c}|f(z)| \mathrm{d} s \leqslant M L
\end{equation}
\end{enumerate}

\autoref{CpxInt_eq2}提供了一种估计复变函数积分的模的方法.

到现在为止,计算复变函数积分只有两种方法,一是定义,二是\autoref{CpxInt_eq1}.有无其他方法呢?

我们注意到,由于积分路径常取光滑曲线(或逐段光滑曲线),所以$ f (z) $沿曲线$C $的积分可归结为$f [z(t)]$关于曲线$C$的参数的积分.