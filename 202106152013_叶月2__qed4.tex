% 狄拉克方程
% dirac function

\subsection{背景}
薛定谔方程并不能描述高速运动的微观粒子,而且由于不满足洛伦兹协变性,无法保证在别的惯性系成立.自然人们希望有满足狭义相对论及量子力学的理论出现.1927年, Klein-Gordon 方程被顺势提出.该方程满足狭义相对论中的质壳条件($E^{2}=\boldsymbol{p}^{2}+m^{2}$),描述的是无自旋的自由粒子:
\begin{equation}
\left(\frac{\partial^{2}}{\partial t^{2}}-\nabla^{2}+m^{2}\right) \phi(\boldsymbol{x}, t)=0
\end{equation}
然而它产生两个无法忽视的问题:

\begin{enumerate}
\item \textbf{负能解}
\item \textbf{负概率密度}
\end{enumerate}
\subsection{狄拉克方程}
\subsubsection{描述对象}
\begin{enumerate}
\item \textbf{XXXX}
\item \textbf{xxxx}
\end{enumerate}
\subsubsection{对费米子负能解的解释——空穴理论}


