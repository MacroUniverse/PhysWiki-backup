% 悬链线

\pentry{双曲函数\upref{TrigH}}

\footnote{参考 Wikipedia \href{https://en.wikipedia.org/wiki/Catenary}{相关页面}.}\textbf{悬链线(catenary)}在物理上是指一条粗细不计的、 质量均匀分布的柔软绳子两端悬挂在相同高度的两个点后(\autoref{Catena_fig1} )当绳子在重力作用下达到平衡后形成的曲线.

\begin{figure}[ht]
\centering
\includegraphics[width=7cm]{./figures/Catena_1.pdf}
\caption{悬链线} \label{Catena_fig1}
\end{figure}
以绳子的最低点作为原点建立坐标系, 水平向右为 $x$ 轴, 竖直向上为 $y$ 轴. 那么悬链线可以用函数 $y(x)$ 来表示为
\begin{equation}
y(x) = \frac{1}{k}\cosh(kx)
\end{equation}
其中 $\cosh$ 是双曲余弦函数双曲函数\upref{TrigH}, 常数 $k$ 满足
\begin{equation}
\sinh(\frac{ka}{2}) = \frac{L}{2}
\end{equation}
其中 $a$ 是两个悬挂点之间的距离, $L$ 是绳子的总长度.

\subsection{推导}
\pentry{曲线的长度\upref{CurLen}, 常微分方程\upref{ODE}}


\centering
\includegraphics[width=8cm]{./figures/Catena_2.png}
\caption{受力分析} \label{Catena_fig2}
\end{figure}
假设原点处的张力为 $T$, 绳的线密度为 $\lambda$, 那么区间 $[0, x]$ 的曲线长度为(\autoref{CurLen_eq1}~\upref{CurLen})
\begin{equation}\label{Catena_eq1}
L(x) = \int_0^x \sqrt{1 + \dot y(x')^2} \dd{x'}
\end{equation}
其中 $\dot y$ 表示函数 $y(x)$ 的导函数, 下文中 $\ddot y$ 则表示二阶导函数). 区间 $[0, x]$ 所受重力为 $G = \lambda L g$. 根据受力分析, $x$ 点的斜率为 $G/T$, 这样就得到了悬链线 $y(x)$ 的微分—积分方程
\begin{equation}
\dot y = \frac{g\lambda}{T} \int_0^x \sqrt{1 + \dot y^2} \dd{x'}
\end{equation}
两边对 $x$ 再次求导得二阶微分方程
\addTODO{如何对积分上限求导?放链接}
\begin{equation}
\ddot y = k \sqrt{1 + \dot y^2}
\end{equation}
其中令 $k = g\lambda/T$. 注意这是一个非线性二阶常微分方程. 可以证明它的通解, 即悬链线为
\begin{equation}\label{Catena_eq2}
y(x) = \frac{1}{k}\cosh(kx)
\end{equation}

那么, 如果已知两个悬挂点之间距离为 $a$, 绳子总长度为 $L$, 以及 $\lambda, g$. 如何求出 $k$ 或拉力 $T$ 呢? 根据\autoref{Catena_eq1} 有限制条件
\begin{equation}
\frac{L}{2} = \int_0^{a/2} \sqrt{1 + \dot y^2} \dd{x'}
\end{equation}
把\autoref{Catena_eq2} 代入有
\begin{equation}
\sinh(\frac{ka}{2}) = \frac{L}{2}
\end{equation}
这样就可以解出 $k$, 进而求出 $T$.
