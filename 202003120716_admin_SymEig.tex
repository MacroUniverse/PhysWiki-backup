% 对称矩阵的本征问题
% 对称矩阵|本征值|本征矢|正交归一|本征方程|线性代数

\pentry{矩阵的本征方程\upref{MatEig}, 厄米共轭\upref{HerMat}}

在物理中, 我们遇到的本征问题中的矩阵往往是对称矩阵或厄米矩阵. % 未完成: 举例, 链接到振动的简正模式
我们下面来证明 $N$ 维对称矩阵 $\mat A$ 存在 $N$ 个两两正交归一的本征矢 $\bvec v_1, \dots, \bvec v_N$, 且本征值都是实数. 另外, 本征矢一般用实数表示, 但也可以乘以一个任意复数.

% 未完成:举例最重要!

\subsection{证明本征值为实数}
本征方程为
\begin{equation}
\mat A \bvec v_i = \bvec v_i
\end{equation}
我们先假设本征值和本征矢都可能是复数, 将本征方程左边乘以 $\bvec v_i$ 得
\begin{equation}
\bvec v_i\Her \mat A \bvec v_i = \lambda_i \bvec v_i\Her \bvec v_i
\end{equation}
将等式两边取厄米共轭(注意矢量也可以看成矩阵), 由\autoref{HerMat_eq2}\upref{HerMat} 和\autoref{HerMat_eq1}\upref{HerMat} 可得
\begin{equation}
\bvec v_i\Her \mat A\Tr \bvec v_i = \bvec v_i\Her \mat A \bvec v_i = \lambda_i^* \bvec v_i\Her \bvec v_i
\end{equation}
对比两式, 得 $\lambda_i = \lambda_i^*$, 所以 $\lambda_i$ 必为实数.

将本征方程记为
\begin{equation}
(\mat A - \lambda_i\mat I)\bvec v_i = \bvec 0
\end{equation}
就会发现这是一个实系数的齐次方程组, 所以必然能把所有线性无关的解都用实数矢量表示. 但注意把任意解 $\bvec v_i$ 乘以一个任意复数, 同样也是方程的解.

\subsection{证明本征矢的正交性}
简并空间内, 我们可以认为地指定正交归一基底, 所以只需要证明不同本征值对应的本征矢正交即可.
\begin{equation}
s = \bvec v_1\Tr (\mat A \bvec v_2) = \bvec v_1\Tr (\lambda_2 \bvec v_2) = \lambda_2 \bvec v_1\Tr \bvec v_2
\end{equation}
使用矩阵乘法结合律\autoref{Mat_eq1}\upref{Mat} 以及\autoref{HerMat_eq2}\upref{HerMat} 得
\begin{equation}
s = (\mat A \bvec v_1)\Tr \bvec v_2 = \lambda_1^* \bvec v_1\Tr \bvec v_2 = \lambda_1 \bvec v_1\Tr \bvec v_2
\end{equation}
以上两矢相等, 因为 $\lambda_1 \ne \lambda_2$, 所以 $\bvec v_1\Tr \bvec v_2 = 0$.

事实上, 厄米矩阵也可以定义为满足
\begin{equation}
\bvec v_1\Tr (\mat A \bvec v_2) = (\mat A \bvec v_1)\Tr \bvec v_2
\end{equation}
的矩阵.
