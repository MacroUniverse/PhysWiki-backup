% 线性映射和线性变换

\begin{issues}
\issueOther{需要给出线性变换的例子.}
\end{issues}

% 未完成:从几何矢量出发, 给出 “线性” 的定义, 给出一些平面线性变换的例子(包括投影变换), 然后再总结出一般的代数形式, 给出矩阵表示

\pentry{矢量空间\upref{LSpace}}

\subsection{线性映射}
\begin{definition}{线性映射}\label{LinMap_def1}
给定域$\mathbb{F}$上的线性空间$V$和$U$.如果有映射$f:V\rightarrow U$满足,对于任意的向量$\bvec{v}_1, \bvec{v}_2\in V$和标量$a_1, a_2\in\mathbb{F}$,都有$f(a_1\bvec{v}_1+a_2\bvec{v}_2)=a_1f(\bvec{v}_1)+a_2f(\bvec{v}_2)$,那么称$f$是$V$到$U$的一个\textbf{线性映射(linear map)}.
\end{definition}


\autoref{LinMap_def1} 的内涵比看上去广一些,不仅仅是对$V$中两个向量的线性组合成立.对于任意的一组有限个向量$\{\bvec{v}_i\}\subseteq V$和一组对应的标量$\{a_i\}\subseteq\mathbb{F}$,都有$f(\sum_i a_i\bvec{v}_i)=\sum_i a_if(\bvec{v}_i)$.

如果$\{\bvec{e}_i\}_{i=1}^n$是$V$的一组基,那么任意$\bvec{v}\in V$都可以唯一地表示为$\bvec{v}=\sum_i c_i\bvec{e}_i$的形式,其中$c_i\in\mathbb{F}$.这样,由于线性性,$f(\bvec{v})=\sum_ic_if(\bvec{e}_i)$.也就是说,只需要知道了基向量被映射到哪里,也就可以计算出任意向量映射到哪里.于是,和线性函数一样,确定一个线性映射的时候,我们最多只能自由选择基向量映射到哪里,只不过这里的函数值不再是数字,而是$U$中的向量.

在矢量空间的表示\upref{VecRep}中我们还会看到,选定两个空间的基以后,一个线性映射也可以看成是多个线性函数的排列,因此线性映射和线性函数性质很相似.

线性函数是一种特殊的线性映射, $V$ 上的所有线性函数组成的矢量空间叫做\textbf{对偶空间}\upref{DualSp}.

\subsubsection{线性映射的例子}

\begin{example}{力映射到加速度}

考虑一个固定在光滑无限长滑杆上的物块,物块的运动是一维的,因此可以把物块所有可能的加速度之集合表示为一个一维实线性空间.如果有除了滑杆之外的动力作用在这个物块上,比如推拉它,那么物块就会在动力作用下产生加速度.所有可能的动力之集合构成了一个三维实线性空间.把动力向量映射为该动力所造成的加速度向量,那么这个映射就是一个三维实线性空间到一维实线性空间的线性映射.

\end{example}


\subsection{线性变换}
线性空间$V$上的一个线性变换$T$,是指把$V$中的任意向量$\bvec{v}$“变成”另一个向量$T\bvec{v}$的\textbf{操作}.这个操作满足线性性,因此被称作线性变换;线性性的好处在于我们只需要讨论基向量被映射到哪里,就知道了任何向量会被映射到哪里.

我们也可以把线性变换理解为一种特殊的线性映射,即线性空间到自身的映射;或者说,每个向量都映射到它所“变成”的那个向量.线性变换的准确定义如下:

\begin{definition}{线性变换}
给定线性空间$V$,如果$T$是$V$到$V$上的线性映射,那么称$T$是一个$V$上的\textbf{线性变换(linear transformation)}.$T$将$\bvec{v}\in V$映射为$T\bvec{v}$.
\end{definition}

要知道一个具体的线性变换的性质,我们只需要知道它把基向量都映射到哪里去了.我们知道,基向量张成线性空间$V$本身,而任意一组基向量变换后所得到的向量组,张成了线性变换的\textbf{像空间},即$V$中全体向量变换后的结果之集合.如果基向量变换后的向量组还是线性无关的,那么这个向量组也是一组基,从而像空间就是$V$本身;但有的时候变换后的向量组线性相关了,这时候的像空间就是$V$的一个真子集了.极端情况下甚至把所有向量都变换为零向量,这样的线性变换的像空间就只有零向量了.

如果一个线性变换的像空间是$V$的一个真子集,那么这个真子集一定是$V$中的一个“过原点的平面”,我们把这种线性变换称为\textbf{退化(degenerate)}的.反过来,$V$中的一个“过原点的平面”可以是某线性变换的像空间.一个线性变换唯一对应一个像空间,但是一个像空间总是对应无穷多个线性变换.

如果$\bvec{v}\in V$在基$\{\bvec{e}_i\}$中表示为$\sum a_i\bvec{e}_i$,那么它变换后的向量就是$T\bvec{v}=T\sum a_i\bvec{e}_i=\sum a_iT\bvec{e}_i$.如果向量组$\{T\bvec{e}_i\}$是线性相关的,那么就意味着存在并非全为零的$\{a_i\}$使得$T\bvec{v}=\sum a_iT\bvec{e}_i=0$;而$\{a_i\}$并非全为零意味着$\bvec{v}$不是零向量.这就是说,退化的线性变换会把一些非零向量变换成零向量.

\begin{exercise}{}
如果线性变换非退化,那么非零向量有可能被变换为零向量吗?提示:退化和非退化的区别在于基做了线性变换后得到的向量组还是不是线性无关的,或者说还是不是一组基.
\end{exercise}

退化的线性变换会把非零向量变成零向量这一事实,引申出了零化子的概念:

\begin{definition}{零化子空间}
给定线性空间$V$上的线性变换$T$.$V$中被变换为零向量的全体向量所构成的集合$\{\bvec{v}\in V|T\bvec{v}=\bvec{0}\}$,称为线性变换$T$的\textbf{零化子空间(null-space)},简称\textbf{零化子}.
\end{definition}

用零化子的语言来说,一个线性变换是非退化的,当且仅当其零化子是$\{\bvec{0}\}$.

\subsubsection{线性变换的例子}

\subsection{总结与拓展}

到目前为止,我们对于向量的描述依然是抽象的,并未涉及许多具体的性质,如向量垂直、向量长度、向量坐标等概念.

在将来的词条中,我们会看到如何在给定了具体的基时用矩阵来描述向量和线性变换,以及当基的选择变化时,这些矩阵该如何变化.

零化子空间是一种子空间,而我们也会在将来的词条中解释子空间的概念.



