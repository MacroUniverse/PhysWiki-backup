% 群
\pentry{逻辑量词, 整数\upref{intger}}

\subsection{基本概念}

\begin{definition}{二元运算}\label{Group_def2}
给定一个非空集合$A$,取$A$中任意两个元素$a, b$(可能是同一个元素). 如果有一条规则使得两个元素可以组合,指向另一个元素 $c$(不一定属于 $A$), 则称这条规则为一个\textbf{二元运算(binary operation)}. 如果 $c\in A$, 那么我们说这个二元运算是\textbf{封闭的(closed)},有的地方也会称作“\textbf{闭合的}”.
\end{definition}

运算的符号可以任意决定,如果使用$\cdot$作为运算符,那么定义中的情况就可以简单写为$a· b=c$. 注意,这里的 $\cdot$ 只是表示某一个运算,不一定是我们通常的乘法或点乘运算.

和二元运算类似,我们也可以称更多元素间相组合的规则为\textbf{多元运算}.

\begin{definition}{群}\label{Group_def1}
一个群$(G, \cdot)$是在集合$G$上赋予了一个二元运算$\cdot$的结构,该运算满足以下要求:\\

\begin{enumerate}
\item \textbf{封闭性(closure)}:$∀x, y∈ G, x· y∈ G$,即任意$G$中元素$x$,$y$满足$x· y$仍是$G$中元素
\item \textbf{结合性(associativity)}:$∀ x, y, z\in G, x·(y· z)=(x· y)· z$
\item \textbf{单位元(identity element)存在性}:$∃ e\in G, ∀ x∈ G, e· x=x· e=x$
\item \textbf{逆元(inverse element)存在性}:$∀ x∈ G, ∃ y∈ G, x· y=y· x=e$.通常我们会把这样的$y$称作$x$的逆元,并记为$x^{-1}$
\end{enumerate}
\end{definition}
严格来说,这样的一个群应该表示为$(G,\cdot)$,而$G$表示的是没有赋予运算的集合.但是为了方便讨论,我们通常也会直接用定义群的这个集合来称呼这个群,比如简单地把上述定义的群叫做\textbf{群$G$.}

实际上,我们可以用更为弱化的公理系统来定义群,比如第 4 条只要求存在左逆元, 即只要求$∀ x∈ G, ∃ y∈ G, y· x=e$. 在这种情况下我们仍然可以证明左逆元都是右逆元(\autoref{Group_the1}).有很多不同的弱化版本公理系统也能等价地定义出群来,但是为了方便理解,我们用了以上对称的公理系统.

集合的元素数量被称为集合的基数或势\upref{Set},而群$G$的元素数量也可以称为群的\textbf{阶(order)},记作$|G|$. 以后如果没有特别说明,默认将群元素$x$的逆元记为$x^{-1}$,将群的单位元记为$e$.

\subsection{群的例子}

\begin{exercise}{二元群}\label{Group_exe1}
定义一个只含有两个元素的集合,记为$\{0, 1\}$.在这个集合上定义运算 “$\cdot$”,由于只有四种运算方式,所以可以通过列举出每一个运算的结果来定义这个运算:
\begin{equation}\label{Group_eq1}
0\cdot0=0 \qquad 0\cdot1=1 \qquad 1\cdot0=1 \qquad 1\cdot1=0
\end{equation}
\begin{itemize}
\item 请用一个 $2\times2$ 的表格表示运算规则
\item 请根据\autoref{Group_def1}验证这个二元集合配上运算 $\cdot$ 构成一个群
\item 尝试改变\autoref{Group_eq1} 的规则使其同样构成一个群
\end{itemize}
\end{exercise}

在以上例子中,尽管$0\cdot1=1\cdot0$,我依然把它们分别单独写了出来,这是因为群的定义不要求\textbf{交换律}成立,也就是说, 群运算允许 $x\cdot y\neq y\cdot x$.群元素选为$0$和$1$没有特殊原因,只是代表这是群里两个不同的元素而已,任何由两个元素构成的群我们都看作同一个. 运算满足交换律的群被称为\textbf{阿贝尔群(abelian group)}或\textbf{交换群(commutative group)}, 否则称为\textbf{非阿贝尔群(non-abelian group)}或\textbf{非交换群(non-commutative group)}. 习惯上,我们把阿贝尔群的运算叫做加法,记为“$+$”,而把非阿贝尔群的运算叫做乘法,记为“$\cdot$”,甚至简化为没有符号,比如$ab\not= ba$.不过,即使是非阿贝尔群中也可能存在两个元素$a$和$b$,使得$ab=ba$;这时我们说$a$和$b$ \textbf{交换(commute)}.

一般地,由于在朴素集合论\upref{Set}中我们最多只讨论了集合的基数问题,集合中的元素具体如何命名是没有约束的,因此在集合论意义下元素数目相同的集合都看作同一个.比如说,我们认为$\{\text{猪},3, K\}$和$\{1,2,3\}$是同一个集合.而现在在集合上定义了一个群运算以后所得到的群,即使构成它们的集合相同,群也可能由群运算的不同而产生不同的结构,从而被看作是不同的群.

\begin{example}{整数加法群}\label{Group_ex1}
所有整数的集合 $\mathbb Z$, 配合通常的整数加法运算构成一个群.
\end{example}

\begin{example}{$n$ 元循环群}\label{Group_ex2}
取一个由$n$个元素组成的集合$G$,由于集合元素命名的任意性,不妨把$G$记为$\{0, 1, \cdots n-1\}$,定义运算为模$n$的加法,即在一个有$n$个整点的钟表上的加法(见“整数\upref{intger}”).那么这个运算构成$G$上的一个群运算,所构成的群$G$称为 \textbf{$n$元循环群(n-element cyclic group)},通常记为$C_n$或者$\mathbb{Z}_n$.

命名为$C_n$是取“cyclic”的含义,而命名为$\mathbb{Z}_n$是为了说明循环群是整数加法群$\mathbb{Z}$的商群(\autoref{Group1_ex4}\upref{Group1}),而商群是将来会提到的重要概念.
\end{example}

\begin{example}{$n$ 元置换群}\label{Group_ex3}
首先给定一个$n$元集合,记作$K=\{1,2, \cdots, n\}$,并将$K$中的元素按现有的顺序编号. 把$K$看作是$n$个桶中分别装了1个写着编号的球,初始状态下球的编号和桶的编号一致.我们可以把球从桶里面拿出来并进行任意的置换,保持每个桶里还是只有一个球,但是球的编号不一定和桶的编号一致了.每一个置换可以详细描述为“把1号桶的球和2号桶的球交换”,“把1号桶的球放入3号桶,3号桶的放入4号桶,4号桶的放入1号桶”等等.

我们用全体“置换”动作来作为元素,构成一个集合,称作$n$个元素的置换集合($n$元置换集),记为$S_n$.\textbf{从原始状态}进行任意置换,所得到的结果状态和置换是一一对应的,所以我们也可以用“从原始状态进行置换$f$所得的结果”来表示置换$f$本身.

置换之间可以定义一个运算“$\circ$”,被称为置换间的复合,它是这样定义的:如果$f$和$g$是两个置换,那么$g\circ f$就是先进行$f$置换,再进行$g$置换.注意先后次序是从右到左进行的.

我们也可以这样来理解一个置换:原始状态下,$n$号桶中的小球为$n$.进行一次$f$置换后,$n$号桶中的小球就变成了$f(n)$,再进行一次$g$置换,那么$n$号桶中现在装的小球就变为$g(f(x))$.这个过程也可以看成是进行了一次$g\circ f$运算,让$n$号桶中的小球变成$g\circ f(n)$.

现在我们有了一个由置换组成的集合以及置换之间的运算,把该集合和运算代入群的定义,会发现四个条件都被满足了.因此,$(K, \circ)$是一个群.注意,\textbf{绝大多数} $S_n$ 是\textbf{不交换(非阿贝尔)}的.

当$K$共有$n$个元素时,习惯上我们把这个群称为$S_n$.

$S_n$一共有$n!$个元素\footnote{$n$个桶各里装了1个小球,1号桶有$n$种装球的可能性,2号桶因此还剩下$n-1$中可能性,以此类推,这$n$个桶一共有$n\times(n-1)\times\cdots\times1$种装球的可能性,每种可能性对应一个从初始状态而来的置换方式.因此,置换的数量一共有$n\times(n-1)\times\cdots\times1=n!$种}.
\end{example}

在以上两个例子中可以看到,尽管元素数量一样,$\mathbb{Z}_{24}$和$S_4$的元素数量都是24,但是前者是阿贝尔群,后者则不交换,显然两个群的运算结构不可能一样.这是一个集合论意义上等价但群论意义上不等价的例子.

\begin{example}{对称群}\label{Group_ex4}
对称性的意思,是在某种条件下保持不变的性质.比如说,平面上的一个正方形,绕着几何中心旋转角度为$\pi$(即角度制下的$180^\circ$)的时候和没有旋转是一样的,那么我们就说正方形是关于角度为$\pi$的旋转对称的.所有能够使得这个正方形不变的平面变换,比如特定角度的旋转、关于特定轴的翻转,配上变换间的复合运算(即先进行一个变换,再进行另一个),构成了一个群,称为这个正方形的\textbf{对称群(symmetry group)}.
\end{example}
对称群的概念非常重要且常见.只要有“不变性”概念存在的地方就可能存在对称群.一个事物的对称群的结构揭示了这个事物的不变性的特点.
\begin{example}{$n$阶可逆方阵群}\label{Group_ex5}
给定域$F$上,全体$n\times n$可逆矩阵\upref{InvMat}构成的集合,配上矩阵乘法就构成了一个非阿贝尔群.%未完成: 链接到“域”论词条.

这样的群被简记为$GL(n, F)$.当不至于混淆时,这里的$F$一般会指全体实数或者全体复数,这时也会把该群简记为$GL(n)$. 多数情况下,$GL(n)$也是不交换的.

\end{example}

\subsection{结构性定理}

群的运算有一个很棒的唯一性,算是我们目前遇到的第一个具体的结构性特征,由以下定理描述:

\begin{theorem}{群运算的唯一性}\label{Group_the2}
给定一个群$G$, 若 对于$ a, b\in G$ 有某个$x\in G$使得 $ax=bx$,那么必然有$a=b$;类似地,如果$xa=xb$,也必然有$a=b$.
\end{theorem}

唯一性定理也可以由以上的逆否命题等价表述为: 群中任意一个元素分别左乘或右乘两个不同的元素, 结果必定不相同. 该定理的证明留作习题.提示:用$x^{-1}$去参与运算试试.

另外,运算的唯一性也被称为\textbf{消去律},原因显而易见,等式两边的相同项可以同时消去.

唯一性是所有群都有的性质,但是等到我们讨论环\upref{Ring}的时候,由于环的乘法不要求逆元一定存在,我们没法对环证明这个唯一性.事实上,很多环都没有唯一性.

\begin{exercise}{单位元的唯一性}
从\autoref{Group_the2} 可知左右单位元分别是唯一的. 请证明左单位 $e_1$ 元等于右单位元 $e_2$ (提示: 将它们相乘).
\end{exercise}

\begin{theorem}{逆元的唯一性}\label{Group_the1}
对任意 $x$, 假设它存在一个左逆元$a$,那么我们必然有$ax=e$.考虑结合性可知,$ae=a=ea=(ax)a=a(xa)$,于是$xa=e$,即$a$也是$x$的右逆元.也就是说,只要左逆元存在,那么右逆元也存在,并且等于左逆元. 反之亦然.
\end{theorem}

\begin{exercise}{}
证明 $(ab)^{-1} = b^{-1}a^{-1}$
\end{exercise}
