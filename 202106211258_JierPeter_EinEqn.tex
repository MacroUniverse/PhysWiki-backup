% 爱因斯坦场方程
% 引力场|引力|gravity|场方程|Ricci张量|测地线|geodesic|广义相对论|相对论|relativity|时空|spacetime|弯曲|曲率

\pentry{引力的弱场近似\upref{WeakG},尘埃云的能动张量\upref{SRFld},曲率张量场\upref{RicciC}}

\subsection{爱因斯坦张量}

我们直接介绍一个有用的性质,在稍后猜测爱因斯坦场方程的时候我们自然会讨论到它的用处.

\begin{definition}{Ricci标量曲率}
给定流形上的Ricci曲率场$R_{ij}$,则有光滑函数$R=g^{ij}R_{ij}$,称之为流形上的\textbf{标量曲率(scalar curvature)}.
\end{definition}

考虑黎曼曲率张量的第二Bianchi恒等式\autoref{RicciC_eq8}~\upref{RicciC}:
\begin{equation}
\nabla_\lambda R_{\mu\nu\rho\sigma}+\nabla_\rho R_{\mu\nu\sigma\lambda}+\nabla_\sigma R_{\mu\nu\lambda\rho}=0
\end{equation}

等式两端同时乘以$g^{\nu\sigma}g^{\mu\lambda}$后,考虑到联络对度量的相容性\footnote{即$\nabla_ag_{ij}=0$.},得到\footnote{注意负号的来源:$g^{\nu\sigma}g^{\mu\lambda}R_{\mu\nu\sigma\lambda}=-g^{\nu\sigma}g^{\mu\lambda}R_{\nu\mu\sigma\lambda}=-g^{\mu\lambda}R_{\mu\lambda}=R$}:
\begin{equation}
\begin{aligned}
0&=g^{\nu\sigma}g^{\mu\lambda}(\nabla_\lambda R_{\mu\nu\rho\sigma}+\nabla_\rho R_{\mu\nu\sigma\lambda}+\nabla_\sigma R_{\mu\nu\lambda\rho})\\
&=\nabla^\mu R_{\mu\rho}-\nabla_\rho R+\nabla^{\nu}R_{\nu\rho}
\end{aligned}
\end{equation}

也就是说
\begin{equation}\label{EinEqn_eq1}
\nabla^\mu R_{\rho\mu}=\frac{1}{2}\nabla_\rho R
\end{equation}

这么一来,如果我们定义一个张量
\begin{equation}
G_{\mu\nu}=R_{\mu\nu}-\frac{1}{2}Rg_{\mu\nu}
\end{equation}
那么\autoref{EinEqn_eq1} 就可以写为
\begin{equation}\label{EinEqn_eq2}
\nabla^\mu G_{\mu\nu}=0
\end{equation}

这个张量$G_{\mu\nu}$就被称为\textbf{爱因斯坦张量(Einstein tensor)},由于度量和Ricci张量的对称性,它也是对称的.\autoref{EinEqn_eq2} 就是我们稍后要用到的有用性质.



\subsection{能动张量}

爱因斯坦场方程的引出过程中,我们考虑的是最简单的宏观模型,即\textbf{尘埃云的能动张量}\upref{SRFld}中所介绍的\textbf{理想流体}.对于任何物质,其四动量的各分量,都会随着参考系的不同而有不同取值,因此这些量只能是流形上某种量在具体坐标系中的坐标分量而已.这样一来,在流形上描述能量质量分布的量,就不能是简单的光滑函数,或者说标量场,而只能是更高阶的张量场.能描述理想流体四动量分布的张量,可以是四动量本身,也可以是能动张量,而我们会选择能动张量,这样才能和Ricci曲率张量的阶数吻合.

%能动张量的讨论尚未确定最终流程




\subsection{爱因斯坦场方程}

\textbf{引力的弱场近似}\upref{WeakG}一节中,我们看到了作为平直时空的微小扰动,带曲率时空中的测地线确实能描述稳定、低速、弱场近似下的引力效应.我们现在希望利用这个原则,将它推广到任意情况下的时空中去.

曲率是引力的体现,而引力是由物质产生的.牛顿引力论中描述物质的引力效应使用的是物质的质量,在牛顿理论中这是时空中的一个光滑函数.问题是,从狭义相对论中我们就知道,描述物质分布时统一的、无视坐标系选择的量,应该是四动量.四动量本身

















