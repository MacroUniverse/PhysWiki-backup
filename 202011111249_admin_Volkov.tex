% Volkov 波函数
% 波函数|本征波函数|偶极子近似|速度规范|库仑势

\begin{issues}
\issueDraft
\issueMissDepend
\end{issues}

速度规范下, Volkov 波函数为
 \begin{equation}
\Psi_{\bvec k}^V(\bvec r, t) = (2\pi)^{-3/2} \exp\qty[\I \bvec k \vdot (\bvec r - \bvec \alpha(t)) - \I Et/\hbar]
\end{equation}
其中 $\bvec \alpha(t)$ 对应的是一个经典点电荷在电场中的位移.
\begin{equation}
\bvec \alpha(t) = -\frac{q}{m} \int_{t_0}^t \bvec A(t') \dd{t'}
\end{equation}
这是以下薛定谔方程的解(速度规范,偶极子近似)
\begin{equation}
\I\hbar \pdv{t} \Psi^V = \qty[\frac{\bvec p^2}{2m} - \frac{q}{m}\bvec A(t) \vdot \bvec p] \Psi^V
\end{equation}

要求长度规范只需要做一个规范变换即可(\autoref{EMHydr_eq3}~\upref{EMHydr} 和\autoref{EMHydr_eq4}~\upref{EMHydr}).

\subsubsection{长度规范}
\begin{equation}
\Psi_{\bvec k}^L(\bvec r, t) = (2\pi)^{-3/2} \exp\qty{[\bvec k + \bvec A(t)]\bvec r - \frac{1}{2}\int_{-\infty}^t [\bvec k + \bvec A(t')]^2 \dd{t'}}
\end{equation}

\subsubsection{加速度规范}
\begin{equation}
\Psi_{\bvec k}^A(\bvec r, t) = (2\pi)^{-3/2} \exp\qty{\bvec k \vdot \bvec r - Et}
\end{equation}
