% 对偶空间
% 线性映射|矢量空间

\pentry{矢量空间\upref{LSpace}, 映射\upref{map}}% 未完成

\subsection{线性函数}

中学数学中一个贯穿始终的研究对象就是\textbf{函数(function)}.函数是\textbf{映射(mapping)}的一种,在现代数学的习惯中通常把从任意集合到一个数字集合\footnote{数字集合就是我们通常叫做数字的元素的集合,比如实数集$\mathbb{R}$,有理数集$\mathbb{R}$,整数集$\mathbb{R}$,复数集$\mathbb{C}$等,甚至$\{0, 1\}$这样的集合都叫数字集合.}的映射就被称为一个函数.特别地,我们把定义线性空间时使用的域$\mathbb{K}$都当作数字集合,多数情况下这些域都是实数域、复数域或者它们的某种等价类划分\footnote{如整数\upref{intger}中提到的模运算,其元素也被称为数字,只不过不是通常意义上的整数.}.

如果给定一个域$\mathbb{K}$上的线性空间$V$,把$V$中的任意向量$\bvec{v}$映射到$\mathbb{K}$中的一个元素$f(\bvec{v})$上,那么映射$f:V\rightarrow\mathbb{K}$就是线性空间$V$上的一个函数.你可以任意指定函数的对应规则,但我们只关心所有可能的函数中最容易研究的一类,线性函数.

\begin{definition}{线性函数}
给定一个域$\mathbb{K}$上的线性空间$V$,如果映射$f:V\rightarrow\mathbb{K}$满足线性性,即对于任意$\bvec{v}, \bvec{u}\in V$以及任意的$a, b\in\mathbb{K}$,都有$f(a\bvec{v}+b\bvec{u})=af(\bvec{v})+bf(\bvec{u})$,那么我们称$f$是$V$上的一个\textbf{线性函数(linear function)}.
\end{definition}











\subsection{有限维}
我们可以定义另一个矢量空间 $X^*$, 称为 $X$ 的\textbf{对偶空间(dual space)}

有限维矢量空间的对偶空间, 

列矢量 $\bvec v$ 对应行矢量 $\bvec v\Her$.

可以看作 $V$ 到 $\mathbb R$ 或 $\mathbb C$ 上的线性映射.

\subsection{无穷维}
矢量不能用有限元素的列矢量表示, 直接定义为所有线性映射的组合.
