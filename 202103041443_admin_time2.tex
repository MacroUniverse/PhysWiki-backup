% 时间的计量 2

\pentry{时间的计量\upref{TimeCa}}}

通过遥远天体的观测等方法决定一个参考系(International Celestial Reference Frame, ICRF), 决定地球在 ICRF 中的转角\textbf{地球旋转角(Earth Rotation Angle, ERA)} $\theta$, 并以此定义 \textbf{UT1 儒略日}为
\begin{equation}\label{time2_eq1}
\theta = 2\pi(0.7790572732640 + 1.00273781191135448 T_u)
\end{equation}
其中 $T_u = \text{UT1 儒略日} - 2451545.0$, 且 $2.7378\times 10^{-3}$ 约等于 $1/365.25$ 即一回归年的太阳日个数.

把 UT1 儒略日的小数部分除以 86400 ($24\times 60\times 60$), 就可以定义 UT1 时间, 即上一个 UT1 正午经过的 UT1 秒数(注意儒略日是从中午开始计算的). 注意这里的 UT1 秒是由\autoref{time2_eq1} 定义的而不符合国际单位. 事实上, 由于地球自转速度会在潮汐力的作用下发生改变, 所以 UT1 中一秒的长度也会缓慢改变. 测量得出, 现在的一天约比 100 年前的 1 天慢 1.7 毫秒.

\subsection{两种太阳时}
\textbf{地方视太阳时(local apparent solar time)}可以定义为一个理想日晷(把太阳看作一个点光源)显示的时间. 视太阳时的问题在于它是不均匀的而是随季节周期性变化. 产生这种不均匀的第一个原因在于地球的公转旋转轨道是椭圆的, 不同季节的公转角速度有所不同.

\textbf{平太阳时(mean solar time)}, 当前的平太阳日约为国际单位的 $86400.002s$.
