% 引力波的几何描述
引力的作用量分为爱因斯坦作用量和物质的作用量$S=S_E+S_M$其中
\begin{align}
S_E = \frac{c^3}{16\pi G} \int d^4 x \sqrt{-g} R~.
\end{align}
能量动量张量的定义如下
\begin{align}
\delta S_M = \frac{1}{2 c} \int d^4 x \sqrt{-g} T^{\mu\nu} \delta g_{\mu\nu} ~. 
\end{align}
对作用量求变分,我们得到了如下的爱因斯坦方程
\begin{align}
R_{\mu\nu} - \frac{1}{2} g_{\mu\nu} R = \frac{8\pi G}{c^4} T_{\mu\nu} ~. 
\end{align}
广义相对论在很大的对称群下都是不变的,比如下面的坐标变换
\begin{align}
x^\mu \rightarrow x'^\mu (x) ~. 
\end{align}
在上面的坐标变换下,度规按照如下规则变换
\begin{align}
g_{\mu\nu} (x) \rightarrow g'_{\mu\nu}  = \frac{\partial x^\rho}{\partial x'^\mu} \frac{\partial x^\sigma}{\partial x'^\nu} g_{\rho\sigma} (x) ~. 
\end{align}
我们把这种对称性叫做广义相对论的规范对称性。现在我们把度规按照如下的方式展开
\begin{equation}\label{Geomet_eq1}
g_{\mu\nu} = \eta_{\mu\nu} + h_{\mu\nu}~, \quad |h_{\mu\nu}| \ll 1 ~. 
\end{equation} 
这样的理论叫做线性理论。在我们感兴趣的物理情形下,总是存在一个参考系使得\autoref{Geomet_eq1}成立。





