% 高斯光束

Wave Eq.
\begin{equation}
\laplacian \bvec E - \frac{1}{c^2} \pdv[2]{\bvec E}{t} = 0
\end{equation}
assume propagation within small angle of $z$ axis
\begin{equation}
\bvec E = \uvec \epsilon E(\bvec r, t) = 2\uvec \epsilon U(x, y, z) \E^{\I (kz - \omega t)}
\end{equation}
$U(x, y, z)$ is the envelope. Plug in, use ‘slowly varying envelope approximation’
\begin{equation}
2\I k\pdv{U}{z} = \pdv[2]{U}{x} + \pdv[2]{U}{y}
\end{equation}

The general solution is a linear combination of the following basis
  
 
       
 ,  ,   
This is called the Hermite-Gauss mode, denoted  .  are Hermite polynomials and   is the Gouy phase-shift,   is the Rayleigh length. The second exp factor makes the wave front a spherical wave with curvature  , because
 




  is the fundamental Gaussian mode.

In cylindrical coordinates, the basis change to Laguerre-Gauss modes  
 
  
This is analogous to solving SHO in polar coordiantes while Hermite-Gauss modes are in Cartesian coordinates. 
