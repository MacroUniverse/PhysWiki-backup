% 机器学习数据类型
\subsection{基本数据类型}
1.定类变量(nominal):变量的不同取值仅仅代表了不同类的事物,这样的变量叫定类变量.问卷的人口特征中最常使用的问题,而调查被访对象的“性别”,就是 定类变量.
2.定序变量(ordinal):变量的值不仅能够代表事物的分类,还能代表事物按某种特性的排序,这样的变量叫定序变量.问卷的人口特征中最常使用的问题“教育程度“,以及态度量表题目等都是定序变量,定序变量的值之间可以比较大小,或者有强弱顺序.
3.定距变量(interval):变量的值之间可以比较大小,两个值的差有实际意义,这样的变量叫定距变量.
4.定比变量(ratio variable):有绝对0点,如质量,高度.定比变量与定距变量在市场调查中一般不加以区分,它们的差别在于,定距变量取值为“0”时,不表示“没有”,仅仅是取值为0.
\subsection{四种基本数据类型的差异}
\begin{table}[ht]
\centering
\caption{四种基本数据类型的差异}\label{DatTyp_tab1}
\begin{tabular}{|c|c|c|c|c|}
\hline
对比项 & 定类变量 & 定序变量 & 定距变量 & 定比变量 \\
\hline
计数\分布 & * & * & * & * \\
\hline
* & * & * & * & * \\
\hline
* & * & * & * & * \\
\hline
* & * & * & * & * \\
\hline
* & * & * & * & * \\
\hline
* & * & * & * & * \\
\hline
* & * & * & * & * \\
\hline
* & * & * & * & * \\
\hline
* & * & * & * & * \\
\hline
* & * & * & * & * \\
\hline
* & * & * & * & * \\
\hline
* & * & * & * & * \\
\hline
\end{tabular}
\end{table}
