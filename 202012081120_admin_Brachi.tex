% 最速降线问题
% 摆线|拉格朗日方程|拉格朗日量|质点|轨迹|动能

\pentry{拉氏方程和极值问题\upref{LagPrb}}

\footnote{参考 Wikipedia \href{https://en.wikipedia.org/wiki/Brachistochrone_curve}{相关页面}.}在一个竖直平面上任意取两个不同的点 $P_1, P_2$, $P_1$ 的高度大于等于 $P_2$. 一个静止的质点在 $P_1$ 处释放, 在恒定的重力的作用下沿某曲线 $\mathcal L$ 无摩擦地运动到 $P_2$, 耗时为 $t$. 若一条曲线 $\mathcal L$ 使得 $t$ 最短, 那么这个曲线就是 $P_1, P_2$ 两点间的\textbf{最速降线}.

\subsection{推导}
\footnote{参考 Wolfram MathWorld \href{https://mathworld.wolfram.com/BrachistochroneProblem.html}{相关页面}, 以及 \cite{Goldstein} 的相关章节.}令 $x$ 轴向右, $y$ 轴向下. 质点由原点出发, 在重力加速度 $g$ 的作用下沿曲线 $y(x)$ 运动. 为书写方便令质量为 $m = 1$. 由动能定理得
\begin{equation}
v = \sqrt{2gy}
\end{equation}
曲线任意一点得斜率为($y$ 轴向下, $x$ 轴向右)
\begin{equation}
\cos\theta = \frac{1}{\sqrt{1 + \dot{y}^2}}
\end{equation}
注意 $\dot{y}$ 代表 $y(x)$ 对 $x$ 求导而不是对时间求导. $x$ 方向速度分量为
\begin{equation}
\dv{x}{t} = v\cos\theta
\end{equation}
即从 $x_1$ 滑落到点 $x_2$ 的时间为\footnote{注意该式对任意 $y(x)$ 都适用.}
\begin{equation}\label{Brachi_eq3}
t_{12} = \int_{x_1}^{x_2} \frac{\dd{x}}{v\cos\theta} = \int_{x_1}^{x_2} \sqrt{\frac{1 + \dot{y}^2}{2gy}} \dd{x}
\end{equation}
要求 $t_{12}$ 关于 $y(x)$ 的极值, 可以使用拉氏方程, 令 “拉格朗日量” 为被积函数
\begin{equation}
L(y, \dot y, x) = \sqrt{\frac{1 + \dot{y}^2}{2gy}}
\end{equation}
注意不显含 $x$, 即 $\pdv*{L}{x} = 0$. 下面来一步步代入拉格朗日方程. 先把 $L$ 对 $\dot{y}$ 求偏导, 注意要把 $y$ 看作常数
\begin{equation}
\pdv{\dot{y}}L(y, \dot{y}, x) = \frac{1}{\sqrt{2gy}}\frac{\dot{y}}{\sqrt{1 + \dot{y}^2}}
\end{equation}
然后对 $x$ 求全导数(把 $y, \dot{y}$ 看成 $x$ 的函数)
\begin{equation}
\dv{x}\pdv{L}{\dot{y}} = \frac{1}{\sqrt{2gy}\sqrt{1+\dot y^2}} \qty(-\frac{\dot y^2}{2y} + \ddot y - \frac{\dot y^2 \ddot y}{1 + \dot y^2})
\end{equation}

同理, 将 $\dot y$ 看作常数, 对 $y$ 求偏导得
\begin{equation}
\pdv{L}{y} = -\frac{1}{2} \sqrt{\frac{1 + \dot{y}^2}{2gy}} \frac{1}{y}
\end{equation}
以上两式代入拉氏方程(\autoref{Lagrng_eq1}~\upref{Lagrng}), 即令两式相等, 并化简得
\begin{equation}\label{Brachi_eq1}
2y\ddot y + \dot y^2 + 1 = 0
\end{equation}
该方程的解为
\begin{equation}
\frac{y}{a} = 1 - \cos\qty[\frac{x + \sqrt{y(2a-y)}}{a}]
\end{equation}
其中 $a$ 是大于零的参数. 令方括号部分为 $\theta$, 该方程能写成 $\theta$ 的参数方程, 该曲线被称为\textbf{摆线(cycloid)}或\textbf{滚轮线}. % 链接未完成
\begin{equation}\label{Brachi_eq2}
\leftgroup{
x &= a(\theta - \sin\theta)\\
y &= a(1 - \cos\theta)
}\end{equation}
并且有
\begin{equation}
\dot y = \frac{\sin\theta}{1 - \cos\theta} \qquad
\ddot y = -\frac{1}{a(1-\cos\theta)^2}
\end{equation}
代入\autoref{Brachi_eq1} 可验证成立.

由于我们一开始假设了起点处 $y_1 = 0$, 在\autoref{Brachi_eq2} 中对应的是滚轮线的任意尖端($\theta_1 = 2\pi n$). 不妨假设起点处 $\theta_1 = 0$, 即 $x_1 = 0$. 接下来, 由于 $(x_2, y_2)$ 必须落在曲线上,  $a$ 的值可以由此确定.

代入\autoref{Brachi_eq3} 得质点沿最速降线滑动的时间为
\begin{equation}
t_{12} = \sqrt{\frac{a}{g}}(\theta_2 - \theta_1) = \sqrt{\frac{a}{g}} \theta_2
\end{equation}
其中 $\theta_2$ 为 $x_2, y_2$ 对应的参数.

该曲线有另一个奇妙的性质: 对于固定的 $a$, 无论从曲线哪个位置释放质点, 质点来回振动的周期都是相同的. 这个结论可以用于制造精确的钟摆, 由惠更斯于 1638 年在 《摆钟论》 中提出, 比牛顿的 《原理》(1686) 早了约 50 年.
