% 皮卡-林德勒夫定理

皮卡-林德勒夫定理 (Picard-Lindelöf theorem) 是分析数学中的一个基本定理, 又称为柯西-李普希茨定理 (Cauchy-Lipschitz theorem). 它断言: 常微分方程 (组) 的初值问题只需要满足一些非常宽泛的条件, 就是唯一可解的. 

由于许多经典物理问题都可以化归为常微分方程组, 所以皮卡-林德勒夫定理可以用来说明这些物理问题的决定论 (deterministic) 特性: \textbf{给定了系统的初始状态之后, 系统的演化就唯一确定了.}

对于不满足皮卡-林德勒夫定理条件的常微分方程组, 尚有皮亚诺存在定理. 后者无法保证解的唯一性.

\subsection{定理的表述与证明}
\begin{theorem}{皮卡-林德勒夫定理}
设$I\subset\mathbb{R}$是开区间, $X$是巴拿赫空间\upref{banach}, $U\subset X$是开集. 设有连续映射$f:U\times I\to X$, 对于$X$变量满足局部李普希茨条件, 即对于任意$x_0\in U$, 都存在$x_0$的小邻域$\bar B_X(x_0,R)\subset U$, 以及一个正数$L>0$, 使得对于$x_1,x_2\in \bar B_X(x_0,R)$和$t\in I$, 总有$$
|f(x_1,t)-f(x_2,t)|_X\leq L|x_1-x_2|_X.
$$

则对于任何$t_0\in I$, $u_0\in U$, 都存在一个正数$T>0$, 使得常微分方程的初值问题
$$
\frac{d}{dt}u(t)=f(u(t),t),\quad u(t_0)=u_0
$$
在区间$[t_0-T,t_0+T]$上有唯一解.
\end{theorem}