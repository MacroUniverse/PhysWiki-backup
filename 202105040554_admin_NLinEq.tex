% 数值解线性方程组(入门)

\pentry{高斯消元法解线性方程组\upref{GAUSS}}

求解线性方程组是科学计算中最普遍也是最为常见的问题,几乎所有与科学计算有关的问题都直接或间接与它有关. 不论是常微分方程,偏微分方程,非线性方程,最优化,甚至是图像和信号处理,机器学习等等问题,最终都会转化成求解线性方程组.因此,线性方程组的解法也是科学计算领域里研究最广泛的问题之一.

线性方程的数值解法按照求解过程可以分为: \textbf{直接法(direct method)}和\textbf{迭代法(iterative method)}. 其中,直接法顾名思义就是直接求得方程组的解,这个解在很多情况下就是方程组的解析解.一般常用直接法为\textbf{高斯消元法(Gauss elimination)}或者是 \textbf{LU 分解(LU decomposition)}.

而相对应的,迭代法则是通过有限次的迭代,将数值解不断逼近解析解的过程.因此,迭代法通常都会引入一定的误差. 这些误差可以通过增加迭代次数和改进方法逐渐逼近于机器精度. 目前常见的迭代法包含了:\textbf{雅可比法(Jacobi method)},\textbf{高斯-赛德尔迭代(Gauss-Seidel method)}, \textbf{Krylov 子空间法(Krylov subspace methods)}等. 由于迭代法对于数值代数的要求较高, 这里就不做过多展开了.有兴趣的同学可以在下面留言, 我会单独开一个子专栏进行讨论.

\subsection{高斯消元法和 LU 分解}

事实上,高斯消元法的过程就是构造 LU 分解的上下三角矩阵的过程. 关于这个高斯消元法的基本算法见 “高斯消元法解线性方程组\upref{GAUSS}” 或参考 Wikipedia \href{https://en.wikipedia.org/wiki/Gaussian_elimination}{相关页面}.

这里我想从更宏观的角度来分析一下高斯消元法和 LU 分解.这个方法的主要思路包含三步,


这个方法的主要思路包含三步,以求解 $Ax=b$ 为例,我们接下来逐一解释.注意,这里的 $A$  是一个 $n\times n$ 的矩阵, $x和b$ 都是 $n\times1$ 的向量.

<ol><li>\textbf{基本的消元运算}</li></ol>通过高斯消元或者LU分解,得到 $A=LU$ , 其中 $L$ 和 $U $ 分别是 $n\times n$ 的上,下三角矩阵.我们将 $A$ 中第 $i $ 行, 第 $j$ 列的元素记做 $a_{i,j}$ ,那么消元的算法(伪代码)如下



经过消元得到新的 $A$ 矩阵实际上就是LU分解中的上三角 $U$ 矩阵.而在消元过程中用于临时存储系数的 $L$ 矩阵,加上一个单位矩阵就可以得到LU分解的下三角矩阵.\textbf{事实上,细心的同学可以发现,这样的LU分解可以直接在} $A$ \textbf{的存储空间上进行,无需额外的内存}.

经过简单的计算,这样的消元过程总共需要进行 $\sum_{k=1}^{n-1}(k+2k^2)\approx \frac{2}{3}n^3$ 次浮点运算.

\textbf{2. 前向替换(Forward substitution)}

把 $Ux$ 看做一个整体 $y $ ,将求解 $Ax=b$ 转化为求解 $LUx=Ly=b$ .由于 $L $ 是下三角矩阵,它的第一行只有一个非0的元素 $L_{1,1}$ ,因此这个求解过程可以简单的从第一行开始,逐行替换.

那么,整个的替换过程需要 $1+3+5+...+(2n-1)=n^2$ 次浮点运算.

\textbf{3. 后向替换(Backward substitution)}

最后,我们利用从\textbf{2.}中得出的 $y$ , 求解 $Ux=y$ ,从而得到原本的未知数 $x$ .这个过程正好和求解下三角矩阵相反,需要从最后一行开始,依次向上.

同样的,它也需要 $n^2$ 次浮点运算.

<h2>小结</h2>这样看起来,我们把求解一个线性方程组的问题转化成了一个LU分解和求解两个线性方程组,但是由于 $L$ 和 $U$ 都是三角矩阵,它们的求解过程非常简单,因此整个过程的总体运算复杂度始终是由LU分解所主导,即为 $\mathcal{O}(n^3)$ .

\textbf{例子1:} 让我们来用下面的代码直接测试一下高斯消元法的运算复杂度.



以及下面的图片

<figure data-size="normal"><noscript> $n$ 的三次方函数.<i>\textbf{有兴趣的同学也可以试试用二次曲线拟合,看看是否符合.}</i>

\textbf{例子2:}假设我们的计算机每秒可以处理 $10^9$ 次浮点运算,即1 giga FLOPS (Floating point operations per second),这其实比现在一般的笔记本电脑都要慢得多.下面这个表分别给出的是对于不同尺寸的问题,进行高斯消元运算和只进行向后替换的理论耗时.其中, $t_f$ 是一次浮点运算所需的时间,即 $10^{-9}~\rm{s}$ 

<figure data-size="normal"><noscript> $10^{17}$ FLOPS.如果用高斯消元法求解一个 $n=10^9$ 矩阵,也需要至少200年.

然而,很多实际应用问所需要求解的线性方程组的尺度经常会大于 $10^9$ ,例如一些三维或者更高维度物理过程的模拟仿真,天气预报,等等.那么它们是如何被求解的呢?显然高斯消元只适用于中小尺寸的问题,对于大尺寸的线性方程组,我们需要其他运算复杂度更低的方法进行求解.这个我会在接下来的几期里陆续给大家介绍.
