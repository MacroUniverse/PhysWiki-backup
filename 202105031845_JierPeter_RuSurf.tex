% 直纹面
% ruled surface

直纹面可以说是直线和曲面的完美结合体,作为一个曲面,即使高斯曲率不为零,也能在任何一个点处找到一条直线,这条直线是曲面的子集.直纹面可以通过让一条直线在空间中扫过来构造,不过我们在这儿肯定是要给出准确的数学描述,以方便深入研究.

空间中的一条直线可以由两个向量组成,一个位置向量$\alpha(t)$和一个方向向量$\omega(t)$.不同的$t$对应不同的直线,而每个$t$对应的直线可以表示为$\{\alpha(t)+v\omega(t)|v\in\mathbb{R}\}$.随着参数$t$变化,直线的位置和方向都连续变化,也就是说,$\alpha$和$\omega$随着$t$连续变化.由此容易想到,可以使用两个参数来描述一个直纹面,将其局部坐标系写成如下形式:
\begin{equation}
\bvec{x}(t, v)=\alpha(t)+\omega(t)v
\end{equation}

这里的两个参数确定直纹面上一个点的过程,可以理解为$t$确定了是在哪一条直线上,而$v$确定了是在直线上的哪个位置.

不失一般性地,为了方便,我们不妨设方向向量$\omega(t)$恒为单位向量.这样的设置能保证$\omega(t)\cdot\omega(t)'=0$恒成立.当$\omega(t)'$为零的时候,得到的直纹面就被称为\textbf{柱形面(cylindrical surface)};当$\omega(t)'$不为零的时候,意味着方向必然在变,得到的就是\textbf{非柱形面(noncylindrical surface)}.

直纹面上的任意一条曲线$\beta(t)$,都可以用一个\textbf{标量函数}$u(t)$来唯一表示:\begin{equation}
\beta(t)=\alpha(t)+\omega(t)u(t)
\end{equation}







