% 单纯形与复形
% 单纯形|simplex|复形|complex|几何|单纯剖分|三角剖分

\subsection{单纯形}

单纯形可以被认为是欧几里得空间中的一种子集,最初的概念来源于对拓扑空间的三角剖分,而单纯形就是剖分出来的每个“三角形”.这里之所以要打引号,是因为单纯形不仅仅指二维的三角形,也包括三维的锥形,以及从零维到任意维的拓展.

\begin{definition}{几何无关点集}
给定欧几里得空间中$n$个点构成的集合$\{\bvec{r}_i\}|_{i=0}^{n-1}$.任取一个$\bvec{r}_k$,我们以它为起点构造出若干向量$\bvec{v}_i=\bvec{r}_i-\bvec{r}_k$.

如果集合$\{\bvec{v}_i\}_{i\not= k}$是一个\textbf{线性无关}向量组,那么我们说$\{\bvec{r}_i\}|_{i=0}^{n-1}$是一个\textbf{几何无关(geometrically independent)}点集.
\end{definition}

注意几何无关点集和线性无关向量组有一点点区别,那就是几何无关点集里多了一个“起点”的位置.这个起点没法简单地排除在外,因为任意一个点都可以做起点,没必要确定谁更特殊.

\begin{definition}{单纯形}
给定欧几里得空间中一个几何无关点集$A=\{\bvec{r}_i\}$.

记$[\bvec{r}_0, \bvec{r}_1, \bvec{r}_2, \cdots, \bvec{r}_q]$为集合$\{\bvec{r}=\sum\lambda_i\bvec{r}_i|\lambda_i\geq 1, \sum\lambda_i=1\}$,称这个集合为由$A$张成的$q$维\textbf{单纯形(simplex)},简称$q$-单形.

\end{definition}

比如说,三维空间里$3$个几何无关点$\bvec{r}_0, \bvec{r}_1, \bvec{r}_2$可以张成一个$2$-单形,就是以这三个点为顶点的平面三角形.类似地,一个$3$-单形就是一个三棱锥.

\begin{definition}{面}

令$L$为单形$[\bvec{r}_0, \cdots, \bvec{r}_q]$,则对于$r\leq q$,$L$的一个$r$维面就是$\{\bvec{r}_0, \cdots, \bvec{r}_q\}$的一个$r$阶子集所张成的单形.

\end{definition}


\begin{definition}{规则相处}
对于两个
\end{definition}


从上面的表述可以看出,尽管我们一开始引入单纯形概念的时候依赖于欧几里得空间的性质,是一种高度几何化的语言,但是表示时我们其实只关心是哪些顶点在构成一个单纯形.这就使得我们可以将单纯形的概念抽象化,将$[\bvec{r}_0, \cdots, \bvec{r}_q]$视作这$q+1$个元素的一个组合,忽视掉几何特征,从而有了.










