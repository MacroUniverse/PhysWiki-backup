% 火车模型与洛伦兹变换

\pentry{狭义相对论的基本假设\upref{SpeRel}}

\subsection{事件}

在和时空相关的理论中,当我们描述一件事的时候,我们并不关心这件事具体是什么,只关心它发生在何时何地.因此为了将来的讨论,我们首先需要定义“事件”的概念.

一个\textbf{事件(event)}是指在时空坐标系中的一个点.事件所发生的时间、地点,就是事件作为一个点的坐标.

\subsection{对事件的观测}

狭义相对论的核心是光.在任何参考系中,光速不变.光具体的性质并不能保证一定不变,如光强等.

除了光速以外,事件本身也是不随惯性系变化的.这就是说,在任何惯性系$K_1$中同时同地发生的事情,在任何惯性系$K_2$中也是同时同地发生的.鉴于我们已经为了简便,将事件简单表达为它所发生的时间和地点,那么同时同地发生的事件都应看成同一个事件.

光速和事件的不变性,是目前我们观测事件的最基本工具.

\subsection{同时性的相对性}

考虑一根的铁轨,向左向右都无限延伸.在这铁轨上取一个点作为原点,向右作为正方向,可以画一个$x_1$轴,用来测量和铁轨静止的参考系中的事件位置,这个参考系称作\textbf{铁轨系}记为$K_1$.

现在,铁轨上开来一辆火车.和火车静止的参考系也可以沿着铁轨画一个$x_2$轴,只不过它是用来描述火车参考系中事件位置的,称作\textbf{火车系},记为$K_2$.

在铁轨系中,如果某时刻看到火车的两个不重叠的轮子同时发光,那么这两道光会在铁轨上的两个发光点的中点相遇,而“相遇”也是一个事件.从发光到相遇,两束光通过了相同的路程,由于光速不变,它们经过了相同的时间,由此反推可知发光的时间是一样的.但是同样的三个事件在火车系看来是不一样的:在火车系中,“相遇”发生在更靠近后轮的位置,也就是说,在火车看来前轮所发的光走过了更长的路程,花了更长的时间,从“相遇”的时间反推回去,可知火车眼里是前轮先发光.

\begin{figure}[ht]
\centering
\includegraphics[width=8cm]{./figures/SRsmt_1.png}
\caption{在铁路系中所看到的三个事件.上图是两个轮子同时发光的两个事件;下图是一段时间以后,火车运动了一段距离,而两束光相遇的事件.} \label{SRsmt_fig1}
\end{figure}

事实上,在$K_1$中同时但不同地发生的事情,在$K_2$中必然不同时发生.“同时”这一概念并非绝对,两个事件是否同时,取决于从什么参考系来观察它们.

\begin{exercise}{火车系中的事件}

上图\autoref{SRsmt_fig1}

\end{exercise}

\subsection{尺缩效应}

