% 匀速曲线运动

\pentry{曲率半径\upref{curvat}, 牛顿运动定律\upref{New3}}

质点沿着曲线运动, 我们希望得到它的加速度. 先来看匀速的情况, 这时候加速度\upref{VnA}完全由速度方向改变而产生, 就像匀速圆周运动那样. 回顾加速度的定义
\begin{equation}\label{PCuvMo_eq2}
\bvec a = \lim_{\Delta t \to 0} \frac{\Delta \bvec v}{\Delta t}
\end{equation}
类似用几何法推导匀速圆周运动的速度\upref{CMVD}那样, 我们可以近似\upref{LimArc}认为当速度矢量 $\bvec v$ 转过一个小角度 $\Delta \theta$ 时, 它的增量 $\Delta \bvec v$ 垂直于 $\bvec v$, 且大小为
\begin{equation}\label{PCuvMo_eq1}
\abs{\Delta \bvec v} = v\Delta\theta
\end{equation}
如何求质点做曲线运动时, 一小段时间 $\Delta t$ 速度方向, 即曲线切线方向的变化呢? 我们可以使用曲率\upref{curvat}的概念. 令质点所在位置的曲率半径为 $R$, 根据曲率半径的定义(\autoref{curvat_eq3}~\upref{curvat}), $\Delta t$ 内质点在曲线上走过的长度为 $\Delta l = v \Delta t$, 所以切线的角度变化为($\Delta t\to 0$ 时取等号)
\begin{equation}
\Delta \theta \approx \Delta l/R = v \Delta t/R
\end{equation}
代入\autoref{PCuvMo_eq1} 再带入\autoref{PCuvMo_eq2} 得加速度大小为
\begin{equation}
\abs{\bvec{a}} = \lim_{\Delta t \to 0} \frac{\abs{\Delta \bvec v}}{\Delta t}
= \lim_{\Delta t \to 0}  \frac{v\Delta\theta}{\Delta t} = \frac{v^2}{R}
\end{equation}
