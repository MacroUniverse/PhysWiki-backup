% test13524

To identify nonadjacent vertices $x$ and $y$ of a graph $G$ is to replace these vertices by a single vertex incident with all the edges which were incident in $G$ with either $x$ or $y$. Let $e = xy\in E(G)$. To contract an edge $e$ of a graph $G$ is to delete the edge and then identify its ends. A graph $H$ is a minor of a graph $G$ if $G$ has a subgraph contractible to $H$; $G$ is called $H$-minor free if $G$ does not have $H$ as a minor. Here we report how to prove the following results.
\begin{theorem}  Let $G$ be a $K_5$-minor free graph.

  \item[(1)] The linear arboricity $la(G) \leq \lceil \frac{\Delta(G)+1}{2}\rceil$;
  \item[(2)] If $\Delta(G)\geq 7$, then the edge chromatic number $\chi'(G) =\Delta(G)$ and the total chromatic number $\chi''(G)\leq \Delta(G)+2$;
  \item[(3)] If $\Delta(G)\geq 9$, then $la(G) =\lceil \frac{\Delta(G)}{2}\rceil$ and $\chi''(G)= \Delta(G)+1$

\end{theorem}
At the same time, we use a structural property of $K_5^-$-minor free graphs to prove that if $G$ is a $K_5^-$-minor free graph and $\Delta (G)\geq 5$, then $\chi'(G) =\Delta(G)$ and $la(G)= \la{\Delta(G)}$.