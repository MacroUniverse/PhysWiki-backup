% 自由群
\pentry{群作用\upref{Group3}}

\subsection{字母和词}

\begin{definition}{字母}
没有赋予任何运算关系的元素,被称为一个\textbf{字母(letter)}.
\end{definition}

给定没有赋予任何运算结构的任意集合$S$,它的每个元素都可以作为一个字母.

\begin{example}{字母的例子}
\begin{itemize}
\item 随便取一个元素,取名为$x$,那么$x$可以看成一个字母.
\item 取$52$个英文字母构成的集合,则每个元素都可以当作字母.
\end{itemize}
\end{example}

字母就是纯粹的符号,除了“指代一个元素”以外没有更多的结构.没有更多结构当然没有研究的意义不过,我们可以拓展字母的用途,就是把若干字母按顺序排列起来,得到新的元素.这种元素不同于字母,我们把它叫做“词”.

\begin{definition}{词}
有限个字按照一定顺序排列构成的元素,称作一个\textbf{词(word)}或\textbf{字}.
\end{definition}

\begin{example}{词的例子}
\begin{itemize}
\item 单元素集合的元素$x$,可以构成词$x$,$xx$,$xxxxxxx$等.
\item $52$个英文字母构成集合$S$,那么$S$中的元素$l, e, t, r$都是字母,它们可以进行有限排列,得到词$letter$.
\end{itemize}

\end{example}

单个字母也可以看成是特殊的词,这样一来,就可以把字的排列拓展成词之间的一种\textbf{运算}.两个词首尾相连,可以构成一个更大的排列,仍然是有限个字的排列,因此得到的还是词.

\begin{definition}{词的运算:首尾相连}
给定集合$S$.用$S$中的元素作为字母所排出的词,构成一个“词集合”,记为$W(S)$.在集合$W(S)$上可以定义一个运算“$\cdot$”如下:设$a, b\in W(S)$,那么$a\cdot b=c$,其中$c$是$a$和$b$首尾相连的结果.
\end{definition}

\begin{example}{词的运算}
还是用英文字母构成的集合$S$.$lov$是一个词,$ely$也是一个词,$lov\cdot ely=lovely$是这两个词进行首尾相连运算的结果.
\end{example}

我们也常常省略运算符号“$\cdot$”.

\subsection{自由生成群}

我们希望改进一下词集合和词运算,构造出一个群.为了做到这一点,我们首先需要扩展一下词集合.

\begin{definition}{字母的逆}
给定集合$S$,它的词集合是$W(S)$.
\end{definition}

