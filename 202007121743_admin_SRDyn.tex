% 相对论动力学假设
\pentry{相对论加速度变换\upref{SRAcc},时空的四维表示\upref{SR4Rep}}

%感觉基本完成了,因为相对论动力学的实例似乎没什么单独拿出来分析的必要,最好是用来进行其它理论的相对论性拓展.所以是否需要进一步讨论,最初笔者暂无好的想法,先这样吧.

\subsection{什么是动力学}
初学牛顿力学时,你也许注意到了,课本将它分为两大部分:运动学和动力学.

运动学的研究范畴,基本上是纯数学的,只讨论了什么是位移、速度、加速度等概念,以及这些概念之间的数学联系.在运动学中有一个看起来很自然的假设,在了解相对论之前常常被忽略,那就是在不同参考系中这些概念和它们的联系是如何变化的.

动力学的研究范畴,则加上了“力”的概念,讨论力是如何影响物体的运动状态、力有什么性质等的.在牛顿力学中,力的性质由牛顿三定律决定,由此可以引申出更深刻的动量守恒和能量守恒等定律.

在狭义相对论中,参考系的变换导致的运动学变换复杂了很多,力的作用也不再像牛顿力学中那样简单.

\subsection{常见误解的辟谣}
一个常见的\textbf{误导}是,许多科普读物和视频会使用光子钟等模型来“推导”出爱因斯坦的明星方程,$E=mc^2$,或者用我们约定的单位制,$E=m$.这些推导的思路是用牛顿第二定律$F=m\dd x/\dd t$来定义“力”,然后利用洛伦兹变换计算在不同惯性系中,物体加速度的变化.当然,在使用伽利略变换的牛顿力学中,任何惯性系下物体的加速度都一样,因此质量是不会变化的;但是在狭义相对论中,由于不同惯性系下加速度一般是不同的,加之相对性原理要求力是不随参考系而变化的\footnote{比如说,一个正在施加张力的弹簧,应该看成是在任何参考系下都产生一样的张力,而不受弹簧尺缩效应的影响.},因此套入牛顿第二定律以后,必须认为质量是可以变化的.这种思路所计算出的质点质量变化,就是$m/\sqrt{1-v^2}$,其中$v$是质点在观察者眼中的运动速度.这个质量,就是所谓的“动质量”.以$v^2$为自变量,对$m/\sqrt{1-v^2}$进行Maclaulin展开(即$v_0^2=0$处的泰勒展开),我们可以得到

\begin{equation}\label{SRDyn_eq1}
\frac{m}{\sqrt{1-v^2}}=m+\frac{1}{2}mv^2+\frac{3}{4}mv^4+\cdots
\end{equation}
其中第二项就是牛顿力学中的动能.在$v$远远小于光速$1$的时候,我们可以只看前两项,也就是$m+mv^2/2$,或者为了更明显地表现量纲,我们把光速添加回去:$mc^2+mv^2/2$.这看上去像是物体的动能加上某种能量.把$mc^2$看成是物体固有的一种能量,那么\autoref{SRDyn_eq1}就变成了

\begin{equation}
\frac{m}{\sqrt{1-v^2/c^2}}c^2=mc^2+\frac{1}{2}mv^2+\frac{3}{4c^2}mv^4+\cdots=\text{固有能量+动能}
\end{equation}
把质点的固有能量+动能理解为质点含有的总能量,记为$E$,那么就有
\begin{equation}
\frac{m}{\sqrt{1-v^2}}c^2=E
\end{equation}
这就是著名的质能方程.

\textbf{但是,这种推导思路是误导的.}


从相对论加速度变换\upref{SRAcc}词条中可以看到,在质点加速度方向垂直和平行于速度方向时,加速度的变化系数是不一样的,这就导致在沿用牛顿第二定律来计算质量的时候,会出现质量随着方向变化的情况.旧版教科书会认为,这意味着不应该把质量简单地看成一个标量,而应该看成一个二阶张量\upref{Tensor}.但现代早已抛弃了“质量随方向变化”的概念,甚至也不再强调“动质量”的概念.

本节将会解释现代的相对论动力学概念.

\subsection{四动量}

设质点静止时的质量是$m$.在参考系$K_1$中,质点的四速度是$\bvec{U}$,定义$\bvec{P}=m\bvec{U}$为质点的\textbf{四动量(4-momentum)}.

具体地,如果在$K_1$中,质点的速度是$(v_x, v_y, v_z)^T$,那么其四速度就是$\gamma(1, v_x, v_y, v_z)^T$,其中$\gamma=1/\sqrt{1-v^2}$.这样,质点的四动量就表示为

\begin{equation}
\frac{m}{\sqrt{1-v^2}}\pmat{1\\v_x\\v_y\\v_z}
\end{equation}

特别地,四动量的时间分量又被称为\textbf{能量}.


\subsection{动力学假设}

在牛顿力学中,动量守恒和能量守恒是三定律的推论.在实验中,这两个守恒定律在很高的精度上一直成立.你可以认为推论的成立是对牛顿定律的证明,也可以换个角度,认为守恒定律才是更基本的规律.

狭义相对论中,加速度的变换复杂多了,因此更难以猜想应该怎么拓展牛顿定律.由于牛顿定律是有效的,我们相信它在限定范围内,也就是宏观低速的情况下是正确的,这也意味着,狭义相对论成立的必要条件之一,是当光速取为无穷大的时候,整个狭义相对论理论就变成了牛顿理论.现在的问题是,该从什么方向拓展牛顿定律呢?

最简洁的可能性,是\textbf{四动量守恒}.也就是说,在任何惯性系看来,一个孤立系统中的若干质点相互作用前后,系统的总四动量守恒.这就是相对论的动力学假设,描述了相互作用的规律,作为牛顿定律的拓展.

需要强调的是,这仅仅是一个假设,无法由别的定律推导出来.也就是说,这个动力学假设是一个公理\upref{axioms},和牛顿第二第三定律的地位一样.如果这个公理所得出的理论和实际相差太大,那么理论物理学家会选择抛弃它,改用别的假设.幸运的是,迄今为止,四动量守恒和实验很好地相符,因此沿用至今.




