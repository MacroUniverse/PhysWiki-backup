% 相对论加速度变换

\pentry{速度变换\upref{RelVel}}

\subsection{一般情况下的加速度变换}
\subsubsection{问题的限制条件}

为了简化讨论,不失一般性,我们可以把情景按以下方式设定:

假设$K_2$相对$K_1$的运动速度是$\left(\begin{matrix}v\\0\\0\end{matrix} \right) $.设在$K_1$中,有一质点以速度$\vec{u}=\left(\begin{matrix}u_x\\u_y\\u_z\end{matrix} \right) \tag{2}$运动,其在$K_2$中的速度是$\vec{u'}=\left(\begin{matrix}u_x'\\u_y'\\u_z'\end{matrix} \right) $.同时,质点在$K_1$中有加速度$\vec{a}=\frac{\dd}{\dd{t}}\vec{u}=\left(\begin{matrix}a_x\\a_y\\a_z\end{matrix} \right)$,在$K_2$中加速度则为$\vec{a'}=\frac{\dd}{\dd{t'}}\vec{u'}=\left(\begin{matrix}a_x'\\a_y'\\a_z'\end{matrix} \right)$.选取两坐标系的原点位置使它们重合、在质点轨迹上,并且设质点一直以匀速运动,直到在$K_1$中测量的时间$t$时才有了非零加速度.

计算的整体思路是,将$\frac{\dd}{\dd{t'}}\vec{u'}$拆成$\frac{\dd}{\dd{t}}\vec{u'}\cdot\frac{\dd{t}}{\dd{t'}}$,分别计算两个微商,然后再乘起来.一阶微分的形式不变性保证了这个思路的合法性.

注意,这里并不能简单地令$t'=\sqrt{1-v^2}t$,因为质点的位置并不总是在$K_2$的原点处,我们需要选所讨论位置的$t'$.

于是有
\begin{equation}

\left\{\begin{matrix}\vec{u'}=\frac{\sqrt{1-v^2}-\frac{1-\sqrt{1-v^2}}{v^2}(\vec{u}\cdot\vec{v})\vec{v}-\vec{v}}{1-\vec{u}\cdot\vec{v}}\\0\end{matrix}


