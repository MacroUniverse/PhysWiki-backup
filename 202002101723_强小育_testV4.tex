% 测试

\subsection{勾股定理}
\textbf{勾股定理}是指直角三角形两条$a$, $b$直接边与斜边$c$之间满足关系
\begin{equation}\label{testV4_eq1}
a^2 + b^2 = c^2
\end{equation}

\subsubsection{证明}
现在来证明\autoref{testV4_eq1}

\begin{figure}[ht]
\centering
\includegraphics[width=6cm]{./figures/testV4_1.png}
\caption{勾股定理的证明} \label{testV4_fig1}
\end{figure}
\autoref{AccTra_eq1}\upref{AccTra}
\begin{theorem}{余弦定理}
对于任意三角形,任何一边$a$的平方等于其他两边$b, c$平方的和减去这两边与它们夹角的余弦的积的两倍
\begin{equation}
a^2 = b^2 + c^2 + 2bc\cdot cos(A~B)
\end{equation}
\end{theorem}
\begin{figure}[ht]
\centering
\includegraphics[width=6cm]{./figures/testV4_2.png}
\caption{余弦定理} \label{testV4_fig2}
\end{figure}