% 一维散射态的正交归一化

\pentry{一维散射(量子)\upref{Sca1D}}
写作参考\href{https://chaoli.club/index.php/4541/last}{这篇帖子}. 类似于平面波, 一维散射态也有不同的归一化方式, 但情况要更为复杂. 

为了方便先假设 $V(x)$ 关于原点对称, 且 $V(x)$ 只在区间 $[-L,L]$ 内不为零. 由于 $V(x)$ 的对称性, 我们必定能找到实值的奇函数和偶函数两种解. 令 $k = \sqrt{2mE} > 0$, 在区间 $[-L,L]$ 外, 波函数就是正弦函数加上一个相移
\begin{equation}\label{ScaNrm_eq3}
\psi_k(x) = A\sin(kx + \phi) \qquad (x > L)
\end{equation}
其中$\phi$ 是 $k$ 的函数, 称为\textbf{相移(phase shift)}. 为方便书写下文把 $\phi(k),\phi(k')$ 分别记为 $\phi, \phi'$.

令对称和反对称散射态分别为实函数 $\psi_{k,e}(x)$ 和 $\psi_{k,o}(x)$ 我们希望通过添加适当的归一化系数后, 波函数能满足归一化条件(\autoref{EngNor_eq3}~\upref{EngNor})% 链接未完成
\begin{equation}\label{ScaNrm_eq1}
\int_{-\infty}^{+\infty} \psi_{k',i'}(x)^* \psi_{k,i}(x) \dd{x} = \delta_{i',i}\delta(k' - k) \qquad (k > 0, i = e, o)
\end{equation}

\begin{theorem}{}
\autoref{ScaNrm_eq1} 对所有性质良好的 $V(x)$ 都成立, 且\autoref{ScaNrm_eq3} 中\textbf{归一化系数和简谐波相同}, 即 $A = 1/\sqrt{\pi}$(\autoref{EngNor_eq5}~\upref{EngNor}).
\end{theorem}

我们以下给出部分证明: 对于奇偶性不同的两个函数, 他们显然式正交的. 首先已知
\begin{equation}
\int_{0}^{+\infty} \sin(k'x)\sin(kx)\dd{x} = \frac{\pi}{2}\delta(k'-k)
\end{equation}
现在添加相位 $\phi(k)$ 后, 有不定积分
\begin{equation}
\int \sin(k'x+\phi')\sin(kx+\phi) \dd{x} = \frac{\sin[(k'-k)x + (\phi'-\phi)]}{2(k'-k)}
- \frac{\sin[(k'+k)x+(\phi'+\phi)]}{2(k'+k)}
\end{equation}
在 $(0,n)$ 做定积分取极限 $n\to\infty$ 后发现比 $\delta(x)$ 多了两项
\begin{equation}
\int_{0}^{+\infty} \sin(k'x+\phi')\sin(kx+\phi) \dd{x} = \frac{\pi}{2}\delta(k'-k)
+ \frac{\sin(\phi'+\phi)}{2(k'+k)} - \frac{\sin(\phi'-\phi)}{2(k'-k)}
\end{equation}
所以在区间 $[0, +\infty)$ 上 $\sin(kx+\phi)$ 一般并不正交.

使用归一化系数 $1/\sqrt{2}$, \autoref{ScaNrm_eq1} 的积分为(利用波函数的奇偶性)
\begin{equation}\label{ScaNrm_eq2}
\begin{aligned}
\braket{\psi_{k'}}{\psi_k} &= 2\int_{0}^{+\infty} \frac{1}{\sqrt{\pi}}\sin(k'x+\phi')\frac{1}{\sqrt{\pi}}\sin(kx+\phi) \dd{x} + 2I(k,k')\\
&= \delta(k'-k) + \frac{\sin(\phi'+\phi)}{\pi(k'+k)} - \frac{\sin(\phi'-\phi)}{\pi(k'-k)} + 2I(k,k')
\end{aligned}
\end{equation}
其中 $2I(k,k')$ 修正了 $[-L,L]$ 区间实际波函数和 $\sin(kx+\phi)$ 的不同
\begin{equation}
I(k,k') = \int_0^L \psi_{k'}(x)^* \psi_k(x) \dd{x}
-\int_{0}^{L} \frac{1}{\sqrt{\pi}}\sin(k'x+\phi') \frac{1}{\sqrt{\pi}}\sin(kx+\phi) \dd{x}
\end{equation}
如果能证明对于任意 $V(x)$, \autoref{ScaNrm_eq2} 的最后的三项之和都为零, 那么我们就证明了\autoref{ScaNrm_eq1} 的正交归一关系. 读者可以尝试用一些具体的例子证明, 如方势垒\upref{SqrPot}.

\subsection{不对称势能}
波函数为
\begin{equation}
\psi_k(x) = \leftgroup{
    &A_+ \sin(kx + \phi_+) &\quad &(x > L)\\
    &A_- \sin(kx - \phi_-) &&(x < -L)
}
\end{equation}
那么归一化系数应该满足
\begin{equation}
\frac{1}{2}\qty(\abs{A_-}^2 + \abs{A_+}^2) = \frac{1}{\pi}
\end{equation}
当两个系数相等时就有上文的 $A_+ = A_- = 1/\sqrt{\pi}$.

\subsection{行波的归一化}
\addTODO{通过 2 乘 2 的酋矩阵变换即可.}
