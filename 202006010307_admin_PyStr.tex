% Python 字符串处理

\pentry{Python 简介\upref{Python}}

\subsection{字符串处理}
\textbf{字符串(string)}可以想i想由字符构成的特殊数组, 一般用于储存一段文字. 一个字符串变量如
\begin{lstlisting}[language=python]
s = 'python String 字符串'
\end{lstlisting}
注意也可以使用双引号 \verb|"..."|. 用 \verb|type(s)| 查看类型得到 \verb|str|, 即 string 的缩写.

\verb|len(s)| 用于获取字符串长度
\begin{lstlisting}[language=python]
print('s = ', s, '; length = ', len(s))
\end{lstlisting}

\subsubsection{截取字符串}
字符串的截取与数组的截取相同
\begin{lstlisting}[language=python]
s = '0123456789'
print(s[0:3]) # 截取第一位到第三位的字符
print(s[:]) # 截取字符串的全部字符
print(s[6:]) # 截取第七个字符到结尾
print(s[:-3]) # 截取从头开始到倒数第三个字符之前
print(s[2]) # 截取第三个字符
print(s[-1]) # 截取倒数第一个字符
print(s[::-1]) # 创造一个与原字符串顺序相反的字符串
print(s[-3:-1]) # 截取倒数第三位与倒数第一位之前的字符
print(s[-3:]) # 截取倒数第三位到结尾
\end{lstlisting}
输出:
\begin{lstlisting}
012
0123456789
6789
0123456
2
9
9876543210
78
789
\end{lstlisting}

\subsubsection{连接字符串}
\begin{lstlisting}[language=python]
delimiter = ','
mylist = ['Brazil', 'Russia', 'India', 'China']
print (delimiter.join(mylist))
print('abc'+'def')
输出
Brazil,Russia,India,China
abcdef
\end{lstlisting}

\subsubsection{字符串搜索}
\begin{itemize}
\item 搜索指定字符串: \verb|s.find('t')|,搜不到返回 \verb|-1|
\item 指定起始位置搜索:\verb|s.find('t', start)|
\item 指定起始及结束位置搜索:\verb|s.find('t', start, end)|
\item 从右边开始查找:\verb|s.find('t')|
\item 搜索到多少个指定字符串:\verb|s.count('t')|
\end{itemize}

\begin{lstlisting}[language=python]
s='python String function'
print('%s find nono=%d' % (s,s.find('nono')))
print('%s find t=%d' % (s,s.find('t')))
print('%s find t from %d=%d' % (s,1,s.find('t',1)))
print('%s find t from %d to %d=%d' % (s,1,2,s.find('t',1,2)))
print('%s rfind t=%d' % (s,s.rfind('t')))
print('%s count t=%d' % (s,s.count('t')))
输出:
python String function find nono=-1
python String function find t=2
python String function find t from 1=2
python String function find t from 1 to 2=-1
python String function rfind t=18
python String function count t=3
\end{lstlisting}

\subsubsection{字符串替换}

替换old为new:\verb|s.replace('old','new')|

替换指定次数的old为new:\verb|s.replace('old','new',maxReplaceTimes)|


\subsubsection{字母处理}
\begin{itemize}
\item 全部变为大写:\verb|s.upper()|
\item 全部变为小写:\verb|s.lower()|
\item 大小写互换:\verb|s.swapcase()|
\item 首字母大写,其余小写:\verb|s.capitalize()|
\item 首字母大写:\verb|s.title()|
\end{itemize}

\begin{lstlisting}[language=python]
s = 'python String 字符串'
print('s =          ', s);
print('lower =      ', s.lower()); # 全部变为小写:
print('upper =      ', s.upper()); # 全部变为大写:
print('swapcase =   ', s.swapcase()); # 大小写互换
print('capitalize = ', s.capitalize()); # 首字母大写,其余小写
print('title =      ', s.title()); # 首字母大写
\end{lstlisting}
输出:
\begin{lstlisting}
s =           python String 字符串
lower =       python string 字符串
upper =       PYTHON STRING 字符串
swapcase =    PYTHON sTRING 字符串
capitalize =  Python string 字符串
title =       Python String 字符串
\end{lstlisting}

\subsubsection{格式化相关}
获取固定长度,右对齐,左边不够用空格补齐:\verb|s.ljust(width)|

获取固定长度,左对齐,右边不够用空格补齐:\verb|s.ljust(width)|

获取固定长度,中间对齐,两边不够用空格补齐:\verb|s.ljust(width)|

获取固定长度,右对齐,左边不足用0补齐

\begin{lstlisting}[language=python]
s='python String'
print('%s ljust=%s' % (s,s.ljust(20)))
print('%s rjust=%s' % (s,s.rjust(20)))
print('%s center=%s' % (s,s.center(20)))
print('%s zfill=%s' % (s,s.zfill(20)))
输出:
python String ljust=python String       
python String rjust=       python String
python String center=   python String    
python String zfill=0000000python String

\end{lstlisting}



\subsubsection{字符串去空格及去指定字符}

去两边空格:\verb|s.strip()|

去左空格:\verb|s.lstrip()|

去右空格:\verb|s.rstrip()|

去两边字符串:\verb|s.strip('d')|,相应的也有\verb|lstrip,rstrip|

按指定字符分割字符串为数组:\verb|s.split(' ')|

\begin{lstlisting}[language=python]
s=' python String function '
print('%s strip=%s' % (s,s.strip()))
s='python String function'
print('%s strip=%s' % (s,s.strip('d')))
\end{lstlisting}



\subsubsection{翻转字符串}
\begin{lstlisting}[language=python]
sStr1 = 'abcdefg'
sStr1 = sStr1[::-1]
print(sStr1)
输出
gfedcba
\end{lstlisting}

关于更多高效的字符串处理方法,感兴趣的读者可以查阅正则表达式的使用方法.
