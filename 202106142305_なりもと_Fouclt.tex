% 傅科摆
% Fouclt|傅科摆|科里奥利力|几何推导
\usepackage[utf8]{inputenc}
\usepackage{xltxtra}
\usepackage{graphics}
\usepackage{amsmath}
\usepackage{amssymb}

\renewcommand{\i}{\vec i}
\renewcommand{\j}{\vec j}
\renewcommand{\k}{\vec k}

\pentry{单摆\upref{Pend}}

傅科摆是首个直接证明地球自转的实验. 试想如果把一个不受任何阻力的单摆放在地球的北极, 那么地球每自转一定角度, 单摆的摆平面不变, 所以以地球为参考系观察, 摆平面将反方向转动, 这样就能证明地球在自转. 现实中, 为了能克服阻力和微扰长时间摆动, 通常使用质量较大, 摆臂较长的摆作为傅科摆.

但若傅科摆被放在北纬 $\alpha$ 角处, 摆平面的将会以怎样的角速度转动呢? 事实证明, 若令地球自转的角速度为 $\omega_0$, 则单摆相对地面转动的角速度 $\omega$ 将等于

\begin{equation}
\omega = \omega_0 \sin\alpha
\end{equation}

\subsection{傅科摆角速度的一种几何推导}
\pentry{球坐标系\upref{Sph}, 连续叉乘的化简\upref{TriCro}}
设 $\bvec R$ 为地心指向傅科摆的矢量, $\hat {\bvec R}$ 是其单位矢量, 当地纬度为 $\alpha$, 地轴指向北的单位矢量为 $\uvec k$, 有
\begin{equation}
\uvec k\vdot \uvec R = \cos(\pi/2 - \alpha) = \sin\alpha
\end{equation}

若把任意矢量 $\bvec P$ 围绕某单位矢量 $\uvec M$ 以右手定则旋转角微元 $\dd{\theta}$, 有
\begin{equation}\label{Fouclt_eq2}
\dd{\bvec P} = \uvec M \cross \bvec P \dd{\theta}
\end{equation}
开始时, 令傅科摆在最低点的速度方向的单位向量为 $\uvec A$ ($\uvec A\vdot \bvec R = 0$), 在傅科摆下方的水平地面上标记单位向量 $\uvec B$, 使开始时 $\uvec B = \uvec A$. 当傅科摆随地球在准静止状态下移动位移 $\dd{\bvec s}$ ($\dd{\bvec s} \vdot\bvec R = 0$)后, 由\autoref{Fouclt_eq2} 可得

\begin{equation}
\dd{\uvec A} = \uvec M\cross \uvec A\vdot \dd{\theta} = 
\frac{\bvec R\cross \dd{\bvec s}}{\abs{\bvec R\cross \dd{\bvec s}}} \cross \uvec A \frac{\dd{s}}{R}
\end{equation}
注意这只是一个比较符合物理直觉的假设, 这里并不给出证明. 当地球转动 $\dd{\theta}$ 时, 上式中 $\dd{\bvec s} = \uvec k \cross \bvec R \dd{\theta}$, 而地面上的标记 $\uvec B$ 也围绕地轴转动, 所以 $\dd{\uvec B} = \uvec k \cross \uvec A \dd{\theta}$.

下面计算 $\dd{\uvec A} - \dd{\uvec B}$. 因为 $\bvec R\vdot \dd{\bvec s} = 0$, 所以 $\abs{\bvec R \cross \dd{\bvec s}} = R\dd{s}$, 所以
\begin{equation}\ali{
\dd{\uvec A} &= \frac{\bvec R\cross \dd{\bvec s}}{R^2}\cross\uvec A =
\frac{1}{R^2}\bvec R\cross(\uvec k\cross\bvec R \dd{\theta})\cross \uvec A\\
&= \uvec R\cross(\uvec k\cross\uvec R)\cross\uvec A \dd{\theta} =
[(\uvec R\vdot\uvec R)\uvec k - (\uvec R\vdot \uvec k)\uvec R] \cross \uvec A \dd{\theta}\\
&= (\uvec k - \uvec R\sin\alpha) \cross \uvec A \dd{\theta}
}\end{equation}
其中第二行使用了 “连续叉乘的化简\upref{TriCro}”.
\begin{equation}
\dd{\uvec A} - \dd{\uvec B} = (\uvec k - \uvec R\sin\alpha)\cross \uvec A \dd{\theta} - \uvec k \cross \uvec A \dd{\theta} = -\sin\alpha \uvec R \cross \uvec A \dd{\theta}
\end{equation}
所以地球转过 $\dd{\theta}$ 角以后, $\uvec A$ 与 $\uvec B$ 之间的夹角为
\begin{equation}
\dd{\gamma} = \abs{\dd{\uvec A} - \dd{\uvec B}} = \sin\alpha \dd{\theta}
\end{equation}
两边除以 $\dd{t}$ 得角速度
\begin{equation}
\omega = \omega_0 \sin\alpha
\end{equation}

\subsection{傅科摆角速度的牛顿力学上的推导}


\title{计算位于武汉的傅科摆摆平面旋转的周期}
\author{杨赖源}
\date{\today}

\begin{document}
\maketitle

\songti
以地面为参考系,以武汉为原点,取地球表面坐标系,如图\ref{fig1}.
\begin{figure}[h]
    \centering
    \includegraphics[scale=0.25]{global.jpg}
    \caption{地球表面参考系}
    \label{fig1}
\end{figure}

设武汉的纬度为$\varphi$,地球的自传角速度为$\omega$,摆的质量为$m$,摆相对于的位矢为$\vec r$,重力加速度为$\vec g$,地球摆所受绳的拉力为$\vec T$,科里奥利力为$\vec{F_{co}}$.其中$\vec g\approx -g \vec k$.则

\begin{align}
    \text{地球角速度}\vec\omega&=-\omega\cos\varphi \vec i +\omega\sin\varphi \j\\
    \text{摆的位矢}\vec r&=x \vec i+y\vec j+z\vec k\\
    \text{摆的速度}\vec v&=\doc x\vec i+\doc y\vec j+\doc z\vec k\\
    \text{摆的加速度}\vec a&=\ddot x \vec i+\ddot y \j+\ddot z \k
\end{align}
\begin{align}
\text{科里奥利力}\vec{F_{co}} &=
        -2m\vec\omega \times \vec v=
        \begin{vmatrix}
        \i &\vec j &\k\\
        -\omega\cos\varphi &\omega\sin\varphi & 0\\
        \dot x &\dot y &\dot z
        \end{vmatrix} \notag \\
        &=2m\omega\left[\dot y\sin\varphi\i-\left(\dot z\cos\varphi+\dot x\sin\varphi \right)\j+\dot y\cos\varphi\k \right]
\end{align}
  设绳长为$L$,当摆到达点$\left( x,y,z \right)$时,所受拉力$\vec T$如图\ref{fig2}
  
  \begin{figure}[h]
      \centering
      \includegraphics[scale=0.25]{T2}
      \caption{拉力$\vec T$}
      \label{fig2}
  \end{figure}
  
  \begin{align}
      T_x&=-T\sin\alpha\cos\beta=-T\frac{\sqrt{x^2+y^2}}{L}\frac{x}{\sqrt{x^2+y^2}}=-\frac{x}{L}T \label{eqTx}\\
      T_y&=-T\sin\alpha\cos\beta=-\frac{y}{L}T \label{eqTy}\\
      T_z&=T\cos\alpha=\frac{L-z}{L}T \label{eqTz}
  \end{align}
 
 所以有
 \begin{align}
     m\ddot x&=-\frac{x}{L}T+2m\omega\dot y\sin\varphi \label{eqmx}\\
     m\ddot y&=-\frac{y}{L}T-2m\omega \left( \dot z\cos \varphi+\dot x\sin \varphi \right) \label{eqmy} \\
     m\ddot z&=\frac{L-z}{L}T+2m\omega\dot y\cos\varphi-mg \label{eqmz}
 \end{align}
 
 因为$z\ll L$,而且$z$方向上的振动幅度很小,所以$\frac{L-z}{L} \approx 1$,$\dot z=\ddot z \approx 0$
 
 于是式\ref{eqmz}可以化为
 \begin{equation}
     0=T+2m\omega\dot y\cos\varphi-mg \label{eqmz2}
 \end{equation}
 又有$2m\omega\dot y\cos\varphi \ll mg$,所以由式\ref{eqmz2}可得
 \begin{equation}
     T\approx mg \label{eqmz3}
 \end{equation}
 
 将式\ref{eqmz3}代入式\ref{eqmx}和式\ref{eqmy},并注意到$\dot z \approx 0$,所以
 \begin{gather}
     \ddot x-2\omega\sin\varphi\dot y+\frac{g}{L}x=0 \label{eqmx2}\\
     \ddot y+2\omega\sin\varphi\dot x+\frac{g}{L}y=0 \label{eqmy2}
 \end{gather}
 式\ref{eqmx2}\times $y$- \text{式}\ref{eqmy2}\times $x$ \text{可得}

 \begin{equation}
     \ddot x y-x\ddot y=2\omega\sin\varphi \left(\dot y y+\dot x x \right)
 \end{equation}
 
 两边积分得
 \begin{equation}
     \dot x y-x\dot y=\omega \sin\varphi \left( x^2+y^2 \right)+C,C\in\mathbb{R} \label{integral}
 \end{equation}
 
 令$x=r\cos\theta, y=r\sin\theta$
 
 所以$\dot x=\dot r\cos\theta-r\dot \theta\sin\theta, \dot y=\dot r \sin \theta +\dot r \cos \theta$
 
 代入式\ref{integral},得
 \begin{equation}
     -r^2\dot \theta=r^2\omega\sin\varphi +C, C\in\mathbb{R}
 \end{equation}
 令$t=0$时,有$r=0$,则$C=0$
 
 所以
 \begin{equation}
     \dot \theta=-\omega\sin\varphi
 \end{equation}
 $\dot\theta$代表摆旋转时的角速度,
 所以摆旋转的周期为
 \begin{equation}
     T=\left|\frac{2\pi}{-\omega\sin\varphi}\right|=\frac{24h}{\sin\varphi}
 \end{equation}
 
 在武汉,$\varphi \approx 30^{\circ}40'$
 所以
 \begin{equation}
     T\approx\frac{24h}{\sin 30^{\circ}40'}=47.1h
 \end{equation}
\end{document}
