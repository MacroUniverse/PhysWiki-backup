% 皮卡-林德勒夫定理

皮卡-林德勒夫定理 (Picard-Lindelöf theorem) 是分析数学中的一个基本定理, 又称为柯西-李普希茨定理 (Cauchy-Lipschitz theorem). 它断言: 常微分方程 (组) 的初值问题只需要满足一些非常宽泛的条件, 就是唯一可解的. 

由于许多经典物理问题都可以化归为常微分方程组, 所以皮卡-林德勒夫定理可以用来说明这些物理问题的决定论 (deterministic) 特性: \textbf{给定了系统的初始状态之后, 系统的演化就唯一确定了.}

对于不满足皮卡-林德勒夫定理条件的常微分方程组, 尚有皮亚诺存在定理. 后者无法保证解的唯一性.

\subsection{定理的表述与证明}
\begin{theorem}{皮卡-林德勒夫定理}
设$X$是巴拿赫空间\upref{banach}. 设有实数$t_0\in\mathbb{R}$, 正数$r,R>0$, 以及$X$中的元素$x_0$. 设映射$f:\bar B_X(x_0,R)\times[t_0-r,t_0+r]\to X$, 满足如下条件:

(1) 存在一正数$M>0$, 使得对于$$
\end{theorem}