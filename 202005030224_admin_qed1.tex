% 洛伦兹群
\pentry{群\upref{Sample}李群\upref{Sample}洛伦兹变换\upref{Sample}算符对易与共同本征函数\upref{Sample}矩阵\upref{Sample}闵可夫斯基空间\upref{Sample}}
\subsection{定义}
所有洛伦兹变换的集合形成一个群,称为洛伦兹群(Lorentz group).
采取度规张量$g_{\mu \nu}=\operatorname{diag}(1,-1,-1,-1)$,引入时空坐标四矢量标记
\begin{equation}
\begin{aligned}
x^{\mu} &=\left(x^{0}, x^{i}\right) \quad(i=1,2,3)\\
&=(t, \bvec{x})
\end{aligned}
\end{equation}
则对于两个惯性系有
\begin{equation}\label{qed1_eq1}\begin{aligned}
S^{2} &=x^{0} x^{0}-x^{i} x^{i}=x^{\mu} x^{\nu} g_{\mu \nu} \\
&=x^{\prime 0} x^{\prime 0}-x^{\prime i} x^{\prime i}=x^{\prime \mu} x^{\prime \nu} g_{\mu \nu}
\end{aligned}\end{equation}
设联系两个惯性系的洛伦兹变换为$\Lambda_{\nu}^{\mu}$,即
\begin{equation}\label{qed1_eq2}x^{\prime \mu}=\Lambda_{\nu}^{\mu} x^{\nu}=\Lambda_{0}^{\mu} x^{0}+\Lambda_{i}^{\mu} x^{i}\end{equation}
保持时空距离不变的要求导致
\begin{equation}\label{qed1_eq3}g_{\rho \sigma}=g_{\mu \nu} \Lambda_{\rho}^{\mu} \Lambda_{\sigma}^{\nu}\end{equation}

\subsection{分类}
设洛伦兹矩阵可以用$L$表示,可以证明洛伦兹矩阵的行列式$detL=\pm1$.\\
取洛伦兹矩阵的00分量,根据\autoref{qed1_eq3}有
\begin{equation}
1=g_{\mu \nu} \Lambda_{\rho}^{0} \Lambda_{\sigma}^{0}= (\Lambda_{0}^{0})^2- (\Lambda_{i}^{0})^2\\
所以 \left|\Lambda_{0}^{0} \right|\geqslant1.
\end{equation}
根据$\Lambda_{0}^{0}$的范围($\Lambda_{0}^{0}\geqslant1$是orthochronous的洛伦兹变换)和$L$($L=1$是proper的变换)的符号可以将洛伦兹变换分为四类
\begin{enumerate}
\item proper orthochronous,对应$L=1\quad \Lambda_{0}^{0}\geqslant1$.
\item proper non-orthochronous,对应$L=-1\quad \Lambda_{0}^{0}\leqslant1$
\item improper orthochronous,对应$L=1 \quad\Lambda_{0}^{0}\geqslant1$
\item improper non-orthochronous,对应$L=-1 \quad\Lambda_{0}^{0}\leqslant1$
\end{enumerate}
\subsection{推导}
采取矩阵记法,设两个惯性系的坐标矢量分别为$X$和$X'$,度规矩阵为$g$,洛伦兹矩阵为$L$.则\autoref{qed1_eq1},\autoref{qed1_eq2},\autoref{qed1_eq3}可以分别写为
\begin{equation}
S^{2}=X^{\mathrm{T}} g X
\end{equation}
\begin{equation}
X'=LX
\end{equation}
\begin{equation}\label{qed1_eq4}
g=L^{\mathrm{T}} g L
\end{equation}
将\autoref{qed1_eq4}两边取行列式,有
\begin{equation}
detg=detL^{\mathrm{T}} detg detL
\end{equation}
命题得证.
\subsection{举例}
(1)转动\\
(2)平动\\
(3)时间反演\\
(4)全反演\\
考虑一个无穷小变换
\begin{equation}\Lambda_{\nu}^{\mu}=\delta_{\nu}^{\mu}+\omega_{\nu}^{\mu}\end{equation}
代入得到$\omega_{\mu \nu}=-\omega_{\nu \mu}$
所以洛伦兹群有六个独立参元素.
\subsection{洛伦兹变换的李群形式}
上述小节已证明,洛伦兹群有6个参数,是反对称矩阵$\omega_{\mu\nu}$的六个独立元素,对应着六个独立生成元.把生成元表示为$J^{\mu\nu}$,因为有反对称指标$(\mu,\nu)$则有$J^{\mu\nu}=-J^{\nu\mu}$.一个广义的洛伦兹群元$\Lambda$可以写为
\begin{equation}
\Lambda=e^{-\frac{\I}{2} \omega_{\mu\nu}J^{\mu\nu}}
\end{equation}
\subsection{四矢量表象下的生成元}
在四矢量表象下,生成元由4×4矩阵$(J^{\mu\nu})_\sigma^\rho$表示.
可以证明,
\begin{equation}\left(J^{\mu \nu}\right)_{\sigma}^{\rho}=i\left(g^{\mu \rho} \delta_{\sigma}^{\nu}-g^{\nu \rho} \delta_{\sigma}^{\mu}\right)\end{equation}
\subsection{洛伦兹群$SO(3,1)$的李代数}
可以证明,$SO(3,1)$的李代数为:
\begin{equation}\left[J^{\mu \nu}, J^{\rho \sigma}\right]=i\left(g^{\nu \rho} J^{\mu \sigma}-g^{\mu \rho} J^{\nu \sigma}-g^{\nu \sigma} J^{\mu \rho}+g^{\mu \sigma} J^{\nu \rho}\right)\end{equation}
通过重组$J^{\mu\nu}$,可以得到两类空间矢量
\begin{equation}J^{i}=\frac{1}{2} \epsilon^{i j k} J^{j k}, \quad K^{i}=J^{i 0}\end{equation}
在这个角度上, 可以写为
\begin{equation}\begin{aligned}
\left[J^{i}, J^{j}\right] &=i \epsilon^{i j k} J^{k} \\
\left[J^{i}, K^{j}\right] &=i \epsilon^{i j k} K^{k} \\
\left[K^{i}, K^{j}\right] &=-i \epsilon^{i j k} J^{k}
\end{aligned}\end{equation}
