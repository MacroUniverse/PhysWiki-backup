% 列维—奇维塔符号
% Levi|Civita

\footnote{本文参考 Wikipedia \href{https://en.wikipedia.org/wiki/Levi-Civita_symbol}{相关页面}.}\textbf{列维—奇维塔符号(Levi-Civita symbol)} 在线性代数, 矢量分析,张量分析等领域有重要的应用.

列维—奇维塔符号是一个函数, 记为 $\epsilon_{i_1, i_2, \dots, i_N}$. 它的自变量是 $N$ 个正整数 $i_1, \dots, i_N$, 每个从 $1, 2, \dots, N$ 中取值, 函数值只能取 $0, 1, -1$ 中的一个. 当 $i_1, \dots, i_N$ 中有任意两个重复时, 函数值为零.

\subsection{三阶定义}
三阶 LC 符号是矢量分析中常用的, 可以直接用穷举法来定义.
\begin{equation}
\epsilon_{123} = \epsilon_{231} = \epsilon_{312} = 1
\end{equation}
\begin{equation}
\epsilon_{321} = \epsilon_{213} = \epsilon_{132} = -1
\end{equation}
当 $i,j,k$ (只能取 1,2,3)中任意两个重复时, $\epsilon_{ijk} = 0$.

\subsection{N 阶}
\begin{equation}
\epsilon_{i_1,i_2,\dots, i_n}
\end{equation}
