% 氢原子薛定谔方程数值解

\pentry{薛定谔方程\upref{TDSE}, 原子单位\upref{AU}, 类氢原子的定态波函数\upref{HWF}}

虽然最直观的方法是使用直角坐标, 但计算效率太低. 实际中一般使用球坐标系, 用球谐函数展开波函数. 如果 Hamiltonian 是轴对称的, 那么只需要 $m = 0$ 的球谐函数, 即勒让德多项式.

\begin{equation}
Ψ(\bvec r, t) = \frac{1}{r}∑_{l',m'} ψ_{l',m'}(r) Y_{l',m'}(\bvec r)
\end{equation}
其中 $ψ_{l,m}(r)$ 是 (scaled)径向波函数.

在薛定谔方程中, 我们往往使用经典的电场, 即令
\begin{equation}
V(\bvec r) = -\frac{Z}{r} - q\bvec E(t) \bvec r
\end{equation}
这叫做长度规范.% 引用未完成
此外我们还可以使用速度规范, 也是等效的.% 引用未完成

\subsection{线性极化场}
若我们取电场极化方向为 $\uvec z$, 则角动量 $L_z$ 是一个守恒量. 假设初始波函数关于 $\uvec z$ 轴对称, 那么在波函数的整个演化过程中, 我们只需要 $m=0$ 的球谐函数展开波函数, 即
\begin{equation}
Ψ(\bvec r, t) = \frac{1}{r}∑_{l'} ψ_{l'}(r) Y_{l', 0}(\uvec r)
\end{equation}
带入薛定谔方程, 并左乘 $\bra{Y_{l,0}}$ 得
\begin{equation}
H_0 ψ_{l}(r) + ∑_{l'} E(t)rF_{l, l'} ψ_{l'}(r) = \I \pdv{ψ_{l}}{t}
\end{equation}
其中
\begin{equation}
H_0 = -\frac{1}{2m} \pdv[2]{r} -\frac{Z}{r} + \frac{l(l+1)}{2mr^2}
\end{equation}
是无场哈密顿算符. 矩阵 $\mat F$ 的计算参考\autoref{SphHar_eq3}\upref{SphHar}.
\begin{equation}
\begin{aligned}
F_{l,l'} &= ∫ Y_{l,0}^*(\uvec r) Y_{1,0}(\uvec r) Y_{l',0} \dd{Ω}\\
&= \sqrt{\frac{3}{4π} (2l+1)(2l'+1)} \pmat{l & 1 & l'\\ 0 & 0 & 0}\pmat{l & 1 & l'\\ 0 & 0 & 0}
\end{aligned}
\end{equation}
可见, 当没有外场的时候每一个项(即每一个分波)都可以独立传播, 而电场将不同的分波耦合起来.

\subsection{算符拆分}
在实际的程序中, 我们可以把传播子 $\exp(-\I H Δ t)$拆成两项. 虽然这么做会引入一定的误差, 但是却大大提高了效率
\begin{equation}
\exp(-\I H Δ t) ≈ \exp(-\I H_0 Δ t)\exp(-\I F Δ t)
\end{equation}
也就是说, 在每个时间步长 $Δ t$ 中, 我们先把薛定谔方程根据
\begin{equation}
H_0 ψ_{l}(r) = \I \pdv{ψ_{l}}{t}
\end{equation}
演化 $Δ t$, 再根据
\begin{equation}
∑_{l'} E(t)rF_{l, l'} ψ_{l'}(r) = \I \pdv{ψ_{l}}{t}
\end{equation}
演化 $Δ t$. 就可以根据

\subsection{网格和演化算法}
\pentry{Crank-Nicolson 算法(一维)\upref{CraNic}}
一个简单使用的算法就是在对径向波函数使用等间距网格, 并使用 Crank-Nicolson 算法演化.
