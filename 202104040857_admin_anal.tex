% 幂级数 2
% 泰勒级数|幂级数|收敛域|收敛半径|解析函数

\pentry{泰勒级数\upref{Taylor}}

在复数域上, 形如
\begin{equation}\label{anal_eq1}
\sum_{n=0}^\infty c_n(z-a)^n
\end{equation}
的级数称为\textbf{幂级数(power series)}, 这里 $c_n$ 皆为复数, 未定元 $z$ 一般也视为复数. 

\subsection{幂级数的收敛域}
\begin{theorem}{}
如果由\autoref{anal_eq1} 给出的幂级数在某点 $z_0\neq a$ 处收敛, 那么它一定在开圆盘 $|z-a|<|z_0-a|$ 上绝对收敛且内闭一致收敛.
\end{theorem}

证明很简单: 如果 $\sum_{n=0}^\infty c_n(z-a)^n$ 在 $z=z_0$ 时收敛, 那么 $c_n(z_0-a)^n\to0$, 从而有一 $M$ 使得 $|c_n(z_0-a)^n|\leq M$ 对任何 $n$ 都成立. 故对于任何固定的 $0<q<1$, 当 $|z-a|<q|z_0-a|$ 时就有
$$
\begin{aligned}
\sum_{n=0}^\infty |c_n(z-a)^n|
&=\sum_{n=0}^\infty |c_n(z_0-a)^n|\frac{|z-a|^n}{|z_0-a|^n}\\
&<\sum_{n=0}^\infty Mq^n.
\end{aligned}
$$
于是幂级数诸项绝对值由收敛的几何级数控制.

由此可见, \autoref{anal_eq1} 给出的幂级数的收敛域或者只是一个点 $a$, 或者至少包含某个以 $a$ 为圆心的开圆盘. 这样的开圆盘中最大者叫做幂级数的\textbf{收敛圆 (disk of convergence)}, 其半径称为\textbf{收敛半径 (radius of convergence)}. 幂级数的收敛半径由柯西-阿达玛公式\upref{CHF}给出. 在收敛圆的边界上, 无法作出幂级数是否收敛的一般论断, 例如幂级数 $\sum_{n=0}^\infty z^n$ 在点 $z=\pm1$ 处皆发散, 但其和函数 $1/(1-z)$ 在 $z=1$ 时为奇异, 在 $z=-1$ 时表现正常.

\subsection{幂级数的运算}
幂级数的四则运算与一般级数的四则运算无异.

\begin{theorem}{幂级数的微分与积分}
幂级数在进行逐项微分与逐项积分后, 其收敛圆不变. 因此, 幂级数的和函数可以在收敛圆内逐项微分, 也可以逐项积分. 幂级数的和函数在收敛圆内是无穷可微的.
\end{theorem}
这是柯西-阿达玛公式\upref{CHF}的直接推论. 

\subsection{解析函数}
由收敛幂级数表示的复变函数称为\textbf{解析函数 (analytic function)}. 它与用复可微性定义的\textbf{全纯函数 (holomorphic function)}\upref{CauRie} 是等价的对象, 尽管这个事实的证明并不平凡 (需要用到柯西积分公式). 如果限制自变量取实数, 那么得到的是\textbf{实解析函数 (real analytic function)}. 等价地, 实解析函数是泰勒级数收敛到其自身的函数. 有如下定理:

\begin{theorem}{}
开区间 $I$ 上的实函数 $f$ 为解析函数, 当且仅当对于 $I$ 的任何闭子区间 $K$, 都有常数 $M_K$ 使得
$$
\max_{x\in K}|f^{(n)}(x)|\leq n!M_K^n.
$$
\end{theorem}

证明是直接的计算: 如果 $f$ 满足此条件, 那么可以估算出其泰勒展开式的余项趋于零; 反过来, 如果 $f$ 由收敛幂级数表征, 那么可以估计其逐项微分得到的幂级数的上界而得到 $f$ 的高阶导数所满足的条件.

这个定理说明: 实解析函数的高阶导数随着其阶数的提升不能增长得太快. 举例来说, 函数 $f(x)=e^{-1/x^2}$ 在任何不包含原点的开区间上是解析函数; 在原点处它的各阶导数都是零, 但是对于任何正数 $M$ (不管有多么大), 都可以找到趋于零的序列 $x_n$, 使得对于充分大的 $n$ 有
$$
|f^{(n)}(x_n)|>n!M^n.
$$
因此 $f$ 在任何包含原点的开区间上都不是解析函数. 用复变函数论的语言, $z=0$ 是函数 $e^{-1/z^2}$ 的本性奇点.

由此可见, 泰勒展开式收敛到其自身的函数是非常特殊的.
