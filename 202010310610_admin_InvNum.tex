% 逆序数
% 逆序数|排列

\pentry{排列\upref{permut}}

\footnote{参考 Wikipedia \href{https://en.wikipedia.org/wiki/Inversion_(discrete_mathematics)}{相关页面}.}我们把集合 $\qty{1,\dots,N}$ 的某种排列记为 $\pi$, 该排列的元素按照顺序分别记为 $\pi_1, \dots, \pi_N$. 对于任意 $i < j$, 如果满足 $\pi_i > \pi_j$ 我们就把 $i, j$ 或者 $\pi_i, \pi_j$ 称为排列 $\pi$ 的一个\textbf{逆序对(inversion)}. 一个排列中所有逆序对的个数就叫\textbf{逆序数(inversion number)}.

逆序数主要的应用有定义行列式\upref{Deter}和列维—奇维塔(Levi-Civita)符号\upref{LeviCi}. 在这两个应用中, 我们只对逆序数的奇偶性感兴趣. 我们把逆序数为奇数的排列叫做\textbf{奇排列}, 逆序数为偶数的排列叫做\textbf{偶排列}. 把按从小到大的顺序排列叫做\textbf{标准排列}. 标准排列逆序数为零, 是一个偶排列.

\begin{example}{}
\begin{itemize}
\item 标准排列 $1,2,3,4$ 中对任意 $i < j$ 都有 $\pi_i < \pi_j$, 不存在逆序对, 逆序数为零.
\item $4,3,2,1$ 中对任意 $i < j$ 都有 $\pi_i > \pi_j$, 任何两个元素都是逆序对, 逆序数等于组合 $C_4^2 = 6$.
\item 排列 $1,6,2,5,3,4$ 中, 逆序对有 $(1,6)$,$(1,2)$, $(1,5)$, $(1,3)$,$(1,4)$, $(2,5)$,$(2,3)$, $(2,4)$, $(3,4)$. 共 $9$ 个.
\end{itemize}
\end{example}
一般情况下, 要计算逆序对, 我们可以依次令 $i = 1, \dots, N$, 对每个 $i$, 逐个检查其右边是否存在小于 $\pi_i$ 的元素, 把它们的数量相加即可. 

\begin{theorem}{}
交换排列中任意两个数, 逆序数奇偶性改变. 
\end{theorem}

\subsubsection{证明}
在排列 $p_n$ 中, 考虑第 $i, j$ ($i < j$)两个元素的置换, 不妨把他们的值分别记为 $m, n$. 不妨假设 $m < n$ ($n < m$ 的讨论同理). 为了方便讨论, 我们把任意两个不同元素都用弧线连起来, 这些弧线有的代表逆序对, 有的不是. 我们可以把这些连线分为 4 类来讨论:
\begin{enumerate}
\item 对于不连接到 $m$ 或 $n$ 的线, 置换不改变它们之中的逆序数.
\item 对于从 $m$ 或 $n$ 连接到 $m$ 左边或 $n$ 右边的线, 置换同样不改变他们之中的逆序数.
\item 对于从 $m$ 或 $n$ 连接到 $m, n$ 之间某个元素 $q$ 的两条线, 只有当 $m < q < n$ 时, 置换后会同时产生 $m,q$ 和 $n,q$ 之间的两条线, 但无论有多少这样的 $q$, 逆序数只改变偶数个.
\item $m,n$ 的连线在置换前不是逆序的, 置换后变为逆序, 使逆序数增加 1, 改变逆序数的奇偶性.
\end{enumerate}
证毕.

\begin{corollary}{}
把奇排列变为标准排列需要置换奇数次, 把偶排列变为标准排列需要置换偶数次.
\end{corollary}
证明显然.
