% 分子电流和分子磁矩

根据物质电结构学说,任何物质(实物)都是由分子、原子组成的,而分子或原子中任何一个电子都不停地同时参与两种运动:一种是环绕原子核的轨道运动,另一种是电子本身的自旋运动.这两种运动都等效于一个电流分布,因而能产生磁效应把分子或原子看作一个整体,分子或原子中各个电子对外界所产生磁效应的总和,可用一个等效的圆电流表示,统称\textbf{分子电流}.这种分子电流具有一定的磁矩,称为分子磁矩(molecular magnetic moment),用符号$\mathbf m_{mole}$表示.
在外磁场B u 作用下,分子或原子中和每个电子相联系的磁矩都受到磁力矩的作用,
由于分子或原子中的电子以一定的角动量作高速转动,这时,每个电子除了保持上述两种运
动以外,还要附加电子磁矩以外磁场方向为轴线的转动,称为电子的进动这与力学中所讲
的高速旋转着的陀螺,在重力矩的作用下,以重力方向力轴线所作的进动十分相似(见图8-
49)