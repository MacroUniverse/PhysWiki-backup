% 椭圆坐标系

\begin{issues}
\issueDraft
\end{issues}

\pentry{椭圆的三种定义\upref{Elips3}、正交曲线坐标系\upref{CurCor}}

\subsection{椭圆坐标系}
\footnote{参考 Wikipedia \href{https://en.wikipedia.org/wiki/Elliptic_coordinate_system}{相关页面}.}
\textbf{椭圆坐标系(elliptic coordinate system)}是一个二维平面上的正交曲线坐标系,它是三种三维正交曲线坐标系定义的基础,这三种正交曲线坐标系为:\textbf{椭圆柱坐标系(elliptic cylindrical coordinate system)}、\textbf{长椭球坐标系(ellipsoidal coordinate system)}和\textbf{扁椭球坐标系(oblate spheroidal coordinate system)}.椭圆坐标系的坐标线为共焦的椭圆和双曲线,椭圆柱坐标系由椭圆坐标系沿垂直于椭圆坐标面的方向投影得到;长(短)椭球坐标系是将椭圆坐标系绕椭圆长(短)轴方向旋转得到.

椭圆坐标系上点的位置由 $(\xi,\eta)$ 这2个有序实数表示.$\xi$的等值曲线为一组共焦椭圆族,焦距为 $2c$;$\eta$ 的等值曲线为一组共焦的双曲线族,其焦点与椭圆族焦点相重.$\xi$、$\eta$由直角坐标定义
\begin{equation}\label{EliCor_eq3}
\left\{\begin{aligned}
&x=c\cosh\xi\cdot\cos\eta\\
&y=c\sinh\xi\cdot\sin\eta
\end{aligned}\right.
\end{equation}
其中 $\xi\geq0,0\leq\eta<2\pi$.

容易看出
\begin{equation}\label{EliCor_eq4}
\frac{x^2}{c^2\cosh^2\xi}+\frac{y^2}{c^2\sinh^2\xi}=1
\end{equation}
\begin{equation}\label{EliCor_eq5}
\frac{x^2}{c^2\cos^2\eta}-\frac{y^2}{c^2\sin^2\eta}=1
\end{equation}
这便是刚刚说的:$\xi$的等值曲线为一组共焦椭圆族,而$\eta$ 的等值曲线为一组共焦的双曲线族.

容易证明椭圆坐标系是一个正交曲线坐标系.在某点 $\bvec{r}$ 处,坐标轴$\xi$ 和 $\eta$ 的方向分别为$\pdv*{\bvec{r}}{\xi}$ 和 $\pdv*{\bvec{r}}{\eta}$.
由\autoref{EliCor_eq3} 
\begin{equation}
\left\{
    \begin{aligned}
    &\dd x=c\sinh\xi\cdot\cos\eta\dd\xi-c\cosh\xi\sin\eta\dd\eta\\
    &\dd y=c\cosh\xi\cdot\sin\eta\dd\xi+c\sinh\xi\cdot\cos\eta\dd\eta\\
    \end{aligned}\right.
\end{equation}
则
\begin{equation}
\left\{
    \begin{aligned}
&\pdv*{\bvec{r}}{\xi}=c\sinh\xi\cdot\cos\eta\uvec{x}+c\cosh\xi\cdot\sin\eta\uvec{y}\\
&\pdv*{\bvec{r}}{\eta}=-c\cosh\xi\sin\eta\uvec{x}+c\sinh\xi\cdot\cos\eta\uvec{y}\\
    \end{aligned}\right.
\end{equation}

令椭圆坐标轴 $\xi$、 $\eta$对应的单位矢量分别为 $\uvec{\xi}$、 $\uvec{\eta}$,由\autoref{CurCor_eq8}~\upref{CurCor}
\begin{equation}
\left\{
    \begin{aligned}
    &\uvec{\xi}=\frac{\pdv*{\bvec{r}}{\xi}}{|\pdv*{\bvec{r}}{\xi}|}=\frac{1}{\sqrt{\sinh^2\xi+\sin^2\eta}}\qty(\sinh\xi\cdot\cos\eta\uvec{x}+\cosh\xi\cdot\sin\eta\uvec{y})\\
    &\uvec{\eta }=\frac{\pdv*{\bvec{r}}{\eta}}{|\pdv*{\bvec{r}}{\eta}}=\frac{1}{\sqrt{\sinh^2\xi+\sin^2\eta}}\qty(-\cosh\xi\sin\eta\uvec{x}+\sinh\xi\cdot\cos\eta\uvec{y})\\
    \end{aligned}\right.
\end{equation}
易求得 $\uvec{\xi}\cdot\uvec{\eta}=0$, 即椭圆坐标系$(\xi,\eta,z)$为正交曲线坐标系.
\subsection{椭圆柱坐标系}
椭圆柱坐标系是在椭圆坐标系的基础上增加一垂直于椭圆坐标面的 $z$ 坐标得到,空间一点坐标用3个有序数 $(\xi,\eta,z)$表示.同样,若用直角坐标系定义椭圆柱坐标系,则
\begin{equation}\label{EliCor_eq1}
\left\{\begin{aligned}
&x=c\cosh\xi\cdot\cos\eta\\
&y=c\sinh\xi\cdot\sin\eta\\
&z=z
\end{aligned}\right.
\end{equation}
其中 $\xi\geq0,0\leq\eta<2\pi,-\infty<z<+\infty$.

显然,\autoref{EliCor_eq4} 、\autoref{EliCor_eq5} 成立.现在情况变成:$\xi$的等值曲面为一组共焦椭圆柱面族,而$\eta$ 的等值曲面为一组共焦的双曲柱面族,$z$的等值面为椭圆坐标面.

只增加垂直于椭圆坐标面的坐标轴$z$意味着,椭圆柱坐标系是一个正交曲线坐标系.其单位矢量为
\begin{equation}
\left\{
    \begin{aligned}
    &\uvec{\xi}=\frac{\pdv*{\bvec{r}}{\xi}}{|\pdv*{\bvec{r}}{\xi}|}=\frac{1}{\sqrt{\sinh^2\xi+\sin^2\eta}}\qty(\sinh\xi\cdot\cos\eta\uvec{x}+\cosh\xi\cdot\sin\eta\uvec{y})\\
    &\uvec{\eta }=\frac{\pdv*{\bvec{r}}{\eta}}{|\pdv*{\bvec{r}}{\eta}}=\frac{1}{\sqrt{\sinh^2\xi+\sin^2\eta}}\qty(-\cosh\xi\sin\eta\uvec{x}+\sinh\xi\cdot\cos\eta\uvec{y})\\
    &\uvec{z}=\frac{\pdv*{\bvec{r}}{z}}{|\pdv*{\bvec{r}}{z}|}=\uvec{z}
    \end{aligned}\right.
\end{equation}

坐标按$(\xi,\eta,z)$排序是由于$\uvec{\xi}\cross\uvec{\eta}=\uvec{z}$(类比直角坐标系$(x,y,z)$中$\uvec x\cross\uvec y=\uvec z$).
\subsection{长椭球坐标系}
设二维椭圆坐标系定义在 $xOz$ 平面上,椭圆长轴与 $z$ 轴重合.将椭圆坐标系绕着 $z$ 轴旋转,便可得到长椭球坐标系(而绕 $x$ 轴旋转则得到扁椭球坐标系.不过,在扁椭球坐标系的情况我们仍选择 $z$ 轴为旋转轴,而将焦点至于 $x$ 轴上).我们将另一坐标记为 $\phi$.

现在,情况是这样的:$\xi$的等值曲面为旋转椭球面,$\eta$的等值曲面为双叶旋转双曲面.由旋转对称性知,任一过旋转轴 $z$ 轴的平面都是椭圆坐标面(因为该平面与原来的椭圆坐标面等价),那么正交性要求 $\phi$ 坐标线必是垂直于旋转轴 $z$ 轴的平面与旋转椭球面的交线.即 $\omega$ 等值面为一组过旋转轴 $z$ 轴的半平面.这意味着,$\omega$ 等值面用直角坐标表示为
\begin{equation}
y=f(\phi)x
\end{equation}
$f(\phi)$ 是 $\phi$等值面与 $xOy$ 面的交线与 $x$轴夹角的正切值,注意坐标零点选取的任意性,那么可取该夹角即为 $\phi$,则
\begin{equation}
y\cos\phi=x\sin\phi
\end{equation}
要该式永远成立,必有
\begin{equation}
y=g(\xi,\eta)\sin\phi,\quad
x=g(\xi,\eta)\cos\phi
\end{equation}

$\xi$ 和 $\eta$ 的等值面用直角坐标可表示为
\begin{equation}
\begin{aligned}
\frac{z^2}{c^2\cosh^2\xi}+\frac{x^2+y^2}{c^2\sinh^2\xi}=1\\
\frac{z^2}{c\cos^2\eta}-\frac{x^2+y^2}{c^2\sinh^2\xi}=1
\end{aligned}
\end{equation}



\addTODO{图}

\addTODO{写出梯度散度旋度, 引用 “正交曲线坐标系中的矢量算符\upref{CVecOp}”}
