% 计算机图形学
% 计算机科学 计算机图形学

\subsection{定义}
计算机图形学是研究计算机在在硬件和软件的帮助下创建计算机图形的科学学科,是计算机科学的一个重要分支领域.
计算机图形学主要研究如何用计算方法来操作视觉和几何信息.它主要聚焦于图像生成和处理的数学和计算基础,而不只是纯艺术方面.

\subsection{分支学科}
几何:研究表面的表示和处理方法

动画:研究运动的表示和处理方法

渲染(绘制):研究模拟光线传递的再现算法

成像:图像获取或图像编辑


\subsection{计算机动画}
计算机动画(Computer animation)通常指场景中任何随着时间推移而发生的视觉变化.除了对象的平移、旋转之外,计算机动画还可以随着时间的进展而改变对象大小、颜色、透明度和表面纹理等.很多计算机动画还需要有真实感的显示.在科学和工程研究中,研究人员常使用随时间而变化的伪彩色或抽象形体来表示物理量,从而有助于理解物理学过程的本质.在影视和娱乐广告中,


\subsection{应用}
计算机图形学广泛应用于多种领域,包括科学、艺术、工程、医药、影视和娱乐等诸多方面.