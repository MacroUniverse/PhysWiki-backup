% 三维空间中的曲面
% 曲面|光滑曲面|正则曲面|surface|regular surface|smooth surface|局部坐标系|parametrization
\pentry{切空间(欧几里得空间)\upref{tgSpaE}}

\subsection{光滑曲面}

在欧几里得$\mathbb{R}^n$中定义一个\textbf{连续函数},对于任意实数$a\in\mathbb{R}$,满足方程$f=a$的全体点的集合$\{(x, y, z)|f(x, y, z)=a\}$就构成一个\textbf{曲面(surface)}.

当$f$是光滑函数时,以上曲面又被称为\textbf{光滑曲面(smooth surface)}.

通常,由于$f$加减一个常数以后不会改变其连续性和光滑性,因此我们常使用$f=0$来作为定义曲面的方式.对于上述$f=a$的曲面,只需要令$g=f-a$,就可以用$g=0$来表示$f=a$的曲面了.

\subsection{三维空间中的正则曲面}

古典微分几何中主要研究的对象是三维空间中的\textbf{正则曲面}.直观来说,正则曲面就是没有折痕、没有尖角、不会和自身相交的曲面,这样我们可以肆无忌惮地在上面应用微积分的工具来进行研究.为了引出正则曲面的严格定义,我们需要先引出以下概念:

\begin{definition}{微分}
给定光滑映射$f:\mathbb{R}^m\to\mathbb{R}^n$,并给定$p\in\mathbb{R}^m$,那么$f$在$p$处的一个\textbf{微分(differential)}被定义为切空间$T_p\mathbb{R}^m$到$T_p\mathbb{R}^n$的映射,且是$f$的Jacobi矩阵.函数$f$在点$p$处的的微分记为$\dd f_p$.
\end{definition}

注意,下标$p$是作为整个$\dd f$符号的下标,而不仅仅是$f$符号的下标.

换一种说法,映射$f$把道路映射为道路,而我们在\textbf{切空间(欧几里得空间)}\upref{tgSpaE}词条里强调过,道路就是切向量,因此这个道路到道路的映射就被称为$f$的微分.从这个角度来理解,还可以很容易地引入曲面之间乃至流形之间的微分,就不赘述了.


\begin{definition}{局部坐标系}
对于集合$V\in \mathbb{R}^3$,如果存在一个映射$\bvec{x}:U\to V$,其中$U$是\textbf{二维平面}$\mathbb{R}^2$的\textbf{单连通开集},满足以下三个条件:
\begin{itemize}
\item $\bvec{x}$看成三个标量函数(即三个分量)时,这三个分量函数都是$\mathbb{R}^2$上偏导连续的.
\item $\bvec{x}$是$U$和$V$的同胚.
\item 对于任意$p\in U$,微分$\mathrm{d}\bvec{x}_p$是一个双射.
\end{itemize}
则称$\bvec{x}$是一个\textbf{局部坐标系(reparametrization)}.
\end{definition}

由于定义中要求$\bvec{x}$的三个分量都是偏导连续的,$U$和$V$同胚,故$V$必然是一个连续可微的曲面,同胚同时还限制了.而第三条要求则限制了折痕、尖角
