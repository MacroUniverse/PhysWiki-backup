% 张量的坐标变换

\pentry{张量\upref{Tensor}, 过渡矩阵\upref{TransM}}

在\textbf{张量}\upref{Tensor}词条中我们看到,张量表示为矩阵依赖于相关的各线性空间中基的选择.本节将讨论基的变换是如何影响张量的矩阵表示的.

\subsection{一阶张量的坐标变换}
一阶张量是将一个向量映射为一个数,因此只涉及一个线性空间,最为简单.

给定$k$维线性空间$V$,及其上一个张量$f:V\rightarrow\mathbb{R}$.如果$V$的基是$\{\bvec{e}_1, \cdots\bvec{e}_k\}$,那么坐标为$(x_1\cdots x_k)\Tr$的向量$\bvec{v}$被映射为:
\begin{equation}
f(\bvec{v})=\sum\limits_{i=1}^k x_if(\bvec{e}_i)
\end{equation}

因此,$f$可以表示为$V$中的一个向量,坐标为$(f(\bvec{e}_1), \cdots, f(\bvec{e}_k))\Tr$.

若取另一个基$\{\bvec{e}_1', \cdots, \\bvec{e}_k'\}$,其中过渡矩阵为$\bvec{Q}$.如果记向量$\bvec{v}$在两个基下的坐标分别为列向量$\bvec{c}$和$\bvec{c}'$,那么$\bvec{Q}\bvec{c}'=\bvec{c}$.

为以下讨论的方便计,我们记$\bvec{Q}$中$i$行$j$列的矩阵元为$a_{ij}$.

在新的基下,$f$的坐标变成
\begin{equation}
\begin{aligned}
(f(\bvec{e}_1'), \cdots, f(\bvec{e}_k'))\Tr&=\\&=
\end{aligned}
\end{equation}




\subsection{二阶张量的坐标变换}




