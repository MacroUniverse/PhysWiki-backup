% 张量
\pentry{线性映射,映射\upref{map}}

%线性映射词条需要从抽象角度重新创作;线性映射词条大概率会引用映射词条,到时候要在预备知识中删除映射词条这一项

\subsection{线性函数}

\begin{definition}{线性函数}
给定域$\mathbb{F}$上的$n$维线性空间$V$,称$f:V\rightarrow \mathbb{F}$为$V$到域$\mathbb{F}$上的一个\textbf{线性函数},如果$f$满足以下线性性:对于任意的$\bvec{v}_1, \bvec{v}_2\in V$和$a_1, a_2\in\mathbb{F}$,都有$a_1f(\bvec{v}_1)+a_2f(\bvec{v}_2)=f(a_1\bvec{v}_1+a_2\bvec{v}_2)$.
\end{definition}

如果把域$\mathbb{F}$本身看成一个一维的线性空间,那么线性函数就是$V$到这个一维空间上的线性映射.因此,我们只需要研究$V$的基向量被$f$映射到哪里,就可以计算出任意的$\bvec{v}\in V$被映射到哪里了.

设$V$有基$\{\bvec{e}_i\}_{i=1}^n$,则在这个基中,$\bvec{e}_1$的坐标为$(1,0,0,\cdots,0)^T$,$\bvec{e}_2$的坐标为$(0,1,0,\cdots,0)^T$,




