% 时间的计算

\subsection{秒的定义}
一秒在历史上有过三种定义, 16 世纪末开始出现以秒为最小单位的钟表.当时秒的定义是将一个太阳日划分为 $24\times60\times60 = 86400$ 等分, 并定义每等分一秒. 而一个\textbf{太阳日(solar day)}可以定义为地球上某处观察到太阳从一天的最高点到第二天最高点的时间间隔. 太阳日的定义今天仍然有效.

第一个定义的问题在于潮汐力的作用, 地球的自转速度并不恒定, 1940 年左右石英钟的精确度已经超过了地球自转所定义的秒. 科学家发现用地球的回归年定义秒会更精确 1956 年, 科学家将一秒钟定义为 1900 年

但, 于是人们将秒重新用铯原子的跃迁时间来定义\upref{SIunit}. 由于地球日在下逐渐变长, 目前一个地球日略大于 86400 秒. 如果不做任何修正, 那么地方时(如格林尼治标准时间)中午 12 点太阳将不会出现在(格林尼治的)最高点.
