% 加速度的坐标系变换

\pentry{速度的参考系变换\upref{Vtrans}}

\subsection{无相对转动}
类比\autoref{Vtrans_eq1}\upref{Vtrans}, 若两个参考系之间只有平移没有转动, 令某时刻点 $P$ 相对于 $S$ 系和 $S'$ 系的加速度分别为 $\bvec a_S$ 和 $\bvec a_{S'}$, 再令两坐标系中任意两个固定点(例如各自的原点)之间的加速度为 $\bvec a_r$, 那么有
\begin{equation}\label{AccTra_eq1}
\bvec a_S = \bvec a_{S'} + \bvec a_r
\end{equation}
同样地, 如果要将该式写成分量的形式, 三个矢量必须使用同一坐标系(见\autoref{Vtrans_ex2}\upref{Vtrans}).

\subsection{一般情况}
类比\autoref{Vtrans_eq2}\upref{Vtrans}, 若两参考系之间有可能存在转动, 我们将以上 $\bvec a_r$ 定义中的两个固定点取 $t$ 时刻点 $P$ 在两参考系中的坐标. 那么是否仍然有 $\bvec a_S = \bvec a_{S'} + \bvec a_r$ 呢? 答案是否定的, 正确的表达式需要添加一项
\begin{equation}\label{AccTra_eq2}
\bvec a_S = \bvec a_{S'} + \bvec a_r + 2 \bvec \omega \cross \bvec v
\end{equation}
其中 $\bvec \omega$ 是 $S'$ 系相对于 $S$ 系的瞬时角速度. 最有一项被称为\textbf{科里奥利加速度}
\begin{equation}
a_c = 2 \bvec \omega \cross \bvec v
\end{equation}
同样, 如果要将\autoref{AccTra_eq2} 写成分量的形式, 所有矢量必须使用同一坐标系.

\begin{exercise}{}
请使用 \autoref{Vtrans_ex1}\upref{Vtrans} 的场景验证\autoref{AccTra_eq2}.
\end{exercise}

\subsection{证明(旋转矩阵)}
\pentry{空间旋转矩阵\upref{Rot3D}}
我们在 $S$ 系中以坐标的形式证明\autoref{AccTra_eq2}, 即式中的矢量都看作是 $S$ 系中的三个坐标. 令点 $P$ 在两系中的坐标分别为 $\bvec r_S(t) = (x, y, z)\Tr$ 和 $\bvec r_{S'}(t) = (x', y', z')\Tr$, 且坐标变换可以用一个旋转矩阵 $\mat R(t)$ 和一个平移矢量 $\bvec d(t)$ 表示为
\begin{equation}
\bvec r_S = \mat R \bvec{r_{S'}} + \bvec d
\end{equation}
两边求二阶导数得
\begin{equation}
\dot{\bvec r}_S = \dot{\mat R} \bvec{r_{S'}} + \mat R \dot{\bvec r}_{S'}+ \dot{\bvec d}
\end{equation}



相对于惯性系 $S$ 绕 $z$ 轴以角速度 $\omega$ 逆时针匀速旋转(右手定则\upref{RHRul}). 由于 $z$ 轴和 $c$ 轴始终重合( $z=c$), 只需要考虑 $x,y$ 坐标和 $a,b$ 坐标之间的关系即可.

令平面旋转矩阵为% 未完成:链接
\begin{equation}
\mat R(\theta) \equiv \begin{pmatrix}
\cos \theta & - \sin \theta \\
\sin \theta & \cos \theta
\end{pmatrix}
\end{equation}
其意义是把坐标逆时针旋转角 $\theta$. 两坐标系之间的坐标变换为
\begin{equation}
\pmat{x\\y}_{S} = \mat R(\omega t) \pmat{a\\b}_{S'}
\qquad
\pmat{a\\b}_{S'} = \mat R(-\omega t) \pmat{x\\y}_{S}
\end{equation}
为了得到质点在惯性系中的加速度,对上面左式的 $(x,y)\Tr$ 求二阶时间导数得\footnote{某个量上方加一点表示对时间的一阶导数,两点表示对时间的二阶导数.} $S$ 系中的加速度(以 $\uvec x, \uvec y$ 为基底)
\begin{equation}\label{AccTra_eq3}
\bvec a_{S} = \pmat{\ddot x \\ \ddot y}_{S} = 
\ddot{\mat R}(\omega t) \pmat{a\\b} + 2\dot{\mat R} (\omega t) \pmat{\dot a \\ \dot b} + \mat R(\omega t)\pmat{\ddot a \\ \ddot b}
\end{equation}
其中\footnote{\autoref{AccTra_eq4} 和\autoref{AccTra_eq5} 相当于用矩阵推导了匀速圆周运动的速度和加速度公式\upref{CMVD}\upref{CMAD}.}
\begin{equation}\label{AccTra_eq4}
\dot{\mat R}(\omega t) = \omega \begin{pmatrix}
\cos(\omega t + \pi /2) &  - \sin(\omega t + \pi /2)\\
\sin(\omega t + \pi /2) & \cos(\omega t + \pi /2)
\end{pmatrix}
= \omega \mat R(\omega t + \pi /2)
\end{equation}
\begin{equation}\label{AccTra_eq5}
\ddot{\mat R} (\omega t)  =  - \omega ^2 \mat R (\omega t)
\end{equation}
 代入\autoref{AccTra_eq3} 得
\begin{equation}
\bvec a_{S} =
- \omega ^2 \mat R(\omega t)\pmat{a\\b} + 2\omega \mat R(\omega t + \pi /2)\pmat{\dot a \\ \dot b} + \mat R(\omega t)\pmat{\ddot a \\ \ddot b}
\end{equation}
上式中的每一项都是以 $\uvec x, \uvec y, \uvec z$ 为基底的坐标.所有坐标乘以 $\mat R(-\omega t)$, 得到以 $\uvec a, \uvec b, \uvec c$ 为基底的坐标
\begin{equation}
\bvec a_{S} =
- \omega^2 \pmat{a\\b}_{S'} + 2\omega \mat R(\pi /2)\pmat{\dot a\\ \dot b}_{S'} + \pmat{\ddot a\\ \ddot b}_{S'}
\end{equation}
所以旋转参考系中的总惯性力(\autoref{Iner_eq1}\upref{Iner})为(以 $\uvec a, \uvec b, \uvec c$ 为基底)
\begin{equation}\label{AccTra_eq6}
\bvec f = m(\bvec a_{S'} - \bvec a_{S})
=  m \omega ^2 \pmat{a\\b}_{S'} - 2m\omega \mat R(\pi /2)\pmat{\dot a\\ \dot b}_{S'}
\end{equation}
其中第一项是已知的离心力\autoref{Centri_eq3}\upref{Centri}, 我们将第二项定义为科里奥利力 $\bvec F_c$. 科里奥利力可以用叉乘记为
\begin{equation}
\bvec F_c = 2m \bvec v_{S'} \cross \bvec \omega
\end{equation}
其中 $\bvec\omega$ 是 $S'$ 系旋转的角速度矢量, $\bvec v_{S'}$ 是质点相对于 $S'$ 系的速度.最后, 我们可以写出\autoref{AccTra_eq6} 的矢量形式
\begin{equation}
\bvec f = m \omega ^2 \bvec r + 2m \bvec v_{S'} \cross \bvec \omega 
\end{equation}
 