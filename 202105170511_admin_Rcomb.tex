% 电阻的串联和并联
% 电阻|串联|并联|欧姆定律|电流|电压

\pentry{欧姆定律\upref{Resist}}
\addTODO{图}
\subsubsection{电阻的串联}
如左边为两电阻的串联.对于串联,流过串联电阻的电流相同,即
\begin{equation}
I_1=I_2=I
\end{equation}
根据欧姆定律\autoref{Resist_eq5}~\upref{Resist}
\begin{equation}
\begin{aligned}
U_1=IR_1\\
U_2=IR_2
\end{aligned}
\end{equation}
所以,两电阻串联后总电阻为
\begin{equation}
R = \frac{U}{I}=\frac{U_1+U_2}{I} = {R_1 + R_2}
\end{equation}

对于$n$个电阻的串联,可采取同样的证明方法,结果有
\begin{equation}
R=\sum_{i=1}^{n}R_i
\end{equation}
或者用数学归纳法证明,我们将此作为习题.
\subsubsection{电阻的并联}
如左边为两电阻的并联.对于并联,两电阻电压相同,即
\begin{equation}
U_1=U_2=U
\end{equation}
根据欧姆定律\autoref{Resist_eq5}~\upref{Resist}
\begin{equation}
\begin{aligned}
I_1=\frac{U}{R_1}\\
I_2=\frac{U}{R_2}
\end{aligned}
\end{equation}
所以,两电阻并联后总电阻为
\begin{equation}
R = \frac{U}{I}=\frac{U}{I_1+I_2} =\frac{R_1R_2} {R_1 + R_2}
\end{equation}
或
\begin{equation}
\frac{1}{R} = \frac{1} {R_1}+\frac{1}{R_2}
\end{equation}

完全一样的做法,对于$n$个电阻的并联,有
\begin{equation}
\frac{1}{R}=\sum_{i=1}^{n}\frac{1}{R_i}
\end{equation}
这同样可用数学归纳法证明,留做习题.
