%test
\subsection{命题及其表示法}



\begin{definition}{}
能表达判断的语言是陈述句, 它称作命题.\\ 
 一个命题,总是具有一个“值”,称为真值.真值只有\textbf{真(True)}和\textbf{假(False)} 两种,记作(真)和 False(假),分别用符号T和F表示只有具有确定真值的陈述句才是命题,一切没有判断内容的句子,无所谓是非的句子,如感叹句、疑问句、祈使句等都不能作为命题.
\\
命题有两种类型:
\begin{enumerate}
\item 原子命题\\不能分解为更简单的陈述语句
\item 复合命题\\由联结词、标点符号和原子命题复合构成的命题
\end{enumerate} 
\end{definition}


\subsection{联结词}

\begin{definition}{否定$\neg$ }\end{definition} 
%\begin{definition}[合取] $\wedge$ \end{definition} 
%\begin{definition}[析取] $\vee$ \end{definition} 
%\begin{definition}[条件] $\rightarrow$ \end{definition} 
%\begin{definition}[双条件] $\leftrightarrow$ \end{definition} 
