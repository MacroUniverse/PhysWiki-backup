% 相对论流体

\pentry{相对论动力学\upref{SRDyn}}

\subsection{均匀等速尘埃云的动量通量}

\subsubsection{均匀等速尘埃云}

在任意参考系中,空间中分布若干质点.这些质点的集合,被称为一片\textbf{尘埃云(dust)},各质点被称为\textbf{尘埃粒子}.如果在某个参考系中,一片尘埃云的各质点都保持静止,那么我们称这个参考系是尘埃云的\textbf{自身系},称尘埃云为\textbf{等速尘埃云},因为这意味着在其它参考系中,尘埃粒子的速度都会是相同的.如果在某个参考系中,尘埃粒子的质量相同、在空间中均匀分布,那么我们称这片尘埃云是\textbf{均匀的}.

任何一片\textbf{均匀尘埃云},都可以看成是许多\textbf{均匀等速尘埃云}的叠加,只要把属于各个速度的尘埃粒子分别拿出来构成尘埃云即可.而任何一片尘埃云,也可以看成是局部均匀的.因此,研究均匀等速尘埃云的性质最为容易,也可以方便地拓展到任意尘埃云的性质中.

\subsection{尘埃云数量通量密度}

假设空间中有一片均匀等速尘埃云,在其自身系中各点的数量密度都是$n$,其中$n$是一个实数.也就是说,在尘埃云的自身系中,在任何体积$V$中,尘埃粒子的数量都是$nV$.

取$K_1$参考系作为观察者,设观察者认为尘埃云的速度是$\bvec{u}=(u_x, u_y, u_z)^T$.


