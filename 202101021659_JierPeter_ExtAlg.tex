% 外代数
% 外代数|外微分|线性空间|格拉斯曼代数|Grassmann|外积|矢量分析|向量分析|外积|外乘|楔积|楔乘
\addTODO{Graded vector space在本文中翻译为“赋次线性空间”,但作者不确定这是否是官方翻译.}
\pentry{张成空间\upref{VecSpn},四元数\upref{Quat},域上的代数\upref{AlgFie}}

外代数是一种利用已有线性空间构造“代数”这一对象的通用方法,同时蕴含了对三维矢量分析中代数结构的本质解释.


给定线性空间$V$,任取$x, y\in V$,定义$x\wedge y\not\in V$是一个新的元素,其中符号$\wedge$称作\textbf{外积(exterior product)},有时也叫做\textbf{楔积(wedge product)},前者是因为这个运算得到的是$V$以外的新元素,后者是由于符号长得像个楔子.注意为了方便,我们没有使用线性代数中常见的粗体正体符号来表示向量.

利用各$x\wedge y$构造新的线性空间:定义$x\wedge y=-y\wedge x$对所有$x, y\in V$成立,这同时意味着$x\wedge x=0$.集合$\{x\wedge y|x, y\in V\}$张成的线性空间,记为$\bigwedge^2 V$.同时,为了统一考虑,记$V=\bigwedge^1 V$.

$\bigwedge^1 V$和$\bigwedge^2 V$之间也可以进行楔积,并且满足\textbf{结合律}:$x\wedge(y\wedge z)=(x\wedge y)\wedge z$,由此可以拿掉结合括号,定义$x\wedge y\wedge z=x\wedge(y\wedge z)=(x\wedge y)\wedge z$.集合$\{x\wedge y\wedge z|x, y, z\in V\}$张成的线性空间,记为$\bigwedge^3 V$.

同理,我们可以构造出任意阶的$\bigwedge^k V$.要注意的是,如果$k>\opn{dim} V$,那么$\bigwedge^k V=\{0\}$.另外,把$V$的基本域$\mathbb{F}$看成一个一维线性空间,记$\mathbb{F}=\bigwedge^0 V$.

不同线性空间之间可以用直和组合在一起,因此以上这些空间也都可以作直和,得到一个$\bigwedge V=\bigoplus_k\bigwedge^k V=\mathbb{F}\oplus\bigwedge^1V\oplus\bigwedge^2V\cdots$.这个$\bigwedge V$,就被称作$V$上的\textbf{外积空间(exterior product space)}或\textbf{楔积空间(wedge product space)}.$\mathbb{F}$是$V$的基域,视为$\mathbb{F}$自身上的一维线性空间.

\begin{theorem}{外代数}
任给域$\mathbb{F}$上的线性空间$V$,则外积$\wedge$是一个$\bigwedge V$上的向量乘法,并且满足结合性,有单位元$1\in \bigwedge^0 V$,故构成一个$\mathbb{F}$上的代数.称这个代数为$V$上的\textbf{外代数(exterior algebra)}或\textbf{格拉斯曼代数(Grassmann algebra)}.
\end{theorem}

外代数中的元素可以有形象的几何理解.$\bigwedge^1 V$中的元素就是$V$中的元素,我们可以想象成箭头.$\bigwedge^2 V$中的元素可以看成箭头对,或者是箭头对表示的平行四边形.同样,$\bigwedge^k V$中的元素都可以看成是$k$个箭头张成的一个$k$维对象.

三维空间$\mathbb{R}^3$中的叉乘实际上就是外积.这是因为,$\opn{dim}\mathbb{R}^3=\opn{dim}\bigwedge^2\mathbb{R}^3$,这样一来,如果给定$\mathbb{R}^3$的标准正交基$\{x, y, z\}$,那么我们可以建立同构$*: \bigwedge^2\mathbb{R}\rightarrow\mathbb{R}^3$,使得$*(x\wedge y)=z, *(y\wedge z)=x, *(z\wedge x)=y$,这样就可以通过这个同构来把外积变成$\mathbb{R}^3$内部的向量积.这一映射也是叉乘的“右手定则”的来源,我们也完全可以规定$*(x\wedge y)=-z, *(y\wedge z)=-x, *(z\wedge x)=-y$,这样定义出来的叉乘就是符合左手定则的了.

三维线性空间是唯一可以构造反交换代数的非平凡空间,就是因为只有三维的$V$才满足$\opn{dim}V=\opn{dim}\bigwedge^2V$,因而可以建立$\bigwedge^2 V$和$V$之间的同构,从而把楔积变成叉积.相应地,比复数更高维的可除代数只有四元数.

外代数是一个“\textbf{赋次线性空间(graded vector space)}”,就是说,它作为一个线性空间,每个向量具有一个“次数”,定义如下:每个$\bigwedge^kV$中的向量,其次数就是$k$;对于
