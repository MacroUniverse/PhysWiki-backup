% 加速度的坐标系变换

\pentry{速度的参考系变换\upref{Vtrans}}

\subsection{无相对转动}
类比\autoref{Vtrans_eq1}\upref{Vtrans}, 若两个参考系之间只有平移没有转动, 令某时刻点 $P$ 相对于 $S$ 系和 $S'$ 系的加速度分别为 $\bvec a_S$ 和 $\bvec a_{S'}$, 再令两坐标系中任意两个固定点(例如各自的原点)之间的加速度为 $\bvec a_r$, 那么有
\begin{equation}\label{AccTra_eq1}
\bvec a_S = \bvec a_{S'} + \bvec a_r
\end{equation}
同样地, 如果要将该式写成分量的形式, 三个矢量必须使用同一坐标系(见\autoref{Vtrans_ex2}\upref{Vtrans}).

\subsection{一般情况}
类比\autoref{Vtrans_eq2}\upref{Vtrans}, 若两参考系之间有可能存在转动, 牵连加速度 $\bvec a_{r}$ 的定义会变得比 $\bvec v_r$ 更微妙, 我们举例解释

\begin{example}{牵连加速度}
我们讨论\autoref{Vtrans_ex1}\upref{Vtrans} 的情景, $t_0$ 时刻点 $P$ 和两坐标系的固定点 $P_{S}$ 和 $P_{S'}$ 都重合
\begin{equation}
\bvec r(t_0) = \alpha t_0 \uvec x' = \alpha t_0 (\cos\omega t_0\, \uvec x + \sin\omega t_0\, \uvec y)
\end{equation}
在 $S$ 系中求二阶导数得 $P_{S'}$ 在 $t_0$ 时相对于 $S$ 系得加速度
\begin{equation}
\bvec a_{S'}^{(S)}(t_0) = - \alpha \omega^2 t_0 (\cos\omega t_0\, \uvec x + \sin\omega t_0\, \uvec y)
\end{equation}

$t_0$ 时刻 $S$ 系得固定点 $P_{S}$ 相对于 $S'$ 系做反方向圆周运动
\begin{equation}
\bvec r(t) = \alpha t_0 \uvec x' = 
\end{equation}


\end{example}

为简单起见, 我们现在只在 $S$ 系中讨论问题. 定义 $t$ 时刻点 $P$ 在 $S'$ 系中的固定点相对于 $S$ 系的加速度为 $\bvec a_{r}$. 那么有(比\autoref{AccTra_eq1} 多出了一项)
\begin{equation}\label{AccTra_eq2}
\bvec a_S = \bvec a_{S'} + \bvec a_{r} + 2 \bvec \omega \cross \bvec v_{S'}
\end{equation}
其中 $\bvec \omega$ 是 $S'$ 系相对于 $S$ 系的瞬时角速度. 最后一项被称为\textbf{科里奥利加速度(Coriolis Acceleration)}
\begin{equation}
\bvec a_c = 2 \bvec \omega \cross \bvec v_{S'}
\end{equation}
再次强调, 所有的矢量的坐标必须都是 $S$ 系基底的坐标.

\begin{exercise}{}
请使用\autoref{Vtrans_ex1}\upref{Vtrans} 的情景验证\autoref{AccTra_eq2}.
\end{exercise}

\subsection{证明(旋转矩阵)}
\pentry{旋转矩阵的导数\upref{RotDer}}
我们在 $S$ 系中以坐标的形式证明\autoref{AccTra_eq2}, 即式中的矢量都看作是 $S$ 系中的三个坐标. 令点 $P$ 在两系中的坐标分别为 $\bvec r_S(t) = (x, y, z)\Tr$ 和 $\bvec r_{S'}(t) = (x', y', z')\Tr$, 且坐标变换可以用一个旋转矩阵 $\mat R(t)$ 和一个平移矢量 $\bvec d(t)$ 表示为
\begin{equation}
\bvec r_S = \mat R \bvec{r_{S'}} + \bvec d
\end{equation}
两边关于时间求导得\footnote{用一点表示时间导数, 两点表示时间二阶导数}
\begin{equation}
\dot{\bvec r}_S = \dot{\mat R} \bvec{r_{S'}} + \mat R \dot{\bvec r}_{S'}+ \dot{\bvec d}
\end{equation}
再求导并整理得
\begin{equation}\label{AccTra_eq3}
\ddot{\bvec r}_S = \mat R \ddot{\bvec r}_{S'} + (\ddot{\mat R} \bvec r_{S'} + \ddot{\bvec d}) + 2 \dot{\mat R} \dot{\bvec r}_{S'}
\end{equation}
下面我们只需证明这三项分别对应\autoref{AccTra_eq2} 的各项即可.

在 $S$ 系中, 显然有 $\bvec a_S = \ddot{\bvec r}_S$. $P$ 在 $S'$ 系中的加速度为 $\ddot{\bvec r}_{S'}$, 乘以旋转矩阵就变换到 $S$ 系中, 所以 $\bvec a_{S'} = \mat R \ddot{\bvec r}_{S'}$.

若 $S'$ 系中的固定点 $\bvec r_{S'}$ 不随时间变化, 则求二阶导数得 $S'$ 系中固定点相对于 $S$ 系中固定点的加速度(在 $S$ 系中的坐标)
\begin{equation}
\bvec a_r = \ddot{\mat R} \bvec r_{S'} + \ddot{\bvec d}
\end{equation}

再来看\autoref{AccTra_eq3} 最后一项, 将\autoref{RotDer_eq4}\upref{RotDer} 带入, 得
\begin{equation}
2\dot{\mat R} \dot{\bvec r}_{S'} = 2\mat\Omega (\mat R \dot{\bvec r}_{S'}) = 2\bvec \omega \cross \bvec v_{S'}
\end{equation}
这就是\autoref{AccTra_eq2} 的最后一项. 证毕.
