% 三维欧几里得空间中的曲线
% 曲线|微分几何|曲率|curvature|扭率|第二曲率|torsion

三维欧几里得空间是我们最为熟悉的空间,其性质也非常好,易于研究.本节讨论的是三维空间中的曲线的性质,引入曲率、挠率等概念.

\subsection{概念}

\begin{definition}{曲线}
令$I$是实数轴$\mathbb{R}$上的一个区间,则称\textbf{连续函数}$f:I\to \mathbb{R}^3$为$\mathbb{R}^3$中的一条\textbf{曲线}.此处连续是指函数的三个分量都是$\mathbb{R}\to\mathbb{R}$的连续函数.
\end{definition}

我们可以任意取定一个坐标系,把向量值函数$f$分为三个标量值函数,简称为$f$的分量,由此来理解连续的含义.你可能自然会想确认,$f$的分量的连续性,和取定坐标系的方式是否有关?答案是无关的.这是因为我们可以用另一种方式来理解此处的“连续”,那就是取集合$I$与$\mathbb{R}^3$,配上通常的拓扑——即$I$取$\mathbb{R}$的子拓扑,$\mathbb{R}^3$取$\mathbb{R}$的乘积拓扑——所得到的拓扑空间,那么$f$就是拓扑空间之间的映射,其连续性取决于拓扑意义上的连续性,和具体坐标系的选择就无关了.








