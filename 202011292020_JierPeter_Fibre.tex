% 纤维丛
% 纤维丛|丛空间|拓扑学|拓扑|向量丛|乘积拓扑

\subsection{定义}

直观来说,纤维丛是指在一个拓扑空间$B$的每一个点都长出来另一个拓扑空间$F$所得到的一个空间.每一个点$x\in B$上的$F$被称为一根\textbf{纤维(fibre)},这些纤维所在的$B$称为\textbf{底空间(base space)},而整个结构$(B, F)$就是一个\textbf{纤维丛(fibre bundle)}.

准确的定义如下所述,其中$E$就是“$B$上每个点都长出一个$F$的丛空间”:

\begin{definition}{纤维丛}
给定拓扑空间$B$和$F$,如果存在一个拓扑空间$E$和一个连续满射$f:E\rightarrow B$,使得对于任意的$x\in B$,都有$f^{-1}(x)\cong F$,那么称这个结构$(E, F, B, f)$为一个\textbf{纤维丛(fibre bundle)},称$E$是这个纤维丛的\textbf{全空间(total space)},$F$是其\textbf{纤维(fibre)},$B$是其\textbf{底空间(base space)},有时也译作\textbf{基空间}.
\end{definition}

如果把$B$想象成一块土地,$F$想象成一棵草,那么$E$就是“土地上长了一片草”这一概念,$E$的每个元素就是某棵草上的一个点.定义中的连续满射$f$的作用是把这样的一个点映射到相应的草所在的地点.






