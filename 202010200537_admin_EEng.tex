% 电场的能量
% 电场|电场能|能量密度

\begin{issues}
\issueDraft
\end{issues}

\pentry{电容\upref{Cpctor}}

\footnote{本文参考 \cite{GriffE}.}首先我们来回顾一下平行板电容器(\autoref{Cpctor_ex2}~\upref{Cpctor})的能量
\begin{equation}
E_p = \frac12 CV^2 = \frac12 \epsilon \frac Sd (Ed)^2 = \frac 12 \epsilon \tau E^2
\end{equation}
其中 $\tau = Sd$ 为平行板间长方体的体积. 这条公式容易让我们联想到电势能储存在电场之中. 在电动力学中, 这种理解是正确的, 我们不妨把电场含有的能量称为\textbf{电场能}. 通过上式, 我们假设空间中任意一点的电场能密度为 $\epsilon E^2/2$, 则总电场能可以通过体积分得到
\begin{equation}
E_p = \frac12 \epsilon \int \bvec E(\bvec r)^2 \dd[3]{r}
\end{equation}
可以证明该式定义的电场能与\autoref{QEng_eq8}~\upref{QEng} 定义的电势能是完全等效的
\begin{equation}\label{EEng_eq1}
E_p = \frac 12 \int V(\bvec r) \rho(\bvec r) \dd[3]{r}
\end{equation}

\subsection{证明}
\pentry{分部积分的高维拓展\upref{IntBP2}}

\addTODO{参考 Griffith 的推导(式 2.43), 使用了分部积分(高维)\upref{IntBP2} 中的结论}

电场的高斯定理\upref{EGauss}
\begin{equation}
\rho = \epsilon_0\div \bvec E
\end{equation}
代入\autoref{EEng_eq1} 得
\begin{equation}
E_p = \frac{\epsilon_0}2 \int V (\div \bvec E) \dd[3]{r}
\end{equation}
由 3 维分部积分\autoref{IntBP2_eq1}~\upref{IntBP2}得
\begin{equation}
E_p = \frac{\epsilon_0}2 \oint \bvec E V \dd{\bvec s} - \frac{\epsilon_0}{2} \int \bvec E \vdot \grad V \dd{V}
\end{equation}
当我们把积分区域拓展到无限大时, 右边第一个面积分趋于 0. 右边第二个积分可以利用电场与电势的关系 $\grad V = -\bvec E$(\autoref{QEng_eq13}~\upref{QEng}). 最后
\begin{equation}
E_p = \frac{\epsilon_0}{2} \int \bvec E \vdot \bvec E \dd{V}
\end{equation}
