% 10

(3)不允许卖空时的最佳投资比例
当不允许卖空时,投资组合各风险资产的权重$W_{i}\geqslant0$,亦即$x_{i}=2W_{i}{\div}\Lambda\geqslant0$,求解方程(6)变成解下列库恩——塔克条件:
$(a) x_{i}{\sigma_{i}^{2}}+{{\sum_{j=1;i\ne j}}^{n}}x_{i}{\sigma_{ij}}-U_{i}=E(R_{i})-r_{f}$
$(b)x_{i}U_{i}=0$
$(c)x_{i}\geqslant0$且$U_{i}\geqslant0$$(i=1,2,\dots,n)$

其中 $U_{i}$是为保证非负性而添加的松弛变量.由(b)可知如果 $x_{i}>0$则必有$U_{i}=0$,即当不存在第 i 种资产卖空时,$U_{i}$从等式(a)中消失. 如果 $U_{i}>0$,则由(b)保证 $x_{i}=0$,因此卖空的资产所占比例可设为零,而不用负数.


当不允许卖空时,凡夏普业绩指数$(R/V_{i})$ 小于它的风险指标$G_{i}$的风险资产,则应当从投资组合中剔除;而$(R/V_{i})$大于它的风险指标$G_{i}$的风险资产,都选入投资组合中.
 将(12)所提供的最优投资比例单位化,即使最优投资比例的权数之和为1,得投资于第 i 种风险资产的标准化最优投资比例为:$z_{i}=x_{i}\div {\sum_{i=1}^{n}x_{i]}}$
 
其中当允许卖空时 m=n;当不允许买空时等于夏普业绩指数$(R/V_{i})$ 大于或等于它的风险指标$G_{i}$的风险资产的个数.




