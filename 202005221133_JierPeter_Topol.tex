% 拓扑空间
% 拓扑|映射|集合|开子集|非开|连续性

\pentry{映射\upref{map},集合\upref{Set}}

\textbf{拓扑空间(topological space)}是能够定义连续性, 连通性, 收敛等性质的最一般化的数学空间. 度量空间和流形等都是拓扑空间的例子.

\subsection{拓扑}

从实函数的连续性\upref{contin}中我们知道开集是讨论连续性的基础,所以拓扑学首先要定义什么是开集.我们提取了开集最重要的特征,然后用这些特征来定义开集.

\begin{definition}{拓扑}\label{Topol_def1}
对于任意给定的集合 $X$, 如果我们按照一定规则将它的子集划分为\textbf{开集(open set)}和\textbf{非开集(closed set)}, 那么所有开集的集合就叫做集合 $X$ 的一个\textbf{拓扑} $\mathcal{T}$. 这个规则是:
\begin{enumerate}
\item 空集 $\varnothing$ 和 $X$ 本身必须是开子集
\item 有限个开子集的交集为开子集
\item 任意个开子集的并集为开子集
\end{enumerate}
\end{definition}

我们可以将 $X$ 的所有不同的子集构成一个集合 $P$, 那么有 $\mathcal{T} \subseteq P$. 集合上的拓扑一般不止一种.

\begin{example}{凝聚拓扑和离散拓扑}
对给定的集合 $X$, 若只定义 $\varnothing$ 和 $X$ 本身为开集, $X$ 的其他子集为非开, 则这个拓扑称为\textbf{凝聚拓扑}或者\textbf{平凡拓扑}, 这是符合\autoref{Topol_def1} 的元素最少的拓扑.

相对地, 若令 $X$ 的任意子集都为开集, 则得到\textbf{离散拓扑}, 这是元素最多的拓扑.
\end{example}

\begin{example}{Sierpinski拓扑}
令集合为$X=\{0, 1\}$,赋予拓扑$\mathcal{T}=\{\varnothing, \{0\}, X\}$,则我们得到了一个Sierpinski空间.
\end{example}

\begin{definition}{拓扑基}
给定拓扑空间$(X, \mathcal{T})$,如果$\mathcal{B}\in 2^X$中任意个元素取并集所生成的集合就是$\mathcal{T}$本身,那么称$\mathcal{B}$是$\mathcal{T}$的\textbf{拓扑基(topological basis)}.
\end{definition}

\begin{example}{度量拓扑}
通常的欧几里得空间$\mathbb{R}^n=\{(x_1, \cdots x_n)|x_i\in \mathbb{R}\}$,记$\{(y_1, \cdots y_n)|y_i\in \mathbb{R}, \sum_i(y_i-x_i)^2\leq r^2\}$为以$(x_1, \cdots x_n)$为球心、$r\geqslant 0$为半径的\textbf{开球(open sphere)}. 那么全体开球的集合$\mathcal{B}$是某个拓扑$\mathcal{T}$的拓扑基,$\mathcal{T}$此时就是一个\textbf{度量拓扑(metric topology)}.这个拓扑就是最常见的实空间上的拓扑,有很直观的几何意义.

由开球生成的这个拓扑,还有一种定义方法:$A\subset \mathbb{R}^n$是一个开集,当且仅当对于任意的$x\in A$,存在一个半径$r$,使得到$x$距离小于$r$的所有点都在$A$内.也就是说,开集的点都是内点.
\end{example}


\begin{definition}{子拓扑}
如果已经给定了一个拓扑$(X, \mathcal{T})$,那么在$X$的一个子集$A$上可以继承一个拓扑$\mathcal{T}|_A$,定义为:
$\mathcal{T}|_A=\{U\cap A|U\in \mathcal{T}\}$.那么$\mathcal{T}|_A$称为$\mathcal{T}$的\textbf{子拓扑(subtopology)}或者\textbf{限制拓扑},$(A, \mathcal{T}|_A)$构成了$(X, \mathcal{T})$的一个\textbf{子拓扑空间(subspace)}.
\end{definition}

\begin{example}{度量空间的子空间}
设$(\mathbb{R}^2, \mathcal{T}_2)$是二维实度量空间,即$x-y$平面;$(\mathbb{R}, \mathcal{T}_1)$是一维实度量空间,即$x$轴.那么$\mathcal{T}_1$刚好是$\mathcal{T}_2$的子拓扑.
\end{example}
