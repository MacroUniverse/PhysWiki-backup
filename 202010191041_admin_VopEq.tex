% 矢量算符常用公式
% 矢量算符|微积分|叉乘|点乘|内积

\footnote{参考教材: David Griffiths, Introduction to Electrodynamics}(未完成: 把 griffiths 书中的全部列出)

\begin{equation}
\curl(U\bvec A) = (\grad U) \cross\bvec A + U \curl\bvec A
\end{equation}

\begin{equation}
\curl(\curl \bvec E) = \grad(\div\bvec E) - \laplacian \bvec E
\end{equation}

\subsection{证明}
理论上, 我们可以直接根据定义, 将各个矢量记为分量的形式证明, 但直接写出来非常繁琐. 一种简单的记号是使用 $\delta_{i,j}$ 和 $\epsilon_{i,j,k}$ 符号, 再结合爱因斯坦求和约定(未完成).

我们另外推荐一种不需要写出分量的推导方法, 见 “一种矢量算符的运算方法\upref{MyNab}”.
