% 拓扑空间
% 拓扑|映射|集合|开子集|非开|连续性

\pentry{映射\upref{map},集合\upref{Set}}

\textbf{拓扑空间(topological space)}是能够定义连续性, 连通性, 收敛等性质的最一般化的数学空间. 度量空间和流形等都是拓扑空间的例子.

\subsection{拓扑}

\begin{definition}{拓扑}\label{Topol_def1}
对于任意给定的集合 $X$, 如果我们按照一定规则将它的子集划分为\textbf{开集(open set)}和\textbf{非开集(closed set)}, 那么所有开集的集合就叫做集合 $X$ 的一个\textbf{拓扑} $\mathcal{T}$. 这个规则是:
\begin{enumerate}
\item 空集 $\varnothing$ 和 $X$ 本身必须是开子集
\item 有限个开子集的交集为开子集
\item 任意个开子集的并集为开子集
\end{enumerate}
\end{definition}

我们可以将 $X$ 的所有不同的子集构成一个集合 $P$, 那么有 $\mathcal{T} \subseteq P$. 集合上的拓扑一般不止一种.

\begin{example}{凝聚拓扑和离散拓扑}
对给定的集合 $X$, 若只定义 $\varnothing$ 和 $X$ 本身为开集, $X$ 的其他子集为非开, 则这个拓扑称为\textbf{凝聚拓扑}或者\textbf{平凡拓扑}, 这是符合\autoref{Topol_def1} 的元素最少的拓扑.

相对地, 若令 $X$ 的任意子集都为开集, 则得到\textbf{离散拓扑}, 这是元素最多的拓扑.
\end{example}

\begin{example}{Sierpinski拓扑}
令集合为$X=\{0, 1\}$,赋予拓扑$\mathcal{T}=\{\varnothing, \{0\}, X\}$,则我们得到了一个Sierpinski空间.
\end{example}

\begin{definition}{拓扑基}
给定集合$X$,如果$$
\end{definition}

\begin{example}{实空间拓扑}
通常的欧几里得空间$\mathbb{R}^n$,
\end{example}


\begin{definition}{子拓扑}
如果已经给定了一个拓扑$(X, \mathcal{T})$,那么在$X$的一个子集上
\end{definition}
