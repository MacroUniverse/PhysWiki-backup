% 校验和

\textbf{校验和(checksum)}是指一类算法, 可以将任意字节的数据转换为一个简短的值. 我们通常也将其称为文件的\textbf{指纹(fingerprint)}或者\textbf{哈希值(hash)}\footnote{严格来说这些不是同义词, 但人们经常混用}. 这可以用于检查两个文件的内容是否相同(即使文件名不同). % 未完成: 文件的结构: 文件名,内容, metadata, 文件系统?

通常来说, 文件在储存或传输的过程中有一定可能会被损坏, 例如网络故障, 黑客攻击, 或者储存介质的个别字节出现故障. 如果我们知道文件原来的校验和, 就可以随时重新做一次校验和, 如果结果和以前不一样, 就说明文件被改变了. 这也是为什么一些网站的提供文件下载时同时也会提供各种校验和, 以便用户下载以后对照校验和确保文件无误.

然而要达到这个目的, 理想的情况就是确保任何文件和它的指纹都是一一对应\upref{map}的. 但实际上, 我们只能确保内容相同的文件得到同样的指纹, 或者说不同的指纹必定对应内容不同的文件. 不同的文件有可能对应相同的指纹


理想状态下, 如果文件和指纹

在 windows 上, 著名压缩软件 7zip % 链接未完成
可以

\subsection{用于对比文件}
\textbf{SHA1(Secure Hash Algorithm 1)}著名文件版本控制软件 Git % 链接未完成
就是使用 SHA1 散列值来检查文件是否发生变化.

\subsection{用于服务器保存密码}
对于一些加密的校验和算法, 几乎不可能从校验和逆像推出原文内容. 这个特性可以用于在服务器中保存密码.如果一个网站的服务器中直接保存用户的密码明文, 那么一旦这些明文密码被泄露, 那么得到密码的人将可以任意访问用户的数据.

更好的做法是服务器仅仅保存每个用户密码的哈希值, 用户每次输入密码时, 服务器先将密码转换为哈希值再验证是否正确. 这样即使这些哈希值泄露, 也不能把他们作为密码直接使用, 或者逆像算出密码原文.
