% 线性映射的结构

\pentry{补空间\upref{DirSum}, 矩阵的秩\upref{MatRnk}}

如果线性映射是一一对应(双射)\upref{map}的, 那么它的结构简单明了, 没有太多值得讨论的. 我们现在来看一个揭示多对一线性映射的结构的重要定理. 考虑线性映射 $A:X\to Y$, 线性空间 $X$ 的零空间(\autoref{LinMap_the1}~\upref{LinMap}) $X_0$ 中的每个矢量都被 $A$ 映射到 $Y$ 空间中的零矢量, 即 $A(X_0) = \qty{0}$.

\begin{figure}[ht]
\centering
\includegraphics[width=9cm]{./figures/MatLS2_1.pdf}
\caption{两个 $X$ 子空间的线性映射: $X_0$ 是零空间, $X_1$ 是 $X_0$ 在 $X$ 中的补空间} \label{MatLS2_fig1}
\end{figure}

\begin{theorem}{}\label{MatLS2_the1}
令 $A:X \to Y$ 的零空间为 $X_0$, $X_1$ 为 $X_0$ 在 $X$ 中任意一个补空间(即 $X_0\oplus X_1 = X$, \autoref{DirSum_def1}~\upref{DirSum}), 令 $Y_1 = A(X)$, 那么映射 $A:X_1\to Y_1$ 是一一对应的(即双射).
\end{theorem}
证明见下文. 我们可以形象地把\autoref{MatLS2_the1} 用\autoref{MatLS2_fig1} 表示, 图中每个三角形代表一个矢量空间, 由于 $X_0, X_1$ 是互补的, 它们只相交于零矢量. 注意对于给定的映射 $A$, $X_0$ 和 $Y_1$ 是确定的, 而 $X_0$ 的补空间 $X_1$ 可以随意选取.

特殊地, 若 $A$ 是一对一映射, 则定理中 $X_0 = \qty{0}$ 是零维空间, 补空间 $X_1 = X$. 此时可以把\autoref{MatLS2_fig1} 中表示 $X_0$ 的三角形去掉.

由这个定理可得多对一线性映射的一个重要性质
\begin{corollary}{}
令线性映射为 $A:X\to Y$ 对应的矩阵为 $\mat A$. $X$ 是 $N_X$ 维矢量空间, $A$ 的零空间为 $X_0$, 是 $X$ 的子空间. 令 $\mat A$ 的列秩($Y_1$ 的维数)为 $R$, 那么
\begin{equation}\label{MatLS2_eq1}
N_X = N_0 + R
\end{equation}
\end{corollary}
证明: 见\autoref{MatLS2_eq1}, 由于 $X_1$ 与 $Y_1$ 一一对应, $X_1$ 的维数也是 $R$, 又因为 $X_0\oplus X_1 = X$, $X$ 的维数等于 $X_0, X_1$ 维数之和(\autoref{DirSum_cor1}~\upref{DirSum}). 证毕.

\begin{corollary}{}
线性映射 $A:X\to Y$ 是双射当且仅当 $A(X) = Y$ 且 $X,Y$ 维数相同.
\end{corollary}
证明: 前者证明后者显然. 后者证明前者: 令\autoref{MatLS2_the1} 中的 $A:X \to Y_1$ 满足 $N_X = R$, 根据\autoref{MatLS2_eq1}, $N_0 = 0$, 即零空间只有零矢量一个元素, 所以 $X = X_1$, 与 $Y_1$ 一一对应(双射). 证毕.

\subsection{证明}
要证明\autoref{MatLS2_the1}, 首先证明 $A(X_1) = A(X)$. 任意 $x\in X$ 可以表示为 $x = x_0 + x_1$, 其中 $x_0\in X_0$, $x_1\in X_1$. $Ax = A x_0 + A x_1 = A x_1$. 这说明 $Y_1 = A(X)$ 中任意元素都能找到 $A(x_1) \in Y_1$, 所以 $A(X_1) = Y_1$.

现在证明 $X_1$ 和 $Y_1$ 一一对应: 令 $u, v \in X_1$, 我们要证明如果 $Au = Av$ 那么 $u = v$. 算符 $A$ 是线性的, 所以 $A(u-v) = 0$, 所以 $u - v \in X_0$. 由封闭性, $u - v \in X_1$. 由于补空间满足 $X_0 \cap X_1 = \qty{0}$, 所以 $u - v = 0$, 即 $u = v$. 证毕.

\addTODO{线性方程组词条\upref{LinEq}中说明: 线性方程组的解空间就是 $X_1$ 中的特解加上齐次解 $X_0$.}
