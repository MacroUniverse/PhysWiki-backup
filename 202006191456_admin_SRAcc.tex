% 相对论加速度变换

\pentry{速度变换\upref{RelVel}}

\subsection{一般情况下的加速度变换}
\subsubsection{问题的限制条件}

为了简化讨论,不失一般性,我们可以把情景按以下方式设定:

假设$K_2$相对$K_1$的运动速度是$\bvec{v}=\left(\begin{matrix}v\\0\\0\end{matrix} \right) $.设在$K_1$中,有一质点以速度$\bvec{u}=\left(\begin{matrix}u_x\\u_y\\u_z\end{matrix} \right) \tag{2}$运动,其在$K_2$中的速度是$\bvec{u'}=\left(\begin{matrix}u_x'\\u_y'\\u_z'\end{matrix} \right) $.同时,质点在$K_1$中有加速度$\bvec{a}=\frac{\dd}{\dd{t}}\bvec{u}=\left(\begin{matrix}a_x\\a_y\\a_z\end{matrix} \right)$,在$K_2$中加速度则为$\bvec{a'}=\frac{\dd}{\dd{t'}}\bvec{u'}=\left(\begin{matrix}a_x'\\a_y'\\a_z'\end{matrix} \right)$.选取两坐标系的原点位置使它们重合、在质点轨迹上,并且设质点一直以匀速运动,直到在$K_1$中测量的时间$t$时才有了非零加速度.

计算的整体思路是,将$\frac{\dd}{\dd{t'}}\bvec{u'}$拆成$\frac{\dd}{\dd{t}}\bvec{u'}\cdot\frac{\dd{t}}{\dd{t'}}$,分别计算两个微商,然后再乘起来.一阶微分的形式不变性保证了这个思路的合法性.

注意,这里并不能简单地令$t'=\sqrt{1-v^2}t$,因为质点的位置并不总是在$K_2$的原点处,我们需要选所讨论位置的$t'$.

于是有
\begin{equation}\label{SRAcc_eq2}
\bvec{u'}=\frac{\sqrt{1-v^2}-\frac{1-\sqrt{1-v^2}}{v^2}(\bvec{u}\cdot\bvec{v})\bvec{v}-\bvec{v}}{1-\bvec{u}\cdot\bvec{v}}
\end{equation}

\begin{equation}\label{SRAcc_eq1}
t'=\frac{t-vu_xt}{\sqrt{1-v^2}}
\end{equation}

从\autoref{SRAcc_eq1}可得,
\begin{equation}\label{SRAcc_eq3}
\frac{\dd t}{\dd t'}=\frac{1}{\dd t'/\dd t}=\frac{\sqrt{1-v^2}}{1-u_xv}
\end{equation}

\begin{exercise}{}

已知对于向量$\bvec{A}$,$\bvec{B}$和$\bvec{C}$有恒等式:$\bvec{C}\times(\bvec{A}\bvec{B})=(\bvec{C}\cdot\bvec{B})\bvec{A}-(\bvec{C}\cdot\bvec{A})\bvec{B}$,结合\autoref{SRAcc_eq2},请证明:

\begin{equation}\label{SRAcc_eq4}

\frac{\dd}{\dd t}\bvec{u'}=\frac{1-v^2}{(1-\bvec{u}\cdot\bvec{v})^2}[\bvec{a}-\frac{1}{\sqrt{1-v^2}+1-v^2}(\bvec{a}\times\bvec{v})\times\bvec{v}+\sqrt{1-v^2}(\bvec{a}\times\bvec{u})\times\bvec{v}]

\end{equation}

\end{exercise}

将\autoref{SRAcc_eq3}和\autoref{SRAcc_eq4}相乘,即得到

\begin{equation}

\bvec{a'}=\frac{\dd}{\dd t}\bvec{u'}\cdot\frac{\dd t}{\dd t'}=\frac{(1-v^2)^{3/2}}{(1-\bvec{u}\bvec{v})^3}[\bvec{a}-\frac{1}{\sqrt{1-v^2}+1-v^2}(\bvec{a}\times\bvec{v})\times\bvec{v}+\sqrt{1-v^2}(\bvec{a}\times\bvec{u})\times\bvec{v}]

\end{equation}

同样地,由于在$K_2$眼中,质点以$\bvec{u'}$运动且$K_1$以$-\bvec{v}$运动,因此可以得到逆变换:

\begin{equation}\label{SRAcc_eq5}

\bvec{a}=\frac{(1-v^2)^{3/2}}{(1+\bvec{u'}\bvec{v})^3}[\bvec{a'}-\frac{1}{\sqrt{1-v^2}+1-v^2}(\bvec{a'}\times\bvec{v})\times\bvec{v}-\sqrt{1-v^2}(\bvec{a'}\times\bvec{u'})\times\bvec{v}]

\end{equation}


\subsection{瞬时自身系中的变换}

对于运动的质点,和质点一直相对静止的参考系称为质点的\textbf{自身系}.如果质点在惯性系中有非零加速度,那么自身系就不是惯性系,不在狭义相对论的讨论范围内.但是,对于质点轨迹上的给定点,我们可以取质点在这里的速度,然后选择一个在这一瞬间和质点相对静止的惯性系.这样在某一点处和质点相对静止的参考系,称为质点的\textbf{瞬时自身系}.

在本节讨论中,如果设$\bvec{u}=\bvec{v}$,即$\bvec{u'}=\bvec{0}$,那么在加速度出现时,$K_2$就是质点在这一事件位置的瞬时自身系.质点在瞬时自身系中的加速度是$\bvec{a'}$,那么由\autoref{SRAcc_eq5},其在$K_1$中的加速度是

\begin{equation}

\bvec{a}=(1-v^2)^{3/2}[\bvec{a'}-\frac{1}{\sqrt{1-v^2}+1-v^2}(\bvec{a'}\times\bvec{v})\times\bvec{v}]

\end{equation}

如果$\bvec{a'}\parallel \bvec{v}$,则有$$\bvec{a}=(1-v^2)^{3/2}\bvec{a'}$$因此$\bvec{a}\parallel\bvec{v}$.

如果$\bvec{a'}\perp \bvec{v}$,则有$$\bvec{a}=(1-v^2)\bvec{a'}$$因此$\bvec{a}\perp\bvec{v}$.


