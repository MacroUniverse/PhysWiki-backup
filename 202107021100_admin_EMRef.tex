% 电磁场的参考系变换
% 参考系变换|洛伦兹变换|电磁场

\begin{issues}
\issueDraft
\end{issues}

\pentry{洛伦兹变换\upref{SRLrtz}, 麦克斯韦方程组\upref{MWEq}, 洛伦兹力\upref{Lorenz}}

若电磁场 $\bvec E, \bvec B$ 都垂直于导线,
\begin{equation}
\bvec E' = \frac{\bvec E - \bvec u \cross \bvec B}{\sqrt{1 - u^2/c^2}}
\end{equation}
\begin{equation}
\bvec B' = \frac{\bvec B - \bvec u \cross \bvec E/c^2}{\sqrt{1 - u^2/c^2}}
\end{equation}



% 让我比较自豪的是,我高中的时候曾经独立推导出了洛伦兹变换,再通过这里给出的例子独立推导出了电磁场的参考系变换. 上大学时我才在新概念电磁学上看到一模一样的变换公式.

到目前位置我们只在同一个参考系中分析电磁学问题, 我们下面以一个例题来分析在不同参考系之间电磁场该如何变换. 我们将会发现, 在讨论电磁场的参考系变换时必须考虑狭义相对论效应才能不发生矛盾, 即麦克斯韦方程组\upref{MWEq}天然与洛伦兹变换而不是伽利略变换兼容. 爱因斯坦创造狭义相对论时所发的论文 《论动体的电动力学》 讨论的就是这类问题.

在 $S$ 参考系中有一个电流为 $I$ 的无限长直导线, 令电流正方向为 $\uvec x$. 作为一个简单对称的导线模型, 我们假设导线中的所有正电荷的线密度为 $\lambda$, 以速度 $v_0\uvec x$ 运动, 导线中的所有负电荷的线密度为 $-\lambda$,  以速度 $-v_0\uvec x$ 运动. 距离导线 $r_0$ 处有一个电荷为 $q$ 的粒子沿 $\uvec x$ 方向以速度 $v$ 运动. 另一个参考系 $S'$ 相对 $S$ 沿 $\uvec x$ 方向运动, 速度为 $u$. 在这两个参考系中, 粒子所受的电磁力是否相同?
\addTODO{图}

\subsubsection{错误的分析}
在 $S$ 参考系中, 电流在导线周围产生的磁场(见\autoref{AmpLaw_ex1}~\upref{AmpLaw})使粒子受到垂直于导线洛伦兹力\upref{Lorenz} $F = qvB$. 而在 $S'$ 系中, 导线同样产生相同的磁场,而粒子的速度却是 $v - u$, 所以洛伦兹力为 $F = q(v-u)B$. 这说明在不同参考系中粒子受的电磁力是不同的.

这显然是不可能的, 例如当 $u = v$ 时, $S$ 参考系中粒子受力而 $S'$ 中粒子不受力, 产生矛盾. 该分析的问题在于使用了牛顿的时空观, 没有考虑相对论效应.

\subsubsection{正确的分析}
上面 $S$ 系的分析是正确的,空间中只有磁场没有电场. 然而在 $S'$ 系中, 由于导线中的异号电荷运动快慢不同,相对论尺缩效应使两种电荷的线密度产生区别, 所以产生的磁场和 $S$ 系中不同, 甚至由于正负电荷不再抵消还会产生垂直导线的电场. 在新的电场与磁场的共同作用下, 使粒子受力与 $S$ 系中相同.
\addTODO{力在不同参考系中是不变的吗?需要引用讨论.}

\subsection{具体计算}
在 $S$ 系中, 导线的电流为
\begin{equation}
I = 2\lambda v_0
\end{equation}
点电荷处磁场大小为
\begin{equation}
B = \frac{\mu_0}{2\pi} \frac{I}{r_0}
\end{equation}
方向右右手螺旋定则决定. 那么它的受力大小为 $F = q v B$, 正方向指向导线.

在 $S'$ 系中, 



(未完成)

\addTODO{如何系统地证明麦克斯韦方程组和洛伦兹变换兼容?}
