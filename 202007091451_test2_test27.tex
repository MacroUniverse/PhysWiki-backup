% 信息熵公式推导
让$H(\frac{1}{n},\frac{1}{n},...,\frac{1}{n})=A(n)$,从上文关于$H$的第三个性质条件,
我们可以将$S^m$中进行一次等概率选择,分解为在$S$中进行$m$次等概率选择,即
\begin{equation}
A(S^m)=mA(s)
\end{equation}
类似的有,
\begin{equation}
A(t^n)=nA(t)
\end{equation}
我们令n为任意大小,并找到一个m满足:
\begin{equation}
S^m\leq t^n\leq s^{m+1}
\end{equation}

