% 电磁场的能量守恒 坡印廷矢量
% 能量守恒|坡印廷矢量|能流密度

%未完成
%(这个概念太神奇了, 要好好介绍其实电荷的势能不是由电荷本身携带, 而是通过电磁场传播的). 举例: 同轴电缆. 不要说介质, 太复杂了, 或者顺带提一下就行)
%“转换速率+流出速率+增加速率=0”

\pentry{麦克斯韦方程组\upref{MWEq}, 电场的能量\upref{EEng}, 磁场的能量\upref{BEng}}

\subsection{结论}

\subsubsection{坡印廷矢量}
真空中电磁场的能流密度为 % 定义链接
\begin{equation}
\bvec s = \frac{1}{\mu_0} \bvec E \cross \bvec B
\end{equation} 
$\bvec s$ 就是坡印廷矢量.

\begin{example}{平面电磁波的能流密度}
我们知道真空中的平面电磁波\upref{VcPlWv}的电场和磁场方向垂直, 所以二者叉乘的模长等于他们各自的模长相乘. 叉乘的方向就是电磁波传播的方向. 考虑到每个位置的磁场 $\abs{B(\bvec r)} = \abs{E(\bvec r)}/c$, 再利用\autoref{VcPlWv_eq1}~\upref{VcPlWv}得
\begin{equation}
s = \frac{1}{\mu_0 c} \abs{\bvec E}^2 = \epsilon_0 c \abs{\bvec E}^2
\end{equation}
但注意在任意一点处, 这个值是随时间以 $\sin^2$ 波动的, 在波节处, 能流密度为零, 而在波峰处为最大, 平均值等于 $1/2$. % 链接未完成
所以任意一点的平均能流密度为
\begin{equation}
\ev{s} = \frac{1}{2} \epsilon_0 c \abs{\bvec E_{max}}^2
\end{equation}
\end{example}

\subsubsection{电磁场能量守恒积分形式}
\begin{equation}
\int_V \dv{w}{t} \dd{V}  + \pdv{t} \int_V \rho_E \dd{V}  + \oint_\Omega  \bvec s \vdot \dd{\bvec a}  = 0
\end{equation} 
选取任意的一个闭合曲面 $\Omega $, 内部空间记为 $V$, 以下三者之和为零.
\begin{enumerate}
\item 电磁场对 $V$ 中所有电荷做功的功率
\item $V$ 中电磁场能量增加的速率
\item 以及通过曲面 $\Omega $ 流出的能量的速率
\end{enumerate}

\subsubsection{电磁场能量守恒微分形式}
\begin{equation}
\dv{w}{t} + \pdv{\rho}{t} + \div \bvec s = 0
\end{equation} 
空间中选取任意一点, 以下三者之和为零.
\begin{enumerate}
\item 电磁场对电荷的功率密度
\item 电磁场能量密度增量
\item 能流密度散度
\end{enumerate}

\subsection{推导}
类比电流的连续性方程\autoref{ChgCsv_eq4}~\upref{ChgCsv}(即电荷守恒),若电磁场不对电荷做功,能量守恒可以写成
\begin{equation}
\pdv{\rho_E}{t} + \div \bvec s = 0
\end{equation} 
的形式.其中 $\bvec s$ 是电磁场的能流密度(也叫\textbf{坡印廷矢量})(参考流密度%未完成:引用
).但若再考虑上电磁场对电荷做功, 则还需要加上一项做功做功功率密度 $\pdv*{w}{t}$, 即单位时间单位体积电磁场对电荷做的功).
\begin{equation}\label{EBS_eq1}
\pdv{w}{t} + \pdv{\rho_E}{t} + \div \bvec s = 0
\end{equation} 

第一项中电磁场对电荷做功即广义洛伦兹力%未完成:引用
做功(功率密度)
\begin{equation}
\pdv{w}{t} = \bvec f \vdot \bvec v = \rho (\bvec E + \bvec v \cross \bvec B) \vdot \bvec v = \bvec j \vdot \bvec E
\end{equation} 
假设电磁场的能量守恒\autoref{EBS_eq1} 成立, 那么 $\bvec j \vdot \bvec E =  - \pdv*{\rho_E}{t} - \div \bvec s$. 等式右边只与场有关, 所以应该把电流密度 $\bvec j$ 用麦克斯韦方程组%未完成:引用
替换成场的表达式, 即
\begin{equation}
\bvec j = \frac{1}{\mu_0} \curl \bvec B - \epsilon_0 \pdv{\bvec E}{t}
\end{equation} 
代入得
\begin{equation}\label{EBS_eq2}
\begin{aligned}
\pdv{w}{t} &= \qty(\frac{1}{\mu_0} \curl \bvec B - \epsilon_0 \pdv{\bvec E}{t}) \vdot \bvec E\\
&= \frac{1}{\mu_0} (\curl \bvec B)\bvec E - \epsilon_0 \pdv{\bvec E}{t} \bvec E
\end{aligned}
\end{equation} 
\autoref{EBS_eq1} 第二项中, $\rho_E$ 是电场能量密度和磁场能量密度之和, 即
\begin{equation}\label{EBS_eq3}
\begin{aligned}
\curl (\bvec B \cross \uvec x) &= \bvec B(\div \uvec x) + (x\bvec\nabla) \vdot\bvec B - \uvec x( \div \bvec B) - (\bvec B\vdot\bvec\nabla)\uvec x \\
&= (x\bvec\nabla)\vdot\bvec B = \pdv{\bvec B}{x}
\end{aligned}\end{equation} 
现在我们可以把\autoref{EBS_eq2},\autoref{EBS_eq3} 代入\autoref{EBS_eq1} 中, 求出 $\div \bvec s$. 
\begin{equation}
\begin{aligned}
\div \bvec s &=  - \pdv{w}{t} - \pdv{\rho_E}{t}\\
&= -\frac{1}{\mu_0} (\curl \bvec B)\bvec E + \epsilon_0 \pdv{\bvec E}{t} \bvec E - \pdv{t} \qty( \frac12 \epsilon_0 \bvec E^2 + \frac12 \frac{\bvec B^2}{\mu_0} )\\
&=  - \frac{1}{\mu_0} (\curl \bvec B)\bvec E - \frac{1}{\mu_0} \pdv{\bvec B}{t}\bvec B
\end{aligned}
\end{equation} 
其中 $(\curl \bvec B) \vdot \bvec E = (\curl \bvec E) \vdot \bvec B - \div (\bvec E \cross \bvec B)$, 因为 $\div (\bvec E \cross \bvec B) = (\curl \bvec E) \vdot \bvec B - (\curl \bvec B) \vdot \bvec E$(Gibbs 算子相关公式%未完成:引用
).代入得
\begin{equation}
\begin{aligned}
\div \bvec s &=  - \pdv{w}{t} - \pdv{\rho_E}{t}\\
&= -\frac{1}{\mu_0} (\curl \bvec B)\bvec E + \epsilon_0 \pdv{\bvec E}{t} \bvec E - \pdv{t} \qty(\frac12 \epsilon_0 \bvec E^2 + \frac12 \frac{\bvec B^2}{\mu_0})\\
&=  - \frac{1}{\mu_0} (\curl \bvec E) \vdot \bvec B - \frac{1}{\mu_0}\pdv{\bvec B}{t} \bvec B + \frac{1}{\mu_0}\div (\bvec E \cross \bvec B)
\end{aligned}
\end{equation} 
其中 $\curl \bvec E =  -\pdv*{\bvec B}{t}$, 代入得
\begin{equation}\label{EBS_eq4}
\div \bvec s = \div \qty(\frac{1}{\mu_0}\bvec E \cross \bvec B)
\end{equation} 
即
\begin{equation}\label{EBS_eq5}
\bvec s = \frac{1}{\mu_0}\bvec E \cross \bvec B
\end{equation} 
这就是电磁场的能流密度.

事实上, 给 $\bvec s$ 再加上任意一个散度为零的场,\autoref{EBS_eq4} 都能满足, 但为了简洁起见, 一般写成\autoref{EBS_eq5}. 


% \subsection{电荷的势能传递}

%未完成
% 其实所谓电荷的势能根本就不是由电荷携带并传递, 而是通过电磁场来传递的呀!


