% 抛物线坐标系中的类氢原子定态波函数

\begin{issues}
\issueDraft
\end{issues}

\pentry{定态薛定谔方程\upref{SchEq}, 抛物线坐标系\upref{ParaCr}}

\footnote{参考 \cite{Brandsen} Chap 3.5.}本文使用原子单位制\upref{AU}. 令原子核不动.

\begin{equation}
-\frac{1}{2m} \laplacian \psi - \frac{Z}{r} \psi = E\psi
\end{equation}
在抛物线坐标系中变为
\begin{equation}
-\frac{1}{2m} \qty{\frac{4}{\xi + \eta} \qty[\pdv{u}{\xi}\qty(\xi\pdv{\xi}) + \pdv{u}{\eta}\qty(\eta\pdv{\eta})] + \frac{1}{\xi\eta}\pdv[2]{u}{\phi}}\psi - \frac{2Z}{\xi + \eta}\psi = E\psi
\end{equation}
分离变量, 令
\begin{equation}
\psi(\xi, \eta, \phi) = f(\xi) g(\eta) \Phi(\phi)
\end{equation}
和球坐标同理, $\Phi(\phi) = \exp(\I m \phi)$. 令另外两个分离变量常数满足
\begin{equation}
\nu_1 + \nu_2 = Z
\end{equation}
有
\begin{equation}
\dv{\xi} \qty(\xi \dv{f}{\xi}) + \qty(\frac{\mu E \xi}{2} - \frac{m^2}{4\xi} + \nu_1) f = 0
\end{equation}
\begin{equation}
\dv{\eta}\qty(\eta \dv{g}{\eta}) + \qty(\frac{mE\eta}{2} - \frac{m^2}{4\eta} + \nu_2)g = 0
\end{equation}
可以化简为 Kummer-Laplace 微分方程.
