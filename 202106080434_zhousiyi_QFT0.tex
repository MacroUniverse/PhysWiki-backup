% 引言
% 量子场论|引言

\pentry{量子力学}

QED可能是我们有的最好的基本理论了.这个理论由一系列简单的方程(麦克斯韦方程和狄拉克方程)组成.方程的形式可以由相对论不变性定出来.这些方程的解决定着宏观和微观的物理.

费曼图给了这个理论同样优雅的计算步骤.假如我们想计算某个过程的发生概率,我们可以遵照如下步骤:

\begin{enumerate}
\item 画出这个过程对应的费曼图
\item 根据图写出散射振幅的表达式
\item 化简表达式
\end{enumerate}

\subsection{最简单的情形}
大多数粒子物理实验跟散射有关.所以,量子场论中最常见的就是散射振幅的计算了.那么我们现在就来做一个QED里面最简单的计算,也就是正负电子湮灭产生一对新的正负轻子(比如说$\mu$子)的过程.相关过程的示意图如下:

\begin{figure}[ht]
\centering
\includegraphics[width=10cm]{./figures/QFT0_1.png}
\caption{$e^+e^-\rightarrow\mu^+\mu^-$过程的费曼图} \label{QFT0_fig1}
\end{figure}

实验上,如果我们把一束电子束射到一束正电子束上,就能够实现这个过程.可观测量是这个过程的\textbf{散射截面}.散射截面是这个过程的值信息能力以及入射电子与出射$\mu$子之间的夹角.

为了简单起见,我们使用质心系.则这四个动量之间有如下关系:
\begin{equation}
\mathbf p' = - \mathbf p~, \quad \mathbf k'=-\mathbf k
\end{equation}
另外,我们需要假设质心系能量远大于电子以及$\mu$子的质量.于是我们有
\begin{equation}
|\mathbf p| = |\mathbf p'| = |\mathbf k| = |\mathbf k'| = E = E_{\rm cm}/2
\end{equation}
我们约定粗体代表三动量,斜体代表四动量.

因为电子和$\mu$子的自旋均为$1/2$,我们必须指定自旋的方向.那么比较方便的做法是取这个例子运动的方向作为坐标轴.那么每个粒子就会有平行于或者反平行于这个坐标轴的自旋.

但是实际上来说,电子和正电子束一般都是没有极化的.$\mu$子探测器也通常探测不到$\mu$子的自旋.所以一般的做法是对电子和正电子的自旋求平均,而对$\mu$子的自旋求和.

微分散射截面是一个很重要的物理量.它的定义如下
\begin{definition}{微分散射截面}
\begin{equation}
\frac{d\sigma}{d\Omega} = \frac{1}{64\pi^2E_{\rm cm}^2}\cdot |\mathcal M|
\end{equation}
\end{definition}
$E$














%\subsection{}

