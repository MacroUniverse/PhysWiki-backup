% 熵
% 热力学|玻尔兹曼|熵|统计力学

\pentry{理想气体状态方程\upref{PVnRT}}
在热力学和统计力学中, \textbf{熵(entropy)}用于描述系统的无序程度, 是一个状态量, 通常记为 $S$. 例如若已知理想气体的 $P, V, n, T$ 等状态量, 就可以确定它的熵. %链接未完成

\subsection{宏观定义}

熵是一个系统的状态参量,它的增量为
\begin{equation}
\mathrm{d} S = \left . \frac{\Delta Q}{T}\right |_{\text{可逆}}
\end{equation}
对于一般的准静态过程, 有
\begin{equation}\label{Entrop_eq1}
\Delta S = \int \frac{\dd{Q}}{T}
\end{equation}

\subsection{微观定义}
\begin{equation}
S = k_B \ln \Omega
\end{equation}

% 举例: 高温物体热传导给低温物体, 损失多少?
 
% 未完成: 热机中熵如何变化? 等压等温绝热过程中熵如何变化?
