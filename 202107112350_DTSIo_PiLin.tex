% 皮卡-林德勒夫定理

皮卡-林德勒夫定理 (Picard-Lindelöf theorem) 是分析数学中的一个基本定理, 又称为柯西-李普希茨定理 (Cauchy-Lipschitz theorem). 它断言: 常微分方程 (组) 的初值问题只需要满足一些非常宽泛的条件, 就是唯一可解的. 

由于许多经典物理问题都可以化归为常微分方程组, 所以皮卡-林德勒夫定理可以用来说明这些物理问题的决定论 (deterministic) 特性: \textbf{给定了系统的初始状态之后, 系统的演化就唯一确定了.}

对于不满足皮卡-林德勒夫定理条件的常微分方程组, 尚有皮亚诺存在定理. 后者无法保证解的唯一性.

\subsection{定理的表述与证明}
\begin{theorem}{皮卡-林德勒夫定理}
设$X$是巴拿赫空间\upref{banach}. 设有实数$t_0\in\mathbb{R}$, 正数$r,R>0$, 以及$X$中的元素$x_0$. 设映射$f:\bar B_X(x_0,R)\times[t_0-r,t_0+r]\to X$, 满足如下条件:

(1) $f$是连续的, 且存在一正数$M>0$, 使得对于$|x-x_0|_X\leq R$, $|t-t_0|\leq r$, 总有$|f(x,t)|_X\leq M$.

(2) $f(x,t)$对$x$满足李普希茨条件, 即存在一正数$L>0$, 使得对于$x_1,x_2\in \bar B_X(x_0,R)$和$t\in[t_0-r,t_0+r]$, 总有$$
|f(x_1,t)-f(x_2,t)|_X\leq L|x_1-x_2|_X.
$$

则对于任何$u_0\in \bar B_X(x_0,R)$, 都存在一个只依赖于$u_0,R,r,M,L$的$T$, 使得常微分方程的初值问题
$$
\frac{d}{dt}u(t)=f(u(t),t),\quad u(t_0)=u_0
$$
在区间$[t_0-T,t_0+T]$上有唯一解.
\end{theorem}