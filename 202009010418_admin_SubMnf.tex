% 子流形

\pentry{流形\upref{Manif}}

\subsection{子流形}

\begin{definition}{子流形}\label{SubMnf_def1}
设$N$是一个$n$维流形,$K$是它的一个子集.如果在$K$上任意点$x_0\in K\subseteq N$,都存在一个$N$的图$(U, \varphi)$,使得$\varphi (U\cap K)$是$\mathbb{R}^n$中“令后$n-k$个坐标为$0$”所得的平面,其中为方便,记$\varphi|_K:U\cap K\rightarrow\mathbb{R}^k$,且$\forall x\in U\cap K, \varphi(x)=\varphi|_K(x)$,那么用各$(U, \varphi|_K)$的图册可以构成集合$K$上的拓扑流形,取该图册的极大图册来赋予$K$所得的光滑流形,即为$N$的子流形.
\end{definition}

简单来说,定义子流形的需要两步.第一步是利用原流形的$N$图册来构建一个子流形$K$的图册,具体方式见\autoref{SubMnf_def1},其中使用的定义方式是为了方便讨论,并非唯一的方式 ;第二步是把这个图册扩充为极大图册.这样,配备了此极大图册的$K$就是我们需要的子流形.


\subsection{等值面/水平集}%level sets




