% 泛函分析笔记3
% 泛函分析|数学分析|空间|Banach 空间|希尔伯特空间

\subsection{3.1 Orthonormal Series}
\begin{itemize}
\item \textbf{正交归一系(orthonormal system)}: $(u_k|u_m) = \delta_{k,m}$ 对所有 $k, m$ 成立

\item 令 $X$ 为实 Hilbert 空间, $\{u_0, u_1,\dots\}$ 为 $X$ 中至多可数的正交归一系. 对有限正交归一系, 如果 $u = \sum_{n=0}^N(u_n|u)u_n$ 对所有 $u\in X$ 成立, 它就叫做\textbf{完备的(complete)}. 对可数的正交归一系, 令 $s_m := \sum_{n=0}^m(u_n|u)u_n$, 如果 $u = \lim_{m\to\infty} s_m$ 对所有 $u\in X$ 成立, 它就叫做\textbf{完备的(complete)}

\item 有限的正交归一系在 Hilbert 空间 $X$ 中是完备的当且仅当它是 $X$ 的一组基底

\item 令 $\{u_n\}$ 为 Hilbert 空间中可数的正交归一系. 如果无穷级数 $u = \sum_{n=0}^\infty c_n u_n$ 对某个 $u\in X$ 收敛, 那么 $c_n = (u_n|u)$ 对所有 $n$ 成立

\item \textbf{Bessel inequality}: $\sum_{n=0}^m \abs{(u_n|u)}^2 \le \abs{u}^2$ 对所有 $u \in X$ 和 $m$ 成立

\item \textbf{Convergence criterion}: 令 $\{u_n\}$ 为 Hilbert 空间 $X$ 中的可数正交归一系. 那么 $\sum_{n=0}^\infty c_nu_n$ 收敛当且仅当 $\sum_{n=0}^\infty \abs{c_n}^2$ 收敛

\item 令 $\qty{u_n}$ 为可数正交归一系, 那么以下两个条件等效: (1) 它在 $X$ 中是完备的. (2) 它的 linear hull (span) 在 $X$ 中是稠密的

\item 对 Hilbert 空间 $X$ 中可数完备的正交归一系 $\qty{u_n}$ : (1) \textbf{Parseval equation}: $(u|v) = \sum_{n=0}^\infty \bar c_n(u) c_n(v)$, (2) $\norm{u}^2 = \sum_{n=0}^\infty \abs{(u_n|u)}^2$, (3) 如果 $(u_n|u) = 0$ 对所有 $n$ 和固定的 $u\in X$ 成立, 那么 $u=0$

\item 对每个 $u\in L_2(-\pi,\pi)$, 傅里叶级数收敛. 即 $\lim_{m\to\infty}\int_{-\pi}^\pi [u(x)-a_0/2-\sum_k a_k\cos kx + b_k \sin kx]^2 \dd{x} = 0$
\end{itemize}

\subsection{3.5 Unitary Operators}
\begin{itemize}
\item 令 $X$ 和 $Y$ 为 $\mathbb K$ 上的希尔伯特空间. 算符 $U: X\to Y$ 叫做 \textbf{unitary} 当且仅当 $U$ 是线性的, 满射的, 且 $(Uv|Uw) = (v|w)$ 对所有 $v, w \in X$ 成立

\item 如果算符 $U$ 是 unitary 的, 那么它就是双射的, 连续的, 且 $\norm{Uv} = \norm{v}$ 对所有 $v\in X$ 成立. 而且, 存在逆算符 $U^{-1}: Y\to X$, 同样是 unitary 的
\end{itemize}

\subsection{3.6 The Extension Principle}
\begin{itemize}
\item 对 Banach 空间 $X$ 和 $Y$, 令线性算符 $A: D\subseteq X\to Y$ 的范数为有限值, $D \subseteq X$ 是稠密线性的. 那么 (1) $A$ 可以唯一地拓展(extended)到 $A:X\to Y$ 上, 范数仍然为有限. (2)如果 $A$ 在 $D$ 上的紧算符, 那么拓展后也是紧算符
\end{itemize}

\subsection{3.7 Applications to the Fourier Transformation}
\begin{itemize}
\item $\mathcal S$ 空间: 包含所有 $C^\infty$ 函数 $u: \mathbb R \to \mathbb C$, 满足 $\norm{u}_{p,q} < \infty$ 对所有 $p, q=0,1,\dots$ 成立. 其中 $\norm{u}_{p,q} := \sup_{x\in\mathbb R} (1+\abs{x}^p) \sum_{n=0}^q\abs{u^{(n)}(x)}$. 极限 $u_n \to u$ 以 $\norm{u}_{p,q}$ 为准

\item $\mathcal S$ 空间中的函数叫做\textbf{在无穷远处 rapidly decreasing}. 且满足 $\abs{\int_{-\infty}^\infty u^{(n)}(x) \dd{x}} < \infty$ 对所有 $n = 0, 1,\dots$ 成立. 且分部积分的边界项为零

\item 傅里叶变换 $F:\mathcal S\to\mathcal S$ 是线性的连续双射的. 反傅里叶变换同样是连续的

\item 傅里叶变换 $F:\mathcal S\to\mathcal S$ 可以唯一地拓展到酉算符 $F: L_2^{\mathbb C}(\mathbb R) \to L_2^{\mathbb C}(\mathbb R)$. 注意 $C_0^\infty(\mathbb R)_{\mathbb C} \subseteq \mathcal S \subseteq L_2^{\mathbb C}(\mathbb R)$
\end{itemize}

\subsection{3.8 The Fourier Transform of Tempered General Functions}
\begin{itemize}
\item $\mathcal S'$ 定义为所有线性连续映射 $T: \mathcal S\to \mathbb C$. $T$ 叫做 \textbf{tempered generalized functions} 或者 \textbf{tempered distributions}

\item 定义 $(FT)(u) := T(Fu)$ ($u \in \mathcal S$)

\item 算符 $F : \mathcal S' \to \mathcal S'$ 是线性且双射的

\item 令 $v : \mathbb R\to\mathbb C$ 为可测的有界函数. 算符 $T(u) = \int_{-\infty}^\infty v(x)u(x) \dd{x}$ ($\forall u \in \mathcal S$). 那么 $T \in \mathcal S'$

\item 令 $y \in \mathbb R$, 定义 \textbf{tempered delta distribution} $\delta_y$: $\delta_y(u) := u(y)$ ($\forall u \in \mathcal S$). 那么 (1) $\delta_y \in \mathcal S'$, (2) $F\delta_y = (2\pi)^{-1/2} “e^{-iky}”$, (3) $(2\pi)^{-1/2} F^{-1} (“e^{-iky}”) = \delta_y$. 其中定义 $“e^{-iky}”(u) := \int_{-\infty}^\infty e^{-iky} u(k) \dd{k}$ ($\forall u\in\mathcal S$)
\end{itemize}

\subsection{Problems}
\begin{itemize}
\item \textbf{Tensor Product} $X\otimes Y$
\end{itemize}