% 朴素集合论
\subsection{公理,定义和定理}
我们是不可能证明所有的命题的,所以任何理论必须有一个出发点,也就是一些基础命题.这些基础命题本身不可证明,它们决定了理论的样貌,理论中一切其它命题都是由这些命题根据逻辑推演得到的.这样的基础命题被称为一个理论的\textbf{公理}(axiom).

有了公理系统以后,我们还需要明确所讨论的对象是什么.比如我用了皮亚诺公理来定义小学四则运算,那么为了讨论“1+1等于几”这样的问题,我首先需要明确“1”和“+”具体指什么.用来明确概念的陈述句,被称为\textbf{定义}(definition).如果说公理系统是创建了一个宇宙的基础参数的话,那么定义就是在给这个宇宙里已经自然存在的事物进行命名,这样才能讨论这些事物.当我们说“把xxx称为”时,也是一种下定义的方式.

最后,任何一个理论的绝大部分内容都是在使用基础命题来进行推演,看哪些命题能成立.这些成立的命题,就叫做\textbf{定理}(theorem). 有时候,根据定理作用的不同,我们也可能称其中一些为\textbf{引理}(lemma)、\textbf{推论}(corollary)等.所有定理加在一起就构成了整个理论.

不同的公理系统可能推演出相同的命题,也可能推演出彼此矛盾的命题,更可能存在一些无法判断是否成立的命题.一个公理系统中所无法判断是否成立的命题,就叫做独立于这个公理系统的命题.如果两个公理系统能够推演出完全一样的命题(定理),那么这两个公理系统就是等价的.如果公理系统A能够推演出公理系统B的一切定理,但是B不能推出A中的一切定理,即A能推演出的一些定理实际上是独立于B的命题,那么可以认为是公理系统A包含了公理系统B.\textbf{在阅读本段话时,请注意命题和定理的区别:定理是在给定公理体系下能被推演出来的命题.}

以上表述是数学的表达方式.在物理学中,\textbf{公理}和\textbf{定理}可以分别被翻译成\textbf{定律}和\textbf{现象}. 

\subsection{集合}

对于物理学习而言,集合论没必要从公理角度来严格理解,所以在此给出的是朴素集合论的解释.

集合(set)是由元素(element)组成的.任何事物和概念都可以成为元素,任何不同的元素都可以放在一起,构成一个集合.可以说,如果我们划定一个讨论的范围,那么这个范围就是一个集合,范围涉及到的事物和概念就是这个集合当中的元素.

表达一个集合的方式有多种,最简单的方式是列出所有集合中的元素.在数学中规定的语法规范是用大括号“\{\}”来列举集合中的一切元素,以逗号“,”隔开彼此.比如,\{猪,牛,狗,羊,猫\}构成了一个具有五个元素的集合,$\{1,2,3,4,\cdots\}$则是全体正整数的集合.第二个例子并没有显然地列举出所有正整数,只是用省略号表达了这个意思;也就是说,表达一个集合的方式并没有死板的规定,只要能让读者理解就可以了.

另一种常见的表达集合的方式是确定一个规则,语法规范是“\{$x|x$需要满足的条件\}”.比如全体正整数的集合,也可以写为\{$x|x$是一个正整数\}.如果有多个条件,也可以列在一起,比如全体正整数的集合:\{$x|x$是一个正数,且$x$是一个整数\}.特别地,如果某条规则是“x属于某集合”,我们通常会将这个条件写到单竖线的前面,如全体正整数的集合:\{$x\in\mathbb{Z}|x$是一个正数}. 这里,$\in$是一个简写的符号,$A\in B$等于说“A是B的元素”.

如果集合A的元素都是集合B的元素,那么称A是B的\textbf{子集}(subset).一切集合都是自身的子集.如果A是B的子集但又和B不同,也就是说A没有包含B的所有元素,那么称A是B的真子集.

有一个特殊的集合,它不含有任何元素,被称为\textbf{空集}(empty set),记作$\phi$. $\phi $是一切集合的子集.

\subsection{属于和包含}
为了简化表达,数学家把集合论中常用的动词表示成简略的形式.

$A\in B$或者$B\ni A$等价B于“A是B的元素”,表达“属于”关系. 

$A\subset B$ 或者 $B\supset A$等价于“A是B的子集”,表达“包含”关系. 

注意区分这两个情况,前一个情况中A是B的元素,后一个情况中A是B的子集.另外,集合本身也可以是别的集合的元素,元素的概念没有限定,任何事物和概念都可以成为元素,包括集合.

其它形式的子集符号,如$\subseteq,\supseteq$等,在不同的文献中可能有不同的含义,所以一般没有特殊说明时不会使用.


\subsection{集合运算}

集合间可以互相操作,生成新的集合,这种操作被称为集合间的运算(operation).

$\cap$表示两个集合的交,意思是将两个集合中共有的元素提取出来,组成一个新的集合.比如说,$\mathbb{N^+}$表示全体自然数的集合,$\mathbb{R^+}$表示全体正实数的集合,$\mathbb{Z}$表示全体整数的集合,那么显然我们可以有$\mathbb{N^+}=\mathbb{R^+}\cap\mathbb{Z}$. 多个集合$A_i$的交集,可以写为$A_0\cap A_1\cap A_2\cap A_3\cdots$,也可以用一个大号的交集符号简记为$\bigcap A_i$,表示“所有形式为$A_i$的集合的交集”.

类似地,将两个集合中都有的元素提取出来,组成一个新的集合的操作,被称为集合的并,用符号$\cup$, 和$\bigcup$表示.注意,如果两个元素中有相同元素,那么在并集中这个元素只出现一次.这是因为我们关心的是每个元素是否出现在集合中,计算集合元素数量时也不会重复计算同一个元素.这是一个并集的例子:{猪,牛,狗,羊,猫\}$\cup\mathbb{N^+}$=$\{$猪,狗,猫,牛,羊, 1,2,3,4,$\cdots\}$. 注意,列举时元素的顺序也不影响集合的本质.

$\ $或者写为$-$是求集合间的差.对于集合A和B,$A-B$所包含的元素是A中全体元素中减去B中元素,如果B还含有A中所没有的元素,那么这部分元素可以忽略掉.例如,如果令$A=\{0,1,2,3\}, B=\{2,3,4\}$,那么$A-B=\{0,1\}$.

$\times$表示集合间的笛卡尔积.对于集合A和B,$A\times B$的元素表示为$(a,b)$,其中$a\in A$, $b\in B$.用集合论的术语表达就是:$A\times B=\{(a,b)|a\in A, b\in B\}$. 例如,如果令$A=\{0,1,2,3\}, B=\{2,3,4\}$,那么$A\times B=\{ (0,2),(0,3),(0,4),(1,2),(1,3),(1,4),(2,2),(2,3),(2,4),(3,2),(3,3),(3,4) \}$.



\subsection{映射}


给定集合A和B,我们可以从A中每一个元素上拉一根线连接到B中的某一个元素,这些线的分布形式就被称为一个从A到B的\textbf{映射}(mapping).将这个映射记为f,A中拉出线的元素组成的集合,叫做f的\textbf{定义域}(domain);B中被线连接到的元素的集合,叫做f的\textbf{值域}(image).我们一般用“$f:A\rightarrow B$”表示“f是从A到B的映射”,也就是说从A的元素上拉线到B的元素上.如果没有特殊说明,这样的表示方法都默认f的定义域是整个A集合.

注意,映射是有方向区分的,比如在上面的例子中,A中每个元素只能拉一根线出去,而且每个元素都要拉一根(默认定义域是整个A),但是B中某些元素是可以被多根线连接的,也可以没有连接(即不在值域中).

如果映射f


\subsection{基数和无穷的概念}

对于集合A,定义$|A|$为A中元素的数量,称为集合A的\textbf{基数}(cardinal number).

显然由于空集不含任何元素,$|\phi|=0$. 对两个集合并集时,由于相同元素不重复计数,所以对于集合A和B,我们有不等式:$|A\cup B|\le|A|+|B|$.

不同的集合有可能有一样多的元素,这个时候它们的基数相等.对于无穷集合,我们没法一个个数出集合中的元素,所以靠数元素数目是没法研究无穷集合的基数的.康托尔注意到,可以通过

对于有限集合A,$|A|=n$,这里$n$是某个非负整数.
