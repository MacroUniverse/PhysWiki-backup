% 复形的单纯同调群
% 同调|homology|复形|complex|单形|simplex|单纯形|复合形|边缘算子|链|链群|chain group|单纯同调


\pentry{单纯形与复形\upref{SimCom},自由群\upref{FreGrp}}


\subsection{复形上的链}

\begin{definition}{可定向单形}
一个单形$[a_0, a_1, \cdots, a_n]$,根据下标是奇排列还是偶排列,可以分为两类.称这样划分出来的两个等价类,是单形的\textbf{有向单形(oriented simplex)}.

偶排列的有向单形记为$a_0a_1a_2a_3\cdots a_n=a_1a_2a_0a_3\cdots a_n=\cdots$,奇排列的有向单形为$a_1a_0a_2a_3\cdots a_n=a_0a_2a_1a_3\cdots a_n=\cdots$.也就是说,有向单形的表示就是去掉单形表示的中括号.
\end{definition}

\begin{definition}{链群}
有向单形作为字母,可以构造自由生成阿贝尔群.其中同一个单形的两个有向单形,互为彼此的逆元.

称这样的自由生成阿贝尔群为一个\textbf{链群(chain group)},它的元素被称为\textbf{链(chain)}.
\end{definition}

链群的名称很直观.考虑$1$-单形$[a_0, a_1]$的有向单形$a_0a_1$和$a_1a_0$,前者可以视为从点$a_0$到点$a_1$的一个箭头,后者可以视为点$a_1$到点$a_0$的一个箭头,而$[a_0, a_1]$就视为两点之间的线段,没有方向区分.这样一来,一维有向单形的链就是箭头的组合,其中首尾相连的部分就像链条一样,所以被称为链.也就是说,链群的群运算,就是链之间的“连接”运算,自然是交换的,因此链群被定义为有向单形的自由生成阿贝尔群.

\begin{definition}{复形上的链群}
给复形$K$的每一个单形都指定一个定向为正定向,则全体正定向的有向单形被称为$K$的\textbf{有向单形基本组}.另一定向的有向单形们,被表示为对应正定向单形的\tex

$K$的全体有向单形的链群,称为
\end{definition}












