% 椭圆轨道的玻尔模型

\begin{issues}
\issueDraft
\end{issues}

\footnote{参考 Wikipedia \href{https://en.wikipedia.org/wiki/Old_quantum_theory}{相关页面}
.}使用原子单位, 氢原子 $n,l,m$ 束缚态中, 总角动量为 $L = \sqrt{l(l+1)}$, 能量为 $E = -Z^2/(2n^2)$. 但不知为何 Sommerfeld 模型中取 $L = l$. 也就是说当 $m = l$ 时角动量可以刚延 $z$ 方向.

Sommerfeld 模型中量子化条件为($T$ 表示轨道的一个周期)
\begin{equation}
\int_T m\dot r \dd{r} = nh
\end{equation}
即
\begin{equation}\label{BohrEc_eq1}
\int_T \sqrt{2m\qty(E - \frac{k}{r} - \frac{L^2}{2mr^2})} \dd{r} = kh
\end{equation}
若取 $L = l$ 可证明 $n = l + k$. 开普勒问题
\begin{equation}
e = \sqrt{1 + \frac{2EL^2}{mk^2}}
\end{equation}
\begin{equation}
a = \frac{k}{2\abs{E}}
\end{equation}
