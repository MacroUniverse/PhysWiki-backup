% 亥姆霍兹分解
\pentry{矢量分析总结\upref{VecAnl}}

在矢量分析中, 三维空间中的亥姆霍兹分解 (Helmholtz decomposition) 可在给定边界条件的情况下, 将一个矢量场唯一地分解为无旋场和无散场的和. 它在流体力学和电磁学中皆有应用.

\subsection{全空间的亥姆霍兹分解 (傅里叶形式)}
设$F(x)$是三维空间$\mathbb{R}^3$上的光滑矢量场, 满足
$$
\int_{\mathbb{R}^3}|F(x)|^2dx<\infty.
$$
则它的傅里叶变换$\hat F$是良好定义的. 由此可定义一个标量函数
$$
\Phi(x)=\int_{\mathbb{R}^3}\frac{i\xi\cdot\hat F(\xi)}{|\xi|^2}e^{i\xi\cdot x}d\xi,
$$
和一个矢量场
$$
A(x)=\int_{\mathbb{R}^3}\frac{i\xi\times\hat F(\xi)}{|\xi|^2}e^{i\xi\cdot x}d\xi.
$$
则有
$$
F=-\nabla\Phi+\nabla\times A.
$$

\subsection{有界区域的亥姆霍兹分解 (强形式)}

\subsection{区域的亥姆霍兹分解 (弱形式)}

