% test2

% Bohr-Sommerfeld atomic model
% Bohr Atomic Model|Semi-classical|Quantum|Angular Momentum|Quantization

\pentry{Bohr atom model\upref{BohrMd} , definite integral\upref{DefInt}}

\footnote{reference Wikipedia \href{https://en .wikipedia .org/wiki/Old_quantum_theory}{related page}
 . }This article uses the atomic unit system \upref{AU} . The Sommerfeld model, like the Bohr atomic model, belongs to the semi-classical atomic model in the early development of quantum mechanics . It improves the Bohr model to better conform to some experimental results (such as the Zeeman effect) . The Bohr model assumes that electrons rotate around the nucleus in a circular orbit, while the Sommerfeld model uses an elliptical orbit . For the sake of simplicity, we also assume that the nucleus is fixed, and the relative coordinates and reduced mass \upref{TwoBD} should be considered for the motion of the nucleus .

In the Sommerfeld model, there are two conditions for orbital quantization
\begin{equation}\label{test2_eq4}
L = l
\end{equation}
\begin{equation}\label{test2_eq3}
\oint m\dot r \dd{r} = kh
\end{equation}
Where $L$ is the modulus length of the orbital angular momentum, $l, k$ are positive integers, $h$ is Planck’s constant (equal to $2\pi$ under the atomic unit system), $r$ is the distance from the electron to the nucleus, $\dot r$ represents the time derivative of $r$, and $\oint$ represents the integral of one circle of the elliptical orbit .

According to these two quantization conditions, the shape and size of the elliptical orbit can be uniquely determined by the quantum numbers $k, l$, and it can be proved that the total energy of the orbit is
\begin{equation}\label{test2_eq5}
E = -\frac{Z^2}{2n^2}
\end{equation}
among them
\begin{equation}\label{test2_eq2}
n = l + k
\end{equation}
Note that since $k$ is a positive integer, $l$ must satisfy $0 <l <n$ .

\subsection{Numerical Verification}

Let's verify that \autoref{test2_eq5} , \autoref{test2_eq2} meets the quantization condition \autoref{test2_eq3} . This is a central force field problem . Substitute \autoref{CenFrc_eq10}~\upref{CenFrc} and \autoref{test2_eq4} into \autoref{test2_eq3} to get
\begin{equation}\label{test2_eq1}
2\int_{ac}^{a+c} \sqrt{2m\qty(E + \frac{Z}{r}-\frac{l^2}{2mr^2})} \dd{r} = kh
\end{equation}
Note that the ring integral in \autoref{test2_eq3} can be replaced with two equal integrals, that is, the integral from the perihelion point $ac$ of the elliptical orbit to the aphelion point $a+c$ (see "Three Definitions of Ellipse\upref {Elips3}”), when integrating from aphelion to perihelion\footnote{"perihelion" and "aphelion" are the customary names in Kepler's problem, regardless of whether the center is perseverance or nucleus . } . In the Kepler problem, the eccentricity $e$ and the parameters $a, c$ of the ellipse can be expressed by energy and angular momentum (\autoref{CelBd_eq7}~\upref{CelBd} , \autoref{CelBd_eq8}~\upref{ CelBd})
\begin{equation}\label{test2_eq6}
e = \frac{c}{a} = \sqrt{1 + \frac{2EL^2}{mZ^2}}
\qquad
a = \frac{Z}{2\abs{E}}
\end{equation}
Substitute \autoref{test2_eq5} , \autoref{test2_eq2} and \autoref{test2_eq6} into \autoref{test2_eq1} . If both sides of the equation are true for all $l,k> 0$, then the quantization condition is satisfied . Since the integration is more complicated, let's use Matlab for numerical integration (the connection is not completed) . Note that $h = 2\pi$ in the atomic unit .
\begin{lstlisting}[language=matlab]
l = 2; k = 1; n = l + k;
Z = 1; h = 2*pi;
E = -Z^2/(2*n^2);
me = 1;
a = abs(k)/(2*abs(E)); b = l/sqrt(2*me*abs(E));
c = sqrt(a^2-b^2);
f = @(r)real(sqrt(2*me*(E + Z ./r-l^2 ./(2*me*r .^2))));
r = linspace(a-c, a+c, 500);
figure; plot(r, f(r));
I = 2*integral(f, a-c, a + c);
rel_err = (I-h*k)/(h*k);
disp('relative error =');
disp(rel_err);
\end{lstlisting}
operation result: 
\begin{lstlisting}[language=matlabC]
relative error =
  1 .4136e-16
\end{lstlisting}
Readers can also replace $k, l$ with other positive integers for verification .

%%eng
