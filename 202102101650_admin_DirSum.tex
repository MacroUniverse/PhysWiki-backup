% 直和 补空间
% 矢量空间|子空间|直和空间|直和

\pentry{子空间\upref{SubSpc}}
\begin{definition}{直和}\label{DirSum_def1}
令域 $\mathbb F$ 上的矢量空间 $V_1$ 和 $V_2$ 为 $V$ 的两个子空间\upref{SubSpc}, 满足
\begin{equation}
V_1 \cap V_2 = \qty{0}
\end{equation}
且任意 ${v} \in V$ 都能表示为 $V_1$ 和 $V_2$ 中矢量的线性组合, 即
\begin{equation}
{v} = c_1 {v_1} + c_2 {v_2}
\qquad
({v_1} \in V_1,\ {v_2} \in V_2,\ c_1, c_2 \in \mathbb F)
\end{equation}
那么空间 $V$ 就是 $V_1$ 和 $V_2$ 的\textbf{直和空间}, 用\textbf{直和(direct sum)}运算记为
\begin{equation}
V = V_1 \oplus V_2
\end{equation}
我们把这两个子空间叫做\textbf{互补的}, 即 $V_2$ 是 $V_1$ 在 $V$ 中的\textbf{补空间(complement space)}, 反之亦然.
\end{definition}

直和空间 $V_1 \oplus V_2$ 中的所有矢量可以分为三组, 分别是 $V_1$ 中的矢量, $V_2$ 的矢量, 以及只能表示为 $V_1$ 和 $V_2$ 中矢量的线性组合的矢量.

\subsection{直和空间的基底}
从基底的角度来看, 若 $V_1$ 和 $V_2$ 中分别有一组基底 ${\alpha_i}$ $(i = 1, \dots, N_1)$ 和 ${\beta_i}$ $(i = 1, \dots, N_2)$, 那么直和空间中的任意矢量可以表示为(系数可以部分或全部为零)
\begin{equation}
{v} = \sum_i a_i {\alpha_i} + \sum_j b_j {\beta_j} \qquad (v\in V)
\end{equation}
可以证明 $\alpha_1, \dots, \alpha_{N_1}, \beta_1, \dots, \beta_{N_2}$ 是线性无关的, 所以是 $V$ 的一组基底. 所以 $V$ 是 $N_1 + N_2$ 维的.

证明: 使用反证法, 若有不全为零的系数使
\begin{equation}
\sum_i a_i {\alpha_i} + \sum_j b_j {\beta_j} = 0
\end{equation}
那么

就是直和空间中的基底. 显然它们是线性无关的: $a_i$ 不在 $V_2$ 内,所以不能用

注意合并后的 $N_1 + N_2$ 个矢量之间不一定线性无关, 所以不一定能构成直和空间中的一组基底. 虽然某个 ${\alpha_i}$ 不能表示为其他 ${\alpha_i}$ 的线性组合, 但却可能可以表示为一些 ${\beta_i}$ 的线性组合, ${\beta_i}$ 亦然.

如果把这些 “多余” 的矢量全部剔除, 使任意 ${\alpha_i}$ 不能用 ${\beta_i}$ 的线性组合表示, 且任意 ${\beta_i}$ 不能用 ${\alpha_i}$ 的线性组合表示, 那么我们就得到了直和空间中的一组基底, 其维数小于或等于 $N_1 + N_2$, 且大于或等于 $N_1$ 和 $N_2$.

\begin{example}{}\label{DirSum_ex1}
若三维空间中有两个不共线的几何矢量 ${v_1}, {v_2}$, 它们张成一个平面, 或二维子空间. 另有一个矢量 ${v_3}$, 独自张成一条直线, 即一维空间.

若 ${v_3}$ 落在 ${v_1}, {v_2}$ 张成的平面内, 则三个矢量的所有线性组合仍然在该平面内, 所以直和空间仍然是该平面.

若 ${v_3}$ 落在平面外, 则三个矢量将会张成整个三维空间, 所以直和就是三维空间. 此时两个子空间在该三维空间中互补.
\end{example}

\begin{example}{}\label{DirSum_ex2}
若三维空间中有两个不共线的几何矢量 ${v_1}, {v_2}$, 它们张成一个平面(二维子空间) $V_{12}$. 另有 $V_{12}$ 平面外的两个不共线且与的几何矢量 ${v_3}$ 和 ${v_4}$, 张成另一个二维子空间 $V_{34}$. 然而四个矢量中只有三个是线性无关的(容易证明任意三个都线性无关), 所以 $V = V_{12} \oplus V_{34}$ 是三维而不是四维几何矢量空间, 四个矢量中的任意三个都可以作为 $V$ 空间的基底.
\end{example}
\addTODO{补是不唯一的(\autoref{DirSum_ex1})}
\addTODO{给出一个空间和子空间, 如何求一个补空间?}
