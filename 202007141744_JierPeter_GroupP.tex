% 群论中的证明和习题解答
% 群论|正规子群|逆序数

\subsection{子群和正规子群}

\begin{example}{对换和逆序数}\label{GroupP_ex1}
问题来源请见\autoref{NormSG_ex5}\upref{NormSG}.

%未完成


\end{example}

\subsection{群作用}

\begin{example}{迷向子群}\label{GroupP_ex2}
问题来源请见\autoref{Group3_exe1}\upref{Group3}.

已知群$G$作用在集合$X$上,且对于$x\in X$有$G$的子集$F_x=\{g\in G|g\cdot x=x\}$.

那么$\forall g\in F_x$,$(g^{-1}g)\cdot x=e\cdot x=x$.又因为$g\cdot x=x$,所以$g^{-1}\cdot x=g^{-1}\cdot(g\cdot x)=(g^{-1}g)\cdot x=x$.

因此,$\forall g_1, g_2\in F_x$,有$(g_1^{-1}g_2)\cdot x=g_1^{-1}\cdot x=x$,也就是说,$(g_1^{-1}g_2)\in F_x$.根据判别式\autoref{Group1_the3}\upref{Group1},$F_x$构成$G$的子群.




\end{example}

\begin{example}{Burnside引理}\label{GroupP_ex3}
首先转写一个表达:$\sum_{g\in G}|X^g|=$“满足$g\cdot x=x$的$(g, x)$数量”$=\sum_{x\in X}|F_x|$.

由\autoref{Group3_cor1}\upref{Group3},$\sum_{x\in X}|F_x|=\sum_{x\in X} |G|/|O_x|=|G| \sum_{x\in X} 1/|O_x|$

现在关键是$\sum_{x\in X} 1/|O_x|$是多少.显然,$\sum_{x\in O_x} 1/|O_x|=1$,也就是说$1/|O_x|$在每一个$O_x$上求和的结果是$1$,那么它在整个$X$上求和的结果,刚好就是所有轨道的数量$|\{O_x|x\in X\}|$.

把以上结果整合,我们有:$\sum_{g\in G}|X^g|=|G|\cdot|\{O_x|x\in X\}|$.
\end{example}

\begin{example}{自由群的一般性质}\label{GroupP_ex4}
问题来源请见\autoref{FreGrp_the1}\upref{FreGrp}

设集合$S$到群$G$上有一个集合间的映射$f$.我们来构造一个$\varphi: F(S)\rightarrow G$的同态.

首先,由于$\varphi$是$f$的扩张,因此要定义$\forall s\in S, \varphi(s)=f(s)$.

其次,由于$\varphi$应为一个群同态,因此要定义$\forall s_i\in S, \varphi({s_1s_2s_3\cdots})=\varphi(s_1)\varphi(s_2)\varphi(s_3)\cdots$.同时,还需要有$\varphi(s^{-1})=\varphi(s)^{-1}$.


\end{example}

