% 调和函数

\begin{issues}
\issueDraft
\end{issues}

\pentry{拉普拉斯方程\upref{LapEq}}

设$\Omega\subset\mathbb{R}^n$是区域. 称$\Omega$上的拉普拉斯方程的解为$\Omega$上的\textbf{调和函数 (harmonic function)}. 它是数学和物理学中常见的对象.


\subsection{一般性质}
\begin{theorem}{刘维尔定理}\label{HarFun_the1}
$\mathbb R^N$ 上的调和函数有界当且仅当它是常数.
\end{theorem}
\addTODO{证明}

\begin{theorem}{最大值定理}
$\mathbb R^N$ 上一个区域内的调和函数的最大值和最小值总出现在该区域的边界处.
\end{theorem}
\addTODO{证明}

\begin{theorem}{光滑}
调和函数是光滑的, 即在定义域处处无穷阶可导.
\end{theorem}
\addTODO{证明}



\subsection{平面区域上的调和函数}


\subsection{调和多项式与球谐函数}
