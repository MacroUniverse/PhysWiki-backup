% 基尔霍夫电路定律

\pentry{电路\upref{Circ}}

\begin{theorem}{基尔霍夫电流定律}
\textbf{基尔霍夫电流定律}又称为\textbf{基尔霍夫第一定律},规定在电路中所有进入某节点的电流的总和等于所有离开这节点的电流的总和. 或者说,假设进入某节点的电流为正值,离开这节点的电流为负值,则所有涉及这节点的电流的代数和等于零.以方程式表达,对于电路的任意节点,有
\begin{equation}
\sum_{k=1}^n i_k =0
\end{equation}
其中,$i_k$是第$k$个进入或离开这节点的电流,是流过与这节点相连接的第$k$个支路的电流,可以是实数或复数.
\end{theorem}

\subsubsection{证明}
考虑电路的某节点,跟这节点相连接有$n$个支路.假设进入这节点的电流为正值,离开这节点的电流为负值,则这节点的总电流$i$等于流过支路$k$的电流$i_k$的代数和:
\begin{equation}
i=\sum_{k=1}^n i_k
\end{equation}
将这方程式对某段时间 $[t_1, t_2]$ 内积分,可以得到这段时间该节点电荷的增加
\begin{equation}
q=\sum_{k=1}^n q_k
\end{equation}
其中 $q = \int_{t_1}^{t_2} i(t) \dd{t}$, $q_k=\int_{t_1}^{t_2} i_k(t) \dd{t}$是流过支路$k$的电荷.

若 $q>0$, 则说明有正电荷会累积于该节点, $q < 0$ 表示负电荷会累积于节点. 在讨论电路时, 我们一般假设任意一点不存在净电荷\upref{Circ}, 所以 $q$ 和 $i$ 都恒为零.

再具体来结合一下电路图说明吧.
\begin{figure}[ht]
\centering
\includegraphics[width=8cm]{./figures/Kirch_1.pdf}
\caption{列节点方程} \label{Kirch_fig1}
\end{figure}
在列出节点方程时,凡流入节点的电流都写在等式的一侧, 凡从节点流出的电流都写在另一侧.但是,求解复杂电路问题时,各支路电流往往是未知量,它们的方向事先并不知道.这时,可以先给每个支路电流假设一个方向,并按照这一方向列出方程.求解方程后,如果求得某支路电流的数值为正,则该电流的实际方向与假设方向相同,否则相反.这个假设的电流方向叫做电流的\textbf{正方向}.给每一支路电流假设(或称“选定")一个正方向之后,就可用代数量描写每条支路的电流,代数量的绝对值反映电流的大小,代数量的正负则反映电流的实际方向.正方向一经选定.节点方程的形式(等号左右两边应写哪些电流)就完全确定.例如,为列出\autoref{Kirch_fig1}中节点$A$的方程,可任意地选定与$A $有关的三个支路电流的正方向如图箭头所示,从而写出如下的节点方程:
\begin{equation}
I_1+I_3=I_2
\end{equation}
也就是
\begin{equation}
I_2-I_1-I_3=0
\end{equation}
\begin{theorem}{基尔霍夫电压定律}
\textbf{基尔霍夫电压定律}又称为\textbf{基尔霍夫第二定律},表明沿着闭合回路所有元件两端的电势差(电压)的代数和等于零.或者,换句话说,沿着闭合回路的所有电动势的代数和等于所有电压降的代数和.以方程式表达,对于电路的任意闭合回路,
\begin{equation}\label{Kirch_eq1}
\sum_{k=1}^m U_k = 0
\end{equation}
其中,$m$是此闭合回路的元件数目,$U_k$是元件两端的电压,可以是实数或复数.
\end{theorem}

\subsubsection{证明}
根据电势差的定义(\autoref{Voltag_eq1}\upref{Voltag})
\begin{equation}
U_{21} = V(\bvec r_2) - V(\bvec r_1) = - \int_{\bvec r_a}^{\bvec r_b} \bvec E_0(\bvec r) \vdot \dd{\bvec r}
\end{equation}
如果路径起点为 $\bvec r_1$, 终点为 $\bvec r_N$, 中途有若干点 $\bvec r_2, \dots, \bvec r_{N-1}$. 那么有可以将路径积分划分为若干段, 总电势差等与每段电势差之和
\begin{equation}
U_{N1} = U_{21} + U_{32} + \dots + U_{N, N-1}
\end{equation}
现在, 如果取一个环路作为积分路径, 即起点终点相接, 即 $\bvec r_1 = \bvec r_N$, $U_{N1} = 0$. 立即可得\autoref{Kirch_eq1}. 证毕.

下面来看一下