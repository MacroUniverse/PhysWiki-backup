% 氢原子的选择定则
% 选择定则|宇称|3j 符号

\pentry{3j 符号\upref{ThreeJ}, 球谐函数\upref{SphHar}, 微扰理论\upref{TDPT}}

矩阵元的选择定则(什么时候为零)决定了某两个径向波函数的微分方程是否 couple. 如果没有 couple, 一阶微扰的 transition rate 为零, 但有可能多次 transition 仍然存在 couple. 即从初态 couple 到中间态, 中间态再 couple 到末态. $n$ 次 couple 需要用至少 $n$ 阶的微扰理论. 如果多次 transition 也被禁止, 那么就是绝对 forbidden 的. % 未完成: 不确定是不是这么定义的

\subsection{利用 3j 符号的对称性推导}
\pentry{3j 符号\upref{ThreeJ}}
相比与算符对易, 3j 符号的好处是不仅能得到选择定则,还可以直接算出 dipole 矩阵元的具体值而无需手动积分\footnote{当然, 手动 3j 符号也比较繁琐, 可以借助 Wolfram Alpha 或 Mathematica, Matlab 中我也写了一个程序(同样可以符号计算), 还没放进百科.}. Length gauge 中电场的哈密顿量为 $\bvec{\mathcal{E}}\vdot \bvec r$. 角向的选择定则可以由
\begin{equation}
\mel{Y_{l,m}}{\bvec r}{Y_{l',m'}} = \mathcal{E}_x \mel{Y_{l,m}}{x}{Y_{l',m'}} + \mathcal{E}_y \mel{Y_{l,m}}{y}{Y_{l',m'}} + \mathcal{E}_z \mel{Y_{l,m}}{z}{Y_{l',m'}}
\end{equation}
来决定. 而
\begin{equation}
x = \sqrt{\frac{2\pi}{3}} r (Y_{1,-1} - Y_{1,1}) \qquad
y = \I \sqrt{\frac{2\pi}{3}} r (Y_{1,-1}+Y_{1,1}) \qquad
z =2\sqrt{\frac{\pi}{3}} rY_{1,0}
\end{equation}

使用\autoref{SphCup_eq1}~\upref{SphCup} 以及 3j 符号的对称性就可以得到选择定则. 
\begin{equation}
\mel{Y_{l,m}}{Y_{1,m_1}}{Y_{l',m'}} = \sqrt{\frac{3(2l+1)(2l'+1)}{4\pi}} \pmat{l & 1 & l'\\ 0 & 0 & 0}\pmat{l & 1 & l'\\ -m & m_1 & m'}
\end{equation}
由选择定则 $-m + m_1 + m' = 0$ 得 $\Delta m = -m_1$. 所以
\begin{equation}
\Delta m =
\begin{cases}
0 & (\text{电场只延 $z$ 方向}) \\
0, \pm 1 & (\text{其他方向电场})
\end{cases}
\end{equation}

由三角约束($\abs{l-l'} \leqslant 1 \leqslant l + l'$)得 $\Delta l = 0, \pm 1$. 但由 parity CG coeff 得 $l + l' + 1$ 为偶数, 即 $l + l'$ 为奇数, 所以只能有
\begin{equation}
 \Delta l = \pm 1
\end{equation}
这是两个常见的选择定则.

注意符合以上选择定则的 dipole 矩阵元仍然有可能为零. 事实上, 由 3j 符号的所有选择定则, 我们可以知道所有 3j 符号等于 0 的情况, 即所有选择定则.

\subsection{物理的角度}
$z$ 方向的电场不会改变 $z$ 方向的角动量, 所以 $L_z$ 守恒, $\Delta m$ 为 0. 但守恒在量子力学中究竟怎么定义呢? 就是一个算符作用在一个态后, 对这个态测量某个物理量时这个态不变.

% 未完成: 补充 3j 符号的所有选择定则. 确定即使符合上面两个常见的选择定则, 也有可能出现 0.

%未完成: Virtual State?
