% 内积

\pentry{矢量空间\upref{InerPd}}

在矢量空间中, 我们可以另外定义任意两个矢量的\textbf{内积}运算, 运算的结果是一个实数或复数. 内积运算不是矢量空间所必须的, 但物理中的矢量空间几乎都定义了内积运算. 我们把定义了内积运算的空间称为\textbf{内积空间}\footnote{满足一定收敛条件的内积空间也叫\textbf{希尔伯特空间(Hilbert space)}.}% 链接未完成

两个矢量内积的定义必须满足
\begin{enumerate}
\item $\braket{u}{v} = \braket{v}{u}^*$ (\textbf{共轭对称})
\item $\braket{w}{a u + b v} = a\braket{w}{u} + b\braket{w}{v}$ (\textbf{线性})
\item $\braket{v}{v} > 0$ (\textbf{正定})
\end{enumerate}

内积一个重要性质就是满足柯西不等式\upref{CSNeq}
\begin{equation}
\abs{\braket{u}{v}}^2 \leqslant \braket{u}{u} \cdot \braket{v}{v}
\end{equation}

\begin{itemize}
\item $\braket{v}{v}$ 必定是一个范数(证明:柯西不等式)
\end{itemize}

\subsection{证明内积必定是范数}
要证
\begin{equation}
\norm{x+y}^2 \leqslant (\norm{x} + \norm{y})^2 = \norm{x}^2 + \norm{y}^2 + 2\norm{x}\norm{y}
\end{equation}
即证
\begin{equation}
\braket{x+y}{x+y} - \braket{x}{x} - \braket{y}{y} \leqslant 2\norm{x}\norm{y}
\end{equation}

