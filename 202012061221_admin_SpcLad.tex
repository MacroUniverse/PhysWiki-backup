% 太空电梯

\pentry{万有引力\upref{Gravty}, 离心力\upref{Centri}}

\subsubsection{简介}
太空电梯是指在地球同步轨道外放置一个空间站

碳纳米管的发现使得太空电梯从的可行性大大提高, 本文中我们来计算太空电梯所需缆绳的质量.

在地球所在的旋转参考系中, 缆绳上某点的张力为
\begin{equation}\label{SpcLad_eq1}
F(r) = F(r_0) + \int_{r_0}^{r} \qty(\frac{GM}{r'^2} - \omega^2 r') \lambda(r') \dd{r'}
\end{equation}
$\lambda(r)$ 是绳的线密度, $r_0$ 是地球半径, $\omega$ 是地球自转角速度, $r$ 是绳索上某点与地心的距离. 我们需要 $r$ 达到地球同步轨道的高度.

绳截面可承受的最大张力压强为 $p$, 那么截面为
\begin{equation}
A(r) = F(r)/p
\end{equation}
绳的密度为常数 $\rho$, 那么线密度为
\begin{equation}
\lambda(r) = \rho A(r) = \rho F(r)/p
\end{equation}
带回\autoref{SpcLad_eq1} 有积分方程
\begin{equation}
F(r) = \frac{GM\rho}{p} \int_{r_0}^{r} \frac{F(r')}{r'^2} \dd{r'} - \frac{\omega^2\rho}{p}\int_{r_0}^r F(r') r' \dd{r'}
\end{equation}
令两个积分前面的常数为 $\alpha, \beta$, 两边求导
\begin{equation}
\dv{F}{r} = \alpha \frac{F(r)}{r^2} - \beta r F(r)
\end{equation}
分离变量, 解得
\begin{equation}
F = C\exp(-\frac{\alpha}{r} + \frac{\beta}{2} r^2)
\end{equation}
令初始条件为 $F(r_0) = F_0$, $F_0$ 是载重. 代入得到方程的解为
\begin{equation}
F(r) = F_0 \exp[\frac{GM\rho}{p} \qty(\frac{1}{r_0} - \frac{1}{r}) - \frac{\omega^2\rho}{2p}(r^2 - r_0^2)]
\end{equation}

目前最强的材料碳纳米管保守估计\footnote{来源:\href{https://www.nature.com/articles/s41467-019-10959-7}{Nature}.} $p = 3\times 10^{10} \Si{Pa}$, 密度使用 $\rho = 1.34\times 10^{3} \Si{kg/m^3}$. 地球半径取 $6.370\times 10^6\Si{m}$, 同步轨道半径为 $4.2164\times 10^7 \Si{m}$. 地球质量 $M = 5.972 \times 10^{24} \Si{kg}$.

代入上式得 $F = 13.5 F_0$. 这是一个很小的比例, 可见碳纳米管的大规模生产完全可以开启新的航天时代.

\subsubsection{Matlab 代码}
\begin{lstlisting}[language=matlab]
G = 6.6743e-11; % 引力常数
r0 = 6.370e6; % 地球半径
r = 4.2164e7; % 同步轨道半径
M = 5.972e24; % 地球质量
rho = 1.34e3; % 碳纳米管密度 0.037-1.34
p = 2.5e10; % 碳纳米管截面压强
w = 2*pi/(24*3600); % 地球角速度

disp(exp(G*M*rho/p*(1/r0 - 1/r) - w^2*rho/(2*p)*(r^2 - r0^2)))
\end{lstlisting}
