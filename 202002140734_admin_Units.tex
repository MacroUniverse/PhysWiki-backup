% 量纲转换

\pentry{几何矢量\upref{GVec}}

我们首先要区分两类变量, 一种是只有数值没有单位, 姑且临时叫做 “数学量”, 另一种既有数值也有单位通常叫做\textbf{物理量}. 我们可以把物理量和一维几何矢量的部分性质类比: 一个一维几何矢量本身也不能用一个数字描述, 而是需要先选取一个矢量基底, 然后用一个数字(即坐标)乘以这个基底表示这个矢量. 若甲选择的基底是乙的基底的两倍, 那么甲的坐标将会是乙的坐标的一半.

用公式表示, 假设某个矢量是 $\bvec v$, 两种不同的基底分别是 $\bvec u_1$ 和 $\bvec u_2$, 则
\begin{equation}\label{Units_eq1}
\bvec v = x_1 \bvec u_1 = x_2 \bvec u_2
\end{equation}
其中 $x_1$ 和 $x_2$ 是数学量. 现在, 如果我们已知 $\bvec u_1$ 和 $\bvec u_2$ 的关系, 例如
\begin{equation}\label{Units_eq2}
\bvec u_1 = c \bvec u_2
\end{equation}
我们就可以直接将这个关系带入\autoref{Units_eq1}, 得
\begin{equation}
x_2 = c x_1
\end{equation}
若给出得关系是
\begin{equation}
\bvec u_2 = b\bvec u_1
\end{equation}
带入\autoref{Units_eq1} 得
\begin{equation}
x_1 = b x_2
\end{equation}
显然 $b = 1/c$.

\begin{example}{长度}
我们来考虑一个表示长度的物理量 $L$. 在我们确定单位(类比矢量基底)以前, 它不能使用任何数字表示. 现在规定单位(即基底) 为 $l_1 = 1\Si{cm}$, $l_2 = 1\Si{m}$, 且有
\begin{equation}
l_2 = 100 l_1
\end{equation}

\begin{equation}
L = 10 l_1 = 
\end{equation}

\end{example}
