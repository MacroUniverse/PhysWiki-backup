% 流形上的代数结构
% 线性空间|切空间|微分几何|张量|张量场|光滑函数|环|模|域

%\pentry{模\upref{Module},切向量场\upref{Vec}}

本节中,设$M$为一光滑实流形.

\subsection{光滑切向量场}

\subsubsection{作为实数域上的线性空间}

对于任意$p\in M$,我们已经讨论了切空间$T_pM$的性质,也就是流形上一点处的代数结构.由于一个切向量场是给流形上每一点赋予一个切向量,我们可以很自然地将一点处的切向量之间的和推广为切向量场之间的和.这就提醒我们,也许可以将向量场整体视为一个向量,构成一个线性空间.

为了得到线性空间,还需要一个运算:数乘.一个向量场整体乘以一个实数,就是每个点的切向量都乘以这个实数.这样,数乘、向量和都有了,$M$上的全体切向量场的集合就构成了一个实数域上的线性空间.

由于我们希望自由地进行微分运算,通常只讨论光滑切向量场.光滑切向量场构成的线性空间,我们在\textbf{切向量场}\upref{Vec}词条中已经提到过了.

\subsubsection{作为光滑函数环上的模}

切向量场上的运算,是由每个点处切向量的运算推广而来的.前面在数乘推广时我们默认了每个点都乘以同一个实数,那么可不可以每个点都乘以不同的数呢?换句话说,数乘时用的“数”,可以是








