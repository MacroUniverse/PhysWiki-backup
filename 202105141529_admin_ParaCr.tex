% 抛物线坐标系
% 抛物线|坐标系|正交曲线坐标|方位角|梯度|散度|旋度

\pentry{抛物线的三种定义\upref{Para3}}

\footnote{参考 \cite{Brandsen} Chap 3.5 和 Wikipedia \href{https://en.wikipedia.org/wiki/Parabolic_coordinates}{相关页面}.}抛物线的极坐标方程为(\autoref{Para3_eq1}~\upref{Para3})
\begin{equation}\label{ParaCr_eq1}
r = \frac{\eta}{1 - \cos \theta }
\end{equation}
若选用不同的半通径 $\xi$ ($\xi > 0$), 将得到一系列缩放的抛物线(\autoref{ParaCr_fig1} 中的绿色). 我们也可以把这些抛物线旋转 180 度, 得到
\begin{equation}
r = \frac{\xi}{1 + \cos \theta }
\end{equation}
这时我们把半通径记为 $\eta$ ($\eta > 0$), 当它取不同的值也得到一系列抛物线(\autoref{ParaCr_fig1} 中的红色). 另外我们再定义 $\eta = 0$ 和 $\xi = 0$ 分别代表极轴的正半轴和负半轴(包括原点). 综上, 通过 $\xi, \eta$ 两个坐标, 我们就能确定平面上的唯一一点.

\begin{figure}[ht]
\centering
\includegraphics[width=10cm]{./figures/ParaCr_1.pdf}
\caption{抛物线坐标系, 极轴向上(来自 Wikipedia)} \label{ParaCr_fig1}
\end{figure}

若把这些曲线绕极轴旋转一周变为一系列抛物面, 那么我们只需要再指定一个方位角 $\phi$ 就可以用坐标 $(\xi, \eta, \phi)$ 确定空间中的任意一点.

\subsection{与直角坐标和球坐标的转换}
一般令极轴与直角坐标的 $z$ 轴重合, 则根据定义有
\begin{equation}
\xi = r(1 + \cos\theta) = \sqrt{x^2 + y^2 + z^2} + z
\end{equation}
\begin{equation}
\eta = r(1 - \cos\theta) = \sqrt{x^2 + y^2 + z^2} - z
\end{equation}
\begin{equation}
\phi = \Arctan(y, x)
\end{equation}
其中 $\Arctan$ 见 “四象限 Arctan 函数\upref{Arctan}”. 可以解得
\begin{equation}\label{ParaCr_eq2}
z = \frac{\xi - \eta}{2}
\end{equation}
\begin{equation}\label{ParaCr_eq3}
x = \sqrt{\xi\eta}\cos\phi \qquad
y = \sqrt{\xi\eta}\sin\phi
\end{equation}

\subsection{正交曲线坐标系}
\pentry{正交曲线坐标系\upref{CurCor}}
可以证明抛物线坐标系是一个正交曲线坐标系, 即\autoref{ParaCr_fig1} 中过任意一点的两条坐标曲线都垂直:

在某点 $\bvec r$, 延 $\eta$ 曲线的切线方向为 $\pdv*{\bvec r}{\xi}$, 延 $\xi$ 曲线的切线方向为 $\pdv*{\bvec r}{\eta}$
\begin{equation}
\dd{x} = \sqrt{\frac{\eta}{\xi}}\frac{\cos\phi}{2}\dd{\xi} + \sqrt{\frac{\xi}{\eta}}\frac{\cos\phi}{2\sqrt{\xi}}\dd{\eta} - \sqrt{\xi\eta}\sin\phi\dd{\phi}
\end{equation}
\begin{equation}
\dd{y} = \sqrt{\frac{\eta}{\xi}}\frac{\sin\phi}{2}\dd{\xi} + \sqrt{\frac{\xi}{\eta}}\frac{\sin\phi}{2}\dd{\eta} + \sqrt{\xi\eta}\cos\phi\dd{\phi}
\end{equation}
\begin{equation}
\dd{z} = \frac{1}{2}\dd{\xi} - \frac{1}{2}\dd{\eta}
\end{equation}
所以三个归一化单位矢量分别为(上式的每一列归一化)
\begin{equation}
\uvec \xi = \sqrt{\frac{\eta}{\xi+\eta}}\cos\phi\,\uvec x + \sqrt{\frac{\eta}{\xi+\eta}}\sin\phi\,\uvec y + \sqrt{\frac{\xi}{\xi+\eta}}\,\uvec z
\end{equation}
\begin{equation}
\uvec \eta = \sqrt{\frac{\xi}{\xi+\eta}}\cos\phi\,\uvec x + \sqrt{\frac{\xi}{\xi+\eta}}\sin\phi\,\uvec y + \sqrt{\frac{\eta}{\xi+\eta}}\,\uvec z
\end{equation}
\begin{equation}
\uvec \phi = -\sin\phi \,\uvec x + \cos\phi \,\uvec y
\end{equation}
容易证明它们两两正交.另外有
\begin{equation}\label{ParaCr_eq4}
\dd{\bvec r} = \frac{1}{2}\sqrt{\frac{\xi + \eta}{\xi}} \uvec \xi \dd{\xi}
+ \frac{1}{2}\sqrt{\frac{\xi + \eta}{\eta}} \uvec \eta \dd{\eta}
+ \sqrt{\xi\eta}\, \uvec \phi \dd{\phi}
\end{equation}
体积元等于式中三个分量相乘
\begin{equation}
\dd{V} = \frac{1}{4} (\xi + \eta) \dd{\xi}\dd{\eta}\dd{\phi}
\end{equation}

\subsection{矢量算符}

结合\autoref{ParaCr_eq4} 和\autoref{CurCor_eq4}~\upref{CurCor}到\autoref{CurCor_eq6}~\upref{CurCor}得
\begin{equation}
\grad u = 2\sqrt{\frac{\xi}{\xi + \eta}}\pdv{u}{\xi} + 2\sqrt{\frac{\eta}{\xi + \eta}}\pdv{u}{\eta} + \frac{1}{\sqrt{\xi\eta}}\pdv{u}{\phi}
\end{equation}
\begin{equation}
\div \bvec A = \frac{2}{\xi + \eta}\qty[\pdv{\xi}\qty(\sqrt{\xi(\xi+\eta)}A_\xi) + \pdv{\eta}\qty(\sqrt{\eta(\xi+\eta)}A_\eta)] + \frac{1}{\sqrt{\xi\eta}} \pdv{A_\phi}{\phi}
\end{equation}
\begin{equation}
\begin{aligned}
\curl \bvec A = &\qty[\frac{2}{\sqrt{\xi+\eta}}\pdv{\eta} (\sqrt{\eta}A_\phi) - \frac{1}{\sqrt{\xi\eta}}\pdv{A_\eta}{\phi}]\uvec \xi\\
&+\qty[\frac{1}{\sqrt{\xi\eta}}\pdv{A_\xi}{\phi} - \frac{2}{\sqrt{\xi+\eta}}\pdv{\xi}(\sqrt{\xi}A_\phi)]\uvec\eta\\
&+\qty[\frac{2\sqrt{\xi}}{\xi+\eta}\pdv{\xi}(\sqrt{\xi+\eta}A_\eta) - \frac{2\sqrt{\eta}}{\xi+\eta}\pdv{\eta}(\sqrt{\xi+\eta}A_\xi)]\uvec\phi
\end{aligned}
\end{equation}
\begin{equation} % \cite{Brandsen}  eq (3.92)
\laplacian u = \frac{4}{\xi + \eta} \qty[\pdv{u}{\xi}\qty(\xi\pdv{\xi}) + \pdv{u}{\eta}\qty(\eta\pdv{\eta})] + \frac{1}{\xi\eta}\pdv[2]{u}{\phi}
\end{equation}
连带拉盖尔多项式(associated Laguerre polynomial).
\addTODO{链接}
