% 黎曼度量与伪黎曼度量
\pentry{内积\upref{InerPd},流形\upref{Manif},切丛\upref{TanBun},余切丛\upref{CotBun},张量场\upref{TenMan}}

本节采用爱因斯坦求和约定.

黎曼度量(Riemannian metric)或伪黎曼度量(pseudo-Riemannian metric)是黎曼几何或伪黎曼几何所要求的基本结构.赋予黎曼度量/为黎曼度量的微分流形被称为黎曼流形/伪黎曼流形,它们既是几何学研究的对象,也是广义相对论得以展开的舞台.

这里将一直设$M$是一个$n$维实微分流形.

\subsection{黎曼度量}
\begin{definition}{黎曼度量}
$M$上的一个黎曼度量$g$是指张量丛$T^*(M)\otimes T^*(M)$的一个对称的正定截面.等价地,给出黎曼度量$g$,就相当于在每一点$p$的切空间$T_pM$上指定一个内积$g_p$.
\end{definition}
给定局部坐标系$\{x^i\}$后,黎曼度量$g$的局部表达式是
$$
g_{ij}(x)dx^i\otimes dx^j,
$$
这里$g_{ij}(x)=g_{ji}(x)$.

\subsection{洛伦兹度量}