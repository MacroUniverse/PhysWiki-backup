% 热容量

\subsection{热容}
一个系统在一定条件下的\textbf{热容量(heat capacity)}定义为\footnote{这个定义可以类比电容量\upref{Cpctor}}
\begin{equation}
C = \dv{Q}{T}
\end{equation}
热熔可能跟温度压强等有关.

定义\textbf{比热容(specific heat capacity)}为热容除以质量
\begin{equation}
c = \frac{C}{m}
\end{equation}

\subsection{理想气体的等压热热容与等体热容}
\pentry{热力学第一定律\upref{Th1Law}}
理想气体等体过程\upref{EqVol}中的热容叫做\textbf{等体热容}
\begin{equation}\label{ThCapa_eq1}
C_V = \frac{i}{2} n R
\end{equation}
而等压过程\upref{EqPre}中的热容叫做\textbf{等压热容}
\begin{equation}\label{ThCapa_eq2}
C_P = \frac{i+2}{2} n R
\end{equation}

\subsubsection{等体热容的推导}
(未完成)
\begin{equation}
C_V = \dv{Q}{T} = \dv{E}{T} = \frac{i}{2} nR
\end{equation}

\subsubsection{等压热容的推导}
(未完成)
\begin{equation}
C_V = \dv{Q}{T} = \dv{E}{T} + \dv{W}{T} = \frac{i}{2} nR + nR = \frac{i+2}{2} nR
\end{equation}

