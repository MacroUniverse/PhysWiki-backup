% 氢原子波函数画图程序
% 氢原子|波函数|Matlab

\begin{issues}
\issueDraft
\end{issues}

\pentry{类氢原子的定态波函数\upref{HWF}, Matlab 画图\upref{MatPlt}}

这里给出\autoref{HWF_fig2}~\upref{HWF} 的画图程序. 需要 “球谐函数数值计算\upref{YlmNum}” 中的 \verb|SphHarm| 函数. 以及 \verb|surfSph| 函数.

\begin{lstlisting}[language=matlab]
Z = 1;
Rnl = @(Z,n,l,r)sqrt((2*Z/n)^3*factorial(n-l-1)/(2*n*factorial(n+l))) *...
        (2*Z*r/n).^l .* laguerreL(n-l-1, 2*l+1, 2*Z*r/n) .* exp(-Z*r/n);

r = linspace(0, 40, 1000);
th = linspace(0, 2*pi, 361);
ph = zeros(size(th));
thresh = [0.3183 7.28e-4 1e-4 1e-5 1e-6];
for n = 4
    for l = 0:n-1
        for m = 0 : l
            figure; set(gcf, 'units','normalized', ...
                'outerposition',[0 0 0.5 1]);
            set(gca, 'color', 'k');
            grid off; hold on;
            r2R2 = @(r) r.^2 .* abs(Rnl(Z,n,l,r)).^2;
            trial = @(rmax) integral(r2R2, 0, rmax) - 0.95;
            r_cut = fzero(trial, [0.1, 100]);
            tmp = linspace(0, 2*pi, 500);
            plot3(r_cut*cos(tmp), -3*ones(size(tmp)), ...
                r_cut*sin(tmp), 'w');
            prob_cut = abs(Rnl(Z,n,l,r_cut))^2 * 5;
            
            prob = abs(Rnl(Z,n,l,r).' .* SphHarm(l,m,th,ph)).^2;
            surfSph(r, th, 0, prob); view(0, 0);
            caxis([0, prob_cut]);
            title([num2str(n) ', ' num2str(l) ', ' num2str(m)]);
            disp([num2str(n) ', ' num2str(l) ', ' num2str(m)]);
            % 未完成: colormap_hydrogen, 使用默认 colormap
            % rgb = colormap_hydrogen(linspace(0,1,100));
            % colormap(rgb);
       end
    end
end
\end{lstlisting}
