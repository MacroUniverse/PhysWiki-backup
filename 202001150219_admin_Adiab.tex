% 绝热过程
% 绝热过程|体积|压强|状态方程|做功

\pentry{理想气体状态方程\upref{PVnRT}, 热力学第一定律\upref{Th1Law}, 压强体积图\upref{PVgraf}}

在系统状态变化过程, 如果和外界没有热量和粒子交换, 这个过程就叫做\textbf{绝热过程}
\begin{equation}
\Delta Q = 0
\end{equation}
由热力学第一定律\autoref{Th1Law_eq1}\upref{Th1Law},
\begin{equation}\label{Adiab_eq1}
W + \Delta E = 0
\end{equation}
即系统对外做功和内能增加之和为零.

\subsection{推导}
考虑一个极短的过程, \autoref{Adiab_eq1} 变为微分形式
\begin{equation}\label{Adiab_eq4}
\dd{W} + \dd{E} = 0
\end{equation}
其中(\autoref{PVgraf_eq1}\upref{PVgraf})
\begin{equation}\label{Adiab_eq2}
\dd{W} = P\dd{V}
\end{equation}
将理想气体状态方程(\autoref{PVnRT_eq1}\upref{PVnRT})两边微分得
\begin{equation}
\dd{P}V + P\dd{V} = nRdT
\end{equation}
将气体的内能公式(\autoref{IdgEng_eq1}\upref{IdgEng})两边微分得
\begin{equation}\label{Adiab_eq3}
\dd{E} = \frac{i}{2}n R\dd{T} = \frac{i}{2} (V\dd{P} + P\dd{V})
\end{equation}


$i$ 是气体分子自由度. 把\autoref{Adiab_eq2} 和\autoref{Adiab_eq3} 带入\autoref{Adiab_eq4} 得 $P$ 和 $V$ 之间得微分方程
\begin{equation}
\gamma P\dd{V} + V \dd{P} = 0
\end{equation}
其中 $\gamma$ 为绝热指数
\begin{equation}
\gamma = \frac{i+2}{i}
\end{equation}

\begin{equation}
P V^\gamma = C
\end{equation}
