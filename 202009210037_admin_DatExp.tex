% 机器学习数据探索
\pentry{机器学习数据类型\upref{DatTyp}}
\subsection{变量标识}
首先要确定哪些数据是输入(Input)的变量,哪些数据是要预测的目标值(output),然后需要确定变量的数据类型,定类变量或者定比变量等
\subsection{单变量分析}
单变量分析是数据分析中最简单的形式,其中被分析的数据只包含一个变量.因为它是一个单一的变量,它不处理原因或关系.单变量分析的主要目的是描述数据并找出其中存在的模式.  
\subsubsection{连续型特征分析方法}
连续特征变量可通过平均值,中位数,众数,最小值,最大值,范围,四分位数,IQR,方差,标准差,偏度,等来探索数据特征.此外,显示单变量数据的一些方法包括频率分布表、柱状图、直方图、频率多边形和饼状图.
\subsubsection{类别型特征分析方法}
对于类别特征的变量长通过频率,直方图来探索.
\subsection{双变量分析}
双变量分析目标是确定两个变量之间的相关性,测量它们之间的预测或解释的能力.
\subsubsection{散点图}
在笛卡尔平面上(想想X和Y轴)将一个变量对另一个变量进行绘图,从而创建散点图如果数据似乎符合直线或曲线,那么这两个变量之间存在关系或相关性.

\subsubsection{热力图}
利用多个变量之间的相关性矩阵做出热力图,以确定两个变量之间的相关性.
\subsubsection{二维列联表}
\subsubsection{堆积柱形图}
\subsubsection{卡方检验}
\subsubsection{Z检验/ T检验}
\subsubsection{方差分析(ANOVA)}