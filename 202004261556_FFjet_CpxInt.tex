% 复变函数的积分

\pentry{定积分\upref{DefInt}}
我们之前已经接触过了实函数的积分,那么我们如何推广到复数上呢?

\subsection{复积分的定义}

实际上,同高等数学一样,也采用“分割”、“作和”、“取极限”的步骤来定义积分.
\begin{definition}{复积分的定义}
设$C $为一条起点在$a $,终点在$b $的有向光滑曲线(或逐段光滑曲线),其方程为
\begin{equation}
z=z(t)=x(t)+\mathrm{i} y(t) \quad,(\alpha \leqslant t \leqslant \beta, a=z(\alpha), b=z(\beta))
\end{equation}
函数$ f (z) $定义在$C $上用一组点$z_{0}=a, z_{1}, z_{2}, \cdots, z_{n-1}, z_{n}=b$沿曲线从$ a $到$b $,对曲线$C$进行分割:
\begin{figure}[ht]
\centering
\includegraphics[width=10cm]{./figures/CpxInt_1.png}
\caption{请添加图片描述} \label{CpxInt_fig1}
\end{figure}
设$\Delta z_{k}=z_{k}-z_{k-1}$,$\zeta_k$为弧$z_{k-1}z_k$上任意一点,作和$\displaystyle S_{n}=\sum_{k=1}^{n} f\left(\zeta_{k}\right) \Delta z_{k}$.

当分割点的数量无限增加,并且分割$C$所得各个弧段长度中的最大值$d \to 0$时,不论对$C$的分法及$\zeta_k$的取法如何,$S_n$存在极限$S$,则称$ f (z)$沿$C $(从$a$到$b$)可积,称$ S $为$ f (z)$沿$C$(从$a$到$b $)的积分,记作
\begin{equation}
S=\int_{C} f(z) \mathrm{d} z
\end{equation}
\end{definition}