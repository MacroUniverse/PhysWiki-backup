% 自然对数底
% 微积分|自然对数|自然对数底|e

\pentry{极限\upref{Lim}, 对数函数}
微积分中有一个重要的极限,极限值是一个无理数,叫做\textbf{自然对数底},记为\footnote{为了与其他变量区分, 本书使用正体字母表示自然对数底.} $\E$. 它是一个无限不循环小数
\begin{equation}\label{E_eq1}
\E \equiv \lim_{x \to 0} (1 + x)^{\frac{1}{x}} = 2.7182818284590452\dots
\end{equation}
注意这里没有指定 $x\to 0$ 的方向, 即无论 $x$ 正数或负数该式都成立. $\E$ 也可以用无穷级数定义为
\begin{equation}\label{E_eq2}
\E \equiv \sum_{n=0}^\infty \frac{1}{n!} = 1 + 1 + \frac{1}{2} + \frac{1}{6} + \dots
\end{equation}
\addTODO{证明两种定义等效}
令 $k$ 为常数, 可以把\autoref{E_eq1} 拓展得
\begin{equation}
\lim_{x \to 0} (1 + kx)^{\frac{1}{kx}} = \E
\end{equation}
当 $k = -1$ 时易得
\begin{equation}\label{E_eq3}
\lim_{x \to 0} (1 - x)^{\frac{1}{x}} = \qty[\lim_{x \to 0} (1 - x)^{\frac{1}{-x}}]^{-1} = \frac{1}{\E} = 0.36787944117144232\dots
\end{equation}

\subsection{数值验证}
这里先用数值的方法验证\autoref{E_eq1} , 首先我们可以画出 $(1+x)^{1/x}$ 在原点附近的函数图, 注意当 $x = 0$ 时, 该函数无定义, 但这并不妨碍极限的存在. 可以看到, 无论 $x$ 从左边还是右边趋近于原点(即左极限和右极限), 结果都相等.
\begin{figure}[ht]
\centering
\includegraphics[width=8cm]{./figures/E_1.pdf}
\caption{$(1+x)^{1/x}$ 的函数图} \label{E_fig1}
\end{figure}

\autoref{E_tab1} 用数值计算验证\autoref{E_eq1} 的右极限.
\begin{table}[ht]
\centering
\caption{极限 $\E$ 数值验证(保留 6 位有效数字)}\label{E_tab1}
\begin{tabular}{|c|c|c|c|c|c|c|}
\hline
$x$ & $10^{-1}$ & $10^{-2}$ & $10^{-3}$ & $10^{-4}$ & $10^{-5}$ & $10^{-6}$ \\
\hline
$(1 + x)^{1/x}$ & $2.59374$ & $2.70481$ & $2.71692$ & $2.71815$ & $2.71827$ & $2.71828$ \\
\hline
\end{tabular}
\end{table}

\subsection{简单应用}
\pentry{随机变量 概率分布函数\upref{RandF}}
为什么说自然对数底是 “自然” 的呢? 我们来看两个例子.

\begin{example}{银行利滚利}
如果一笔数量为 $x$ 的钱存入某银行后, 银行的年利率是 $\lambda$, 那么一年后取出来连本带利共得 $(1+\lambda)x$. 并且银行规定, 在 $y$ 年时取出来($y$ 取任意正实数), 则利率按照 $\lambda y$ 来计算. 例如半年取出得到 $(1+\lambda/2)x$, 若取出立刻存入, 再过半年可得 $(1+\lambda/2)^2 x = (1 + \lambda + \lambda^2/4)x$, 比直接存一年要多. 事实上可以证明, 存取越频繁, 一年的总利息就越多, 简单来说是因为利滚利会越多.

那么如果存取的时间 $y \to 0$, 一年后连本带利的极限是多少呢? 答案是
\begin{equation}
x_1 = \lim_{y\to 0}(1 + \lambda y)^{1/y}
= \qty[\lim_{y\to 0}(1 + \lambda y)^{1/(\lambda y)}]^\lambda
= \E^\lambda
\end{equation}
这样, 
\end{example}

\begin{example}{彩票中奖概率分布}
假设某种彩票从 $t = 0$ 开始每 $\Delta t$ 开奖一次, 开奖时每张彩票有 $\Delta P$ 的概率中奖, 每张彩票可以参加任意多次抽奖\footnote{也可以假设每张彩票只能抽一次, 但每次抽奖后就立即购买一张新彩票.}. 令 $\lambda = {\Delta P}/{\Delta t}$, 对每种彩票, $\lambda$ 是一个常数. 为了让问题更连续, 考虑极限 $\Delta t\to 0$ 的情况, 求第一次中奖时间概率分布\upref{RandF} $f(t)$ 以及数学期望值.

先考虑离散情况, 画出概率树, 在第 $N$ 次开奖时第一次中奖的概率为(注意要保证前 $N-1$ 次未中奖)
\addTODO{概率树图}
\begin{equation}
\begin{aligned}
P(N) &= (1-\Delta P)^{N-1} \Delta P\\
&= (1-\lambda\Delta t)^{\frac{t}{\Delta t}-1} \lambda\Delta t\\
&= [(1-\lambda\Delta t)^{\frac{1}{\lambda\Delta t}}]^{\lambda t} \frac{\lambda\Delta t}{1-\lambda\Delta t}
\end{aligned}
\end{equation}

接下来, 我们假设该彩票开奖的时间越来越短, 但始终保持 $\lambda$ 不变. 那么根据\autoref{E_eq3}, 在极限 $\Delta t\to 0$ 下, 上式中方括号等于 $1/\E$, 而右边的分式变为 $\lambda \Delta t$.
\begin{equation}
P(N) = \lambda\E^{-\lambda t}\Delta t
\end{equation}
所以第一次中奖时间的概率分布为
\begin{equation}
f(t) = \lim_{\Delta t\to 0} \frac{P(N)}{\Delta t} = \lambda\E^{-\lambda t}
\end{equation}
\addTODO{画图}
可以验证归一化条件
\begin{equation}
\int_0^{+\infty} f(t) \dd{t} = \int_0^{+\infty} \lambda\E^{-\lambda t} \dd{t} = 1
\end{equation}
进而可以计算第一次中奖时间的数学期望为
\begin{equation}
T = \int_0^{+\infty} t f(t) \dd{t} = \int_0^{+\infty}t \lambda\E^{-\lambda t} \dd{t} = \frac{1}{\lambda}
\end{equation}
也可以计算在时间 $[0, t]$ 内中奖的概率为
\begin{equation}
P(t) = \int_0^{t} f(t') \dd{t'} = 1 - \E^{-\lambda t}
\end{equation}
\begin{table}[ht]
\centering
\caption{中奖概率}\label{E_tab2}
\begin{tabular}{|c|c|c|c|c|c|}
\hline
累计时间 $t$ & $T/3$  & $T/2$  & $T$     & $2T$ &     $3T$   &  $4T$   & $5T$\\
\hline
概率 $P(t)$ & $0.283$ & $0.393$ & $0.632$ & $0.865$ & $0.950$ & $0.982$ & $0.993$\\
\hline
\end{tabular}
\end{table}
\end{example}
