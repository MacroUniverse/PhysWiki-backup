% 正交曲线坐标系
% 多元微积分|坐标系|柱坐标系|球坐标系|矢量|内积|内积|导数|偏导数|曲线坐标系|正交曲线坐标系

\begin{issues}
\issueTODO
\end{issues}

\pentry{球坐标系\upref{Sph}, 柱坐标系\upref{Cylin}}

\footnote{本文参考 Wikipedia \href{https://en.wikipedia.org/wiki/Curvilinear_coordinates}{相关页面}.}如果 $u, v, w$ 是三维空间中某曲线坐标系的三个坐标, 空间任意一点的位置矢量\upref{Disp} $\bvec r$ 都是它们的函数 $\bvec r(u, v, w)$. 那么定义任意一点处三个单位矢量为
\begin{equation}\label{CurCor_eq8}
\uvec u = \frac{\pdv*{\bvec r}{u}}{\abs{\pdv*{\bvec r}{u}}}\qquad
\uvec v = \frac{\pdv*{\bvec r}{v}}{\abs{\pdv*{\bvec r}{v}}}\qquad
\uvec w = \frac{\pdv*{\bvec r}{w}}{\abs{\pdv*{\bvec r}{w}}}
\end{equation}
注意一般来说, 这三个矢量会随着 $\bvec r$ 改变. 形象地说: 当我们分别只把 $u, v, w$ 增加一点时, $\bvec r$ 会分别沿 $\uvec u, \uvec v, \uvec w$ 方向移动(请以球坐标系和柱坐标系为例思考).

若对空间中任意一点, \autoref{CurCor_eq8} 中的三个矢量都两两正交, 那么这个曲线坐标系就是\textbf{正交曲线坐标系(orthogonal curvilinear coordinate system)}. 常见的例子除了球坐标系\upref{Sph}, 柱坐标系\upref{Cylin} 还有抛物线坐标系\upref{ParaCr}.
\begin{exercise}{}
练习:试着用\autoref{CurCor_eq8} 计算球坐标和柱坐标中的单位矢量, 例如球坐标中
\begin{equation}
\bvec r = r\sin\theta\cos\phi\,\uvec x + r\sin\theta\sin\phi\,\uvec y + r\cos\theta\,\uvec z
\end{equation}
然后用点乘\upref{Dot}证明它们的单位矢量总是两两垂直. 即球坐标系和柱坐标系都是正交曲线坐标系.
\end{exercise}

\addTODO{以下内容合并到相关词条}
\subsection{柱坐标系}
我们先来分析柱坐标系\footnote{由于极坐标系可以看做柱坐标系 $z = 0$ 的情况, 我们不单独讨论}, 位置矢量 $\uvec r$ 在直角坐标系中展开为
\begin{equation}\label{CurCor_eq1}
\bvec r(r, \theta, z) = r\cos\theta\, \uvec x + r\sin\theta\, \uvec y + z\uvec z
\end{equation}
柱坐标系中三个单位矢量 $\uvec r, \uvec \theta, \uvec z$ 的方向被定义为每个坐标增加时 $\bvec r$ 增加的方向, 即以下偏导数的方向
\begin{equation}\label{CurCor_eq2}
\leftgroup{
\pdv{\bvec r}{r} &= \cos\theta\, \uvec x + \sin\theta\, \uvec y\\
\pdv{\bvec r}{\theta} &= -r\sin\theta\, \uvec x + r \cos\theta\, \uvec y\\
\pdv{\bvec r}{z} &= \uvec z
}\end{equation}
将这三个矢量归一化% 未完成:相关词条中可以给一道例题
, 就得到三个单位矢量
\begin{equation}\label{CurCor_eq3}
\begin{cases}
\uvec r = \cos\theta\, \uvec x + \sin\theta\, \uvec y\\
\uvec \theta = -\sin\theta\, \uvec x + \cos\theta\, \uvec y\\
\uvec z = \uvec z
\end{cases}
\end{equation}

可见柱坐标系和直角坐标系中的 $\uvec z$ 相同, 而 $\uvec r, \uvec \theta$ 分别是 $\uvec x, \uvec y$ 绕 $z$ 轴逆时针旋转 $\theta$ 角所得. 所以尽管柱坐标系中的三个单位矢量的方向取决于坐标, 但它们始终两两垂直.

我们把单位矢量始终保持两两垂直的坐标系叫做\textbf{正交曲线坐标系}, 或简称为\textbf{曲线坐标系}. 我们熟知的直角坐标系显然就是一个正交曲线坐标系, 稍后我们会看到球坐标系也是正交曲线坐标系.

现在我们可以将\autoref{CurCor_eq1} 和\autoref{CurCor_eq2} 用柱坐标中的三个单位矢量来表示.
\begin{equation}
\bvec r = r\uvec r + z\uvec z
\end{equation}
\begin{equation}
\pdv{\bvec r}{r} = \uvec r \qquad \pdv{\bvec r}{\theta} = r\uvec \theta \qquad \pdv{\bvec r}{z} = \uvec z\label{CurCor_eq5}
\end{equation}

与极坐标的情况\upref{DPol1} 类似, 将\autoref{CurCor_eq3} 对 $\theta$ 求偏导可以得到单位矢量的偏导
\begin{equation}
\pdv{\uvec r}{\theta} = \uvec \theta \qquad
\pdv{\uvec \theta}{\theta} = -\uvec r \qquad
\pdv{\uvec z}{\theta} = \bvec 0
\end{equation}
根据\autoref{CurCor_eq5} 和矢量函数的全微分%未完成: 链接
, 柱坐标系中一段微小位移可记为
\begin{equation}\label{CurCor_eq7}
\dd{\bvec r} = \pdv{\bvec r}{r}\dd{r} + \pdv{\bvec r}{\theta}\dd{\theta} + \pdv{\bvec r}{z}\dd{z} = \dd{r}\uvec r + r\dd{\theta} \uvec \theta + \dd{z} \uvec z
\end{equation}
