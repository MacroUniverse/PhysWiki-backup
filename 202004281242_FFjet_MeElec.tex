% 力电类比

电磁振荡和机械振动的规律非常相似,所以运用力电类比就可以把电磁振荡和机械振动对应起来,只要知道一种振动的解,就可以用类比方法得到另一种振动的解.虽然机械振动比较直观,但由于电学的迅速发展,人们对交变电路规律的熟悉程度已经超过机械振动.因此,在工程上,常常把复杂的机械振动问题用力电类比方法化成交变电路问题,然后通过计算或实验测定,找出它们的解机械振动和电磁振荡对应的物理量列在下表中.

\begin{table}[ht]
\centering
\caption{机械振动和电磁振荡对应的物理量}\label{MeElec_tab1}
\begin{tabular}{cccc}
\hline  机械振动  & &  电磁振荡(  串联电路 ) &\\ \hline  位移  & $x$ &  电荷  & $q$ \\ \hline  速度  & $v $&  电流  & $i$ \\ \hline  质量  & $m$ &  电感  & $L$ \\ \hline  劲度系数  & $k$ &  电容的倒数  & $\dfrac1C$ \\ \hline  阻力系数  & $\gamma$ &  电阻  & $R$  \\ \hline  驱动力  &$F$ &  电动势  & $\mathscr E$  \\ \hline  弹性势能  &  $\dfrac12 k x^2$ &  电场能量  & $\dfrac12 \dfrac{q^2}C$ \\ \hline  动能  & $\dfrac12 m v^2$ &  磁场能量  &  $\dfrac12 L i^2$ \\ \hline
\end{tabular}
\end{table}