% 电场的高斯定律
% 高斯定律|高斯定律|麦克斯韦方程组|库仑定律|闭合曲面

\pentry{电场\upref{Efield}, 散度 散度定理\upref{Divgnc}}

\footnote{本文参考 Wikipedia \href{https://en.wikipedia.org/wiki/Gauss's_law}{相关页面}.}\textbf{电场的高斯定律(Gauss's law for electric field)}是麦克斯韦方程组\upref{MWEq}中四条方程中的一条, 是经典电磁学的基本假设之一, 无法推导. 用电场线来形象地描述该定理, 就是正电荷会产生电场线, 负电荷会消除电场线. 产生和消除的电场线和电荷量成正比.

在静电学条件下, 即空间中净电荷不随时间变化,且电流恒定不变时, 高斯定律和库伦定律\upref{ClbFrc}是等价的. 然而在非静电学条件下, 高斯定律依然成立而库伦定律却不成立\footnote{作为一个反例, 我们在电动力学简介\upref{EM0}中已经讨论过: 若一个静止的电荷在原地快速抖动一下, 电磁场会以光速把这个扰动向周围传播, 所以远处的电荷并不能立即感受到电场的变化. 这显然超出了库伦定律描述的范围.}. 所以高斯定律是更普适的定理, 库仑定律可以看作是高斯定律在静电学条件下的特殊形式.

\subsubsection{积分形式}
在空间中任意选取一个闭合曲面 $\mathcal S$,电场在这个曲面上从内向外的通量等于被曲面包围的总电荷量除以真空中的介电常数.
\begin{equation}\label{EGauss_eq2}
\oint_{\mathcal S} \bvec E(\bvec r) \vdot \dd{\bvec s} = \frac{1}{\epsilon_0} \int_{\mathcal S} \rho(\bvec r) \dd{V}
\end{equation}
作为另一种形象的理解, 我们可以想象正电荷会以固定速率向周围释放或吸收一种不可压缩的流体, 负电荷则吸收这种流体. 释放和吸收的速率和电荷绝对值成正比, 电场可以看作该流体的速度场. 所以无论取什么形状的闭合曲面, 单位时间流经曲面的总流量都只取决于曲面内部的电荷总量.

\subsubsection{微分形式}
空间任意一点的电场散度等于电荷密度除以真空介电常数.
\begin{equation}\label{EGauss_eq1}
\div \bvec E = \frac{\rho}{\epsilon_0}
\end{equation}
根据散度定理\upref{Divgnc}, 该式和\autoref{EGauss_eq2} 是完全等价的.

\subsection{电通量}
如果把电场想像成是某种不可压缩液体的速度场(液体流动的速度矢量在空间上的分布),那么对于电场中假想的一个不闭合的空间曲面,电通量就相当于单位时间流过该曲面的体积.
类比流量和通量%未完成:引用
中得到的公式
\begin{equation}
\dv{V}{t} = \oint \bvec v \vdot \dd{\bvec s}
\end{equation} 
电通量为(符号为 $\Phi $, 国际单位 $\Si{N m^2/C}$)
\begin{equation}
\Phi  = \oint \bvec E \vdot \dd{\bvec s}
\end{equation} 
即电通量就是电场在所选曲面上的通量.

\subsection{高斯定律的积分形式}

高斯定律的积分形式说的是,选取任意闭合曲面为高斯面(由内向外为正方向),高斯面上的电通量等于高斯面内的总电荷除以常数.即
\begin{equation}\label{EGauss_eq5}
\oint \bvec E \vdot \dd{\bvec s}  = \frac{1}{\epsilon_0} \int \rho \dd{V}
\end{equation} 
其中常数 $\epsilon_0$ 为\textbf{真空中的电介质常数(vacuum permitivity)},又叫\textbf{真空中的电容率}(原因见 “电容\upref{Cpctor}”). 为了便于理解和记忆,可以把电场想像成某种流体的场,这种流体从正电荷流出,流入负电荷,流出和流入的速率(单位时间的体积)和电荷的大小成正比,比例系数为 $1/\epsilon_0$. 从数学上来讲,有电荷的地方电场就有\textbf{源}, $\rho /\epsilon_0$ 就叫做电场的\textbf{源密度}.

% 这里要举几个静电场应用高斯定律的例子
\begin{example}{点电荷和球壳的电场}\label{EGauss_exe1}
令点电荷的电荷量为 $Q$, 以点电荷为圆心选取半径为 $r$ 的球形高斯面,如\autoref{EGauss_fig5} . 
\begin{figure}[ht]
\centering
\includegraphics[width=3cm]{./figures/EGauss_5.pdf}
\caption{利用点电荷周围的高斯球面求电场} \label{EGauss_fig5}
\end{figure}
由系统的对称性可知, 高斯面上的电场大小必处处相等. 所以高斯面的电场通量(\autoref{EGauss_eq5} 左边)等于
\begin{equation}
\oint \bvec E \vdot \dd{\bvec s} = E \cdot 4\pi r^2
\end{equation}
\autoref{EGauss_eq5} 的右边等于 $Q/\epsilon_0$, 无需积分. 两边除以 $4\pi r^2$ 得高斯面上电场的大小为
\begin{equation}
E = \frac{Q}{4\pi\epsilon_0 r^2}
\end{equation}
与库仑定律的结论一致. 这说明高斯定律可以推导出库伦定律.

再来看均匀带电球壳产生的电场,如\autoref{EGauss_fig6} , 若在其外部作球形的高斯面, 得到的结论与点电荷一致, 但若在球内部作球形高斯面, 由于高斯面内没有电荷, 根据对称性可知球壳内部电场为零.
\begin{figure}[ht]
\centering
\includegraphics[width=5cm]{./figures/EGauss_6.pdf}
\caption{均匀带电球壳内外的电场} \label{EGauss_fig6}
\end{figure}
\end{example}


\begin{example}{带电平面和平行板电容器的电场}

空间中有一块无限大的厚度不计的均匀带电平面, 面电荷密度为 $\sigma$, 求空间中的电场.

% 图未完成
如\autoref{EGauss_fig7} , 取一个柱形的高斯面, 上下底面离平面的距离都为 $z$.
\begin{figure}[ht]
\centering
\includegraphics[width=6cm]{./figures/EGauss_7.pdf}
\caption{无限大均匀带电平面的电场} \label{EGauss_fig7}
\end{figure}
系统对称方式有:关于带电平面的镜面对称、旋转轴沿垂直于平面方向的旋转对称及二者的任意组合(乘积).空间产生的电场也应具有同样的对称性,显然这样的对称性要求距离带电平面相同位置的电场强度大小相同,方向垂直于平面指向平面外( $\sigma>0$ )或内( $\sigma<0$ ).所以所作高斯圆柱侧面的通量为0,而上下底面通量为 $ES$,故
\begin{equation}
\oint \bvec{E}\vdot \dd\bvec{s}=2ES=\frac{\sigma S}{\epsilon_0}
\end{equation}
所以
\begin{equation}
E=\frac{\sigma}{2\epsilon_0}
\end{equation}

%\addTODO{图, 推导}
\end{example}

\begin{example}{均匀带电球的电场}\label{EGauss_exe3}
\begin{figure}[ht]
\centering
\includegraphics[width=3cm]{./figures/EGauss_8.pdf}
\caption{均匀带电球内外的电场} \label{EGauss_fig8}
\end{figure}
令球的半径为 $R$, 电荷密度为 $\rho$, 令无穷远处为零势点, 求均匀带电球内外的电场分布. 

带点球的总电荷量为 $Q = 4\pi R^3\rho/3$, 用与\autoref{EGauss_exe1} 类似的方法可知球外电场与点电荷的电场相同. 再来看球内的电场, 以半径为 $r$ ($r < R$)作高斯面, 高斯面内的电荷为 $q = 4\pi r^3\rho/3$, 高斯面上的电场为
\begin{equation}
\bvec E = \frac{q}{4\pi\epsilon_0 r^2} \uvec r = \frac{\rho}{3\epsilon_0}\bvec r
\end{equation}
\begin{figure}[ht]
\centering
\includegraphics[width=3cm]{./figures/EGauss_9.pdf}
\caption{均匀带电球内高斯面作法} \label{EGauss_fig9}
\end{figure}
\end{example}

\begin{example}{无限长直导线}
令无限长直导线上线电荷密度为 $\lambda$, 以导线为轴线, $r_0$ 为半径作一个高为 $L$ 的圆柱体. 
\begin{figure}[ht]
\centering
\includegraphics[width=5cm]{./figures/EGauss_10.pdf}
\caption{无限长直导线周围的电场} \label{EGauss_fig10}
\end{figure}
系统的对称方式为:以直导线为旋转轴的旋转对称性和垂直于直导线的平面为反射面的镜面对称性,这要求在距离直导线相同距离的各点电场大小相同,方向沿径向指向内(负电荷)外(正电荷).所以所作高斯圆柱面上下底电场通量为0,而侧面通量为 $2\pi r_0L$.现在,高斯定律\autoref{EGauss_eq2} 变为
\begin{equation}
E\cdot2\pi r_0L=\frac{\lambda L}{\epsilon_0}\Rightarrow E=\frac{\lambda}{2\pi \epsilon_0r_0 }
\end{equation}

%\addTODO{图, 推导}
%\begin{equation}
%E=\frac{\lambda}{2 \pi \varepsilon_{0} a}
%\end{equation}
这与\autoref{Efield_ex1}~\upref{Efield} 的结论相同.
\end{example}

\subsection{高斯定律的微分形式}

根据矢量分析中的散度定理\upref{Divgnc}, 若对任意闭合曲面, 一个标量场在曲面内的体积分等于一个矢量场在曲面上的面积分, 该标量场就是该矢量场的散度. 把该结论用于上式,可得电场的散度为
\begin{equation}
\div \bvec E = \rho / \epsilon_0
\end{equation}
\addTODO{直接使用\autoref{Efield_eq9}~\upref{Efield} 证明微分形式}

\subsection{由库仑定律推导高斯定律(静电学)}
由\autoref{EGauss_exe1} 可知静电学情况下, 由高斯定律可以轻易得出库伦定律. 现在我们从库仑定律证明高斯定律. 这里先给出一个符合几何直觉的幼稚证明, 便于理解. 严谨的证明见 “电场的高斯定律证明\upref{EGausP}”.

首先只考虑高斯面内一个点电荷 $Q$ 对一个面元 $\dd{\bvec s}$ 的通量.若元 $\dd{\bvec s}$ 与相对于点电荷的位矢 $\bvec r$ 的夹角为 $\theta$, 通过该面元的电通量为
\begin{equation}
\dd{\Phi} = \bvec E \vdot \dd{\bvec s} = \frac{kQ}{r^2} \dd{s} \cdot \cos \theta 
\end{equation} 
下面证明保持以 $\dd{\bvec s}$ 为底面, $Q$ 为顶点的圆锥的任意
一个截面的电通量都是一样的(即\autoref{EGauss_fig3} 中 $\dd{\bvec s}$ 和 $\dd{\bvec s'}$ 的电通量一样).要证明这点,首先考虑两个与 $Q$ 距离相
同,但角度不同的截面(见\autoref{EGauss_fig4} ),其中 $\dd{\bvec s}$ 垂直于$\bvec r$. 它们的电通量分别为
\begin{figure}\label{EGauss_fig3}[ht]
\centering
\includegraphics[width=8cm]{./figures/EGauss_1.pdf}
\caption{图 1}
\end{figure}
\begin{figure}\label{EGauss_fig4}[ht]
\centering
\includegraphics[width=7cm]{./figures/EGauss_2.pdf}
\caption{图 2}
\end{figure}

\begin{equation}
\dd{\Phi_a} = E \cdot \dd{s_a} \cdot \cos \theta 
\end{equation} 
\begin{equation}
\dd{\Phi_b} = E \cdot \dd{s_b}
\end{equation}
然而显然有 $\dd{s_b} = \cos\theta \ \dd{s_a} $ (注意圆锥非常细长的时候,母线近似都平行),所以有
\begin{equation}
\dd{\Phi_a} = \dd{\Phi_b}
\end{equation} 
即,原地改变截面的角度,电通量不变.下面再考虑角度垂直但与电荷距离不同的情况(\autoref{EGauss_fig2} 中的 $\dd{s_1}$ 与 $\dd{s_2}$ ) 它们的电通量分别为
\begin{figure}\label{EGauss_fig2}[ht]
\centering
\includegraphics[width=7cm]{./figures/EGauss_3.pdf}
\caption{图 3}
\end{figure}

\begin{equation}
\dd{\Phi_1} = \frac{kQ}{r_1^2} \dd{s_1}
\end{equation}
\begin{equation}
\dd{\Phi_2} = \frac{kQ}{r_2^2} \dd{s_2}
\end{equation}
然而根据几何关系,有
\begin{equation}
\frac{\dd{s_1}}{r_1^2} = \frac{\dd{s_2}}{r_2^2}
\end{equation} 
所以仍然有.
\begin{equation}
\dd{\Phi_1} = \dd{\Phi_2}
\end{equation} 

若把整个闭合曲面划分成无数个小面元(如\autoref{EGauss_fig1} ),每个小面元 $\dd{\bvec s_i}$ 根据上面推理,都等效为半径 $R$ 上的一块垂直面元 $\dd{\bvec s_i}$. 这样,这些等效面元可以重新组成一个半径为 $R$ 的球体.而球体的通量为
\begin{figure}\label{EGauss_fig1}[ht]
\centering
\includegraphics[width=7cm]{./figures/EGauss_4.pdf}
\caption{图 4}
\end{figure}

\begin{equation}
\Phi  = \frac{kQ}{R^2} \cdot 4\pi R^2 = 4\pi kQ
\end{equation} 
令真空中的介电常数 $\epsilon_0 = 1/(4\pi k)$, 就得到高斯定律
\begin{equation}
\Phi  = Q/\epsilon_0 = \frac{1}{\epsilon_0} \int \rho \dd{V}
\end{equation} 
若高斯面内有许多点电荷或电荷分布,则根据电场叠加原理,对每个点电荷应用上述结论再相加即可.另外,若出现高斯面重叠的情况(如图 1 中的 $\dd{\bvec s_1}$,  $\dd{\bvec s_2}$,  $\dd{\bvec s_3}$ ),高斯定律仍然成立.这是因为 $\dd{\bvec s_1}$ 和 $\dd{\bvec s_2}$ 的电通量大小相等,然而方向相反,对总电通量的贡献抵消,只有剩下的 $\dd{\bvec s_3}$ 才对总电通量有贡献,这和不重叠的情况一样.

最后来证明高斯面外部的点电荷对高斯面的总电通量没有贡献.因为从高斯面外的点电荷(电场源)“流入”高斯面的电通量全部“流出”,这是由于从电荷出发的所有圆锥要么不被高斯面截断,要么被高斯面截断两次,一次进一次出,且两个界面上的电通量大小相同,总电通量贡献为零.
 