% 用 GDB 调试 C++ 程序
% GDB|c++|调试|断点

\subsection{程序调试}

对于 Matlab 或 Python 这样的动态语言, 由于它们的解释器(interpreter)执行代码的方式就是从代码中读一行执行一行, 所以对它们的来说, \textbf{调试(debug)}是一件很自然的事情, 当某个地方出错了, 解释器会告诉你出错的原因, 位置, 以及函数调用的顺序(也就是说如果错误出现在某个函数中, 解释器就会告诉你例如程序的第几行调用了 A 函数, A 函数中第几行调用了 B 函数, B 函数的第几行哪里出错了). 然而对于编译语言, 由于在编译过程中代码被转换成二进制命令, 程序在执行二进制命令时如果没有额外的调试信息, 则没有办法对应到代码的相应位置.

无论使用什么编程语言, 调试程序通常有两种方法, 一种叫做 \textbf{print debug}, 就是在代码中插入许多额外的输出命令, 运行的过程中在屏幕上不断输出信息, 如果哪里出错了, 我们就可以根据最后输出的信息对应到出错代码的位置. 第二种方法是使用某种 \textbf{debugger}, 例如下面要介绍的 GNU Debugger, 简称 gdb.

下面来分析两种方式的优劣. print debug 的劣势显而易见, 如果你的代码本来没有任何输出, 那么程序运行过程中崩溃的时候你将完全不知道发生了什么, 只能先在代码的一些关键地方插入一些输出命令, 重新编译, 复现错误, 把出错的位置缩小一些, 然后再在这个小范围进一步插入更详细的输出命令, 重新编译, 复现错误, 直到找到最深层的原因. 更糟糕的是, 如果一个函数被层层调用, 即使你知道这个函数中出错了, 也可能不知道是谁调用它的时候出错的. 如果你的程序有嵌套循环以及层层调用的函数, 那输出的调试信息可能

如果使用 debugger, 你不需要修改程序就能找到出错的位置. 

\subsection{简介}
如果你用过 Matlab 或者 Python 等动态语言, 你可能会对断电
