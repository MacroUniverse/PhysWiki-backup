% 电偶极子
% 偶极子|电场|电荷

% 这是一级词条, 先介绍两个点电荷

\pentry{电场\upref{Efield}}

\subsection{偶极子的电场}

令空间中两个位置不同的点电荷具有等量的异号电荷, 则他们构成一对\textbf{电偶极子(electric dipole)}. 令他们的电荷量分别为 $q_1$ 和 $q_2$ ($q_1 + q_2 = 0$), 位置矢量分别为 $\bvec r_1$ 和 $\bvec r_2$, 则它们的总电场为两个电荷各自电场的矢量和(见\autoref{Efield_eq2}~\upref{Efield})

\begin{equation}
\bvec E(\bvec r) = \frac{1}{4\pi\epsilon_0} \qty(\frac{q_1(\bvec r - \bvec r_1)}{\abs{\bvec r - \bvec r_1}^3} + \frac{q_2(\bvec r - \bvec r_2)}{\abs{\bvec r - \bvec r_2}^3})
\end{equation}
总电势同样是两个点电荷的电势之和% 链接未完成: N 个点电荷的电势
\begin{equation}
V(\bvec r) = \frac{1}{4\pi\epsilon_0} \qty(\frac{q_1}{\abs{\bvec r - \bvec r_1}} + \frac{q_2}{\abs{\bvec r - \bvec r_2}})
\end{equation}

我们常常讨论的是偶极子远处的电势和电场分布, 即 $r \gg \abs{\bvec r_2 - \bvec r_1}$ 的情况. 定义\textbf{电偶极矩(electric dipole moment)}为
\begin{equation}\label{eleDpl_eq1}
\bvec p = q_1 \bvec r_1 + q_2 \bvec r_2 = q_2 (\bvec r_2 - \bvec r_1)
\end{equation}
则电势分布为
\begin{equation}
V_d(\bvec r) = \frac{1}{4\pi\epsilon_0} \frac{\bvec p\vdot \bvec r}{r^3} = \frac{1}{4\pi\epsilon_0} \frac{\bvec p \vdot \uvec r}{r^2}
\end{equation}
由此可以求出远处的电场分布
\begin{equation}
\bvec E_d(\bvec r) = \frac{1}{4\pi\epsilon_0} \frac{1}{r^3} [3(\bvec p \vdot \uvec r) \uvec r - \bvec p]
\end{equation}
注意这两个量分别随 $r$ 的平方反比和三次方反比下降.

\subsection{偶极子远处电势推导}

\subsection{偶极子远处电场推导}

\begin{equation}
\bvec E_d(\bvec r) = \div V_d(\bvec r) = \frac{1}{4\pi\epsilon_0} \div  \frac{\bvec p\vdot \bvec r}{r^3}
\end{equation}
