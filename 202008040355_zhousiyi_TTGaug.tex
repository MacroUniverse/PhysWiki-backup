% TT规范
在源的外面,我们有$T_{\mu\nu} = 0$,所以我们有如下方程
\begin{equation}\label{TTGaug_eq1}
\Box \bar h_{\mu\nu} = 0 ~.
\end{equation}
其中$\Box = - (1/c^2) \partial_0^2 +\nabla^2$. 这个方程说明了引力波以光速传播。因为\autoref{Geomet_eq3}没有完全确定规范,在没有源的地方我们能够极大地固定规范。

如果$\Box \xi_\mu = 0$,那么变换$x^\mu \rightarrow x^\mu+\xi^\mu$保持$\partial^\nu \bar h_{\mu\nu}$不变。从$\Box \xi_\mu = 0$可以推出
\begin{equation}
\Box \xi_{\mu\nu} = 0~, \quad \xi_{\mu\nu} \equiv \partial_{\mu} \xi_\nu +\partial_\nu \xi_\mu - \eta_{\mu\nu} \partial_\rho\xi^\rho~. 
\end{equation}
我们可以使用TT规范
\begin{equation}
h^{0\mu} = 0~, \quad h^i_i = 0~, \quad \partial^j j_{ij} = 0~.
\end{equation}
加上洛伦兹条件把十个自由度的对称矩阵$h_{\mu\nu}$变为六个自由度。剩余的四个满足$\Box \xi_\mu = 0$的$\xi^\mu$把它降到了两个自由度。我们把TT规范下的度规记作$h_{ij}^{TT}$. 

注意,TT规范在含有源的时候是不能取的,这是因为$\Box \bar h_{\mu\nu} \neq 0$. \autoref{TTGaug_eq1}有如下的平面波解
\begin{equation}
h_{ij}^{TT} (x) = e_{ij} (\mathbf k) e^{ikx} ~.  
\end{equation}
其中$k^\mu = (\omega/c,\mathbf k)$, $\omega/c = |\mathbf k|$。

















