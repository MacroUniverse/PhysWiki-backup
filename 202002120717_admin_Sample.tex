% 词条示例
% 小时物理百科|词条编辑|latex|宏包

\pentry{二项式定理\upref{BiNor}}
\begin{figure}[ht]
\centering
\includegraphics[width=5cm]{./figures/Sample_2.png}
\caption{上传测试} \label{Sample_fig2}
\end{figure}
本词条需要与 LaTeX 源码对照阅读, 请使用\href{http://wuli.wiki/editor}{在线编辑器}中打开.

\subsection{正文}
蓝色的小标题通过 \lstinline|\subsection| 实现, 黑色的小小标题通过 \lstinline|\subsubsection| 实现.

正文必须使用中文的括号,逗号,引号,冒号,分号,问号,感叹号,以及全角实心句号\footnote{科技文献中常用实心句号, 参考中华人民共和国国家标准标点符号用法 GB\_T15834\_1995}. 禁止使用英文标点. 像 “牛顿—莱布尼兹公式” 中的横线必须用中文单破折号. 所有的标点符号前面不能有空格, 后面尽量有空格. 正文\textbf{粗体}用 \lstinline|\textbf|. 方便的办法是先全部使用中文标点, 最后再把所有空心句号替换成全角实心句号. 正文中禁止使用 \lstinline|\\| 换行, 以及 \lstinline|\noindent|, \lstinline|\phantom| 等命令强制修改格式.

\subsection{公式}
行内公式用单个美元符号,且两边要有空格,例如 $a^2+b^2=c^2$, 后面有标点符号的除外.

独立公式的 label 必须要按照 “词条标签\_eq编号” 的格式, 只有需要引用的公式才加标签, 标签编号无需和编译后的编号相同, 保证标签不重复即可. 图表的标签分别把 \lstinline|eq| 改成 \lstinline|fig| 和 \lstinline|tab| 即可,例题用 \lstinline|ex|, 习题用 \lstinline|exe|.但凡是有 \lstinline|\caption| 命令的,\lstinline|\label| 需要紧接其后. 公式严禁使用 MathType 等软件自动生成, 必须手打.
\begin{equation}\label{Sample_eq1}
(a+b)^n = \sum_{i=0}^n C_n^i a^i b^{n-i} \quad (\text{$n$ 为整数})
\end{equation}

公式中的空格从小到大如 $a\, b\; c\quad d\qquad e$, 注意大括号不可省略. 公式中三点省略号用 $\dots$, 如果要强制在下方, 用 $\ldots$. 实部和虚部如 $\Re[z], \Im[z]$. 双重极限如
\begin{equation}
\lim_{\substack{\Delta x_i\to 0\\ \Delta y_i\to 0}} \sum_{i, j} f(x_i,y_i) \Delta x_i \Delta y_j
\end{equation}
注意行内公式的 \lstinline|\lim| 和 \lstinline|\sum| 如果要在正上方或正下方写东西, 需要加 \lstinline|\limits|, 如 $\lim\limits_{x\to 0}$.

如果要强制分式正常大小显示, 用 \lstinline|\dfrac| 命令, 如果 \lstinline|\frac| 后面只有两个字符的代码,可以写成如 \lstinline|\frac12|, \lstinline|\frac ab|, \lstinline|\frac1a| 等. 斜分数线后面若多于一个变量需要加括号, 如 $ab/(cd)$.

行内分式如 $a/b$, 不允许行内用立体分式. 矢量如 $\bvec a$(被重新定义为黑体和正体), 尽量不要使用带箭头的矢量如 $\vec a$. 

行间公式换行及对齐用 aligned 环境, 注意该环境可嵌套.
\begin{equation}
\begin{aligned}
(a-b)^2 &= a^2+b^2 - 2ab \\
& = a^2+b^2+2ab-4ab\\
& = (a+b)^2-4ab
\end{aligned}
\end{equation}

用 \lstinline|\begin{enumerate}[resume]|  来继续上一个 enumerate 的编号

可变化尺寸的斜分数线如下
\begin{equation}
\left. \dv[2]{X}{x} \middle/ X + \dv[2]{Y}{y} \middle/ Y + \dv[2]{Z}{z} \middle/ Z  = \frac{1}{c^2}  \dv[2]{T}{t} \middle/ T\right.
\end{equation}
希腊字母如下
\begin{equation}
\begin{aligned}
&\alpha (a), \beta (b), \chi (c), \delta (d), \epsilon/\varepsilon (e), \phi (f), \gamma (g),
\eta (h), \iota (i), \varphi (j), \kappa (k), \lambda (l), \mu (m),\\
& \nu (n), o (o), \pi (p), \theta (q), \rho (r), \sigma (s), \tau (t), \upsilon (u), \varpi (v), \omega (w), \xi (x), \psi (y), \zeta (z)
\end{aligned}
\end{equation}
电介质常数一律用 $\epsilon$ 而不是 $\varepsilon$.

以下是 script 字母,只有大写有效.所谓大写 $\varepsilon$ 其实是花体的 $E$. 
\begin{equation}
\begin{aligned}
&\mathcal{A, B, C, D, E, F, G, H, I, J, K, L, M,}\\
&\mathcal{N, O, P, Q, R, S, T, U, V, W, X, Y, Z}
\end{aligned}
\end{equation}
以下是双线字母, 只有大写有效
\begin{equation}
\begin{aligned}
&\mathbb{A, B, C, D, E, F, G, H, I, J, K, L, M,}\\
&\mathbb{N, O, P, Q, R, S, T, U, V, W, X, Y, Z}
\end{aligned}
\end{equation}


\subsection{使用 physics 宏包}
目前仅支持 Physics 宏包的以下命令: 绝对值如 $\abs{a}$, 微分符号如 $\dd{x}$,

可变尺寸的小中大括号可以用 \lstinline|\qty|, 如
\begin{equation}
\qty(\frac ab)^2 \qquad \qty[\frac ab]^2 \qquad \qty{ \frac ab}
\end{equation}
矢量内积如 $\bvec A \vdot \bvec B$ (\lstinline|\vdot| 不可省略),矢量叉乘如 $\bvec A\cross\bvec B$.

常用三角函数和自然指数对数函数后面的小括号(中括号不可以!)会自动改变尺寸,若要给他们加幂,用中括号
\begin{equation}
\sin(\frac ab) \qquad \sin[2](\frac ab)
\end{equation}

导数和偏导可以用
\begin{equation}
\dv{x} \quad \dv{f}{x} \quad \dv[2]{f}{x} \quad \dv*[2]{f}{x} \quad
\pdv{x} \quad \pdv{f}{x} \quad \pdv[2]{f}{x} \quad \pdv{f}{x}{y} \quad \pdv*[2]{f}{x}
\end{equation}

定积分求值如 $\eval{x^2}_0^1$ (自动尺寸).

对易算符如 $\comm{\Q A}{\Q B}$ 或 $\comm*{\Q A}{\Q B}$, 前者自动尺寸, 后者强制小尺寸. 泊松括号如 $\pb{\frac12}{B}$ 和 $\pb*{\frac12}{B}$.

梯度散度旋度拉普拉斯如 $\grad V$,$\div\bvec A$, $\curl\bvec A$, $\laplacian V$.

狄拉克符号(加 * 号强制小尺寸)如 $\bra{a}, \ket{b}, \braket{a}, \braket{a}{b}, \mel{a}{Q}{b}$. 平均值(加 * 号强制小尺寸)如 $\ev*{Q}, \ev{Q}{\psi}$.

\subsection{自定义的命令}
本模板自定义的命令如下, 如果不使用, 要保证编译后效果相同.

引用公式和图表都统一使用 \lstinline|\autoref| 命令, 注意前面不加空格后面要加空格(后面是标点符号除外), 例如\autoref{Sample_eq1}. 如果要引用其他词条中的公式,可以引用 “其他词条\upref{Sample}” 的\autoref{Sample_eq1} 也可以用 “\autoref{Sample_eq1}\upref{Sample}”, 为了方便在纸质书上使用, 词条页码是不能忽略的.

正文中粗体用 \lstinline|\textbf|. 单位矢量如 $\uvec a$.

写量纲用 \lstinline|\Si{}| 命令, 如 $a = 100\Si{m/s^2}$, 这个命令只能出现在公式环境内.

 \lstinline|\sinc| 函数如 $\sinc x, \sinc(x), \sinc^2(x+y)$,但后面的括号没有自动尺寸.
 
 大于等于或小于等于必须用 $\leqslant, \geqslant$ 而不是 $\le, \ge$. 自然对数底如 $\E$, 复数如 $u+\I v$ 虚数单位不能用 $i$,复共轭如 $z\Cj$.
 
 矩阵 $\mat A$,转置 $\mat A \Tr$,厄米共轭用 $\mat A \Her$. 常见的几种矩阵括号如
\begin{equation}
\pmat{1&2\\3&4} \qquad
\vmat{1&2\\3&4} \qquad 
\bmat{1&2\\3&4} \qquad
\Bmat{1&2\\3&4} \qquad
\pmat{1&2\\3&4}\Tr \qquad
\pmat{1&2\\3&4}\Her
\end{equation}
行内的列矢量用行矢量的转置表示,如 $(1,2,3)\Tr$. 张量如 $\ten T$.

单独一个粗体的 $\nabla$ 用 $\Nabla$.

\begin{equation}
\begin{aligned}
k_1 &= f(y_n, t_n) 
& k_2 &= f \qty(y_n + h\frac{k_1}{2}, t_n + \frac h2 )\\
k_3 &= f \qty( y_n + h\frac{k_2}{2}, t_n + \frac h2 ) \qquad
& k_4 &= f(y_n + hk_3, t_n + h)
\end{aligned}
\end{equation}

左大括号用 \lstinline|cases| 环境, 如
\begin{equation}
\begin{cases}
d+e+f = \int \frac{a}{b} \dd{x} &(a > 0)\\
a+b = c &(b > 0)
\end{cases}
\end{equation}
但是注意这个环境里面的符号都是小尺寸的(与行内公式相同), 且只能用一次 \lstinline|&|. 如果需要使用大尺寸, 可以用自定义的 \lstinline|\leftgroup{}| 命令
\begin{equation}
\leftgroup{
&d+e+f = \int \frac{a}{b} \dd{x} &&(a > 0)\\
&a+b = c &&(b > 0)
}
\end{equation}
这相当于一个可变尺寸的 \lstinline|{| 加上 \lstinline|aligned| 环境
\begin{equation}
\left\{
\begin{aligned}
&d+e+f = \int \frac{a}{b} \dd{x} &&(a > 0)\\
&a+b = c &&(b > 0)
\end{aligned}
\right.
\end{equation}

表格中若用 \lstinline|\dfrac|, 需要在行首加上 \lstinline|\dfracH| 命令.% 未完成: 为什么?

量子力学算符如 $\Q a$(一般可以不加, 只有必要的时候加), 矢量量子算符如 $\Qv p$.

\subsection{图表}

现在来引用一张图片, 位图必须使用 png 格式, 矢量图(推荐)必须以 pdf 和 svg 格式\footnote{pdf 格式用于 pdf 版的百科, svg 用于网页显示}放在 figures 文件夹中. 文件名是词条 label 加图片序号, 即使只有一张图片也要加上 1. 生成 svg 时, 字体一律用 outline, 更多选项中保留 7 位小数, 如果矢量图中包含位图, 要把 Image Location 选成 embed. 代码中使用 pdf 图片.图片宽度一律用 cm 为单位.
\begin{figure}[ht]
\centering
\includegraphics[width=7cm]{./figures/Sample_1.pdf}
\caption{例图} \label{Sample_fig1}
\end{figure}
在\autoref{Sample_fig1} 中,label 只能放在 caption 的后面,否则编号会出错.由于图片是浮动的,避免使用“上图”,“下图”等词.

再来看一个表格,如\autoref{Sample_tab1}. 注意标签要放在 caption 后面.
\begin{table}[ht]
\centering
\caption{极限 $\E$ 数值验证}\label{Sample_tab1}
\begin{tabular}{|c|c|c|c|c|c|c|}
\hline
$x$ & ${10^{ - 1}}$ & ${10^{ - 2}}$ & ${10^{ - 3}}$ & ${10^{ - 4}}$ & ${10^{ - 5}}$ & ${10^{ - 6}}$ \\
\hline
$(1 + x)^{1/x}$ & 2.59374 & 2.70481 & 2.71692 & 2.71815 & 2.71827 & 2.71828 \\
\hline
\end{tabular}
\end{table}

定义如\autoref{Sample_def1}.
\begin{definition}{名称(可省略)}\label{Sample_def1}
 定义导数为
\begin{equation}
f'(x) = \lim_{h \to 0} \frac{f(x + h) - f(x)}{h}
\end{equation}
\end{definition}

引理如\autoref{Sample_lem1}.
\begin{lemma}{名称(可省略)}\label{Sample_lem1}
三角形内角和为 $\pi$
\end{lemma}

定理如\autoref{Sample_the1}.
\begin{theorem}{名称(可省略)}\label{Sample_the1}
内错角相等
\end{theorem}

推论如\autoref{Sample_cor1}
\begin{corollary}{名称(可省略)}\label{Sample_cor1}
1 + 2 = 3
\end{corollary}

例子如\autoref{Sample_ex1}. 
\begin{example}{名称(可省略)}\label{Sample_ex1}
在例子中,我们的字体可以自定义,包括公式的字号会保持与内容一致.
\begin{equation}
(a+b)^n = \sum_{i=0}^n C_n^i a^i b^{n-i} \quad (\text{$n$ 为整数})
\end{equation}
\end{example}

习题如\autoref{Sample_exe1}. 
\begin{exercise}{名称(可省略)}\label{Sample_exe1}
已知 $b$ 和 $c$, 求 $a^2 + b^2 = c^2$.
\end{exercise}

\subsection{代码}
在行内显示代码用 \lstinline|\lstinline| 命令. 注意这个命令比较特殊, 可以用任何两个相同的字符作为定界符, 如 \lstinline|\lstinline*some code*|, \lstinline|\lstinline Isome codeI| 都会显示为 \lstinline|some code|. 注意定界符必须是代码中没有的字符. 编辑器目前暂不支持 \lstinline|\lstinline{...}| 格式.

\subsubsection{显示  Command Window 中的代码}
显示  Command Window 中的代码用 lstlisting 环境
\begin{lstlisting}[language=MatlabCom]
s = 'abc'; % 一些评论
A =
     1     1    -1     1     3
     2     2    -2     1     7
     1     1     0     2     3
     2     2    -1     5     4
\end{lstlisting}

\subsubsection{显示 m 文件中的代码}

Matlab 代码文件(.m)中如果含有中文注释, 默认是 GB2312 编码, 而 LaTeX 用的是 UTF-8 编码, 直接 input 这些文件会产生乱码. 所以必须先把 m 文件转换成 UTF-8 编码. 推荐用 Visual Studio Code, 底部状态栏可以自动检测编码, 点击可转换. 转换完放在 codes 文件夹中. 用 \lstinline|\Code{}| 命令将代码导入正文. 较长的代码文件必须含文件名, 文件名需要反映代码的内容而不是与词条 label 同名.

一旦使用了 UTF-8 编码, 在 Matlab 中打开后中文注释会显示乱码, 但不影响运行. 编辑注释可以用 Visual Studio Code 打开. wuli.wiki 中提供的代码压缩包下载必须是 GB2312 编码.

如果一个图片的制作使用了代码, 则代码文件必须与图片同名同目录保存. 如果一个词条中的数据用到了一个代码但这个代码却不用出现在书中, 那么这个代码就以图片相同的方式命名并与词条的 .tex 文件保存在同一目录. 

\Code{sample}

C++ 代码可以使用 \lstinline|\Cpp{}| 命令(用法与 \lstinline|\Code{}| 相同)或者 \lstinline|\cpp{}| 命令, 但是要包含文件后缀名. 少量的 C++ 代码例如, 行内代码例如 \lstinline|abc_123|
\begin{lstlisting}[language=cpp]
template <class T, class T1, class T2,
MY_IF(is_scalar<T>() && is_scalar<T1>() && is_scalar<T2>())>
void Plus(T &v, const T1 &v1, const T2 &v2)
{ v = v1 + v2; }
\end{lstlisting}

其他不支持的代码也可以使用 lstlisting 环境, 但不能使用方括号(可以在前面用注释说明语言)
% language=julia
\begin{lstlisting}
a = rand(3, 3)
c = 1 + 2im
b = c * a
\end{lstlisting}


\subsection{文献引用}
每章都有一个独立的参考文献列表的词条, 需要在主文件中每章最后一个 \lstinline|\entry| 后面加入 \lstinline|\bibentry| 命令插入(见本词条后面). bibentry 的参数是 bibliographies 文件夹中的文件名, 每个文件与章节导航的 label 同名. 注意全书的词条标签不能有重复. 词条中引用文献格式如\cite{PhysWiki}\cite{PhysWikiEng}.

网址的超链接如\href{http://wuli.wiki}{本书网站}.
