% 乘积拓扑
\pentry{拓扑空间\upref{Topol},集合\upref{Set}}

\subsection{乘积拓扑的定义}
\subsubsection{有限维乘积拓扑}

给定两个拓扑空间$(X_1, \mathcal{T}_1)$和$(X_2, \mathcal{T}_2)$,那么我们可以在乘积集合$X_1\times X_2$中定义一个拓扑$(\mathcal{T_\times})$,其拓扑基为$\{O_1\cap O_2|O_1\in\mathcal{T_1}, O_2\in\mathcal{T_2}\}$.就是说,乘积集合的拓扑基,是两个空间中开集的笛卡尔积的集合.

由拓扑基生成拓扑的方式(取任意并),容易发现,如果上述$O_1$和$O_2$取的只是$X_1$和$X_2$空间的某两个拓扑基,也能得到一样的定义.

一般地,$N$个拓扑空间的集合做笛卡尔积,这个笛卡尔积集合上的\textbf{乘积拓扑}定义为:

\begin{definition}{有限维乘积拓扑}

设$(X_n, \mathcal{T}_n)$是若干拓扑空间,$n$取值范围为$[1, N]\cap\mathbb{Z}$.那么乘积空间$X_1\times X_2\times X_3\times\cdots\times X_N=\prod\limits_{n=1,2,\cdots,N}X_n$中的拓扑由拓扑基$\mathcal{B}$生成,其中$\mathcal{B}=\{\prod\limits_{n=1,2,\cdots,N}O_n|\forall n, O_n\in\mathcal{T}_n\}$.

\end{definition}

\subsubsection{任意维乘积拓扑}

如果用于进行笛卡尔积的拓扑空间数量大于等于$\aleph_0$,那么我们常用的乘积拓扑定义会和有限维情况的说法略有不同.在这里,我们使用\textbf{子基(sub-basis)}来定义乘积拓扑:

\begin{definition}{任意维乘积拓扑}

设$(X_\alpha, \mathcal{T}_\alpha)$是若干拓扑空间,$\alpha$不再是整数指标,而是用一个无穷集合$\Lambda$中的元素来表达的指标:$\alpha\in\Lambda$.这样的无穷集合$\Lambda$称为一个\textbf{指标集(set of indexes)}.

空间$\prod\limits_{\alpha\in\Lambda}X_\alpha$的

\end{definition}

