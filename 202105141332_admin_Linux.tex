% Linux 基础
% 编程|Linux|Ubuntu|操作系统|命令行

\begin{issues}
\issueTODO
\end{issues}

虽然大多数人更熟悉 Windows 系统, 但由于 Linux 系统的免费开源稳定, 许多科学计算软件都是主要支持 Linux, 而许多研究组的服务器和计算机集群也已 Linux 系统为主. 所以本书使用 Linux 作为编译和运行环境. 注意我们只介绍 Linux 的命令行(\textbf{terminal})操作, 不涉及任何图形界面. 另外, Linux 有不同的版本, 本书选用 Ubuntu.

\subsection{同时使用 Windows 和 Linux}
如果你的电脑已经有了 Windows 系统, 要使用 Linux 命令行可以有几种方式:
\begin{enumerate}
\item 远程链接 Linux 服务器
\item Windows Subsystem for Linux (WSL)\footnote{注意只有 Windows 10 支持 WSL} 或者 WSL2\footnote{WSL2 预计 2020 年发布.}
\item Cygwin 或者 MinGW
\item Docker
\item 安装虚拟机, 如 VirtualBox
\item 安装双系统
\end{enumerate}
其中前 4 种方案可以在使用 Windows 的同时使用 Linux 命令行(注意使用虚拟机需要占用较多资源, 可能导致电脑卡顿), 而双系统方案一次启动只能进入一个系统.

\subsection{Windows 远程软件}
在 Windows 下使用 Linux 的另一种方案就是远程连接到另一台装有 Linux 系统的电脑
\begin{itemize}
\item putty, kitty, MobaXterm 远程命令行
\item WinSCP 文件传输
\end{itemize}

\subsection{基本命令}

暂时先列出最基础的命令行命令, 请自行搜索学习
\addTODO{有待完善}
\begin{itemize}
\item \verb|pwd| 当前目录
\item \verb|ls| 查看某目录的文件(\verb|-v| 序号排序)
\item \verb|cd| 改变当前目录
\item \verb|mkdir| 创建目录 (\verb|-p| 创建多层)
\item \verb|rmdir| 删除目录
\item \verb|cp| 复制文件(夹)
\item \verb|mv| 移动文件(夹)
\item \verb|touch| 创建空文件或更改文件日期
\item \verb|rm| 删除文件或文件(夹)
\item \verb|man|, \verb|--help|
\item \verb|echo| 重复文字
\item \verb|>, <| 和 “|”
\item \verb|cat| 显示文本文件内容
\item \verb|vim| 编辑文本文件
\item \verb|sudo| 超级管理员权限
\item \verb|du|, \verb|df| 文件(夹)大小, 硬盘容量
\item \verb|top| 进程管理, 资源占用
\item \verb|free| 检查内存
\item \verb|tar|, \verb|zip| 压缩与解压
\item \verb|ln -s| 符号链接(类似 windows 的快捷方式)
\item \verb|uname| 系统信息
\item \verb|apt-get| 安装软件
\item \verb|chmod| 修改文件权限
\item \verb|chown| 修改文件所有者
\item \verb|hostname|, \verb|hostname -I| 本机名称, IP 地址
\item \verb|ping| 检查是否可以连接到网络地址
\item \verb|reboot| 重启
\item \verb|ssh| 远程命令行. \verb|ssh-keygen|, \verb|ssh-copy-id|
\item \verb|sshfs| 远程传文件
\item \verb|find| 搜索文件或目录 \verb|-exec| 对搜索结果执行命令
\item \verb|grep| 搜索字符串或文件内容
\item \verb|locate| 搜索计算机的文件
\item \verb|nohup| 不挂起, 即退出终端不会影响程序的运行
\item \verb|&| 后台运行命令
\end{itemize}

\subsection{快捷键}
\begin{itemize}
\item \verb|Ctrl+R| 可以搜索命令历史, 再次按下 `Ctrl+R` 可以跳到上一个匹配命令, 按回车运行, 按左右键编辑
\item \verb|Ctrl+C| 种植正在运行的程序
\item \verb|Ctrl+Z| 后台运行正在运行的程序
\end{itemize}

\subsection{进阶命令}
\addTODO{有待完善}
\begin{itemize}
\item \verb|sed|
\item \verb|dd| 硬盘读写, 如创建随机文件, 全 0 文件.
\item \verb|tree| 显示文件结构
\item \verb|sha1sum|,\verb|md5sum| 等指纹
\item \verb|rsync| (若文件中含有转义字符需要三个反斜杠 \verb|\\\|)
\item 环境变量
\item \verb|split| 切割文件
\item \verb|LVM|
\end{itemize}
