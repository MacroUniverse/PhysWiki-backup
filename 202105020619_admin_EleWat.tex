% 电路和水路的形象对比

\addTODO{图}

形象来说, 我们可以把电路中的导线类比为水平面上的水管, 管道中充满某种不可压缩的非粘稠流体, 即该流体不会因自身的摩擦损耗能量. 导线中自由电子的电荷可以类比流体的质量, 电流则类比液体单位时间流经管道横截面的质量, 我们姑且把后者叫做\textbf{水流}, 注意这区别于\textbf{速度}. 没有电阻的理想导线类比为内壁无摩擦的管道.

在该模型中, 电势可以理解为管道内某点处的绝对水压. 事实上把整个管道系统的绝对水压处处增加一个常数并不会影响水流动的方式和快慢, 所以我们主要关心的是管道两点之间的绝对压强之差, 正如在电路中我们关心的不是绝对电势而是电势差(电压).

我们可以认为一般情况下水流的\textbf{速度}极小, 正如一般情况电子在电路中移动的速度同样极小. 这使得在出现压强差的一瞬间水就可以从静止加速度某个速度. 如果管道中没有任何阻力, 那么可以认为在其两端加上压强差的一瞬间水流就会变为无穷大\footnote{这并不是数学意义上的无穷大, 物理中的无穷大一般是指远远大于一般讨论情况的数值.}.

\subsubsection{电阻}
在以上类比中, 电路中的电阻\upref{Resist}可以看成在管道中塞入一块海绵用于阻碍水流. 海绵越密对水的阻力就越大, 只有通过增加海绵两侧的压强差才能让水流变快. 这样, 欧姆定律\upref{Resist} 类比过来就可以理解为水流和海绵两侧的压强差成正比.

在电路中, 当电阻两端出现压强差, 移动的速度其实非常慢. 

\subsection{电源}
在该类比中, 电源可以看成一个水泵, 如果




下面我们会看到

某处离水平地面的高度, 而电压/电势差可以理解为管道不同两点的高度只差.
