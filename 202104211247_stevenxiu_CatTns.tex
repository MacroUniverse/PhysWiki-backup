% 张量的分类
% 张量|型张量|张量类型|张量指标|指标升降

\begin{issues}
\issueDraft
\end{issues}

\pentry{协变和逆变\upref{CoCon}}

协变和逆变\upref{CoCon}词条中提到,尽管张量是多个任意的线性空间 $V$ 之间的多重线性映射,我们通常只考虑用 $V$ 和其对偶空间 $V^*$ 定义的张量,这样只需要定义一个空间的基就可以得到所有空间的基了.

回忆张量\upref{Tensor}词条中所教的判断张量阶数的方法:看一共有多少线性空间参与映射,不过是作为自变量的一部分还是像的一部分.在协变和逆变\upref{CoCon}词条中我们介绍了定义一个张量时给线性空间分类的方法,即分为 $V$ 和 $V^*$,这样就可以进一步把张量分出类型来,比阶数更细致一些.

\begin{definition}{}
一个 $(m, n)$ 型张量是涉及了 $n$ 个逆变向量和 $m$ 个协变向量的 $(m+n)$ 阶张量.
\end{definition}

一个 $(m, n)$ 型张量可以把 $n$ 个逆变向量和 $m$ 个协变向量映射为一个数字,也可以是映射成低阶张量.比如说,如果 $h^{ab}_c$ 是一个 $(2, 1)$ 型张量,那么它乘以一个逆变向量 $v^c$,就得到一个 $(2, 0)$ 型张量 $g^{ab}=v^ch^{ab}_c$;同样,它乘以两个协变向量后就得到一个 $(0, 1)$ 型张量,也就是一个协变向量.









