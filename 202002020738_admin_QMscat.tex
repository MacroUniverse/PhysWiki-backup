% 量子散射
\pentry{散射\upref{Scater}, 概率流密度\upref{PrbJ}, 球坐标中的薛定谔方程\upref{RYTDSE}}

\footnote{本文参考 Physics of Atoms and Molecules, Bransden}我们讨论一个粒子在\textbf{中心势能} $V(r)$ 下的散射. 核心思路: 我们需要 $E > 0$ 的所有本征波函数(也叫散射态), 入射的初始波包可以由这些散射态叠加而成, 之后的时间演化就是这些散射态分别乘以 $\exp(-\I E t)$ 再叠加.

\subsection{散射截面}
在三维情况下, 每个能量 $E$ 都是无穷维简并的. 且根据不同的边界条件我们可以获得不同的正交归一基底. 常见的边界条件如平面波入射. 这是一种物理意义较强的选择, 因为如果一个无穷远处的入射波包具有很窄的能量带宽, 我们就可以把它近似看作是平面波.
\begin{equation}\label{QMscat_eq1}
\lim_{r \to \infty} \psi_{\bvec k}^{(+)}(\bvec r) = (2\pi)^{-3/2} \exp(\I \bvec k \vdot \bvec r) + f(k, \uvec r) \frac{\exp(\I k r)}{r}
\end{equation}
$f(k, \bvec r)$ 是\textbf{散射幅}.

正交归一条件为
\begin{equation}
\int \psi_{\bvec k'}^{(+)}(\bvec r)^* \psi_{\bvec k}^{(+)}(\bvec r) \dd[3]{r} = \delta(\bvec k - \bvec k')
\end{equation}


\textbf{微分截面(differential cross section)}是能量的函数, 与概率流密度的关系
\begin{equation}
\dv{\sigma}{\Omega} = \lim_{r\to\infty}\frac{\bvec j_{out} r^2}{\bvec j_{in}}
\end{equation}
总\textbf{散射截面(scattering cross section)}可以形象理解为挡住概率流密度 $\bvec j$ 的面积
\begin{equation}
\sigma = \int\dv{\sigma}{\Omega} \dd{\Omega}
\end{equation}

\begin{equation}
\bvec j = \frac{\hbar}{2m\I}\qty[\psi^*(\grad \psi) - (\grad \psi*)\psi]
\end{equation}
径向概率流密度
\begin{equation}
\bvec j \vdot \uvec r = \frac{\hbar}{2m\I} \qty(\psi^*\pdv{\psi}{r} - \pdv{\psi*}{r} \psi)
\end{equation}
可得微分截面与散射幅的关系为
\begin{equation}
\dv{\sigma}{\Omega} = \abs{f(k, \uvec r)}^2
\end{equation}
注意只有\autoref{QMscat_eq1} 中的第一项, 即入射平面波对散射截面没有贡献.

由概率守恒对散射幅的约束, 可以得出\textbf{光学定理 Optical Theorem}
\begin{equation}
\sigma = \frac{4\pi}{k} \Im[f(k, \uvec k)]
\end{equation}
理解: 球面波往其他方向发射出去的总概率流等于在 $\uvec k$ 方向抵消的概率流.

\subsection{分波展开}
除了\autoref{QMscat_eq1} 的边界条件外, 我们也可以找到符合另一种常用边界条件的正交归一散射态, 每个能量同样由无穷维简并
\begin{equation}
\lim_{r\to 0} \psi(\bvec r) = 0
\end{equation}
球坐标中的解具有形式
\begin{equation}
\psi(\bvec r) = \frac{1}{r} \sum_l u_l(r) Y_{l,0}(\uvec r)
\end{equation}
其中每一项就是一个分波, 满足上述边界条件. 在实际计算中我们一般使用球坐标. 因为在中心势能 $V(r)$ 下, 薛定谔方程可以在球坐标中分离变量, 使解偏微分方程变为解常微分方程, 即径向方程
\begin{equation}
-\frac{\hbar^2}{2m} \dv[2]{\psi_{l,m}}{r} + \qty[V(r) + \frac{\hbar^2}{2m}\frac{l(l + 1)}{r^2}]\psi_{l,m} = \I \pdv{t} \psi_{l,m}
\end{equation}

\subsection{相移}
如果 $V(r)$ 在 $r\to\infty$ 时可以忽略, $u_l(r)$ 得渐进表达式就完全可以由相移 $\delta_l(k)$ 来描述. 由于微分截面也是在无穷远处定义的, 我们可以直接由相移计算微分截面.

我们定义相移满足
\begin{equation}
u_l(k, r) \to \sin[kr - \frac{l\pi}{2} + \delta_l(k)]
\end{equation}
为什么这么定义呢? 因为相移是相对的, 需要一个基准径向函数. 而我们把自由粒子, 即把 $V(r) \equiv 0$ 时的相位作为基准
\begin{equation}
u_l(k, r) = kr j_r(kr) \to \sin[kr - \frac{l\pi}{2}]
\end{equation}
定义 $S$ 矩阵为
\begin{equation}
S_l(k) = \exp[2\I\delta_l(k)]
\end{equation}
定义 $K$ 矩阵为
\begin{equation}
K_l(k) = \tan \delta_l(k)
\end{equation}
二者关系为
\begin{equation}
S_l(k) = \frac{1 + \I K_l(k)}{1 - \I K_l(k)}
\end{equation}
