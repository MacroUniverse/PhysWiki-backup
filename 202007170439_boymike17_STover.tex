% 弦论概述
弦论的基本思想是:组成物质的基本单元不是粒子,而是弦.弦有开弦和闭弦两种,形状如下图所示:
\begin{figure}[ht]
\centering
\includegraphics[width=5cm]{./figures/STover_1.png}
\caption{开弦} \label{STover_fig1}
\end{figure}
\begin{figure}[ht]
\centering
\includegraphics[width=5cm]{./figures/STover_2.png}
\caption{闭弦} \label{STover_fig2}
\end{figure}
弦的激发给出了各种不同的粒子.粒子在时空中运动的轨迹叫做世界线(world line).弦在时空中运动的轨迹叫做世界面(world sheet).世界面由两个参数$\sigma$和$\tau$来参数化.函数$x^\mu(\tau,\sigma)$ 把世界面上的坐标$\sigma$和$\tau$映射到时空坐标$x$上.

