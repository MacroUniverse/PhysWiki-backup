% 欧几里得环
% Euclid Ring|Euclid环|整环|多项式|因式分解|欧几里得整环|Euclid Domain

\pentry{真因子树\upref{FctTre}}

\begin{definition}{欧几里得环}
给定整环$R$,如果存在映射$\delta:R\to\mathbb{Z}^+\cup\{0\}$\footnote{即给每一个环中元素赋予一个非负整数.},使得对于任意$a, b\in R$且$b\not=0$,都存在$q, r\in R$使得$a=qb+r$,且$\delta(r)<\delta(b)$,则称$R$为一个\textbf{欧几里得环(Euclid Ring)},或者\textbf{欧几里得整环(Euclid Domain)}.
\end{definition}

直观来说,欧几里得环就是“可以做辗转相除法的环”.这从定义就可以看出来:任取$a$,再用任意的$b$去尝试除以它,总能得到$qb+r$的形式,其中$r$相当于除法的余数.虽然任意环中的元素都可以这么做分解,而欧几里得环就特殊在它还关系到一个非负整数赋值,使得余数的赋值总是小于非零除数$b$的,这样就使得我们可以多次分解,也就是进行辗转相除.

为了严格证明以上说法成立,我们还需要讨论欧几里得环的一个性质:

\begin{theorem}{}\label{EuRing_the1}
设$R$是一个欧几里得环,那么对于$r\in R$,有:

$r=0\iff \delta(r)=0$.
\end{theorem}

\textbf{证明}:

任选$a\in R$且$a\not=0$,那么$a=1a+0$,故由欧几里得环的定义可推知,对于任意$a\in R-\{0\}$,必有$\delta(0)<\delta(a)$,由此得证.

\textbf{证毕}.

有了



