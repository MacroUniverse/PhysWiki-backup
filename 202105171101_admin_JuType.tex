% Julia 的数据类型

\subsection{变量类型}

\begin{itemize}
\item Julia 是动态类型的语言. 具体类型不能互相作为子类, 只能作为抽象类的子类.
\item Julia 中所有的值都是对象. C++ 中 float, double 等不是对象, 只有 struct, class 定义的才是对象. Julia 中的对象没有成员函数.
\item 可以在 literal、 变量、 表达式、 函数定义后面加上 \verb|::变量类型| 用于确认它具有该类型, 如果类型不符会产生异常. 如果 \verb|变量类型| 是抽象的, 那么表达式只需要是它的一个 sub type.
\item 只有 “值” 有类型, 而不是变量具有类型. 变量只是个名字而已
\item 抽象和具体的类都可以使用其他类作为参数
\end{itemize}

常见类型
\verb|Int8|, \verb|UInt8|, \verb|Int16|, \verb|UInt16|, \verb|Int32|, \verb|UInt32|, \verb|Int64|, \verb|UInt64|, \verb|Int128|, \verb|UInt128|, \verb|Float16|, \verb|Float32|, \verb|Float64|, \verb|ComplexF16|, \verb|ComplexF32| 和 \verb|ComplexF64|

\verb|BigInt| 是任意大小的整数, \verb|BigFloat| 是任意精度浮点数.

\begin{lstlisting}[language=julia]
julia> BigFloat(2.1) # 2.1 here is a Float64
2.100000000000000088817...

julia> BigFloat("2.1") # the closest BigFloat to 2.1
2.099999999999999999999999999...99999999999999999999986

julia> BigFloat("2.1", RoundUp)
2.100000000000000000000000000...00000000000000000000021

julia> BigFloat("2.1", RoundUp, precision=128)
2.100000000...00000000000007
\end{lstlisting}

使用 \verb|typeof(var)| 来询问变量类型, 用 \verb|isa(var, type)| 来确定类型.

\subsection{抽象类型}
声明抽象类型
\begin{lstlisting}[language=julia]
abstract type 子类名 <: 母类名 end
\end{lstlisting}
例子
\begin{lstlisting}[language=julia]
abstract type Number end
abstract type Real     <: Number end
abstract type AbstractFloat <: Real end
abstract type Integer  <: Real end
abstract type Signed   <: Integer end
abstract type Unsigned <: Integer end
\end{lstlisting}
最高级的抽象类是 \verb|Any|, 这里的 \verb|Number| 就是 \verb|Any| 的子类.

\begin{figure}[ht]
\centering
\includegraphics[width=13cm]{./figures/JuType_1.png}
\caption{Julia 自带类型结构} \label{JuType_fig1}
\end{figure}
以及
\begin{lstlisting}[language=julia]
Complex{T<:Real} <: Number
Rational{T<:Integer} <: Real
Irrational{sym} <: AbstractIrrational
\end{lstlisting}


\subsection{原始类型}
\textbf{原始类型(primitive type)}是有 bit 组成的具体类型. Julia 可以自定义原始类型. 定义原始类型的格式为
\begin{lstlisting}[language=julia]
primitive type 类名 比特数 end
primitive type 类名 <: 父类名 比特数 end
\end{lstlisting}
例如
\begin{lstlisting}[language=julia]
primitive type Float16 <: AbstractFloat 16 end
primitive type Float32 <: AbstractFloat 32 end
primitive type Float64 <: AbstractFloat 64 end

primitive type Bool <: Integer 8 end
primitive type Char <: AbstractChar 32 end

primitive type Int8    <: Signed   8 end
primitive type UInt8   <: Unsigned 8 end
primitive type Int16   <: Signed   16 end
primitive type UInt16  <: Unsigned 16 end
primitive type Int32   <: Signed   32 end
primitive type UInt32  <: Unsigned 32 end
primitive type Int64   <: Signed   64 end
primitive type UInt64  <: Unsigned 64 end
primitive type Int128  <: Signed   128 end
primitive type UInt128 <: Unsigned 128 end
\end{lstlisting}

\subsection{复合类型}
\textbf{复合类型(composite)} 类似于 C++ 中的 \verb|struct|
\begin{lstlisting}[language=julia]
struct Foo
           bar # 抽象类型是 Any
           baz::Int
           qux::Float64
       end
\end{lstlisting}
创建该类型的对象
\begin{lstlisting}[language=julia]
foo = Foo("Hello, world.", 23, 1.5) # 叫做 constructor
\end{lstlisting}
查看成员名称用 \verb|fieldnames(Foo)|

要获取成员的值, 用 \verb|foo.bar|, \verb|foo.baz|, \verb|foo.qux| 等. 用 \verb|struct| 声明的复合类型值在生成后就不能被改变. 如果 struct 里面有矩阵等, 那么矩阵元仍然是 mutable 的(没有 lower level const).

如果要支持成员改变, 那么用 \verb|mutable struct| 来声明即可. mutable struct 通常存在 heap 中而不是在 stack 中.

\subsection{Type Unions}
\begin{lstlisting}[language=julia]
IntOrString = Union{Int,AbstractString}
1 :: IntOrString
"Hello!" :: IntOrString
1.0 :: IntOrString # 异常
\end{lstlisting}
但是, 这并不能看出和使用 \verb|<:| 有什么区别.

一个应用的例子是 \verb|Union{T, Nothing}| 作为函数参数, 这样这个参数就可以忽略了.

\subsection{含参类型}

复数类型 \verb|ComplexF16|, \verb|ComplexF32| 和 \verb|ComplexF64| 是 \verb|Complex{Float16}|, \verb|Complex{Float32}| 和 \verb|Complex{Float64}| 的别名. \verb|Complex| 就是一个\textbf{含参类型(parametric type)}

例如
\begin{lstlisting}[language=julia]
julia> struct Point{T}
           x::T
           y::T
       end
\end{lstlisting}
那么 \verb|Point{Float64}| 等都是合法的具体类型.

\verb|Point| 本身是一个抽象类, 是其具体类的母类
\begin{lstlisting}[language=julia]
julia> Point{Float64} <: Point
true
julia> Point{AbstractString} <: Point
true
\end{lstlisting}
\verb|<:| 相当于一个二元算符, 输出 \verb|Bool|.

例子:
\begin{lstlisting}[language=julia]
function norm(p::Point{Real})
    sqrt(p.x^2 + p.y^2)
end
\end{lstlisting}
constructor
\begin{lstlisting}[language=julia]
p = Point{Float64}(1.0, 2.0)
\end{lstlisting}

\subsection{含参抽象类型}
\textbf{含参抽象类型(Parametric Abstract Types)}

