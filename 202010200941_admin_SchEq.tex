% 定态薛定谔方程
% 量子力学|波函数|薛定谔方程|哈密顿|束缚态

\begin{issues}
\issueTODO
\issueAbstract
\end{issues}

\pentry{量子力学基本假设\upref{QMPos}, 矢量算符\upref{VecOp}}

\subsection{定态薛定谔方程}
定态薛定谔方程就是哈密顿算符的本征方程, 本征值就是能量 $E$.
\begin{equation}
H \psi(\bvec r) = E \psi(\bvec r)
\end{equation}

单个粒子问题中, 哈密顿算符对应粒子的总能量, 总能量算符可以表示为动能算符和势能算符之和
\begin{equation}
H = T + V
\end{equation}

\subsection{一维定态薛定谔方程}
一维运动的单个质点, 波函数是坐标 $x$ 的函数 $\Psi(x)$
\begin{equation}
T = -\frac{\hbar^2}{2m} \dv[2]{x} \qquad V = V(x)
\end{equation}
所以定态薛定谔方程为
\begin{equation}\label{SchEq_eq1}
-\frac{\hbar^2}{2m} \dv[2]{\Psi}{x} + V(x)\Psi = E \Psi
\end{equation}

\subsubsection{束缚态}
这是一个二阶线性常微分方程. 数学上来看无论 $E$ 是多少, 必有两个线性无关的解, 它们的线性组合也是解(二维解空间). 然而物理上可能存在的波函数必须要可归一化: 即随着 $\abs{x} \to \infty$,  $\abs{\psi}^2$ 在极限 $\abs{x}\to\infty$ 的过程中下降得比 $1/x$ 要快. 可以证明只有对于某些离散的 $E$ 我们才能解出这些波函数. 我们把每个 $E_i$ 叫做\textbf{能级(energy level)}, 对应的波函数叫做\textbf{束缚态(bound state)}.

所有的能级都处于能量区间
\begin{equation}\label{SchEq_eq2}
\min V(x) < E_i < \min V(\pm\infty)
\end{equation}
其中 $\min V(x)$ 是函数 $V(x)$ 的最小值(可以是负无穷), $\min V(\pm\infty)$ 表示 $V(x)$ 的正无穷极限和负无穷极限中较小的一个. 因为当 $E$ 太小时, 波函数必定会在无穷远处爆炸(无穷大), 而当 $E$ 太大时, 波函数虽然不会爆炸但也不会趋于零. \autoref{SchEq_eq2}, 要求势能函数 $V(x)$ 中存在某种形状的凹陷, 又称为\textbf{势阱(potential well)}.

例子: 无限深势阱\upref{ISW},有限深势阱\upref{FSW}, 简谐振子(升降算符)\upref{QSHOop}.

\subsubsection{散射态}
当 $E > \min V(\pm\infty)$ 时, 虽然波函数不满足归一化, 但它们仍然有重要的应用. 我们把它们叫做\textbf{散射态(scattering states)}. 计算散射态时, 通常我们要求两个极限 $V(\pm \infty)$ 都存在. 这样在满足无穷远处 $E > V$ 的方向, 波函数在无穷远处将会是简谐波. 最直接的例子是平面波函数. % 链接未完成

虽然散射态本身不能归一化, 但是它们的线性组合却可以. 例如 $V(x) \equiv 0$ 势能中的高斯波包\upref{GausPk} 就可以看作由平面波线性叠加而来, 即反傅里叶变换.

某种意义上, 给定一个势能函数, 所有的束缚态和散射态可以构成一组正交归一的函数基底, 用于展开所有可归一化的波函数.

\subsubsection{简并}
简并是指一个能量本征值 $E$ 对应多个线性无关的归一化波函数的情况. 可以证明一维束缚态不存在简并, 散射态存在双重简并.

\subsection{多维定态薛定谔方程}
二维或三维的情况下, 波函数是位置矢量\upref{Disp}的函数 $\Psi(\bvec r)$
\begin{equation}
T = -\frac{\hbar^2}{2m} \laplacian \qquad V = V(\bvec r)
\end{equation}
定态薛定谔方程为
\begin{equation}
-\frac{\hbar^2}{2m} \laplacian {\Psi} + V(\bvec r)\Psi = E \Psi
\end{equation}
例子: 三维简谐振子(球坐标)\upref{SHOSph}.

\addTODO{为什么束缚态必须是实函数(或者乘以一个相位因子)? 为什么实函数的平均动量必须等于 0(实函数的傅里叶变换对对称函数)? 物理上意味着什么? 如果势能是偶函数,那么波函数为什么一定是基函数或者偶函数?这可以通过哈密顿算符和宇称算符的对易性来解释.}

\subsection{波函数的对称性}
\pentry{宇称算符\upref{Parity}}
若哈密顿算符和宇称算符 $\Pi$ 对易, 则它们具有一组共同的本征波函数, 其中每个都具有奇宇称或者偶宇称. 例如对于一维定态薛定谔方程, 若势能函数是偶函数, 即 $V(x) = V(-x)$, 则波函数必定是奇函数或者偶函数, 即
\begin{equation}\label{SchEq_eq3}
\psi(-x) = \pm \psi(x)
\end{equation}
