% 巴拿赫定理
\pentry{巴拿赫空间\upref{banach}}

在巴拿赫空间的理论中, \textbf{巴拿赫定理(Banach theorems)}是一组关于巴拿赫空间上的有界线性算子的定理. 它们都依赖于巴拿赫空间的完备性, 并且在分析数学中有广泛的应用. 可参考 Kôsaku Yosida: Functional analysis. Grundlehren der mathematischen Wissenschaften 123, Springer-Verlag, 1980 (6th ed.).

\subsection{一致有界原理 (共鸣定理)}
\begin{theorem}{一致有界原理 (共鸣定理, Uniform Boundedness Principle)}
设 $X$ 是巴拿赫空间, $Y$是赋范线性空间, $\mathfrak{F}\subset\mathfrak{B}(X,Y)$是有界线性算子的族. 如果对于任何$x\in X$皆有$\sup_{T\in\mathfrak{F}}\|Tx\|_Y<\infty$, 那么实际上必有
$$
\sup_{T\in\mathfrak{F}}\|T\|_{\mathfrak{B}(X,Y)}
=\sup_{T\in\mathfrak{F}}\left(\sup_{\|x\|_X\leq 1}\|Tx\|_Y\right)<\infty.
$$
也就是说, 从巴拿赫空间出发的线性算子的族, 如果是逐点有界的, 则一定是一致有界的. 
\end{theorem}
这个定理的证明用到了贝尔纲定理(Baire category theorem), 而这依赖于$X$的完备性. 如果$X$不完备, 则定理不成立. 一个反例如下: 取$X$为仅有有限项非零的序列组成的空间, 并赋以范数$\|x\|=\max_{k}|x_k|$. 这个空间不是完备的. 定义$T_n:X\to X$为
$$
T_nx=(x_1,2x_2,...,nx_n,0,...),
$$
则$\|T_n\|=n$, 因此族$\{T_n\}$不是一致有界的, 但对于任何固定的 $x\in X$, 当$n$充分大时, $T_nx$将恒等于常值, 因此对于任何固定的$x\in X$, 都有$\sup_n\|T_nx\|<\infty$.

该定理的一个应用是说明: 连续函数的傅里叶级数不必逐点收敛. 

\subsection{开映像原理}
\begin{theorem}{开映像原理 (Open Mapping Theorem)}
设$X,Y$是巴拿赫空间, $T:X\to Y$是连续线性算子, 而且是满映射. 则$T$是开映射, 即$X$中的开集经$T$变为$Y$中的开集.
\end{theorem}

这个定理的证明也用到了贝尔纲定理 (Baire category theorem), 而这依赖于$X,Y$的完备性. 如果$X$不完备, 则定理不成立. 实际上有很简单的反例可以作为说明: 如果$Z\subset X$是稠密的真子空间, 那么自然的嵌入映射$Z\to X$就显然不是开映射.

开映像原理有如下的直接推论: 如果$T$是巴拿赫空间$X,Y$之间的连续线性算子, 且值域$\text{Ran}(T)$是$Y$的闭子空间, 则商映射$X/\text{Ker}(T)\to \text{Ran}(T)$是同构. 另外, 开映像原理还说明: 一个线性空间上的两个范数若都是完备的, 则这两个范数必然等价.

\subsection{闭图像定理}
\begin{theorem}{闭图像定理 (Closed Graph Theorem)}
设$T:X\to Y$是巴拿赫空间之间的闭算子. 则$T$是连续的.
\end{theorem}
这个定理实际上可作为开映像原理的推论: 如果$\text{G}(T)\subset X\times Y$是闭子空间, 则$X$在图范数$[|x|]_T=\|x\|_X+\|Tx\|_Y$下成为巴拿赫空间. 但巴拿赫空间之间的映射$I:(X,[|\cdot|]_T)\to(X,\|\cdot\|_X),\,I(x)=x$是连续的单满射, 从而根据开映像原理它也是开映射, 因此其逆映射是连续的. 这表示$T$是有界算子.
