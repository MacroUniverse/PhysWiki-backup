% 数据结构:带对角矩阵
% BLAS|CBLAS|矩阵|数据结构|科学计算|线性代数

\pentry{密矩阵\upref{MatSto}}

在 CBLAS 中, 我们可以用\textbf{带对角矩阵(band diagonal matrix)}的数据结构来储存矩阵, 减少不必要的零元素的储存和运算. 和密矩阵一样, 带对角矩阵同样分为行主序和列主序两种.

要把一个密矩阵转换成列主序带对角矩阵, 就把每个对角线都排成一行, 并保持每个元素的列标不变即可. 同理, 要转换成行主序带对角矩阵, 就把每个对角线都排成一列, 保持每个元素的行标不变.

举一个例子, 若列主序密矩阵(\autoref{BanDmt_tab1})中只有几个对角线不为零, 当这样的矩阵很大时, 就会在左下角和右上角出现大量的矩阵元为零,不仅占用内存, 矩阵运算时也效率不高. 为了方便我们不妨把不为零的矩阵元依次编号\footnote{这并不是标准的编号方式, 只是这里为了讲解临时使用的.}.

列主序带对角矩阵如\autoref{BanDmt_tab2}, 列主序带对角矩阵如\autoref{BanDmt_tab3}.

\begin{table}[ht]
\centering
\caption{密矩阵}\label{BanDmt_tab1}
\begin{tabular}{|c|c|c|c|c|c|}
\hline
01  & 04  &    &    &    &   \\
\hline
02  & 05  & 08  &    &    &   \\
\hline
03  & 06  & 09  & 12  &    &   \\
\hline
   & 07  & 10 &  13  & 15  &   \\
\hline
   &    & 11 &  14  & 16  & 17 \\
\hline
\end{tabular}
\end{table}

\begin{table}[ht]
\centering
\caption{列主序带对角矩阵}\label{BanDmt_tab2}
\begin{tabular}{|c|c|c|c|c|c|}
\hline
   & 04  & 08  & 12 &  15 &  17 \\
\hline
01  & 05  & 09  & 13 &  16 &    \\
\hline
02  & 06  & 10 & 14 &     &    \\
\hline
03  & 07  & 11 &    &     &    \\
\hline
\end{tabular}
\end{table}

\begin{table}[ht]
\centering
\caption{行主序带对角矩阵}\label{BanDmt_tab3}
\begin{tabular}{|c|c|c|c|c|c|}
\hline
   &     &  01  &  04 \\
\hline
   &  02  &  05  &  08 \\
\hline
03  &  06  &  09  &  12 \\
\hline
07  &  10 &  13 &  15 \\
\hline
11 &  14 &  16 &  17 \\
\hline
\end{tabular}
\end{table}

\subsection{索引转换}

令\autoref{BanDmt_tab1} 到\autoref{BanDmt_tab3} 中矩阵的行数和列数分别为 $N_1, N_2$, 行标分别为 $i, i_c, i_r$, 列标分别为 $j, j_c, j_r$. 单索引分别为 $k, k_c, k_r$. 再令 $i_{diag}$ 为\autoref{BanDmt_tab2} 中对角线所在的行标, $j_{diag}$ 为\autoref{BanDmt_tab3} 中对角线所在的列标, 那么有以下转换公式

\begin{equation}
\leftgroup{
&i_c = i-j + i_{diag}\\
&j_c = j\\
&k_c = i_c + N_1 j_c = i + i_{diag} + (N_1-1) j
}\end{equation}

\begin{equation}
\leftgroup{
&i_r = i\\
&j_r = j-i + j_{diag}\\
&k_r = N_2 i + j = (N_2-1) i + j + j_{diag}
}\end{equation}
