% 自由群
\pentry{群同态\upref{Group2}}
%基本完成

\subsection{字母和词}

\begin{definition}{字母}
没有赋予任何运算关系的元素,被称为一个\textbf{字母(letter)}.
\end{definition}

给定没有赋予任何运算结构的任意集合$S$,它的每个元素都可以作为一个字母.

\begin{example}{字母的例子}
\begin{itemize}
\item 随便取一个元素,取名为$x$,那么$x$可以看成一个字母.
\item 取$52$个英文字母构成的集合,则每个元素都可以当作字母.
\end{itemize}
\end{example}

字母就是纯粹的符号,除了“指代一个元素”以外没有更多的结构.没有更多结构当然没有研究的意义不过,所以我们考虑拓展字母的用途,比如把若干字母按顺序排列起来,得到新的元素.这种元素不同于字母,我们把它叫做“词”.

\begin{definition}{词}
有限个字按照一定顺序排列构成的元素,称作一个\textbf{词(word)}或\textbf{字}.在计算机理论中,也称为\textbf{字符串}.
\end{definition}

\begin{example}{词的例子}
\begin{itemize}
\item 单元素集合的元素$x$,可以构成词$x$,$xx$,$xxxxxxx$等.
\item $52$个英文字母构成集合$S$,那么$S$中的元素$l, e, t, r$都是字母,它们可以进行有限排列,得到词$letter$.
\end{itemize}

\end{example}

单个字母也可以看成是特殊的词,这样一来,就可以把字的排列拓展成词之间的一种\textbf{运算}.两个词首尾相连,可以构成一个更大的排列,仍然是有限个字的排列,因此得到的还是词.

\begin{definition}{词的运算:首尾相连}
给定集合$S$.用$S$中的元素作为字母所排出的词,构成一个“词集合”,记为$W(S)$.在集合$W(S)$上可以定义一个运算“$\cdot$”如下:设$a, b\in W(S)$,那么$a\cdot b=c$,其中$c$是$a$和$b$首尾相连的结果.
\end{definition}

\begin{example}{词的运算}
还是用英文字母构成的集合$S$.$lov$是一个词,$ely$也是一个词,$lov\cdot ely=lovely$是这两个词进行首尾相连运算的结果.
\end{example}

我们也常常省略运算符号“$\cdot$”,这时$lov$和$ely$的连接运算就直接写成$lovely$了,和运算的结果形式上一样,非常简练.

\subsection{自由生成群}

我们希望改进一下词集合和词运算,构造出一个群.为了做到这一点,我们首先需要扩展一下词集合.

\begin{definition}{字母的逆}
给定集合$S$.给$S$中的每一个元素$x$都赋予一个“逆字母”,记为$x^{-1}$.
\end{definition}

字母的逆是用来满足群的“逆元存在性”的.如果一个词中出现某个字母和对应的逆字母相连接,那么它们就必须被“消除”.比如说,$lsdd^{-1}l^{-1}m^{-1}m$被认为和$lsl^{-1}$是相同的.特别地,对于任何字母$x, y$,我们认为$xx^{-1}$,$x^{-1}x$,$yy^{-1}$,$y^{-1}y$都是同一个词,称为\textbf{空词(empty word)}或\textbf{空字}.

有了这个规则,我们就可以构造出一个群了:

\begin{theorem}{自由生成群}
给定集合$S$,用它的元素作为字母.$S$中的全体字母和它们的逆字母可构成词,并且构词过程中把相邻的互逆字母消除,这样得到的词所构成的集合记为$F(S)$.在$F(S)$中定义运算为词的首尾相连,那么$F(S)$配合该运算构成一个群,称为\textbf{由集合$S$生成的自由群(free group)},或\textbf{自由生成群}.
\end{theorem}

自由群的单位元就是\textbf{空词}.一个词的字母都取逆以后反序排列,就得到了这个词的逆元.比如说,$le^{-1}tter$的逆元素就是$r^{-1}e^{-1}t^{-1}t^{-1}r^{-1}el^{-1}$.

自由群是结构复杂度最高的群,这是因为所有的群都可以看成某个自由群的商群,而商群是把原来的群中一些细节特征忽略掉的结果.这个结论是由以下定理保证的:

\begin{theorem}{自由群的一般性质}\label{FreGrp_the1}
给定任意的集合$S$和群$G$.如果$f:S\rightarrow G$是两个集合间的任意映射,那么总可以把$f$拓展成群同态$\varphi:F(S)\rightarrow G$,并且这种拓展是唯一的.
\end{theorem}

证明见\autoref{GroupP_ex4}~\upref{GroupP}.

考虑集合$G$到群$G$上的映射$f:f(g)=g, \forall g\in G$,那么由$f$拓展而来的同态$\varphi$就是$F(G)$到$G$的满射.结合\textbf{群同态基本定理}(\autoref{Group2_exe1}~\upref{Group2})可推知,这意味着$G$是$f(G)$的商群.

