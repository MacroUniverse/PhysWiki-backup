% 原子的观念

在《生活大爆炸》的第3季第10集中,莱纳德邀请伯纳黛特参观他正在做的验证AB效应(全称是阿哈朗诺夫-玻姆效应)的实验.佩妮不想落于人后,也想在聚餐的时候谈论莱纳德的实验,于是央求谢尔顿教她物理.

但物理是没法速成的,要讲就得从古希腊开始.

“假想在一个炎热的夏季的夜晚,你刚刚在阿戈拉(集市)买完东西,\dots\dots”

“但,这与莱纳德的研究有什么关系呢?”

谢尔顿的回答是:“科学是个2600年的旅程,从古希腊开始,经由牛顿,到玻尔(旧量子论),然后薛定谔(波动力学),到哥本哈根学派”,最后我们才能谈论莱纳德的实验.

我们也采用类似的路径,首先让我们回到2600年前雅典的阿戈拉,一个炎热的夏季夜晚,那里正有人在发表关于原子的理论.

\begin{figure}[ht]
\centering
\includegraphics[width=10cm]{./figures/AtomId_1.png}
\caption{公元前2世纪雅典鸟瞰图} \label{AtomId_fig1}
\end{figure}

\subsection{柏拉图的四元素说}

关于物质的理论,自古就有,比如古希腊的泰勒斯曾说“万物皆水”,后来又有人说万物是四种元素“水、气、土、火”构成.这就是所谓四元素说.

四元素说本来是古希腊人的常识(common sense),但柏拉图给这种关于物质的学说理论化,系统化了.这些内容被记载在柏拉图(427 BC — 347 BC)的宇宙论《蒂迈欧篇》中.

这里我们可以给出一个论证的概要:

\begin{itemize}
\item 万物或者是可见的,或者是可以触摸的.可见是因为光,可以触摸是因为坚硬.光是火的性质,坚硬是土的性质,这样我们就有了“火和土”两种元素.

\item 万物是三维的,我们需要像“胶水”一样的元素把“火和土”按比例混合起来成为三维的物体. 这里柏拉图是通过构造如下数列来论证的:
\begin{equation}
1=1^2=1^3, 2, 4=2^2, 8=2^3, 16=4^2, 32, 64=4^3,...
\end{equation}
这里$1=1^3$,$8=2^3$,$64=4^3$,...,叫做立方数. 每两个立方数之间正好是两个数,比如1和8之间是2和4;而8和64之间是16和32. 柏拉图因此论证说需要两种元素在“火和土”之间调和使生成万物,这两种元素就是水和气.

\item 不论是可见,还是可以触摸,都可以归结为形状,而形状可以归结为多边形的拼合,多边形则可以归结为三角形的拼合. 三角形是研究形的基础,或说三角形是研究几何学的基础.

\item 柏拉图提出了两种基本的直角三角形:(1)等腰直角三角形,记做$T_{45}$;(2)一个锐角为$30^o$的直角三角形,记做$T_{30}$.

\item 利用两种基本的直角三角形可以拼成四种正多面体,即正四面体、正八面体、正六面体和正二十面体.
\end{itemize}

\begin{figure}[ht]
\centering
\includegraphics[width=8cm]{./figures/AtomId_2.png}
\caption{五种正多面体,也称柏拉图多面体.这里面的正十二面体没有出现在对元素的构造中,这是柏拉图理论中欠缺美感的地方.} \label{AtomId_fig2}
\end{figure}

\begin{enumerate}
\item 柏拉图认为立方体,即正六面体最稳定,因此把土的形定为正六面体.
\item 正四面体有最锐利的尖角,因此是最活跃的.火的形就是正四面体.
\item 剩下的还有气和水.因为气比水活跃,因此气的形就是正八面体.
\item 还剩下最后一个,正二十面体,是水的形.
\end{enumerate}

这些形都是很小的,用今天的话说就是“水气土火”的原子.并且这些原子还可以相互转化.

\begin{table}[ht]
\centering
\caption{柏拉图的“元素说”}\label{AtomId_tab1}
\begin{tabular}{|c|c|c|c|}
\hline
元素 & 正$n$面体 & 等腰直角三角形 & $30^o$直角三角形 \\
\hline
土 & 正六面体 & 12 & 0  \\
\hline
火 & 正四面体 & 0 & 8 \\
\hline
气 & 正八面体 & 0 & 16 \\
\hline
水 & 正二十面体 & 0 & 40\\
\hline
\end{tabular}
\end{table}

\begin{itemize}
\item 只有土是由等腰直角三角形围成的,因此土最稳定,它会被火溶解,可以被气或水分解,但不会转变成其它元素.

\item “火、气、水”都是由$30^o$的直角三角形围成的,因此可相互转换.比如,我们可以写出如下的反应方程式:
\end{itemize}

\begin{equation}
1 Water \to 2 Gas + 1 Fire 
\end{equation}

反应前有40个$T_{30}$,而反应后有$2 \times 16 + 8 = 40$个$T_{30}$.我们可以把上式与电解水的反应方程式比较:

\begin{equation}
2 H_2 O \to 2 H_2 \uparrow + O_2 \uparrow
\end{equation}

从思维的角度这是属于同一母型(Prototype)的.如果我们说古代思想会对近代科学有启迪作用,或我们说人总是在几种思维母型里打转转就毫不奇怪了.实际上量子力学的创建者,比如海森堡自幼就熟读《蒂迈欧篇》,如果说这些这些直观的图像会对他有潜移默化的影响可能并不夸张.

以上是柏拉图元素论的概要,当然这些都是非常粗糙的论证,而且经不起实验的定量检测,但元素之间可以互相转化,这个想法还是启迪了后来的炼金术,并最终导致了近代化学的出现.

\subsection{伊壁鸠鲁的原子论}

原子在古希腊语中不可再分的意思,这个不可再分固可以做物理上的拆分理解,亦可作逻辑上的不可再分(析)理解.比如刚才介绍的柏拉图关于“水气火土”四元素的理论就是逻辑上无法再分析的范例.

柏拉图的理论在古代是显学,2000多年来一直有稳定的传承,但它在古代并非没有对手,比如从德谟克利特、留基伯到伊壁鸠鲁、卢克莱修的原子论.但说实话这两种理论区别并不大,他们争执不休,以至于柏拉图都准备带弟子去焚烧德谟克利特的著作,并非是他们对自然真有什么本质的看法不同,毕竟都是一个时代对自然的看法.

关键的分歧是他们对伦理学和政治学的观点不一样,而古人的学问是个整体,伦理学的基础是哲学和科学,要驳倒对方,争取听众,釜底抽薪,攻击对方的科学观点是一个潜在的选择.

古代原子论者是今天所说的唯物论者,他们不相信灵魂,把生命看做是一堆原子具有功能性、活性或协调一致运动的集合.就好像是一把调谐良好的里拉琴,只要弦不断就能奏出美妙的音乐,对应于人的生命状态,而弦断则对应人亡,灵魂在这里是无所栖身的.没有灵魂,自然就没有死后世界,所有传统的道德说理就被架空了,Polis将处于危机之中.柏拉图派对唯物论者(或自然哲学家)的反感和敌对就在这里.

当然这并非我们现在的主题,我也不再继续展开讨论,而仅仅强调一点,即当我们讲到古代原子论的时候,我们应把柏拉图派的很多观点、理论也置于古代原子论的范畴内,而不仅仅是介绍自然哲学家的原子论.实际上很多柏拉图派的理念,比如天球的理念和近代原子模型是很接近的.对玻尔、海森堡这些哲学倾向很强的物理学家,很难想象他们没有阅读过《理想国》和《蒂迈欧篇》,而如果读过的话,他们一定对柏拉图的“数学-几何学”的宇宙模型印象深刻.

下面我将扼要地介绍伊壁鸠鲁的原子论,伊壁鸠鲁(341 BC — 270 BC)的《致希罗多德书信》是现存最早的关于原子论的记录,再早的比如德谟克利特和留基伯的就都已经失传了.

\begin{itemize}
\item 在视觉上存在最小的点,再小我们就看不见了,这个视觉上的最小的点是有大小的. 由此类比:物质也是由最小的不可再分的最小单元构成的,再小是不存在的.就这一点而言,伊壁鸠鲁的方案和柏拉图的方案是不同的,柏拉图的基础三角形没有明确地说存在最小、最基础的三角形.今天的人大多会认为物质存在最小的不可再分的单元是很抽象的,或很难想象,但在古人那里也许未必.比如在中国古代是通过计量“肥而美”的“黑小米”来建立度量衡制度的,而在古代西方也有类似的比喻,比如卢克莱修在《物性论》中把水想象为大量罂粟籽的集合,比如阿基米德在《数沙者》中假想整个宇宙都充满了沙子,然后计算了整个宇宙中有多少粒沙子.古人的数学观念是由“1,2,3,\dots\dots”出发开始建立的,对他们来说想象一个由原子构成的世界其实是更简单的.这个原子就好比构成宇宙、万物的基础砖石,我们先寻求生命的结构和功能,进而问生命的可能的意义是什么?

\item 原子很小,因为我们谁也没见过原子.但我们没法假设原子的大小是绝对的0,因为如果那样的话,很多原子就无法构成万物了,而万物在我们的眼里是有体积的.

\item 原子如果有大小,那它就有部分,有形状.所以这里我们说的“不可再分”其实是物理意义下的,因为从几何学的角度,原子既然有部分,有形状,那它就可以在想象中继续分解.
\end{itemize}
