% 爱因斯坦求和约定
% 爱因斯坦求和|线性代数|求和符号|下标|上标

\begin{issues}
\issueTODO
\issueOther{本词条需要重新创作和整合,融入章节逻辑体系.}
\end{issues}



爱因斯坦求和约定是一种物理学中常用的表达方式,用于简化线性代数的表示.

\subsection{定义和例子}

用一句话来总结爱因斯坦求和约定,就是:\textbf{当式子中任何一个角标出现了两次,并且一次是上标、一次是下标时,那么该式表示的实际上是对这个角标一切可能值的求和}.换言之,如果角标$i$作为上标和下标各出现了一次,那么式子相当于添加了一个关于$i$的求和符号.

我们举例来说明:

\begin{example}{线性函数}
从张量\upref{Tensor}中我们知道,一个$1$-线性函数可以表示为一个向量,这样的向量常被称为\textbf{余向量}、\textbf{补向量}或者\textbf{$1$-形式}.通常,我们用下标来表示一个余向量的各分量:$\bvec{\alpha}=(\alpha_1, \alpha_2, \alpha_3)$;而用上标来表示一个通常的几何向量:$\bvec{v}=(v^1, v^2, v^3)$.注意,上标不是乘方.

$\bvec\alpha$和$\bvec v$的内积是$$\sum\limits_{i=1, 2, 3}\alpha_i v^i$$
用爱因斯坦求和约定,$\bvec\alpha$和$\bvec v$的内积就可以写为$\alpha_i v^i$.
\end{example}

\begin{example}{矩阵运算}
对于矩阵$\bvec{A}$,我们把其第$i$行第$j$列的元素表示为$A^i_j$.
\begin{itemize}
\item 矩阵乘法表示为:如果$\bvec{A}=\bvec{B}\bvec{C}$,那么$A^i_j=B^i_k C^k_j$.
\item 矩阵$\bvec{A}$的迹为$A^i_i$.

\end{itemize}
\end{example}

由于重复出现而实际上应该是求和的指标,被称为\textbf{赝指标}或者\textbf{哑指标(dummy index)},因为它们不是真正的指标,而是可以用任意字母代替的.没有求和的指标是固定的,是真正的指标.比如说,$B^i_k C^k_j$中$k$可以是任何字母,但是$i$和$j$是不可以替换成别的字母的,因为它们由$A^i_j$决定了.在这里,哑指标实际上是表示遍历全部可能的真指标.

求和的哑指标不一定是遍历$1, 2, 3$,也可能是对更多或者更少的指标求和,甚至是无穷集合.

\subsection{若干约定俗成的表示}

矩阵$\bvec{A}$的第$i$行$j$列的元素记为$A^i_j$.

向量的分量用上标表示,余向量的分量用下标表示.

在四维时空中,$x^0$表示时间坐标$t$,而$x^1$,$x^2$和$x^3$分别表示空间坐标$x$,$y$和$z$.




