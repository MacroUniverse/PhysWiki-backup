% 亥姆霍兹定理
% 散度|旋度|调和场|亥姆霍兹

参考 \cite{GriffE}

任何矢量场都可以分解为一个无旋场 $\bvec F_{div}$ 和一个无散场 $\bvec F_{curl}$, 也可以选择性地添加一个调和场 $\bvec F_{harm}$.
\begin{equation}
\bvec F = \bvec F_{div} + \bvec F_{curl} + \bvec F_{harm}
\end{equation}
无旋场总能表示为某个势能函数的梯度, 而无散场总能表示为另一个场的散度, 所以
\begin{equation}
\bvec F_{div} = \grad V
\end{equation}
\begin{equation}
\bvec F_{curl} = \curl \bvec A
\end{equation}
调和场既没有散度也没有旋度, 所以可以合并到前两项中任意一个中. 所以亥姆霍兹分解可以记为
\begin{equation}
\bvec F = \grad V + \curl \bvec A
\end{equation}

\begin{equation}
\bvec F = \int \frac{(\curl \bvec F)\cross \bvec R}{4\pi R^3} \dd[3]{r'} + \int \frac{(\div \bvec F)\bvec R}{4\pi R^3} \dd[3]{r'} + \bvec h(\bvec r)
\end{equation}

其中 $\bvec h(\bvec r)$ 是调和场.
