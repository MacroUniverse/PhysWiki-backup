% 经典场论基础
% 经典场


\subsection{拉格朗日场论}
这一节里面,我们复习一下经典场的知识,为后面的量子场论做铺垫.首先要复习的一个重要的量就是拉式量了,定义如下
\begin{equation}
S = \int L dt = \int \mathcal L(\phi,\partial_\mu \phi)d^4 x
\end{equation}
经典场论的重要原理是变分原理$\delta S = 0$.
\begin{equation}
\begin{aligned}
0 &=\delta S \\
&=\int d^{4} x\left\{\frac{\partial \mathcal{L}}{\partial \phi} \delta \phi+\frac{\partial \mathcal{L}}{\partial\left(\partial_{\mu} \phi\right)} \delta\left(\partial_{\mu} \phi\right)\right\} \\
&=\int d^{4} x\left\{\frac{\partial \mathcal{L}}{\partial \phi} \delta \phi-\partial_{\mu}\left(\frac{\partial \mathcal{L}}{\partial\left(\partial_{\mu} \phi\right)}\right) \delta \phi+\partial_{\mu}\left(\frac{\partial \mathcal{L}}{\partial\left(\partial_{\mu} \phi\right)} \delta \phi\right)\right\}
\end{aligned}
\end{equation} 
最后一项是一个表面项,这里我们考虑边界条件是$\delta \phi$为零的构型,这一项就可以忽略.现在我们看前两项.因为对于任意的$\delta \phi$这个式子都为零,所以我们必须让$\delta \phi$前面的系数为零,这样,我们就推出了著名的欧拉-拉格朗日方程
\begin{equation}
\partial_\mu \bigg( \frac{\partial \mathcal L}{\partial(\partial_\mu\phi)} \bigg) - \frac{\partial \mathcal L}{\partial \phi} = 0 
\end{equation}

\subsection{哈密顿场论}
拉式量的方法的优点是所有的量都是明显洛仑兹不变的.哈密顿场论的优点是更容易过度到量子力学.

对于一个分立系统,我们可以定义共轭动量
\begin{definition}{共轭动量}
对于每个动力学变量$q$,我们可以定义它的相应的共轭动量
\begin{equation}
p \equiv \frac{\partial L}{\partial \dot q}
\end{equation}
\end{definition}
那么哈密顿量的定义如下
\begin{definition}{哈密顿量}
\begin{equation}
H \equiv \sum p \dot q - L
\end{equation}
\end{definition}
上面的定义也可以推广到连续系统.只要假设空间坐标$\mathbf x$是分立的就可以了,这样对于连续系统,我们可以定义如下的共轭动量
\begin{definition}{连续系统的共轭动量}
\begin{equation}
\begin{aligned}
p(\mathbf x) & \equiv \frac{\partial L}{\partial \dot \phi(\mathbf x)} = \frac{\partial}{\partial \dot \phi(\mathbf x)} \int \mathcal L(\phi(\mathbf y),\dot \phi(\mathbf y)) d^3 y \\
& \sim \frac{\partial}{\partial \dot \phi(\mathbf x)} \sum_{\mathbf y} \mathcal L(\phi(\mathbf y,\dot \phi(\mathbf y))) d^3 y=\pi(\mathbf x) d^3 x
\end{aligned}
\end{equation}
其中
\begin{equation}
\pi(\mathbf x) \equiv \frac{\partial \mathcal L}{\partial \dot \phi(\mathbf x)}
\end{equation}
是与$\phi(\mathbf x)$共轭的哈密顿量密度.
\end{definition}
因此哈密顿量为
\begin{equation}
H = \int d^3 x\,\, p(\mathbf x) \dot \phi(\mathbf x) - L
\end{equation}
现在我们来看一个简单的例子.
\begin{align}\nonumber
\mathcal L & = \frac{1}{2} \dot \phi^2 - \frac{1}{2} (\nabla \phi)^2 - \frac{1}{2} m^2 \phi^2 \\
& = \frac{1}{2} (\partial_\mu\phi)^2 - \frac{1}{2} m^2 \phi^2
\end{align}
根据这个拉式量可以写出运动方程
\begin{equation}
\bigg( \frac{\partial^2}{\partial t^2} - \nabla^2 +m^2 \bigg)\phi = 0~,\quad (\partial^\mu\partial_\mu+m^2)\phi = 0
\end{equation}
这就是克莱因戈登方程.这个标量场对应的哈密顿量为
\begin{equation}
H =  \int d^3x \mathcal H = \int d^3 x \bigg[ \frac{1}{2} \pi^2 + \frac{1}{2} (\nabla \phi)^2 + \frac{1}{2} m^2 \phi^2 \bigg] 
\end{equation} 

\subsection{诺特定理}
\begin{theorem}{诺特定理}
每个\textbf{连续对称性}都有着\textbf{相应的守恒定律}.
\begin{itemize}
\item 物理系统的\textbf{空间平移不变性}(物理定律不随着空间中的位置而变化)给出了\textbf{动量守恒}律;
\item \textbf{转动不变性}给出了\textbf{角动量守恒}律;
\item \textbf{时间平移不变性}给出了\textbf{能量守恒}定律.
\end{itemize}
\end{theorem}
现在考虑标量场$\phi$的无穷小变换
\begin{equation}
\phi(x) \rightarrow \phi'(x) +\alpha \Delta \phi (x)
\end{equation}
这里$\alpha$是一个无穷小参数,$\Delta \phi$是场的变化.如果这个变换\textbf{令$\phi$场的运动方程保持不变}的话,我们就把这个变换称为一个\textbf{对称性}.因为拉式量的不变性总是跟运动方程的不变性相联系的,所以我们也可以说,如果这个变换令拉式量保持不变的话,我们就说.













