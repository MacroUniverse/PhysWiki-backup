% 自然单位制
% 单位制|原子单位制|转换常数

\begin{issues}
\issueDraft
\end{issues}

\pentry{原子单位制\upref{AU}}

\footnote{参考 Wikipedia \href{https://en.wikipedia.org/wiki/Natural_units}{相关页面}.}在高能物理和场论中, 我们往往使用一套单位制, 使得普朗克常数 $\hbar = 1$ 以及真空光速 $c = 1$, 下面我们对这种单位制进行说明.

在 “原子单位制\upref{AU}” 的第一节中, 为了使 $\beta_L = \hbar$ (也就是所谓的 “令 $\hbar = 1$”), 我们得到(\autoref{AU_eq6}~\upref{AU})
\begin{equation}\label{NatUni_eq1}
\beta_t = \frac{\beta_m \beta_x^2}{\hbar}
\end{equation}
令动能公式 $E_k = mv^2/2$ 成立, 有(\autoref{AU_eq7}~\upref{AU})
\begin{equation}
\beta_E = \frac{\beta_m \beta_x^2}{\beta_t^2} = \frac{\hbar}{\beta_t}
\end{equation}

现在为了让 $c = 1$, 即规定速度的转换常数为光速 $\beta_v = c$. 如果我们希望满足 $x = vt$, 那么必须有 $\beta_x = \beta_v \beta _t$. 将\autoref{NatUni_eq1} 代入得
\begin{equation}
\beta_x \beta_m = \frac{\hbar}{c}
\end{equation}
此时以上出现的转换常数中只剩一个自由度. 最后令做功公式 $E = Fs$ 成立, 令万有引力公式为(即 “令引力常数 $G = 1$”)
\begin{equation}
F = \frac{m_1 m_2}{r^2}
\end{equation}
得
\begin{equation}
\beta_E\beta_x = \beta_m^2
\end{equation}

\begin{table}[ht]
\caption{自然单位转换常数表}\label{NatUni_tab1}
\begin{tabular}{|c|c|c|c|}
\hline
物理量 & $\beta$ & 描述 & 数值(国际单位)\\
\hline
质量 $m$ & $\sqrt{\hbar c} \Si{kg^{-1/2} m^{-3/2} s}$ & - & $1.778068270201313\e{-13}$ \\
\hline
\dfracH 长度 $x$ & $\dfrac{1}{c}\sqrt{\dfrac{\hbar}{c}} \Si{m^{1/2}kg^{-1/2}}$ &  1 光秒 & $299792458$ \\
\hline
\dfracH 速度 $v$ & $c$ & 光速 & $299792458$ \\
\hline
时间 $t$ & $\dfrac{1}{c^2}\sqrt{\dfrac{\hbar}{c}} \Si{m^{3/2} kg^{-1/2} s^{-1}}$ & 1 秒 & $6.599124520025956\e{-39}$\\
\hline
\dfracH 能量 $E$ & - & - & $1.054571817646156\e{-34}$ \\
\hline
角动量 $L$ & - & - & $1.054571817646156\e{-34}$ \\
\hline
\end{tabular}
\end{table}

电磁学有不同的定义
(以下未完成)
\begin{table}[ht]
\caption{电磁学}\label{NatUni_tab2}
\begin{tabular}{|c|c|c|c|}
\hline
电荷 $q$ & $e$ 或 $q_e$ & 电子电荷 & $1.6021766208\e{-19}$\\
\hline
\dfracH 电场强度 $\mathcal{E}$ & $\dfrac{e}{4\pi \epsilon_0 a_0^2}$ & 基态轨道电场强度 & $5.1422067070\e{11}$ \\
\hline
\dfracH 磁感应强度 $B$ & $\dfrac{\hbar}{ea_0^2}$ &  & $2.350517567\e5$\\
\hline
\dfracH 电势 $V$ & $\dfrac{e}{4\pi\epsilon_0 a_0}$ & 基态轨道电势 & $27.211386019$ \\
\hline
\end{tabular}
\end{table}
