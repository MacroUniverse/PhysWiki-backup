% 抛物线坐标系

\begin{issues}
\issueTODO
\end{issues}

\pentry{抛物线的三种定义\upref{Para3}}

抛物线的极坐标方程为(\autoref{Para3_eq1}~\upref{Para3})
\begin{equation}
r = \frac{\xi}{1 - \cos \theta }
\end{equation}
若选用不同的半通径 $\xi$ ($\xi > 0$), 将得到一系列缩放的抛物线(\autoref{ParaCr_fig1} 中的绿色). 我们也可以把这些抛物线旋转 180 度, 得到
\begin{equation}
r = \frac{\eta}{1 + \cos \theta }
\end{equation}
这时我们把半通径记为 $\eta$ ($\eta > 0$), 当它取不同的值也得到一系列抛物线(\autoref{ParaCr_fig1} 中的红色). 这样, 通过 $\xi, \eta$ 两个坐标, 我们就能确定平面上的唯一一点.

\begin{figure}[ht]
\centering
\includegraphics[width=10cm]{./figures/ParaCr_1.pdf}
\caption{抛物线坐标系, 极轴向上(来自 Wikipedia)} \label{ParaCr_fig1}
\end{figure}

若把这些曲线绕极轴旋转一周变为一系列曲面, 那么我们只需要再指定一个方位角 $\phi$ 就可以用坐标 $\xi, \eta, \phi$ 确定空间中的任意一点.

\subsection{与直角坐标和球坐标的转换}
一般令极轴与直角坐标的 $z$ 轴重合, 则有
\begin{equation}
\xi = r(1 + \cos\theta) = \sqrt{x^2 + y^2 + z^2} + z
\end{equation}
\begin{equation}
\eta = r(1 - \cos\theta) = \sqrt{x^2 + y^2 + z^2} - z
\end{equation}
\begin{equation}
\phi = \Arctan(y, x)
\end{equation}

\subsection{矢量算符}
\begin{equation}
\laplacian = \frac{4}{\xi + \eta} \qty[\pdv{\xi}\qty(\xi\pdv{\xi}) + \pdv{\eta}\qty(\eta\pdv{\eta})] + \frac{1}{\xi\eta}\pdv[2]{\phi}
\end{equation}
