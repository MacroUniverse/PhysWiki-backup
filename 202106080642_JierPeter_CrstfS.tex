% Christoffel符号
% 克里斯托费尔|克氏符|克氏|Christoffel|测地线|geodesic|广义相对论|relativity

\pentry{联络形式与结构定理\upref{ConFom}}

流形的特点是局部与我们熟悉的欧几里得空间同胚.尽管我们经常讨论的是流形的内禀性质,不涉及具体的图或者嵌入,但是在实际应用的时候,比如计算广义相对论的现象时,我们却要关心特定图中的数值关系.本节引入的是著名的Christoffel符号,它描述了在特定图中联络的性质.

本节中默认$(M, \nabla)$是一个带仿射联络的流形.


\subsection{Christoffel符号}

对于$M$的任意一个图$(U, \varphi)$,由于$\varphi(U)$是一个欧几里得空间,即实数坐标空间,因此它的光滑向量场集合自带一组标准正交基$\{\frac{\partial}{\partial x^i}\}$.为方便计,我们可以将每个导子$\frac{\partial}{\partial x^i}$简记为$\partial_i$.

点$\varphi(p)=(\varphi(p)_1, \varphi(p)_2, \cdots)\in\varphi(U)$处和$\partial_i$相对应的道路,可以取$c_i:I\to \varphi(U)$为代表,其中$c_i(0)=\varphi(p)_i$,且$\frac{\dd}{\dd t}c_i(t)|_{t=0}=1$.

回忆\textbf{切空间(欧几里得空间)}\upref{tgSpaE}中的讨论,导子和道路都是“切向量”这一对象的等价描述.上面给出导子后又给了对应的道路,是为了提示你该如何将$\varphi(U)$中的切向量通过$\varphi^{-1}$映射为$U\subseteq M$上的切向量.

每个图唯一对应一个量,称为Christoffel符号,如\autoref{CrstfS_def1} 所示.

\begin{definition}{Christoffel符号}\label{CrstfS_def1}

对于$M$的图$(U, \varphi)$,易知$\{\partial_i\}$是$\varphi(U)$上光滑向量场的基,因此存在一组光滑函数$\Gamma^k_{ij}$,使得
\begin{equation}
\nabla_{\partial_i}\partial_j=\Gamma^k_{ij}\partial_k
\end{equation}

称$\Gamma^k_{ij}$为联络$\nabla$在图$(U, \varphi)$上的\textbf{Christoffel 符号(symbol)},简称\textbf{克氏符}.
\end{definition}

回忆\textbf{爱因斯坦求和约定}\upref{EinSum}的规定,$\Gamma^k_{ij}$是由$\Gamma$类型的元素构成的嵌套矩阵,这里每个元素都是一个\textbf{光滑函数}.如果固定$i$和$j$,那么$\Gamma^k_{ij}$就是一个光滑函数构成的列矩阵,用来表示$\partial_k$线性组合出$\nabla_{\partial_i}\partial_j$的系数.

Christoffel符号的分量由所选择的图来决定,因此并不是流形上不变的量,这就把它和张量场区分开来.张量场的定义不依赖于图的选择,我们讨论的时候也都可以摆脱图来讨论,只不过当选定图了以后,一个张量场总可以表示为光滑函数的嵌套矩阵;但Christoffel符号就是依赖图来定义的,根本就没有脱离图的概念,所以要注意区分\footnote{记住,不是所有有上下标的东西都是张量场的.上下标的表示法,只是对嵌套矩阵的简化表达而已.}.


下面是一个重要的性质.

\begin{theorem}{无挠对称性}
流形$(M, \nabla)$是无挠的,\textbf{当且仅当}在任意图中,$\Gamma^k_{ij}=\Gamma^k_{ji}$.
\end{theorem}

\textbf{证明}:

$\Rightarrow$:

因为无挠,故$\nabla_{\partial_i}\partial_j-\nabla_{\partial_j}\partial_i=[\partial_i, \partial_j]$.而由于欧几里得空间中偏微分算子的交换性,$[\partial_i, \partial_j]=0$,故$\nabla_{\partial_i}\partial_j=\nabla_{\partial_j}\partial_i$.

又因为$\{\partial_k\}$是线性不相关的,因此$\Gamma^k_{ij}=\Gamma^k_{ji}$.

$\Leftarrow$:

由\autoref{affcon_exe1}~\upref{affcon}可知,$T(f\partial_i, g\partial_j)=fgT(\partial_i, \partial_j)$.任意向量场都可以表示为$f^i\partial_i$的形式,其中$f^i$的类型是“光滑函数”.

将任意两个光滑向量场分别表示成$f^i\partial_i$和$g^j\partial_j$,于是有$T(f^i\partial_i, g^j\partial_j)=f^ig^jT(\partial_i, \partial_j)=f^ig^j(\Gamma^k_{ij}-\Gamma^k)$

\textbf{证毕}.












