% 仿射联络(流形)
% affine|联络|connection|方向导数|平行移动|directional derivative|parallel transport

\subsection{平行移动}

地球的表面是一个二维实流形.如果我站在赤道上,沿着地球表面,保持向东匀速运动,那么我最终会绕赤道一圈回到原点.由于我被限制在地面,在我看来自己的速度是没有变化的,但如果你站在月球上看我,你会认为我的速度方向一直在变化.我们的答案都没有问题,只不过分别处在不同的视角中.站在月球上的你是从整个三维空间来观察的,也就是$\mathbb{R}^3$,但限制在地球上的我并不知道宇宙和地幔的存在,对我来说整个世界就是地球表面,也就是$S^2$.

在你看来,我处在不同位置上的速度向量是不一样的,但在我看来是一样的.这种情况就被称为一种“平行移动”,即给定流形上的一个切向量,沿着某条道路移动切点,过程中保持切向量不变.例子中的速度向量,其变化率是一直垂直地面的,而对于地面上的我来说是不存在这一方向的,因此在我的计算里我的速度并没有变化.

仔细琢磨以上例子,会发现很多值得注意的点.首先,平行移动是“在沿着某条道路移动过程中”保持切向量不变,也就是说此概念是依赖于道路而定义的,光有两个点可不行.我们可以通过一个例子来理解这一点:假设我手里有一个箭头,我在运动过程中保持这个箭头方向不变.我先绕赤道半圈,再沿着经线抵达北极,记录箭头的指向;回到原点重新开始,这次我直接沿着经线抵达北极,记录箭头的指向.liang









