% 本章导航
% laserdog|量子力学讲义

本章选自 laserdog 编写的量子力学的讲义,%未完成
并由小时进行了重新排版和一些格式调整. 以下是讲义的前言.

\subsection{前言}

量子力学的教材,整体结构利用了矩阵力学的形式,主要目的为弥补国内教材和量子力学课的匮乏.

原稿由英文写成,但是在制作讲义的时候放弃了英文:我写的讲义肯定是不如Sakurai的大作,或者其他类似的书的.既然这样的话,为什么要放弃我本就不多的目的——方便中国的读者们阅读呢?

因此,本讲义为中文的,当然如果有时间的话我会把我的英文原稿也弄出来,不过意义并不很大.

本讲义主要参考文献为 Sakurai\footnote{J. N. J. J. Sakurai, Modern Quantum Mechanics, 2nd ed., edited by L. Kenny (Addison-Wesley, 2011).},也适当的参考了我自己学习量子力学的时候各个老师写的讲义.
