% 正交子空间
% 正交|子空间|内积|矢量空间|高等代数

\pentry{内积\upref{InerPd}, 直和\upref{DirSum}}

\begin{definition}{正交子空间}
一个内积空间 $V$ 中, 如果两个子空间 $V_1$ 和 $V_2$ 任意各选一个矢量 $\ket{v_1}$ 和 $\ket{v_2}$ 都有 $\braket{v_1}{v_2} = 0$, 那么我们就说者两个子空间是\textbf{正交}的.
\end{definition}

构造正交子空间的一种简单的方法是, 在 $V$ 中找到两组矢量 $x_1, \dots, x_m$ 和 $y_1, \dots, y_m$, 确保对任意 $x_i$ 和 $y_j$ 正交, 那么 $x_1, \dots, x_m$ 张成的子空间必定和 $y_1, \dots, y_m$ 张成的子空间正交.%未完成: “张成” 提过吗?

\begin{theorem}{}
从基底的角度来看, 两个空间正交的充分必要条件是: 如果从两空间各选一组基底 $\ket{\alpha_i}$ $(i = 1, \dots, N_1)$ 和 $\ket{\beta_i}$ $(i = 1, \dots, N_2)$, 有对任意 $i, j$ 都有 $\braket{\alpha_i}{\beta_j} = 0$.
\end{theorem}

\begin{exercise}{}
证明两个正交子空间中, 只有零矢量是共同矢量.
\end{exercise}

\subsection{正交子空间的直和}

若两个正交子空间的维数分别为 $N_1$ 和 $N_2$, 它们之和等于母空间的维数 $N$, 那么就说它们是\textbf{互补(complementary)}的. 若分别在这两个空间中取一组基底, 那么将他们合并起来就得到了母空间中的一组基底.

特殊地, 如果 $V_1$ 和 $V_2$ 的两组基底合并后仍然正交归一, 那么合并后就得到了直和空间 $V = V_1 \oplus V_2$ 中的一组正交归一基底. 但注意直和空间中的任意一组正交归一基底未必可以划分为 $V_1$, $V_2$ 空间中的两组基底.

\begin{example}{}
三维几何矢量空间中, 建立直角坐标系, 那么 $\uvec x$ 和 $\uvec y$ 张成的二维矢量空间(平面)与 $\uvec z$ 张成的一维矢量空间(直线)正交.
\end{example}

\begin{example}{}
虽然 $xy$ 平面和 $xz$ 平面是两个正交的平面, 但它们并不是两个正交子空间. 例如矢量 $\uvec x$ 是两个平面共同的矢量, 但 $\uvec x$ 和它本身不正交.
\end{example}

\begin{exercise}{}
请将\autoref{DirSum_ex1}~\upref{DirSum} 和\autoref{DirSum_ex2}~\upref{DirSum} 稍作修改以适用于本文.
\end{exercise}

\subsection{正交补}
在 $V$ 空间中, 若 $V_1$ 和 $V_2$ 正交且 $V = V_1 \oplus V_2$, 那么 $V_1$ 和 $V_2$ 互为对方的\textbf{正交补(Orthogonal complement)}.
