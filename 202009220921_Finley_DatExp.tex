% 机器学习数据探索
\pentry{机器学习数据类型\upref{DatTyp}}
首先要确定哪些数据是输入(Input)的变量,哪些数据是要预测的目标值(output),然后需要确定变量的类型,定类变量或者定比、变量等.
探索数据特征常采用以下方法:
\subsection{单变量分析}
单变量分析是数据分析中最简单的形式,其中被分析的数据只包含一个变量.因为它是一个单一的变量,它不处理原因或关系.单变量分析的主要目的是描述数据并找出其中存在的模式.  
\subsubsection{连续型特征分析方法}
连续特征变量可通过平均值,中位数,众数,最小值,最大值,范围,四分位数,IQR,方差,标准差,偏度,等来探索数据特征.此外,显示单变量数据的一些方法包括频率分布表、柱状图、直方图、频率多边形和饼状图.
\subsubsection{类别型特征分析方法}
对于类别特征的变量常通过频率,直方图来探索.
\subsection{双变量分析}
双变量分析目标是确定两个变量之间的相关性,测量它们之间的预测或解释的能力.
\subsubsection{散点图}
在笛卡尔平面上将一个变量对另一个变量进行绘图,从而创建散点图,如果数据似乎符合直线或曲线,那么这两个变量之间存在关系或相关性.
\begin{figure}[ht]
\centering
\includegraphics[width=12cm]{./figures/DatExp_1.png}
\caption{散点图} \label{DatExp_fig1}
\end{figure}
\subsubsection{热力图}
利用多个变量之间的相关性矩阵做出热力图,以确定两个变量之间的相关性.在建模的时候衡量变量相关性我们一般都是计算变量两两之间的皮尔逊相关系数(Pearson correlation coefficient),并以热力图的方式展示.
\begin{figure}[ht]
\centering
\includegraphics[width=12cm]{./figures/DatExp_2.png}
\caption{热力图} \label{DatExp_fig2}
\end{figure}
\subsubsection{二维列联表}
\subsubsection{堆积柱形图}
\subsubsection{卡方检验}
\subsubsection{Z检验/ T检验}
\subsubsection{方差分析(ANOVA)}