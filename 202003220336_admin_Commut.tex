% 算符对易与共同本征函数
% 算符|本征矢|本征函数|量子力学|对易

\pentry{厄米矩阵,%未完成链接
本征函数的简并}%未完成, 要引入希尔伯特子空间的概念, 说明子空间中的任意一个函数都是本征函数, n重简并的子空间是n维的, 即可以有n个线性无关的本征函数张成. 一般选取两两正交的波函数作为基底, 基底有无穷多种选法, 任何基底经过任意幺正变换以后仍然是子空间的基底. 类比一下

\subsection{命题}
以下两个条件互为充分必要条件
\begin{enumerate}
\item 两个厄米算符 $\Q A$ 和 $\Q B$ 互相对易.
\item 算符 $\Q A$ 和 $\Q B$ 的本征方程存在一整套共同的本征函数 $\psi_i$.
\end{enumerate}

\subsection{证明条件 $2 \to 1$}
设算符 $\Q A$ 和 $\Q B$ 有一组共同的本征函数 $\psi_i$,  则它们同时满足 $\Q A$ 和 $\Q B$ 的本征方程
\begin{equation}
\begin{cases}
\Q A \psi_i = a_i \psi_i\\
\Q B \psi_i = b_i \psi_i
\end{cases}
\end{equation}
对任何 $\psi_i$,  都有
\begin{equation}
\Q A (\Q B \psi_i) = \Q A (b_i \psi_i) = b_i\Q A \psi_i = a_i b_i \psi_i
\end{equation}
\begin{equation}
\Q B (\Q A \psi_i) = \Q B (a_i \psi_i) = a_i \Q B \psi_i = a_i b_i \psi_i
\end{equation}
所以 $\Q A\Q B \psi_i = \Q B\Q A \psi_i$ 即
\begin{equation}
\comm*{\Q A}{\Q B} = \Q A\Q B - \Q B\Q A = 0
\end{equation}
即两算符对易.证毕.

\subsection{证明条件 $1 \to 2$}
要证明 $1 \to 2$,  只需证明 $\Q A$ 的一套本征函数都满足 $\Q B$ 的本征方程即可.

\subsubsection{算符 $\Q A$ 非简并情况( $\Q B$ 是否简并没关系)}
先解出算符 $\Q A$ 的本征方程 $\Q A \psi_i = a_i \psi_i$,  如果 $\Q A$ 算符不发生简并(见本征函数的简并%未完成链接
  )那么本征值各不相同, 且给定一个本征值 $a_i$ 其解只可能是 $\psi_i$ 或者 $\psi_i$ 乘以一个任意复常数(注释:其实也可以再相乘一个算符 $\Q A$ 不涉及的物理量的函数, 例如总能量算符 $\Q H$ 的本征函数还可以再成一个时间因子 $\E^{\I\omega t}$ ).

因为算符对易, 有
\begin{equation}
\Q A (\Q B \psi_i) = \Q B (\Q A \psi_i) = a_i (\Q B \psi_i)
\end{equation}
把式中的 $\Q B \psi_i$ 看成一个新的波函数, 上式说明 $\Q B \psi_i$ 是算符 $\Q A$ 和本征值 $a_i$ 的另一个本征函数. 根据以上分析, $\Q B \psi_i$ 必定是 $\psi_i$ 乘以某个复常数(命名为 $b_i$ ), 即
\begin{equation}
\Q B \psi_i = b_i \psi_i
\end{equation}
而这正是 $\Q B$ 的本征方程(而 $\Q B$ 也是厄米矩阵, 所以作为本征值 $b_i$ 的数域从复数缩小到实数). 证毕.

\subsubsection{算符 $\Q A$ 简并情况}
假设算符 $\Q A$ 的所有本征值为 $a_i$ (各不相同), 任意一个 $a_i$ 有 $n_i$ 重简并. 若 $n_i = 1$,  对应唯一一个 $\psi_i$,  那么根据上文对非简并情况的推理, $\psi_i$ 就已经是 $\Q B$ 的本征函数了. 若 $n_i > 1$,  存在一个 $n_i$ 维希尔伯特子空间, 里面任何一个函数都是 $a_i$ 对应的本征函数, 所以要在子空间中寻找共同本征函数, 只需在子空间中寻找 $\Q B$ 的本征函数即可. 令 $\phi_i$ 为本征值为 $a_i$ 的子空间中的任意函数, 利用对易关系
\begin{equation}
\Q A (\Q B \phi_i) = \Q B (\Q A \phi_i) = a_i (\Q B \phi_i)
\end{equation}
这条式子说明 $\Q B \phi_i$ 是 $\Q A$ 和 $a_i$ 的一个本征函数, 即 $\Q B \phi_i$ 仍然在 $a_i$ 的简并子空间中.所以 $\Q B$ 对子空间来说是一个闭合的厄米算符, 所以必有 $N$ 个线性无关的本征函数. 证毕.%(厄米算符性质 $x$,  未完成)

以下的内容应该归到厄米算符里面讲(厄米算符在希尔伯特空间中是无穷维的矩阵, 但是如果一个厄米算符在一个子空间中闭合,那么就可以通过以下方法找到N个线性无关的本征函数.%厄米算符在
先在空间中任意选取 $n_i$ 个线性无关的正交本征函数 $\psi_{i1}, \psi_{i2}\dots \psi_{i n_i}$ 作为子空间的基底(本征函数的简并%未完成链接
)),并可以用基底 $\psi_{i1}, \psi_{i2}\dots \psi_{i n_i}$ 展开.

令 $\Q B \psi_{ij} = \sum_{k=1}^{n_i} W_{jk}\psi_{ik}$ ( $W_{jk} = \bra{\psi_{ij}} \Q B \ket{\psi_{ik}}$, 可以是复数), 则 $\Q B$ 在该子空间可以表示成一个 $n_i$ 维的方形矩阵(记为 $W$ ).

以 $\psi_{i1}, \psi_{i2} \dots \psi_{i n_i}$ 为子空间的基底, 子空间内任意函数 $\phi  = x_1 \psi_{i1} + x_2 \psi_{i2}\dots$ 可以记为 $\ket{\phi} = (x_1, x_2, \dots, x_{n_i})\Tr$. 根据算符的矩阵表示\upref{OpMat}
, $\Q B$ 在子空间的矩阵元就是系数 $W_{jk}$, 
\begin{equation}
W = \begin{pmatrix}
W_{11} & \ldots & W_{1 n_i}\\
\vdots & \ddots & \vdots \\
W_{n_i 1} & \ldots & W_{n_i n_i}
\end{pmatrix}
\end{equation}
所以 $\Q B$ 在子空间范围内的本征方程的矩阵形式就是
\begin{equation}
W \ket{\phi_k} = b_{ik} \ket{\phi_k}
\end{equation}
所以 $\Q B$ 在子空间的本征值就是 $W$ 的本征值, 本征函数就是 $W$ 的本征矢对应的波函数.

最后要证明的就是 $W$ 矩阵必然存在 $n_i$ 个本征矢.由于 $\Q B$ 是厄米算符,  $W$ 必然是厄米矩阵, 而 $n_i$ 维的厄米矩阵必然存在 $n_i$ 个两两正交的复数本征矢和实数本征值(厄米接矩阵%链接未完成
).

综上所述, 对每一个 $n_i$ 重简并的 $a_i$,  都存在 $n_i$ 个两两正交的本征函数作为 $\Q A$,  $\Q B$ 算符的共同本征函数.证毕.
