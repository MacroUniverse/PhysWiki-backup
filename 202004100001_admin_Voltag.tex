% 电压

\pentry{电势 电势能\upref{QEng}}

\textbf{电压(voltage)} 就是电势差的同义词, 通常在讨论电路时使用, 本词条只讨论电路中的电压. 电势差的定义为(\autoref{QEng_eq1}\upref{QEng})
\begin{equation}\label{Voltag_eq1}
U_{21} = V(\bvec r_2) - V(\bvec r_1) = - \int_{\bvec r_1}^{\bvec r_2} \bvec E(\bvec r) \vdot \dd{\bvec r}
\end{equation}
在 “电势 电势能\upref{QEng}” 中, 我们强调了要定义电荷的电势能或电势, 我们必须要使用无旋的电场(保守场), 而电路中一般来说既包含无旋场也包含有旋场. 所以我们规定



注意这个广义定义只在静电学中有意义, 即要求净电荷分布和电流分布不随时间变化.

事实上, “电压” 的概念一般在讨论电路时才会出现. 但电路中的电压的定义却与电磁学中的一般定义有所不同.

在我们讨论电路时, 可能不满足静电学的条件(例如交流电, 以及磁生电等情况), 但仍然可以用上式定义电路中任意两点之间的瞬时电压, 但是积分路径必须要沿着电路.

\begin{exercise}{磁生电}
假设我们有一个 $N$ 匝的不闭合线圈, 两端接在理想电压表上. 若线圈中有磁铁在上下运动, 使线圈中的磁通量随时间变化, 那么根据高中的知识我们知道电压表会显示读数.

然而这并不是一个静电学问题, 上下运动的磁铁会沿着线圈产生涡旋电场, 显然这种电场不是一个保守场, 所以一般来说我们无法在空间中给出\textbf{与路径无关}的电势的定义. 但如果我们只讨论导线
\end{exercise}