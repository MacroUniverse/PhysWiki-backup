% 切丛
% 向量丛|切向量|截面|光滑截面
\begin{issues}
\issueDraft
\end{issues}

\pentry{纤维丛\upref{Fibre}}

“切丛”是“切向量丛”的简称.顾名思义,它是切向量构成的向量丛,而这里的切向量是在流形上定义的,因此切丛的底空间是流形.

\begin{definition}{切丛、}
给定流形$M$,以$M$为底空间,把各$p\in M$上的$T_pM$视为该点处的一根纤维,得到的纤维丛就称为流形$M$上的\textbf{切丛(tangent bundle)}.
\end{definition}

一个切向量场可以视为切丛的一种特殊的子集,称为“\textbf{截面(section)}”.使用这个术语是为了强调这种子集的特殊性,它在每一根纤维上都取且仅取一个点,看起来就像是纤维丛的一个截面.同样地,一个光滑切向量场有时也被称作切丛上的一个\textbf{光滑截面}.

局部来看,流形$M$上切丛的每根纤维是一个线性空间;整体来看,每个光滑截面都可以看成一个向量,构成一个$M$整体上的线性空间,记为$\mathfrak{X}(M)$.


