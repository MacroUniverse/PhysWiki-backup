% test
% test|测试|编辑器

\section {误差动力学方程与通信结构的关系推导}
每辆车的动力学方程
\begin{equation}
\begin{bmatrix}
\dot{p}_{i} \\
\dot{v}_{i} \\
\dot{a}_{i}
\end{bmatrix} =\begin{bmatrix}
 0 & 1 & 0\\
 0 & 0 & 1\\
 0 & 0 &-\frac{1}{\tau}
\end{bmatrix}\begin{bmatrix}
p_{i} \\
v_{i} \\
a_{i}
\end{bmatrix} +\begin{bmatrix}
0 \\
0 \\
\frac{1}{\tau}
\end{bmatrix}u_{i}
\end{equation}
即$\dot{x}=Ax+Bu$

如果引入集总状态向量,那么
\begin{equation}
\begin{bmatrix}
\dot{x} _{1} \\
\dot{x} _{2} \\
\dot{x} _{3} \\
\vdots  \\
\dot{x} _{N}
\end{bmatrix}=\begin{bmatrix}
Ax _{1}+Bu_{1} \\
Ax _{2}+Bu_{2}\\
Ax _{3}+Bu_{3} \\
\vdots  \\
Ax _{N}+Bu_{N}
\end{bmatrix}=\begin{bmatrix}
 A &  &  &  & \\
  & A &  &  & \\
  &  & A &  & \\
  &  &  & \ddots  & \\
  &  &  &  &A
\end{bmatrix}\begin{bmatrix}
x _{1} \\
x _{2}\\
x _{3} \\
\vdots  \\
x _{N}
\end{bmatrix}+\begin{bmatrix}
 B &  &  &  & \\
  & B &  &  & \\
  &  & B &  & \\
  &  &  & \ddots  & \\
  &  &  &  &B
\end{bmatrix}\begin{bmatrix}
u _{1} \\
u _{2}\\
u _{3} \\
\vdots  \\
u _{N}
\end{bmatrix}
\end{equation}
记为$\dot{X} =\mathbb{A} X+\mathbb{B} U$

\subsection {线性控制器}
假如我们设计的第$i$辆车的线性控制器结构是
\begin{equation}
\begin{aligned}u_{i}(t)&=-\sum_{j\in \mathbb{I}_{i} }^{} [k_{p}(p_{i}(t)-p_{j}(t)-d_{ij})+k_{v}(v_{i}(t)-v_{j}(t))+k_{a}(a_{i}(t)-a_{j}(t))]\\
&=-\sum_{j\in \mathbb{I}_{i} }^{}k_{ij}^T(e _{i}(t)-e_{j}(t))
\end{aligned}
\end{equation}
这里的$k_{ij}^T=[k_{ij,p}\ k_{ij,v}\ k_{ij,a}]$,$e _{i}(t)=x_{i}(t)-x_{0}(t)-\tilde{d} _{i}$


其中
\begin{equation}
\begin{aligned}
U(t)=\begin{bmatrix}
u_{1}(t) \\
u_{2}(t)  \\
\vdots  \\
u_{N}(t)
\end{bmatrix}
&=\begin{bmatrix}
-\sum_{j\in \mathbb{I}_{1} }^{}(k_{1j}^T(e _{1}(t)-e_{j}(t)) \\
-\sum_{j\in \mathbb{I}_{2} }^{}(k_{2j}^T(e _{2}(t)-e_{j}(t))\\
\vdots   \\
-\sum_{j\in \mathbb{I}_{N} }^{}(k_{Nj}^T(e _{N}(t)-e_{j}(t))
\end{bmatrix}\\
&=
-\begin{bmatrix}
p_{1}k_{10}^T+\sum_{j=1}^{N}a_{1j}k_{1j}^T   & -a_{12}k_{12}^T & \cdots  & -a_{1N}k_{1N}^T\\
-a_{21}k_{21}^T  &  p_{2}k_{20}^T+\sum_{j=1}^{N}a_{2j}k_{2j}^T& \cdots  &-a_{2N}k_{2N}^T \\
  \vdots & \vdots  & \ddots  & \vdots \\
-a_{N1}k_{N1}^T  & -a_{N2}k_{N2}^T & \cdots  &p_{N}k_{N0}^T+\sum_{j=1}^{N}a_{Nj}k_{Nj}^T
\end{bmatrix}\begin{bmatrix}
e _{1}(t) \\
e_{2}(t) \\
\vdots  \\
e_{N}(t)
\end{bmatrix}\\
&=-(\mathcal{L} +\mathcal{P})\otimes k^T\cdot E(t)
\end{aligned}
\end{equation}
因此如果将系统的动力学方程用表示拓扑结构的矩阵表示,就可以得到$\dot{X} =\mathbb{A}X+\mathbb{B} U=\mathbb{A}X-\mathbb{B}(\mathcal{L} +\mathcal{P})\otimes k^T\cdot E$.

这里的$X=[x _{1}^T\ x _{2}^T\ \cdots \ x _{N}^T]^T$是集总状态向量,$E=[e _{1}\ e _{2}\ \cdots \ e _{N}]^T$是集总误差状态向量,所以为了统一起见,最好将动力学方程改为误差动力学方程.(但如果选择其他的状态变量的话或者别的控制器结构也可以,但对应的形式肯定会有所不同)


误差动力学方程的推导:
%$X=\begin{bmatrix}
% p_{1} & v_{1} &a_{1} \\ p_{2}  &v_{2} &a_{2} \\ \vdots  & \vdots  &\vdots  \\p_{N}  &  v_{N} &a_{N}
%\end{bmatrix}$,故$\dot{X} =\begin{bmatrix}
%v_{1} &a_{1}&\dot{a} _{1} \\v_{2} &a_{2}&\dot{a} _{2} \\
%\vdots  &\vdots &\vdots \\v_{N} &a_{N}&\dot{a} _{N}
%\end{bmatrix}$


$E=\begin{pmatrix}
p_{1}(t)-p_{0}(t)-d _{10} \\ v_{1}(t)-v_{0}(t)\\ a_{1}(t)-a_{0}(t)\\ p_{2}(t)-p_{0}(t)-d _{20}\\ v_{2}(t)-v_{0}(t)\\ a_{2}(t)-a_{0}(t)\\
\vdots \\
p_{N}(t)-p_{0}(t)-d _{N0} \\ v_{N}(t)-v_{0}(t) \\ a_{N}(t)-a_{0}(t)  \\
\end{pmatrix}$,
故$\dot{E} =\begin{pmatrix}
v_{1}(t)-v_{0}(t)\\ a_{1}(t)-a_{0}(t)\\\dot{a} _{1}(t)-\dot{a}_{0}(t)\\v_{2}(t)-v_{0}(t)\\a_{2}(t)-a_{0}(t)\\ \dot{a} _{2}(t)-\dot{a}_{0}(t)\\
\vdots \\
v_{N}(t)-v_{0}(t) \\a_{N}(t)-a_{0}(t)  \\\dot{a} _{N}(t)-\dot{a}_{0}(t)
\end{pmatrix}$
不难得到$\dot{E} =\mathbb{A} E+\mathbb{B} U$,

将$U=-(\mathcal{L} +\mathcal{P})\otimes k^T\cdot E$代入误差动力学特性方程,得到

$\dot{E} =(I_{N}\otimes \mathbb{A} -I_{N}\otimes \mathbb{B}\cdot(\mathcal{L} +\mathcal{P})\otimes k^T)\cdot E=(I_{N}\otimes \mathbb{A} -(\mathcal{L} +\mathcal{P})\otimes\mathbb{B}  k^T)\cdot E$
\subsection {MPC控制器}
在MPC里,邻接矩阵$\mathcal{A}$,拉普拉斯矩阵$\mathcal{L}$和牵引矩阵$\mathcal{P}$很少用到的,更多的是利用这些矩阵对应的集合,称为邻域集$\mathbb{N} _{i}$,可达性集合$\mathbb{P} _{i}$和可使用的邻域信息$\mathbb{I} _{i}=\mathbb{N} _{i} \cup  \mathbb{P} _{i}$.
\subsection {分布式MMPC—匀质队列—模仿}
第i辆车的动力学方程:
$\begin{pmatrix}
\dot{p} _{i}(t) \\
\dot{v} _{i} (t)\\
\dot{a} _{i}(t)
\end{pmatrix}=\begin{pmatrix}
 0 & 1 &0\\
 0 & 0 & 1\\
 0 & 0 &-\frac{1}{\tau }
\end{pmatrix}\begin{pmatrix}
p _{i}(t) \\
v _{i} (t)\\
a _{i}(t)
\end{pmatrix}+\begin{pmatrix}
0 \\
0\\
\frac{1}{\tau }
\end{pmatrix}u_{i}(t) $


以T为采样周期离散化后的动力学方程为:
$\begin{pmatrix}
p_{i}(k+1) \\
v_{i} (k+1)\\
a_{i}(k+1)
\end{pmatrix}=\begin{pmatrix}
 1 & T &0\\
 0 & 1 & T\\
 0 & 0 &1-\frac{T}{\tau }
\end{pmatrix}\begin{pmatrix}
p _{i}(k) \\
v _{i} (k)\\
a _{i}(k)
\end{pmatrix}+\begin{pmatrix}
0 \\
0\\
\frac{T}{\tau }
\end{pmatrix}u_{i}(k)$\\

令$x_{i}(k)=\begin{pmatrix}
p_{i}(k) \\
v_{i} (k)\\
a_{i}(k)
\end{pmatrix}$,得到$x_{i}(k+1)=Ax_{i}(k)+Bu_{i}(k)$


文章中有$u_{i}(k)=u_{i}(k-1)+\Delta u_{i}(k)$,所以模仿一下,将状态空间模型重新写成:
\begin{equation}
  \begin{bmatrix}
x_{i}(k+1) \\
u_{i}(k)
\end{bmatrix}=\begin{bmatrix}
A  &B\\
 0 &I
\end{bmatrix}\begin{bmatrix}
x_{i}(k) \\
u_{i}(k-1)
\end{bmatrix}+\begin{bmatrix}
B \\
I
\end{bmatrix}\Delta u_{i}(k)
\end{equation}
令$\mathcal{X} _{i}(k)=\begin{bmatrix}
x_{i}(k) \\
u_{i}(k-1)
\end{bmatrix}$,故得
\begin{equation}
\mathcal{X} _{i}(k+1)=\mathcal{A} \mathcal{X}_{i}(k)+\mathcal{B}\Delta u_{i}(k)\end{equation}\\
由于MMPC每次只更新一个控制输入,其他的控制输入保持不变,故得:
\begin{equation}
\mathcal{X} _{i}(k+1)=\mathcal{A} \mathcal{X}_{i}(k)+\mathcal{B}_{\sigma (k)}\Delta \hat{u} _{i}(k)
\end{equation}
但是这里有个问题,$\sigma(k)=(k\mod{m})+1$,这里的$m$是控制输入的个数,但是一辆车只有一个控制输入,所以$m=1$,但是这样就没有意义了.






