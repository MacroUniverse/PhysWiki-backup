% RNS 超弦

我们使用Ramond-Neveu-Schwarz(RNS)形式来修改玻色弦理论以引入费米子.这个方法在世界面上具有超对称.随后我们会使用具有时空超对称的Green-Schwarz形式.当时空维度是10的时候,这两个方案是等价的.

首先我们考虑共形规范下的Polyakov作用量.
\begin{equation}
S = - \frac{T}{2} \int d^2 \sigma \partial_\alpha X^\mu \partial^\alpha X_\mu~.
\end{equation}
加入自由费米子$\psi$之后,作用量如下
\begin{equation}
S = - \frac{T}{2} \int d^2\sigma (\partial_\alpha X^\mu \partial^\alpha X_\mu - i \bar\psi^\mu \rho^\alpha \partial_\alpha \psi_\mu)~.
\end{equation}
$\rho^\alpha$是世界面上的狄拉克矩阵.因为世界面是1+1维的,所以$\rho^\alpha$也是1+1维的狄拉克矩阵.有两个这样的矩阵
\begin{equation}
\rho^0 = \begin{pmatrix}
0 & -i \\
i & 0
\end{pmatrix}~, \quad \rho^1 = \begin{pmatrix}
0 & i \\
i & 0
\end{pmatrix}~.
\end{equation}

\subsubsection{Majorana旋量}
$\psi^\mu = \psi^\mu(\sigma,\tau)$是两分量Majorana旋量.我们把它记作
\begin{equation}
\psi = \begin{pmatrix}
\psi_- \\
\psi_+
\end{pmatrix}~.
\end{equation}
在洛伦兹变换下,这些场按照矢量的规则变换.我们可以定义$\bar\psi^\mu$
\begin{equation}
\bar\psi^\mu = (\psi^\dagger)^\mu \rho^0~.
\end{equation}






