% 使用数字货币钱包

\subsection{安全事项}
\begin{itemize}
\item 建议使用不易中病毒的操作系统生成钱包(如 iOS, Mac, Linux 等), 如果你使用 windows, 确保你的系统和软件都是正版的, 没有录屏软件, 并先进行杀毒.
\item 强烈建议使用系统自带的浏览器和输入法(在 iOS 上, 即使使用第三方输入法, 也不要打开全部权限)进行任何数字货币有关的操作.
\end{itemize}

\subsection{比特币钱包种类}
我们接下来主要以比特币为例讲解, 其他数字货币也大同小异. 比特币最原始的钱包形式就是一个地址和一个私钥. 一对真实有效的例子如
\begin{lstlisting}[language=bash]
地址: 15Vq6G2x7pZkGnLZBgqCCaruG5UZCtn8dr
私钥: Kzr3NAC9z5sE71MWrzdqoYz4wfmYhu6zainJHgQRBW6hcYZ5Xh25
\end{lstlisting}
一个用于生成地址和私钥的常用网站为 \href{https://www.bitaddress.org/}{bitaddress.org}. 需要注意的是地址和私钥是拥有比特币的唯一凭证, 任何持有两者的人就等同于持有该地址中的所有比特币. 比特币一旦被盗几乎不可能追回.

\subsection{关于隐私}
理论上, 当一个钱包地址生成以后, 如果只单独挖矿或者和随机的地址之间转账, 那么无法得知该地址的持有者是谁. 但如果该地址直接接收交易所的转账, 且交易所中有地址持有者的身份信息, 那么由于所有的转账都是公开的, 那么这个地址的身份也可能暴露. 但是网上也一种提供混淆转账功能的热钱包, 把不同的随机汇款混合da

\subsection{123}
以下大致列出钱包的类型.
\begin{itemize}
\item \textbf{纸钱包}: 顾名思义, 把地址和私钥写在纸上, 或者打印出来(通常包含二维码), 并保存到安全的地方(如保险柜)就叫纸钱包. 纸钱包的缺点是使用起来不太方便, 以及纸容易被烧毁, 泡烂, 褪色等.
\item \textbf{钢钱包}: 为了克服纸钱包的脆弱, 市面上也可以买到一些专用的小钢板, 可以在上面以打点或拼图的方式记录地址和私钥. 钢钱包是保存比特币最安全最稳定的方式, 不需要依赖任何第三方数字设备或软件, 无法被黑客盗取.
\item \textbf{硬件钱包}: 类似于银行的 U 盾, 通常带有 USB 接口, 地址和私钥信息储存在钱包中, 需要转账时将其插入电脑即可. 原则上该钱包不会把私钥以任何方式联网, 但这要求你信任该 app 的开发者.
\item \textbf{手机离线钱包}: 把地址和私钥加密保存在手机钱包 app 中, 通常附带转账功能. 这同样要求你信任该开发者.
\item \textbf{热钱包}: 常用于交易所. 钱包的开发者拥有所有的地址和私钥, 且用户通常无法获得. 用户通过钱包网站的注册信息登录并进行转账或交易等操作. 热钱包是最不安全的钱包, 此前已经有一些热钱包被黑客破解的案例. 注意黑客只是攻击了热钱包网站并获得了比特币的地址和私钥, 但比特币网络本身却从来没有被破解过.
\end{itemize}

\subsection{123}
