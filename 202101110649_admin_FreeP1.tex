% 一维自由粒子(量子)

\begin{issues}
\issueDraft
\end{issues}

\pentry{薛定谔方程\upref{TDSE}, 傅里叶变换(指数)\upref{FTExp}, 原子单位制\upref{AU}}

本文使用原子单位制\upref{AU}. 当含时薛定谔方程\upref{TDSE}中势能函数 $V(x) = 0$ 时, 有
\begin{equation}\label{FreeP1_eq1}
-\frac{1}{2m}\pdv[2]{x}\Psi(x,t) = \I \pdv{t} \Psi(x,t)
\end{equation}
一般用分离变量法解该方程, 通解为(\autoref{TDSE_eq3}~\upref{TDSE}). 但首先要解出对应的定态薛定谔方程
\begin{equation}
-\frac{1}{2m}\dv[2]{x}\psi_E(x) = E \psi_E(x)
\end{equation}
只有 $E > 0$ 时有可归一化的解, 也就是熟悉的平面波
\begin{equation}
\psi_E(x) = \E^{\I k x}
\end{equation}
其中 $k = \sqrt{2mE}$. 令 $\omega = E$, 则\autoref{FreeP1_eq1} 的通解为
\begin{equation}
\Psi(x,t) = \int A(\omega) \E^{\I (k x - \omega t)} \dd{\omega}
\end{equation}
做变量替换 $\omega = k^2/(2m)$, 通解也可以记为
\begin{equation}
\Psi(x,t) = \frac{1}{\sqrt{2\pi}} \int C(k) \E^{\I (k x - \omega t)} \dd{k}
\end{equation}
这恰好是反傅里叶变换\upref{FTExp}. 当 $t = 0$ 时有
\begin{equation}
C(k) = \frac{1}{\sqrt{2\pi}} \int \Psi(x,0) \E^{-\I k x} \dd{x}
\end{equation}
这样我们就从初始条件得到了系数.

一个简单的粒子见 “一维高斯波包(量子)\upref{GausWP}”.
