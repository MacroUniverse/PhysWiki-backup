% 原子的观念

在《生活大爆炸》的第3季第10集中,莱纳德邀请伯纳黛特参观他正在做的验证AB效应(全称是阿哈朗诺夫-玻姆效应)的实验.佩妮不想落于人后,也想在聚餐的时候谈论莱纳德的实验,于是央求谢尔顿教她物理.

但物理是没法速成的,要讲就得从古希腊开始.

“假想在一个炎热的夏季的夜晚,你刚刚在阿戈拉(集市)买完东西,\dots\dots”

“但,这与莱纳德的研究有什么关系呢?”

谢尔顿的回答是:“科学是个2600年的旅程,从古希腊开始,经由牛顿,到玻尔(旧量子论),然后薛定谔(波动力学),到哥本哈根学派”,最后我们才能谈论莱纳德的实验.

我们也采用类似的路径,首先让我们回到2600年前雅典的阿戈拉,一个炎热的夏季夜晚,那里正有人在发表关于原子的理论.

\begin{figure}[ht]
\centering
\includegraphics[width=10cm]{./figures/AtomId_1.png}
\caption{公元前2世纪雅典鸟瞰图} \label{AtomId_fig1}
\end{figure}

\subsection{柏拉图的四元素说}

关于物质的理论,自古就有,比如古希腊的泰勒斯曾说“万物皆水”,后来又有人说万物是四种元素“水、气、土、火”构成.这就是所谓四元素说.

四元素说本来是古希腊人的常识(common sense),但柏拉图给这种关于物质的学说理论化,系统化了.这些内容被记载在柏拉图(427 BC — 347 BC)的宇宙论《蒂迈欧篇》中.

这里我们可以给出一个论证的概要:

\begin{itemize}
\item 万物或者是可见的,或者是可以触摸的.可见是因为光,可以触摸是因为坚硬.光是火的性质,坚硬是土的性质,这样我们就有了“火和土”两种元素.

\item 万物是三维的,我们需要像“胶水”一样的元素把“火和土”按比例混合起来成为三维的物体. 这里柏拉图是通过构造如下数列来论证的:
\begin{equation}
1=1^2=1^3, 2, 4=2^2, 8=2^3, 16=4^2, 32, 64=4^3,...
\end{equation}
这里$1=1^3$,$8=2^3$,$64=4^3$,...,叫做立方数. 每两个立方数之间正好是两个数,比如1和8之间是2和4;而8和64之间是16和32. 柏拉图因此论证说需要两种元素在“火和土”之间调和使生成万物,这两种元素就是水和气.

\item 不论是可见,还是可以触摸,都可以归结为形状,而形状可以归结为多边形的拼合,多边形则可以归结为三角形的拼合. 三角形是研究形的基础,或说三角形是研究几何学的基础.

\item 柏拉图提出了两种基本的直角三角形:(1)等腰直角三角形,记做$T_{45}$;(2)一个锐角为$30^o$的直角三角形,记做$T_{30}$.

\item 利用两种基本的直角三角形可以拼成四种正多面体,即正四面体、正八面体、正六面体和正二十面体.

\begin{figure}[ht]
\centering
\includegraphics[width=8cm]{./figures/AtomId_2.png}
\caption{五种正多面体,也称柏拉图多面体.这里面的正十二面体没有出现在对元素的构造中,这是柏拉图理论中欠缺美感的地方.} \label{AtomId_fig2}
\end{figure}
\end{itemize}
