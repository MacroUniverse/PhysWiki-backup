% 弹簧的串联和并联
% 弹簧|弹性系数|串联|并联

\begin{issues}
\issueDraft
\end{issues}

\pentry{胡克定律}
\begin{figure}[ht]
\centering
\includegraphics[width=9cm]{./figures/Spring_2.pdf}
\caption{弹簧的串联(左)、并联(右)} \label{Spring_fig2}
\end{figure}
两个劲度系数为 $k_1, k_2$ 的弹簧的串联后劲度系数为 $k$, 那么
\begin{equation}
\frac{1}{k} = \frac{1}{k_1} + \frac{1}{k_2}
\quad \text{或} \qquad
k = \frac{k_1 k_2}{k_1 + k_2}
\end{equation}
若并联后劲度系数为 $k$, 那么
\begin{equation}
k = k_1 + k_2
\end{equation}
可见弹簧的串联和并联分别类似于电阻的并联和串联\upref{Rcomb}. 类似地多个弹簧串联或并联有
\begin{equation}\label{Spring_eq2}
\frac{1}{k} = \sum_i \frac{1}{k_i}
\end{equation}
\begin{equation}\label{Spring_eq3}
k = \sum_i k_i
\end{equation}

\addTODO{推导, 类比电阻}
\subsection{推导}
\subsubsection{弹簧的串联}
先考虑两弹簧串联的情况,
如\autoref{Spring_fig2} 左,对于弹簧的串联,设弹簧系统两边受力$F$达到平衡.对弹簧$1$受力分析,由平衡条件,其受弹簧$2$的力大小 $F_1$ 等于外力 $F$,由牛顿第三定律,弹簧 $2$ 受弹簧 $1$作用力大小$F_2$也为 $F$.即
\begin{equation}
F_1=F_2=F
\end{equation}
由胡克定律$F=-k\Delta x$,上式可写为
\begin{equation}
k_1\Delta x_1=k_2\Delta x_2=-F
\end{equation}
那么对于弹簧系统,其劲度系数 $k$ 为
\begin{equation}
k=\frac{-F}{\Delta x}=\frac{-F}{\Delta x_1+\Delta x_2}=\frac{-F}{\frac{-F}{k_1}+\frac{-F}{k_1}}=\frac{k_1k_2}{k_1+k_2}
\end{equation}
或者
\begin{equation}
\frac{1}{k}=\frac{1}{k_1}+\frac{1}{k_2}
\end{equation}

现在,我们用数学归纳法推导 $n(n\in \mathbb N)$ 个弹簧的串联公式\autoref{Spring_eq2} .

1)对于$n=2$的情形,已由上面所证明;

2)假设对任意给定的 $n=l$,\autoref{Spring_eq2} 成立,那么对 $n=l+1$情形,可看成前 $l$ 个弹簧串联的弹簧系统(劲度系数 $k_l$ )与第 $l+1$ 个弹簧(劲度系数$ k_{l+1}$ )串联的情形.即对于 $n=l+1$ 个弹簧串联的系统,其劲度系数 $k$ 满足
\begin{equation}\label{Spring_eq1}
\frac{1}{k}=\frac{1}{k_l}+\frac{1}{k_{l+1}}
\end{equation}

由假设
\begin{equation}
\frac{1}{k_l}=\sum\limits_{i=1}^{l}\frac{1}{k_i}
\end{equation}
代入\autoref{Spring_eq1} ,得
\begin{equation}
\frac{1}{k_l}=\sum\limits_{i=1}^{l+1}\frac{1}{k_i}
\end{equation}

\subsubsection{弹簧的并联}
\subsection{弹簧的切割}
如果把一根均匀弹簧切割成原长的 $\lambda$ ($\lambda < 1$)倍, 那么它的劲度系数变为
\begin{equation}\label{Spring_eq4}
k' = \frac{k}{\lambda}
\end{equation}

证明: 我们可以把弹簧原长分割成 $n$ 等分, 由于弹簧是均匀的, 每份的劲度系数都为 $k_0$, 那么根据\autoref{Spring_eq2} 有
\begin{equation}
k_0 = nk
\end{equation}
然后再把其中连续的 $m$ ($m < n$)等分串联, 有
\begin{equation}
k' = \frac{n}{m}k
\end{equation}
由于以上的 $m, n$ 可以任取, 我们可以使 $m/n \to x$ (当 $x$ 是有理数时取等号).所以有\autoref{Spring_eq4}.

\begin{example}{}
一根弹性绳劲度系数为 $k$, 固定在水平相距为 $L$ 的两点之间, 绳子原长远小于 $L$. 在距离绳一端$x$处固定一个质点, 质点受重力下沉后使其平衡静止, 求下沉的深度 $h$.
\begin{figure}[ht]
\centering
\includegraphics[width=10cm]{./figures/Spring_1.pdf}
\caption{受力分析} \label{Spring_fig1}
\end{figure}
假设质点左边部分的原长占总原长的比例为 $\lambda$, 右边部分的原长占 $1-\lambda$, 则有
\begin{equation}
\leftgroup{
&\frac{k_1}{k_2} = \frac{1-\lambda}{\lambda}\\
&\frac{1}{k} = \frac{1}{k_1} + \frac{1}{k_2}
}
\end{equation}
根据\autoref{Spring_eq4} 
\begin{equation}
k_1 = \frac{k}{\lambda} \qquad
k_2 = \frac{k}{1-\lambda}
\end{equation}
受力分析
\begin{equation}
\leftgroup{
&T_1\sin\theta_1 + T_2\sin\theta_2 = mg\\
&T_1\cos\theta_1 = T_2\cos\theta_2
}
\end{equation}
其中
\begin{equation}
\tan\theta_1 = \frac{h}{x}
\qquad
\tan\theta_2 = \frac{h}{L-x}
\end{equation}
\begin{equation}
T_1 = \frac{x}{\cos\theta_1} k_1 \qquad
T_2 = \frac{L-x}{\cos\theta_2} k_2
\end{equation}
解得
\begin{equation}
h = \frac{mg}{Lk\qty(\frac{1}{x} + \frac{1}{L-x})}
\end{equation}
\end{example}
