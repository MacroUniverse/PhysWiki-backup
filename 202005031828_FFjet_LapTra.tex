% 拉普拉斯变换

拉普拉斯变换是用来干什么的呢?拉普拉斯变换常用于初始值问题,即已知某个物理量在初始时刻$t=0$的值$f(0)$,而求解它在初始时刻之后的变化情况$f(t)$.至于它在初始时刻之前的值,我们就让它都等于$0$,也就是说
\begin{equation}
f(t)=0 \quad(t<0)
\end{equation}
为了获得较宽的变换条件,构造一个函数$g(t)$,
\begin{equation}
g(t)=\mathrm{e}^{-\sigma t} f(t)
\end{equation}
这里$e^{-\sigma t}$为收敛因子.我们需要选一个充分大的正实数$\sigma$,用来保证$g(t) $在区间$(-\infty,+\infty)$上绝对可积.于是,可以对$g(t) $做傅里叶变换:
\begin{equation}
G(\omega)=\frac{1}{2 \pi} \int_{-\infty}^{\infty} g(t) \mathrm{e}^{-\omega t} \mathrm{d} t=\frac{1}{2 \pi} \int_{0}^{\infty} f(t) \mathrm{e}^{-(\sigma+\mathrm{i} \omega) t} \mathrm{d} t
\end{equation}
将$\sigma+i \omega$记作$p$,并将$G(\omega)$改记作$\bar f(p) / 2 \pi$则
\begin{equation}
\bar{f}(p)=\int_{0}^{\infty} f(t) \mathrm{e}^{-p t} \mathrm{d} t
\end{equation}