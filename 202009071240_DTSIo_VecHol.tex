% 和乐群 (向量丛)
\pentry{联络 (向量丛)\upref{VecCon}, 曲率 (向量丛)\upref{VecCur}, 平行性 (向量丛)\upref{VecPar}}

本节采用爱因斯坦求和约定.

设$M$是$n$维微分流形, $E$是其上秩为$k$的光滑向量丛. 设给定了$E$上的联络$D$.

\subsection{和乐群}
固定一点$p\in M$. 考虑纤维$E_p$上所有形如$P_\gamma:E_p\to E_p$的平行移动, 其中$\gamma$是起终点都在$p$处的$C^1$道路. 容易验证:$P_\gamma\cdot P_\eta=P_{\gamma\cdot\eta}$,
其中$\gamma\cdot\eta$表示道路的复合. 也容易验证沿着常值道路的平行移动是恒等变换, 以及$P_\gamma^{-1}=P_{\gamma^{-1}}$. 因此所有形如这样的变换构成了$E_p$上的一般线性群$GL(E_p)$的子群, 叫做$D$的基点在$p$处的\textbf{和乐群 (holonomy group)}, 常记为$\text{Hol}_p(D)$. 

如果$q\in M$是另外一点, 那么$\text{Hol}_q(D)$与$\text{Hol}_p(D)$是同构的: 只需要构造任何连接$p$和$q$的光滑道路就可以了. 如果限于考虑沿着零伦的闭道路的平行移动构成的群, 则得到\textbf{限制和乐群 (restricted holonomy group)}, 记为$\text{Hol}^0_p(D)$. 容易看出商群$\text{Hol}_p(D)/\text{Hol}_p^0(D)$可作为基本群$\pi_1(M)$的商群.

群$\text{Hol}_p(D)$和$\text{Hol}_p^0(D)$都作为一般线性群$GL(E_p)$的子群赋予拓扑. 如果$\gamma:S^1\to M$是基点为$p$的零伦闭道路, 那么存在同伦$\gamma_u(t):[0,1]\times S^1\to M$使得$\gamma_0(t)=\gamma(t)$, $\gamma_1(t)\equiv p$. 则每一个$P_{\gamma_u}$都是$\text{Hol}_p^0(D)$的元素, 而且根据常微分方程的基本理论, $u\to P_{\gamma_u}$是连接$P_\gamma$和$\text{id}$的道路. 这样一来$\text{Hol}_p^0(D)$是$GL(E_p)$的道路连通子群. 山边英彦 (Yamabe) 定理说: 李群的道路连通子群是李子群, 因此\textbf{$\text{Hol}_p^0(D)$是$GL(E_p)$的李子群}. 同理$\text{Hol}_p(D)$也是李子群. $\text{Hol}_p^0(D)$实际上是$\text{Hol}_p^0(D)$的单位分支.

\subsection{和乐定理}
$\text{Hol}_p^0(D)$和$\text{Hol}_p(D)$的李代数是一样的, 因为二者的商群是可数的离散群. $\text{Hol}_p^0(D)$的李代数叫做联络$D$的基点为$p$的\textbf{和乐代数 (holonomy algebra)}, 记为$\mathfrak{h}_p(D)$.  它是$\text{End}(E_x)$的李子代数. 

嘉当 (E. Cartan) 给出了曲率算子的一个直观解释, 揭示了曲率算子与和乐群之间的关系. 固定两个切向量$X,Y\in T_pM$. 考虑从二维三角形$\Delta_2$ (二维方块$[0,1]^2$的左下角) 到$M$的光滑映射$f:\Delta_2\to M$, 使得$f(0,0)=p$, $\partial_xf(0,0)=X$, $\partial_yf(0,0)=Y$. 令$f_u(x,y):=f(ux,uy)$, 这给出了从周线$f_{\partial\Delta_2}:\partial\Delta_2\to M$到点$p$的一个收缩同伦. 设$t$是周线$f_{\partial\Delta_2}$的弧长参数, 并以$\gamma_u$表示参数为$u$的同伦闭道路. 则$\gamma_u(t)=f(ux(t),uy(t))$, 而且容易算出
$$
\left.\frac{d}{du}P_u\right|_{u=0}=0,\,\left.\frac{d^2}{du^2}P_u\right|_{u=0}=-2R(X,Y).
$$
粗略来说, 这表示"绕小周线一周"得到的平行移动算子实际上是由曲率算子给出的. 将这个命题精细化, 就可看出和乐群$\text{Hol}_p^0(D)$的无穷小生成元 (也就是$\mathfrak{h}_p(D)$的元素) 是曲率算子. 由此可得到和乐定理 (holonomy theorem):

\begin{theorem}{Ambrose-Singer 和乐定理, 向量丛版本}
设$R$是联络$D$的曲率算子. 则对于任何一点$p\in M$, 和乐代数$\mathfrak{h}_p(D)$是由变换$\{R(X,Y):X,Y\in T_pM\}$的集合生成的.
\end{theorem}