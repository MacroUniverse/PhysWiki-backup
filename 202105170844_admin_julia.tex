% Julia 笔记

\begin{issues}
\issueDraft
\end{issues}

\begin{lstlisting}[language=julia]
println("hello world")
\end{lstlisting}

\begin{lstlisting}[language=julia]
function sphere_vol(r)
    return 4/3*pi*r^3
end
\end{lstlisting}

\subsection{变量类型}



Julia 是动态类型的语言. 具体类型不能互相作为子类, 只能作为抽象类的子类.

可以在 literal, 变量或者表达式后面加上 \verb|::变量类型| 用于确认它具有该类型, 如果类型不符会产生异常. 如果 \verb|变量类型| 是抽象的, 那么表达式只需要是它的一个 sub type.

\verb|Int8|, \verb|UInt8|, \verb|Int16|, \verb|UInt16|, \verb|Int32|, \verb|UInt32|, \verb|Int64|, \verb|UInt64|, \verb|Int128|, \verb|UInt128|, \verb|Float16|, \verb|Float32|, \verb|Float64|, \verb|ComplexF64|

\verb|ComplexF16|, \verb|ComplexF32| 和 \verb|ComplexF64| 是 \verb|Complex{Float16}|, \verb|Complex{Float32}| 和 \verb|Complex{Float64}| 的别名.

\verb|BigInt| 是任意大小的整数, \verb|BigFloat| 是任意精度浮点数.

\begin{lstlisting}[language=julia]
BigFloat(2.1) # 2.1 here is a Float64

BigFloat("2.1") # the closest BigFloat to 2.1

BigFloat("2.1", RoundUp)

julia> BigFloat("2.1", RoundUp, precision=128)
\end{lstlisting}

随机数
\begin{lstlisting}[language=julia]
rand(ComplexF64, Nr1, Nr2, Npw)
\end{lstlisting}

hash
\begin{lstlisting}[language=julia]
hash(矩阵)
\end{lstlisting}

当前时间 \verb|time()|, 零向量 \verb|zeros(整数)|

矩阵切割 \verb|Psi[:, j, :]|

\verb|size(Psi, 维度)|
