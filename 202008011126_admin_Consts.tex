% 国际单位制
% 物理常数|国际单位|测量

% 未完成: 只是写了基本单位的定义, 还要提及 “物理单位前缀\upref{UniPre}”

\textbf{国际单位制(SI Units)}.

物理单位的一个重要性质就是可测量, 至少理论上可测量. 以下的数值除了有特殊说明, 都是精确值(无限位小数用省略号表示), 不存在误差.

\subsubsection{秒(s)}
铯原子133基态的超精细能级之间的跃迁辐射的电磁波周期的 $9, 192, 631, 770$ 倍. 

说明: 我们知道原子中的电子具有不同的能级, 当电子从一个能级跃迁到一个更低的能级时, 会放出一个光子. 光子的频率为 $\nu  = \varepsilon /h$,   其中 $\varepsilon $ 是光子的能量, $h$ 为普朗克常数.

\subsubsection{米(m)}
真空中, 光在 $1/299792458$ 秒内传播的距离.

说明: 由于真空中的光速是物质和信息能传播的最快速度(见狭义相对论\upref{SpeRel} 相关内容), 且在任何参考系中都相同, 所以可以作为一个精确的标准. 结合秒的定义, 就可以通过实验得到一米的长度. 根据米的定义, 一秒中光可以在真空中传播
\begin{equation}
c = 299792458 \Si{m/s}
\end{equation}

\subsubsection{千克(kg)}
使得普朗克常数精确等于 $\hbar = 6.62607015\times10^{-34}\Si{Js}$.

说明: 2019 年 5 月开始, 千克根据普朗克常数定义(见量子力学\upref{QMIntr}相关内容). 这个定义可以类比“米”的定义(使光速精确地等于 $299792459\Si{m/s}$): $\hbar$ 的单位 $\Si{Js}$ 也可以表示为 $\Si{kg\cdot m^2/s}$, 我们已经定义了“米”和“秒”, 所以通过测量普朗克常数, 我们就可以定义千克.

历史上, 千克最初在 1795 年被定义为一升水的质量, 但在实际操作中会遇到许多困难使结果不太精确. 1799 年使用国际公斤原器的质量来定义, 并复制若干份分别存放, 但经过长时间后被发现和复制品存在细微误差.

\subsubsection{牛顿(N)}
等于使 $1\Si{kg}$ 物体获得 $1\Si{m/s^2}$ 加速度的力.

说明: 该定义符合牛顿第二定律(\autoref{New3_eq1}~\upref{New3}).

\subsubsection{焦耳(J)}
$1\Si{N}$ 的恒力将受力物体沿力的方向移动 $1\Si{m}$, 做功为 $1\Si{J}$.

\subsubsection{库仑(C)}
若真空中两个相同的点电荷相距一米, 产生的相互作用力数值上等于 $1/(4\pi\epsilon_0)$, 则该点电荷为 1 库仑. 其中真空介电常数精确地定义为
\begin{equation}
\epsilon_0 = \frac{1}{c^2 \mu_0} = 8.8541878176\dots \times 10^{-12} \Si{F/m}
\end{equation}
其中 $c$ 是真空中的光速, $\mu_0$ 是真空中的磁导率, 精确定义为
\begin{equation}
\mu_0 = 4\pi \times 10^{-7} \Si{N/A^2}
\end{equation}

说明: 1. 该定义符合库伦定律(\autoref{ClbFrc_eq1}~\upref{ClbFrc}). 2. $\epsilon_0$ 和 $\mu_0$ 的定义使得 $c = 1/\sqrt{\epsilon_0 \mu_0}$ 精确成立.

\subsubsection{伏特(V)}
$1\Si{C}$ 的电荷增加 $1\Si{V}$ 电势, 需要 $1\Si{J}$ 的能量.

\subsubsection{法拉(F)}
法拉是电容量的单位, 一个 $1\Si{F}$ 的电容器两端施加 $1\Si{V}$ 电压, 可以储存 $1\Si{C}$ 净电荷.

\subsubsection{安培(A)}
以下两种定义精确等效:
\begin{enumerate}
\item 每秒钟经过横截面的电荷量为 1 库仑的电流就是 1 安培.
\item 两根相距一米的无限长平行细导线流入 1 安培电流后, 相互作用力是 $2 \times 10^{-7}\Si{N/m}$. 
\end{enumerate}

说明: 第二种定义的等效性可以由安培环路定理(\autoref{AmpLaw_ex1}~\upref{AmpLaw}) 以及安培力(\autoref{FAmp_eq1}~\upref{FAmp})证明.

\subsubsection{亨利(H)}
(未完成) 电感的单位.

\subsubsection{开尔文(K)}
开尔文温度的定义应该使得玻尔兹曼常数精确等于
\begin{equation}
k_B = 1.3806505\times 10^{−23} \Si{J/K}
\end{equation}
例如, 理想气体中分子平均动能为 $\ev{E_k} = 3 k_B T/2$, 理论上我们可以根据测其动能来定义温度.
