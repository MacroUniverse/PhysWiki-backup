% test100
$\mathrm{若五阶行列式的展开式中项}a_{13}a_{2k}a_{34}a_{42}a_{5l}\mathrm{带负号},\mathrm 则k,l\mathrm{的值分别为}(\;).$

A.$k=1,l=5$   B.$k=5,l=1$   C.$k=2,l=5$   D.$k=5,l=2$

E.   F.   G.   H.
$\begin{array}{l}\begin{array}{l}\mathrm{根据行列式的定义},k,l\mathrm{只能取}1或5,\\若k=5,l=1,则\tau(35421)=8,\end{array}\\\begin{array}{l}若k=1,l=5,则\tau(31425)=3,\\\mathrm{所以}k=1,l=5.\\\end{array}\end{array}$


$\mathrm{在四阶行列式的展开式中},\mathrm{下列各项中带正号的是}(\;).$

A.$a_{13}a_{24}a_{32}a_{41}$   B.$a_{14}a_{23}a_{32}a_{41}$   C.$a_{14}a_{22}a_{33}a_{41}$   D.$a_{11}a_{23}a_{32}a_{44}$

E.   F.   G.   H.
$a_{14}a_{23}a_{32}a_{41}\mathrm{的列标的逆序数为}1+2+3=6,\mathrm{故此项前边带有正号}.$


$\mathrm{下列哪一个不是五阶行列式的展开式中的项}(\;).$

A.$a_{21}a_{13}a_{34}a_{55}a_{42}$   B.$a_{21}a_{13}a_{24}a_{55}a_{42}$   C.$a_{11}a_{23}a_{34}a_{55}a_{42}$   D.$a_{13}a_{34}a_{51}a_{42}a_{25}$

E.   F.   G.   H.
$a_{21}a_{13}a_{24}a_{55}a_{42}\mathrm{在第二行取了两个元素},\mathrm{与行列式定义不符}.$


$在n\;\mathrm{阶行列式中},\mathrm{关于主对角线与元素}a_{ij}\mathrm{对称的元素为}(\;).$

A.$a_{ij}$   B.$a_{ji}$   C.$a_{(i-1)(j-1)}$   D.$a_{(j-1)(i-1)}$

E.   F.   G.   H.
$\mathrm{由行列式定义知},\mathrm{与元素}a_{ij}\mathrm{关于主对角线对称的元素行列刚好互换},\mathrm{即为}a_{ji}.$


$若a_{1i}a_{23}a_{35}a_{4j}a_{54}\mathrm{为五阶行列式的展开式中带正号的一项},则\;i,j\;\mathrm{分别为}(\;).$

A.$i=1,j=2$   B.$i=3,j=2$   C.$i=2,j=1$   D.$i=2,j=3$

E.   F.   G.   H.
$\begin{array}{l}\begin{array}{l}\mathrm{由题意可知},i,j\mathrm{只能分别取值}1,2\\\mathrm{不妨设}i=1,j=2,\mathrm{则排列}13524\mathrm{的逆序数为}3,\mathrm{故此排列为奇排列},\mathrm{对应项带负号},\mathrm{不合题意},\end{array}\\\mathrm{因此}i=2\;,j=1.\end{array}$


$若-a_{32}a_{1r}a_{25}a_{4s}a_{53}\mathrm{是五阶行列式展开式中的一项},则\;r,s\;\mathrm{的值分别为}(\;).$

A.$r=1,s=1$   B.$r=1,s=4$   C.$r=4,s=1$   D.$r=4,s=4$

E.   F.   G.   H.
$\begin{array}{l}r和s\mathrm{只能取值为}1,4\\当r=4,s=1时,\mathrm{判断}a_{32}a_{14}a_{25}a_{41}a_{53}即a_{14}a_{25}a_{32}a_{41}a_{53}\mathrm{的列标的逆序数为}7(\mathrm{奇数}),\mathrm{故其项的符号为负}.\end{array}$


$若a_{13}a_{2k}a_{34}a_{42}a_{5l}\mathrm{是五阶行列式的展开式中带有正号的项},则\;k,l\mathrm{分别为}(\;).$

A.$k=1,l=5$   B.$k=5,l=1$   C.$k=1,l=4$   D.$k=4,l=1$

E.   F.   G.   H.
$\mathrm{根据行列式的定义},k\;,l\mathrm{只能取值}1或5,若k=5,l=1,则\tau(\;35421)=8,若k=1,l=5,则\tau(\;31425)=3,\mathrm{所以}k=5,l=1.$


$\mathrm{六阶行列式的展开式中共有}(\;)项.$

A.$36$   B.$720$   C.$240$   D.$160$

E.   F.   G.   H.
$\mathrm{由行列式的定义可知}n\mathrm{阶行列式表示所有取自不同行不同列}n\mathrm{个元素乘积的代数和},即n!\mathrm{项的代数和}.$


$\mathrm{在四阶行列式的展开式中},\mathrm{下列带负号的是}(\;).$

A.$a_{13}a_{24}a_{32}a_{41}$   B.$a_{14}a_{23}a_{32}a_{41}$   C.$a_{12}a_{24}a_{33}a_{41}$   D.$a_{13}a_{21}a_{32}a_{44}$

E.   F.   G.   H.
$a_{13}a_{24}a_{32}a_{41}\mathrm{的列标的逆序数为}2+3=5,\mathrm{所以此项前面带有负号}.$


$\mathrm{四阶行列式展开式中带负号且包含元素}a_{12}和a_{21}\mathrm{的项为}(\;).$

A.$a_{12}a_{21}a_{33}a_{44}$   B.$a_{12}a_{21}a_{34}a_{43}$   C.$a_{12}a_{21}a_{32}a_{44}$   D.$a_{12}a_{21}a_{33}a_{41}$

E.   F.   G.   H.
$\mathrm{根据行列式的定义},\mathrm{四阶行列式中包含元素}a_{12}和a_{21}\mathrm{的项为}a_{12}a_{21}a_{33}a_{44}或a_{12}a_{21}a_{34}a_{43},而a_{12}a_{21}a_{33}a_{44}\mathrm{带有负号}.$


$n\mathrm{阶行列式}\begin{vmatrix}a_1&0&\cdots&0\\0&a_2&\cdots&0\\\cdots&\cdots&\cdots&0\\0&0&\cdots&a_n\end{vmatrix}=(\;).$

A.$0$   B.$1$   C.$a_1+a_2+\cdots+a_n$   D.$a_1a_2\cdots a_n$

E.   F.   G.   H.
$\mathrm{根据行列式的定义},\begin{vmatrix}a_1&0&\cdots&0\\0&a_2&\cdots&0\\\cdots&\cdots&\cdots&0\\0&0&\cdots&a_n\end{vmatrix}=a_1a_2\cdots a_n.$


$\mathrm{四阶行列式}\begin{vmatrix}0&0&2&0\\-1&0&3&0\\6&0&0&-3\\2&2&5&4\end{vmatrix}=().$

A.$0$   B.$4$   C.$12$   D.$-12$

E.   F.   G.   H.
$\mathrm{由行列式的定义可知}\begin{vmatrix}0&0&2&0\\-1&0&3&0\\6&0&0&-3\\2&2&5&4\end{vmatrix}=(-1)^{\tau(3142)}\;\;2\times(-1)\times(-3)\times2=-12.$


$\mathrm{设多项式}f(x)=\begin{vmatrix}x&x&0&2\\3&x&1&1\\2&1&x&0\\5&4&2&x\end{vmatrix},则f(x)中x^4\mathrm{的系数为}(\;).$

A.$1$   B.$3$   C.$2$   D.$4$

E.   F.   G.   H.
$\mathrm{根据行列式定义可以得出},含x^4\mathrm{的项只有主对角线元素的乘积},\mathrm{因此系数为}1.$


$n\mathrm{阶行列式}\begin{vmatrix}1&0&0&...&0\\0&2&0&...&0\\0&0&3&...&0\\...&...&...&...&...\\0&0&0&...&n\end{vmatrix}=(\;).$

A.$n!$   B.$0$   C.$n$   D.$-n!$

E.   F.   G.   H.
$\mathrm{原式}=1\times2\times3....\times n=n!.$


$\mathrm{四阶行列式}D=\begin{vmatrix}a_{11}&a_{22}&a_{32}&a_{13}\\a_{21}&a_{12}&a_{31}&a_{14}\\a_{33}&a_{41}&a_{24}&a_{34}\\a_{44}&a_{43}&a_{42}&a_{23}\end{vmatrix},\mathrm{下列哪一个不是}D\mathrm{的展开式中的项}().$

A.$a_{11}a_{22}a_{33}a_{44}$   B.$a_{32}a_{12}a_{44}a_{34}$   C.$-a_{21}a_{22}a_{23}a_{24}$   D.$a_{11}a_{12}a_{23}a_{24}$

E.   F.   G.   H.
$a_{11}a_{22}a_{33}a_{44}\mathrm{不是}D\mathrm{中的项},\mathrm{因为}a_{11},a_{22}\mathrm{在同一行}.$


$设\begin{vmatrix}-1&1&1\\1&-1&x\\1&1&-1\end{vmatrix}\mathrm{是关于}x\mathrm{的一次多项式},\mathrm{该多项式一次项系数为}(\;).$

A.$-1$   B.$1$   C.$-2$   D.$2$

E.   F.   G.   H.
$含x\mathrm{的项为}x(-1)^{2+3}\begin{vmatrix}-1&1\\1&1\end{vmatrix}=2x,\mathrm{故一次项系数为}2.$


$\mathrm{五阶行列式的展开式中},\mathrm{带负号的项是}(\;).$

A.$a_{12}a_{25}a_{31}a_{43}a_{54}$   B.$a_{12}a_{31}a_{25}a_{54}a_{43}$   C.$a_{11}a_{22}a_{33}a_{44}a_{55}$   D.$a_{12}a_{24}a_{31}a_{43}a_{55}$

E.   F.   G.   H.
$\mathrm{对于项}a_{12}a_{24}a_{31}a_{43}a_{55},\tau(\;24135)=3.$


$\mathrm{四阶行列式}\begin{vmatrix}0&0&1&0\\0&1&0&0\\0&0&0&1\\1&0&0&0\end{vmatrix}=().$

A.$0$   B.$1$   C.$-1$   D.$-4$

E.   F.   G.   H.
$\begin{vmatrix}0&0&1&0\\0&1&0&0\\0&0&0&1\\1&0&0&0\end{vmatrix}=(-1)^{\tau(3241)}a_{13}a_{22}a_{34}a_{41}=1.$


$\mathrm{四阶行列式}\begin{vmatrix}0&0&0&a\\0&0&b&0\\0&c&0&0\\d&0&0&0\end{vmatrix}=(\;).$

A.$0$   B.$a$   C.$abcd$   D.$-abcd$

E.   F.   G.   H.
$\begin{vmatrix}0&0&0&a\\0&0&b&0\\0&c&0&0\\d&0&0&0\end{vmatrix}=(-1)^{\tau(4321)}abcd=(-1)^6abcd=abcd.$


$\mathrm{设多项式}f(x)=\begin{vmatrix}2x&3&1&2\\x&x&0&1\\2&1&x&4\\x&2&1&4x\end{vmatrix},则f(x)中x^4\mathrm{系数为}(\;).$

A.$2$   B.$4$   C.$8$   D.$6$

E.   F.   G.   H.
$\mathrm{根据行列式定义可以得出},含x^4\mathrm{的项只能是主对角线上元素的乘积},\mathrm{因此系数为}2\times4=8.$


$\mathrm{四阶行列式}\begin{vmatrix}a&0&0&0\\0&0&b&0\\0&c&0&0\\0&0&0&d\end{vmatrix}=(\;).$

A.$0$   B.$1$   C.$abcd$   D.$-abcd$

E.   F.   G.   H.
$\begin{vmatrix}a&0&0&0\\0&0&b&0\\0&c&0&0\\0&0&0&d\end{vmatrix}=(-1)^{\tau(1324)}abcd=-abcd.$


$\mathrm{四阶行列式的展开式中含有因子}a_{11}a_{23}\mathrm{的项为}(\;).$

A.$-a_{11}a_{23}a_{32}a_{44}和a_{11}a_{23}a_{34}a_{42}$   B.$a_{11}a_{23}a_{32}a_{44}和a_{11}a_{23}a_{34}a_{42}$   C.$-a_{11}a_{23}a_{32}a_{44}和-a_{11}a_{23}a_{34}a_{42}$   D.$a_{11}a_{23}a_{32}a_{44}和-a_{11}a_{23}a_{34}a_{42}$

E.   F.   G.   H.
$\begin{array}{l}\mathrm{由定义知},\mathrm{四阶行列式的一般项为}:(-1)^\tau a_{1p_1}a_{2p_2}a_{3p_3}a_{4p_4},\mathrm{其中}\tau 为p_1p_2p_3p_4\mathrm{的逆序数},\mathrm{由于}p_1=1,p_2=3\mathrm{已固定},p_1p_2p_3p_4\mathrm{只能形如}13xx,即1324或1342,\\其\tau\mathrm{分别为}0+0+1+0或0+0+0+2=2,\mathrm{所以}-a_{11}a_{23}a_{32}a_{44}和a_{11}a_{23}a_{34}a_{42}\mathrm{为所求}.\\\end{array}$


$\mathrm{多项式}f(x)=\begin{vmatrix}x&3&1&2\\x&2x&0&1\\2&1&x&4\\0&2&1&x\end{vmatrix},则f(x)中x^4\mathrm{系数为}(\;).$

A.$2$   B.$1$   C.$4$   D.$8$

E.   F.   G.   H.
$\mathrm{由于行列式是不同行不同列元素乘积的和},\mathrm{故含有}x^4\mathrm{项只有行列式主对角线上元素的乘积},\mathrm{即系数为}2.$


$若n\mathrm{阶行列式中有}n^2-n\mathrm{个以上的元素为零},\mathrm{则该行列式为}(\;).$

A.$0$   B.$n^2-n$   C.$n$   D.$\mathrm{无法确定}$

E.   F.   G.   H.
$\begin{array}{l}\mathrm{如果}n\mathrm{阶行列式中有}n^2-n\mathrm{个以上元素为零},\mathrm{则至多有}n-1\mathrm{个不为零元素}\\\mathrm{由于}n\;\mathrm{阶行列式的每一项为}n\mathrm{个不同元素的乘积},\mathrm{从而每一项均为零},\mathrm{故该行列式为为零}.\end{array}$


$\mathrm{四阶行列式}\begin{vmatrix}0&0&1&0\\0&1&0&0\\0&0&0&1\\1&0&0&0\end{vmatrix}和\begin{vmatrix}1&1&1&0\\0&1&0&1\\0&1&1&1\\0&0&1&0\end{vmatrix}\mathrm{的值分别为}(\;).$

A.$1,0$   B.$0,0$   C.$1,1$   D.$0,1$

E.   F.   G.   H.
$\begin{array}{l}(1)\begin{vmatrix}0&0&1&0\\0&1&0&0\\0&0&0&1\\1&0&0&0\end{vmatrix}=(-1)^{\tau(3241)}=1,\\(2)\begin{vmatrix}1&1&1&0\\0&1&0&1\\0&1&1&1\\0&0&1&0\end{vmatrix}=(-1)^{\tau(1243)}+(-1)^{\tau(1423)}=0.\end{array}$


$设\begin{vmatrix}x&x&1\\2x&x&1\\3&2&x\end{vmatrix},\mathrm{则展开式中}x^2\;\mathrm{的系数为}(\;).$

A.$1$   B.$-1$   C.$0$   D.$2$

E.   F.   G.   H.
$\begin{vmatrix}x&x&1\\2x&x&1\\3&2&x\end{vmatrix}=\begin{vmatrix}x&x&1\\0&-x&-1\\3&2&x\end{vmatrix}=\begin{vmatrix}x&0&1\\0&-x&-1\\3&-1&x\end{vmatrix}\mathrm{展开式不会出现}x^2.$


$\mathrm{设多项式}f(x)=\begin{vmatrix}3&-1&x\\x&2&5\\1&4&x\end{vmatrix},则f(x)是()\mathrm{次多项式}.$

A.$0$   B.$1$   C.$2$   D.$3$

E.   F.   G.   H.
$f(x)=\begin{vmatrix}3&-1&x\\x&2&5\\1&4&x\end{vmatrix}=\begin{vmatrix}3&-1&x\\x&2&5\\-2&5&0\end{vmatrix}取a_{13}a_{21}a_{32}\mathrm{就会出现}x\mathrm{的最高次数}2.$


$\mathrm{用行列式的定义计算五阶行列式}\begin{vmatrix}a_{11}&a_{12}&a_{13}&a_{14}&a_{15}\\a_{21}&a_{22}&a_{23}&a_{24}&a_{25}\\a_{31}&a_{32}&0&0&0\\a_{41}&a_{42}&0&0&0\\a_{51}&a_{52}&0&0&0\end{vmatrix}=(\;).$

A.$1$   B.$a_{11}a_{12}$   C.$0$   D.$a_{11}a_{22}a_{33}a_{44}a_{55}$

E.   F.   G.   H.
$\begin{array}{l}\mathrm{由于在五阶行列式的展开式中每一项的五个元素均分布在不同的行和不同的列},\mathrm{显然任意的这样五个元素中都至少有一个为零}\\\mathrm{从而行列式的每一项为零},\mathrm{故所求行列式为零}.\end{array}$


$\mathrm{用行列式的定义计算五阶行列式}\begin{vmatrix}0&a_{12}&a_{13}&0&0\\a_{21}&a_{22}&a_{23}&a_{24}&a_{25}\\a_{31}&a_{32}&a_{33}&a_{34}&a_{35}\\0&a_{42}&a_{43}&0&0\\0&a_{52}&a_{53}&0&0\end{vmatrix}=(\;\;\;).$

A.$0$   B.$a_{11}a_{22}a_{33}a_{44}a_{55}$   C.$-a_{11}a_{22}a_{33}a_{44}a_{55}$   D.$a_{22}a_{33}$

E.   F.   G.   H.
$\begin{array}{l}设D\mathrm{中第}1,2,3,4,5\mathrm{行的元素分别为}a_{1p_1},a_{2p_2},a_{3p_3},a_{4p4},a_{5p_5},\mathrm{则由}D\mathrm{中第}1,2,3,4,5\mathrm{行可能的非零元素分别得到}p_1=2,3;p_2=1,2,3,4,5;p_3=1,2,3,4,5\\p_4=2,3;p_5=2,3;\;p_1,p_2,p_3,p_4,p_5\mathrm{在上述可能的代码中},\mathrm{一个}5\mathrm{元排列也不能组成},故D=0.\\\\\end{array}$


$\mathrm{设多项式}f(x)=\begin{vmatrix}3&-1&x\\x&2&5\\1&4&x\end{vmatrix},则f(x)\mathrm{一次项的系数为}(\;).$

A.$-2$   B.$-4$   C.$2$   D.$4$

E.   F.   G.   H.
$\mathrm{一次项为}(-1)^{\tau(321)}x\times2\times1+3\times2\times x=4x,\mathrm{故系数为}4.$


$\mathrm{按行列式的定义计算}n\mathrm{阶行列式}\begin{vmatrix}0&0&...&0&a_{1n}\\0&0&...&a_{2(n-1)}&a_{2n}\\...&...&...&...&...\\0&a_{(n-1)2}&...&a_{(n-1)(n-1)}&a_{(n-1)n}\\a_{n1}&a_{n2}&...&a_{n(n-1)}&a_{nn}\end{vmatrix}=(\;\;\;).$

A.$0$   B.$-a_{1n}a_{2(n-1)}a_{3(n-2)}....a_{(n-1)2}a_{n1}$   C.$a_{1n}a_{2(n-1)}a_{3(n-2)}....a_{(n-1)2}a_{n1}$   D.$(-1)^{\textstyle\frac{n(n-1)}2}{\textstyle{}^{}}a_{1n}a_{2(n-1)}...a_{(n-1)2}a_{n1}$

E.   F.   G.   H.
$\mathrm{原式}=(-1)^{\tau((n-1)....123)}a_{1n}a_{2(n-1)}...a_{(n-1)2}a_{n1}=(-1)^\frac{n(n-1)}2a_{1n}a_{2(n-1)}...a_{(n-1)2}a_{n1}.$


$n\mathrm{阶行列式}\begin{vmatrix}0&0&...&0&1\\0&0&...&2&0\\...&...&...&...&...\\0&n-1&...&0&0\\n&0&...&0&0\end{vmatrix}=(\;).$

A.$n!$   B.$-n!$   C.$(-1)^{{\textstyle\frac12}n(n-1)}n!$   D.$(-1)^{\textstyle(n-1)}n!$

E.   F.   G.   H.
$\begin{array}{l}\mathrm{行列式的展开式中除了}(-1)^{\textstyle\tau(n(n-1)....321)}n!外,\mathrm{其余各项都为零},\mathrm{由于}(-1)^{\textstyle\tau(n(n-1)....321)}n!=(-1)^{{\textstyle\frac12}n(n-1)}n!,\mathrm{故原行列式}\\\mathrm{等于}(-1)^{\textstyle\frac12n(n-1)}n!.\end{array}$


$\mathrm{设多项式}f(x)=\begin{vmatrix}2x&x&1&2\\1&2x&1&-1\\3&2&x&1\\1&1&1&x\end{vmatrix},则f(x)中x^4与x^3\mathrm{的系数分别为}(\;).$

A.$4,1$   B.$4,-1$   C.$1,2$   D.$1,-2$

E.   F.   G.   H.
$\begin{array}{l}含x^4\mathrm{的项只能由}a_{11}a_{22}a_{33}a_{44}\mathrm{组成},\mathrm{其系数为}4,\\含x^3\mathrm{的项只能由}a_{12}a_{21}a_{33}a_{44}\mathrm{组成},\mathrm{其系数为}-1.\end{array}$


$n\mathrm{阶行列式}\begin{vmatrix}a_{11}&...&...&a_{1(n-1)}&a_{1n}\\a_{21}&...&...&a_{2(n-1)}&0\\a_{31}&...&...&0&0\\...&...&...&...&...\\a_{n1}&...&...&0&0\end{vmatrix}=(\;\;).$

A.$(-1)^{{\textstyle\frac12}n(n-1)}a_{1n}a_{2(n-1)}....a_{n1}$   B.$a_{1n}a_{2(n-1)}....a_{n1}$   C.$-a_{1n}a_{2(n-1)}....a_{n1}$   D.$(-1)^na_{1n}a_{2(n-1)}....a_{n1}$

E.   F.   G.   H.
$\begin{array}{l}\mathrm{根据行列式的定义},\mathrm{行列式中唯一可能不为零的项为}(-1)^{\tau(n(n-1)....321)}a_{1n}a_{2(n-1)}....a_{n1}\\\tau(n(n-1)....321)=\frac{n(n-1)}2,\mathrm{故原行列式为}(-1)^\frac{n(n-1)}2a_{1n}a_{2(n-1)}....a_{n1}.\end{array}$


$\mathrm{行列式}D=\begin{vmatrix}0&0&...&0&1&0\\0&0&...&2&0&0\\...&...&...&...&...&...\\0&2015&...&0&0&0\\2016&0&...&0&0&0\\0&0&...&0&0&2017\end{vmatrix}=(\;\;\;).$

A.$-2017!$   B.$2017!$   C.$0$   D.$2017$

E.   F.   G.   H.
$\begin{array}{l}\mathrm{第一行的非零元素只有}a_{1,2016},故p_1\mathrm{只能取}2016,\mathrm{同理第},3,4,...2016\mathrm{可知}p_2=2001,p_3=2000...j_{2016}=1,p_{2017}=2017,\mathrm{于是在可能取的数码中},\\\mathrm{只能组成一个}2017\mathrm{级排列},\mathrm{故中非零项只有一项},即\;\;\;D=(-1)^{\tau(2016\;2015....2\;1\;2017)}\;a_{1,2016}a_{2,2015}\;.....a_{2016,1}a_{2017,2017}=2017!\;.\;\;\;\;\;\end{array}$


$\mathrm{四阶行列式}\begin{vmatrix}-1&0&x&1\\1&1&-1&-1\\1&-1&1&-1\\1&-1&-1&1\end{vmatrix}\;,\mathrm{则展开式中}x\mathrm{的系数为}(\;)\;.$

A.$-4$   B.$4$   C.$2$   D.$-2$

E.   F.   G.   H.
$\begin{array}{l}x\mathrm{的系数应为}(-1)^{1+3}\begin{vmatrix}1&1&-1\\1&-1&-1\\1&-1&1\end{vmatrix}=-4.\\\end{array}$


$n\mathrm{阶行列式}D_n=\begin{vmatrix}0&0&...&0&1&0\\0&0&...&2&0&0\\...&...&...&...&...&...\\n-1&0&...&0&0&0\\0&0&...&0&0&n\end{vmatrix}=(\;\;\;).$

A.$(-1)^\frac{(n-1)(n-2)}2\;\;\;n!$   B.$n!$   C.$(-1)^nn!$   D.$(-1)^\frac{(n+1)(n+2)}2\;\;n!$

E.   F.   G.   H.
$\begin{array}{l}D_n=(-1)^\tau a_{1(n-1)}a_{2(n-2)}...a_{nn}=(-1)^\tau n!,\mathrm{其中}\tau=\tau((n-1)(n-2)...21n)=0+1+2+....+(n-2)+0=\frac{(n-1)(n-2)}2\\故D_n=(-1)^\frac{(n-1)(n-2)}2n!.\end{array}$


$n\mathrm{阶行列式}\begin{vmatrix}a_{11}&&&&a_{1n}\\&a_{22}&&&\\&&...&&\\&&&...&\\0&&&&a_{nn}\end{vmatrix}=\begin{vmatrix}a_{11}&&&&0\\&a_{22}&&&\\&&...&&\\&&&...&\\a_{n1}&&&&a_{nn}\end{vmatrix}=(\;).$

A.$0$   B.$a_{11}a_{22}...a_{nn}$   C.$-a_{11}a_{22}...a_{nn}$   D.$1$

E.   F.   G.   H.
$\begin{vmatrix}a_{11}&&&&a_{1n}\\&a_{22}&&&\\&&...&&\\&&&...&\\0&&&&a_{nn}\end{vmatrix}=\begin{vmatrix}a_{11}&&&&0\\&a_{22}&&&\\&&...&&\\&&&...&\\a_{n1}&&&&a_{nn}\end{vmatrix}=a_{11}a_{22}...a_{nn}.$


$n\mathrm{阶行列式}\begin{vmatrix}0&&&&a_{1n}\\&&&a_{2(n-1)}&\\&&...&&\\&...&&&\\a_{n1}&&&&0\end{vmatrix}=(\;).$

A.$a_{1n}a_{2(n-1)}.....a_{n1}$   B.$-a_{1n}a_{2(n-1)}.....a_{n1}$   C.$(-1)^{\textstyle\frac{n(n-1)}2}a_{1n}a_{2(n-1)}.....a_{n1}$   D.$(-1)^{\textstyle\frac n2}a_{1n}a_{2(n-1)}.....a_{n1}$

E.   F.   G.   H.
$\begin{vmatrix}0&&&&a_{1n}\\&&&a_{2(n-1)}&\\&&...&&\\&...&&&\\a_{n1}&&&&0\end{vmatrix}=(-1)^{\tau(n(n-1)....1)}a_{1n}a_{2(n-1)}.....a_{n1}=(-1)^{\textstyle\frac{n(n-1)}2}a_{1n}a_{2(n-1)}.....a_{n1}.$


$\mathrm{双曲面}x^2-y^2/4-z^2/9=1\mathrm{与平面}y=4\mathrm{交线为}(\;).$

A.$\mathrm{双曲线}$   B.$\mathrm{椭圆}$   C.$\mathrm{抛物线}$   D.$\mathrm{一对相交直线}$

E.   F.   G.   H.
$将y=4\mathrm{代入到双曲面方程得}x^2-z^2/9=5,\mathrm{所以相交的曲线为双曲线}.$


$\mathrm{若五阶行列式的展开式中项}a_{13}a_{2k}a_{34}a_{42}a_{5l}\mathrm{带负号},\mathrm 则k,l\mathrm{的值分别为}(\;).$

A.$k=1,l=5$   B.$k=5,l=1$   C.$k=2,l=5$   D.$k=5,l=2$

E.   F.   G.   H.
$\begin{array}{l}\begin{array}{l}\mathrm{根据行列式的定义},k,l\mathrm{只能取}1或5,\\若k=5,l=1,则\tau(35421)=8,\end{array}\\\begin{array}{l}若k=1,l=5,则\tau(31425)=3,\\\mathrm{所以}k=1,l=5.\\\end{array}\end{array}$




