% 加速度规范

\begin{issues}
\issueDraft
\end{issues}

\pentry{速度规范\upref{LVgaug}}

本文使用原子单位制\upref{AU}以及偶极子近似\upref{DipApr}. 首先注意加速度规范并不是一种规范而只是薛定谔方程的一种变换, 说它是规范只是习惯上的叫法. 改变换也叫做 \textbf{Kramers-Henneberger 变换} 或 \textbf{K-H 变换}. 另一种说法是, 我们使用 \textbf{K-H 参考系}.

一个静止粒子在电磁波到来之前处于静止, 那么接下来它在电磁波作用下的位移为
\begin{equation}
\bvec \alpha(t) = q\int_{-\infty}^t \bvec A(t') \dd{t'}
\end{equation}
令所谓的
\begin{equation}
\chi(\bvec r, \bvec t) = \bvec \alpha(t) \vdot \bvec p
\end{equation}
注意这并不是一个函数而是算符, 

\begin{equation}
\Psi_V(\bvec r, t) = \E^{-\I \bvec \alpha \vdot \bvec p} \Psi_A(\bvec r, t)
\end{equation}
这里的 $\bvec p$ 应该是广义动量算符…… 但这岂不是需要无穷阶导数?

哈密顿量变为
\begin{equation}
H_A = -\frac{\bvec p^2}{2m} + V[\bvec r + \bvec \alpha(t)]
\end{equation}
其中 $V(\bvec r)$ 是不含时的势能函数(例如原子核对电子的库仑势能).

薛定谔方程为
\begin{equation}
H_A \Psi_A = \I \pdv{t} \Psi_A
\end{equation}
