% 密度矩阵(量子力学)

\pentry{矩阵的迹\upref{trace}, 投影算符\upref{projOp}}

\footnote{参考 Shankar, Principles of Quantum Mechanics 2ed}若一个系综中的 $N$ 个系统中, 有 $n_i$ ($i = 1,2,\dots,k$) 个在状态 $\ket{i}$ (这里假设 $\ket{i}$ 是正交归一的). 那么这个系综可以用\textbf{密度矩阵(density matrix)}(或算符)描述
\begin{equation}
\rho = \sum_i p_i\ket{i}\bra{i}   
\end{equation}
其中 $p_i = n_i/N$ 是随机选一个系统, 处于状态 $\ket{i}$ 的概率. 若所有系统都处于同一个 $\ket{i}$, 那么这个系综就是\textbf{纯的(pure)}, 否则就是\textbf{混合的(mixed)}.

对于某个物理量对应的算符 $\Omega$, 它的\textbf{系综平均值(ensemble average)}为
\begin{equation}
\ev{\bar\Omega} = \sum_i p_i \mel{i}{\Omega}{i}
\end{equation}
这个平均值既包含了每个 $\ket{i}$ 的平均, 又包含了对每个系统的平均.

系综平均也可以用迹表示为 $\opn{tr}(\Omega\rho)$. 根据迹的定义,
\begin{equation}
\opn{tr}(\Omega\rho) = \sum_j \mel{j}{\Omega\rho}{j} = \sum_{i,j} p_i\mel{j}{\Omega}{i} \braket{i}{j} = \sum_{i} p_i\mel{i}{\Omega}{i} = \ev{\bar\Omega}
\end{equation}
证毕.

对于纯态, 获得测量值 $\omega$ 的概率可以看作投影算符 $\mathbb P_\omega$ 的平均值
\begin{equation}
P(\omega) = \abs{\braket{\omega}{\psi}}^2 = \mel{\psi}{\mathbb P_\omega}{\psi}
\end{equation}
所以对于混合态, 测量值 $\omega$ 的概率为
\begin{equation}
\overline{P(\omega)} = \opn{tr}(\mathbb P_\omega\rho)
\end{equation}

\subsection{密度矩阵的性质}
\begin{itemize}
\item 密度矩阵是厄米算符(自伴算符)

\item 密度矩阵的迹为 1
\end{itemize}

