% 弦论的种类
\subsection{五种弦论}

弦论有五种.这五种弦论通过\textbf{对偶(duality)}联系在一起.这五种弦论分别是

1. 玻色弦论
这个弦论里面只有玻色子.没有超对称.因为这种弦论里面没有费米子,所以这种弦论不能描述物质.这只是一个玩具理论.它包含了开弦,闭弦.需要26维的时空.

2. 第一型弦论
这个版本的弦论包含了玻色子和费米子.粒子相互作用包含了超对称和规范群SO(32).这个理论有十维.

3. 第二A型弦论
这个版本的弦论同样包含了超对称,开弦和闭弦.第二A型弦论的端点能够attach到高维物体(D-膜)上面.这个理论里面费米子不是chiral的.

4. 第二B型弦论
跟二A型弦论差不多,只不过这种弦论里面的费米子是chiral的.

5. Heterotic弦论
包含了超对称,只允许闭弦.规范群是$E_8\times E_8$. 左行和右行模式需要不同的时空维度.Heterotic弦论有两种.

\subsection{M-理论}
不同的duality把这几种不同的弦论联系在了一起.这些duality的名字分别是S duality和T duality.所以一个大胆的猜测是,在这五种弦论的背后,还有一个更加基本的理论:M理论.

\subsection{p-膜}
p-膜的意思是,具有p个空间维度的物体.D-膜在弦论里面非常重要,因为开弦的端点能够attach到它上面.比如Yang-Mills理论,就涉及到attach到D-膜的开弦. 引力并不attach到D-膜上.这也就解释了为什么引力这样弱.

\subsection{高维}
这个idea首先是Kaluza和Klein首先提出来的. 在五维的空间,引力和电磁力可以很好地统一到同一个框架下. 他们提出了compactification这个idea.
意思就是,当额外的维度足够小的时候,多余的维度就探测不到了. 
