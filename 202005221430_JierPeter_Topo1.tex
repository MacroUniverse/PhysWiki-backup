% 连续映射和同胚
\pentry{映射\upref{map},拓扑空间\upref{Topol},函数的连续性\upref{contin}}
给定两个拓扑空间$(X, \mathcal{T}_X)$和$(Y, \mathcal{T}_Y)$,定义一个映射$f:X\rightarrow Y$.在集合论意义上,这个映射$f$可以有任何可能的形式,但是在拓扑学意义上,我们只研究满足特定条件的一类映射,称为连续映射.那么什么是连续映射呢?

\subsection{连续映射}
\begin{definition}{连续映射}
给定两拓扑空间$(X, \mathcal{T}_X)$和$(Y, \mathcal{T}_Y)$.映射$f:X\rightarrow Y$在某一点$x_o\in X$上连续$\iff$对于任意$Y$中的开集$U\ni f(x_0)$,存在$X$中的开集$V\ni x_0$,使得$f(V)\subset U$\footnote{这个定义是类比实变函数中逐点连续的$\epsilon-\delta$表达.}.

如果$f$在$X$中的任意一点上都连续,那么称$f$是一个\textbf{连续映射(continuous mapping)}.
\end{definition}

\begin{exercise}{连续映射的等价定义}\label{Topo1_exe1}
给定两拓扑空间$X$,$Y$,那么$f:X\rightarrow Y$是连续映射,当且仅当对于任意的$U\in\mathcal{T}_Y$,有$f^{-1}(U)\in\mathcal{T}_X$.即开集的逆映射还是开集.

证明这一点.
\end{exercise}

\autoref{Topo1_exe1}中的等价定义,可以简单记为“开集的逆映射还是开集”,类似地,容易证明它等价于“闭集的逆映射还是闭集”.这个定义和微积分中实函数的连续性\autoref{contin_the1}本质上是一样的.在微积分和实变函数中,这个等价定义的用途似乎没那么多,但是在点集拓扑中要更为常用.

\subsection{同胚}

在代数学中我们提到了同构和同态的概念