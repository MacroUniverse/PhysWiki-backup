% 麦克斯韦—玻尔兹曼分布
% 麦克斯韦|玻尔兹曼|速度分布|理想气体|动能|能量分布
% 未完成: 推导: 非常简单, 参考新概念热学

\pentry{随机变量的变换\upref{RandCV}, 气体分子的速度分布\upref{VelPdf}}
理想气体分子的速率分布由\textbf{麦克斯韦—波尔兹曼}分布来描述
\begin{equation}
f(v) = 4\pi \qty(\frac{m}{2\pi kT})^{3/2} v^2 \exp(-\frac{mv^2}{2kT})
\end{equation}
这是一个\textbf{概率分布函数}\upref{RandF}, 即速度模长在某个区间 $v \in [v_a, v_b]$ 的概率为
\begin{equation}
P_{ab} = \int_{v_a}^{v_b} f(v) \dd{v}
\end{equation}

平均速度为
\begin{equation}
\bar v = \sqrt{\frac{8kT}{\pi m}}
\end{equation}
速度平方平均值为
\begin{equation}
\overline {v^2} = \frac{3kT}{m}
\end{equation}
概率最大的位置为
\begin{equation}
v_p = \sqrt{\frac{2kT}{m}}
\end{equation}
动能分布为
\begin{equation}
f(E) = \frac{2}{kT}\sqrt{\frac{E}{\pi kT}} \exp(-\frac{E}{kT})
\end{equation}
