% 矩阵与矢量空间(与 MatSpc 重复!)
% 线性代数|矩阵|矢量|矢量空间|代数矢量|映射|线性

我们从矢量空间的角度来看\autoref{Mat_eq5} 的意义, $n$ 维代数矢量 $\bvec x$ 可以看做任意 $n$ 维矢量空间中的一个矢量在任意基底 $\{\bvec \alpha_i\}$ 上的坐标, 而 $m$ 维代数矢量 $\bvec y$ 可以表示任意 $m$ 维矢量空间中的一个矢量在任意基底 $\{\bvec \beta_i\}$ 上的坐标. 所以一个矩阵定义了从第一个空间到第二个空间的一个\textbf{映射}, 即把第一个空间的任意一个矢量对应到第二个空间中的一个矢量. 

我们先来看第一空间的基底 $\bvec \alpha_1$ 如何映射到第二空间. $\bvec\alpha_1$ 对应的列矢量是 $(1, 0, \dots)\Tr$, 作为 $\bvec x$ 输入矩阵 $\mat A$ 得 $\bvec y$ 等于 $\mat A$ 的第一列. 同理, 矩阵的第 $i$ 列就是 $\bvec\alpha_i$ 映射到第二个空间的矢量在基底 $\{\bvec \beta_i\}$ 上的坐标.

显然, 这里讨论的映射是\textbf{线性}的, 即第一空间中任意几个矢量的线性组合的映射等于这几个矢量先映射到第二空间再进行同样的线性组合, 这从矩阵的乘法分配律易证. 所以要将第一空间任意矢量的映射到第二空间, 就先将它表示成第一空间基底的线性组合, 然后再对矩阵每一列在第二空间对应的矢量做同样的线性组合即可.

映射有一一映射

