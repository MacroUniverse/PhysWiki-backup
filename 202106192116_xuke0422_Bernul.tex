% 伯努利方程
% 流体|密度|压强|功率|横截面

\begin{issues}
\issueDraft
\end{issues}

\footnote{参考 Wikipedia \href{https://en.wikipedia.org/wiki/Bernoulli-principle}{相关页面}.}不可压缩流体的方程.

\begin{equation}
\frac{v^2}{2} + gz + \frac{p}{\rho} = \text{常数}
\end{equation}
其中 $g$ 是重力加速度, $z$ 是高度, $p$ 是压强, $\rho$ 是液体的密度(液体不可压缩, 是常数).

\subsection{推导}
如图, 一根管子的粗细不同两部分的横截面面积分别为 $A_1, A_2$, 压强分别为 $p_1, p_2$, 速度分别为 $v_1, v_2$.

穿过横截面 $i = 1, 2$ 的功率分别为
\begin{equation}
P_i = p_i A_i v_i + \frac{1}{2} (\rho v_i A_i) v_i^2 + \rho v_i A_i g z_i
\end{equation}
而且我们要求
\begin{equation}\label{Bernul_eq1}
P_1 = P_2
\end{equation}

另外由于液体不可压缩, 有
\begin{equation}
A_1v_1 = A_2 v_2
\end{equation}
所以\autoref{Bernul_eq1} 两边同时除以 $\rho v_i A_i$ 得
\begin{equation}
\frac{v_1^2}{2} + gz_1 + \frac{p_1}{\rho} = \frac{v_2^2}{2} + gz_2 + \frac{p_2}{\rho}
\end{equation}

\addTODO{这样的推导如何拓展到开放空间的情况呢? 举例: 水龙头下的乒乓球, 香蕉球, 机翼, 两张纸中间吹气}
