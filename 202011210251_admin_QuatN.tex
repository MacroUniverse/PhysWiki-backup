% 四元数与旋转矩阵
% 线性代数|矩阵|绕轴旋转矩阵|旋转矩阵|四元数|基底|基底变换矩阵

\pentry{绕轴旋转矩阵\upref{RotA}, 旋转矩阵的导数\upref{RotDer},四元数\upref{Quat}}

四元数可以用来简洁地表示三维空间中的旋转,极大地减少了计算量.

三维空间中的一个向量表示为标部为$0$的四元数:$v=\pmat{0, \bvec{v}}$.如果我们绕着一个单位向量$\hat{\bvec{n}}$把$\bvec{v}$旋转一个角度$\theta$,所得的结果应该是哪个向量呢?取四元数$q=\pmat{\cos{\theta/2}, \hat{\bvec{n}}\sin{\theta/2}}$,那么旋转后的向量就可以表示为
\begin{equation}
qvq^{-1}
\end{equation}

比如说,取$v=(0, 1, 0, 0)$,它代表一个$x$轴上的单位向量.如果想要把它绕着$z$轴上的单位向量转$\pi/2$,那么结果向量的四元数表示应该是$(0, 0, 1, 0)$.按照我们定义的规则,旋转表示为四元数$q=(\sqrt{2}/2, 0, 0, \sqrt{2}/2)$,按照四元数的乘法规则易得$qvq^{-1}=(0,0,1,0)$.

\subsection{证明}







% 我们可以用\textbf{四元数(quaternions )} $\bvec q = [s, \bvec v]$ 来表示绕轴旋转矩阵, 其中
% \begin{equation}
% s = \cos(\theta/2) \qquad v = \abs{\bvec v} = \sin(\theta/2)\uvec A
% \end{equation}
% 则绕轴旋转矩阵可以表示为
% \begin{equation}\label{QuatN_eq2}
% \mat R(\theta) =
% \begin{pmatrix}
% 1 - 2v_y^2 - 2v_z^2 & 2v_xv_y - 2sv_z  & 2v_x v_z + 2s v_y\\
% 2v_x v_y + 2sv_z & 1 - 2v_x^2 - 2v_z^2 & 2v_y v_z - 2s v_x\\
% 2v_x v_z - 2s v_y & 2v_y v_z + 2s v_x & 1 - 2v_x^2 - 2v_y^2
% \end{pmatrix}
% \end{equation}

% 四元数的乘法运算可以表示两个旋转矩阵相乘, 即把两次旋转合并为一次旋转
% \begin{equation}
% [s_1, \bvec v_1] [s_2, \bvec v_2] = [s_1 s_2 - \bvec v_1 \vdot \bvec v_2, s_1 \bvec v_2 + s_2 \bvec v_1 + \bvec v_1 \cross \bvec v_2]
% \end{equation}
% % 推导未完成

% 若从坐标系 B 到坐标系 A 的基底变换矩阵为 $\mat R$, 当 B 绕原点以角速度 $\bvec \omega$ 旋转时有
% \begin{equation}
% \dv{\mat R}{t} = \mat \Omega \mat R
% \end{equation}
% 其中 $\mat\Omega$ 乘以任意位置矢量 $\bvec r$ 等于 $\bvec \omega \cross \bvec r$
% \begin{equation}
% \mat \Omega = \pmat{
% 0 & -\omega_z & \omega_y\\
% \omega_z & 0 & -\omega_x\\
% -\omega_y & \omega_x & 0
% }\end{equation}
% 若旋转矩阵 $\mat R$ 对应的四元数为 $\bvec q$, 则
% \begin{equation}\label{QuatN_eq1}
% \dot {\bvec q}(t) = \frac12 [0, \bvec \omega(t)] \bvec q(t)
% \end{equation}

% 推导大概就是旋转微小角度然后取极限
