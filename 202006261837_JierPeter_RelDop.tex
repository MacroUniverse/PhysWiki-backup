% 光的多普勒效应
% 相对论|光速|多普勒|频率|波长

\pentry{多普勒效应\upref{Dopler},斜坐标系表示洛伦兹变换\upref{SROb}}

不论是牛顿力学框架下还是相对论意义下,光的频率都可能在不同参考系之间有所不同.由于观察者和光源之间相对运动,造成观察者所测得的光的频率与光源的频率不一致,这个现象被称为光的多普勒效应.由于光速不变原理,在相对论框架下讨论光的多普勒效应反而更为简单.


\subsection{光的多普勒效应的推导}
由于在任何参考系中,光的速度都一样,因此只需要知道光的波长就可以根据以下公式得到光的频率:
\begin{equation}
f=\frac{1}{\lambda}
\end{equation}
其中,$f$是光的频率,$\lambda$是光的波长,$1$是光速.

假设光源所在的参考系为$K_1$,观察者所在的参考系为$K_2$,而$K_2$相对$K_1$的运动速度为$\vec{v}=(v, 0, 0)^T$.设光的频率在$K_1$为$f$,在$K_2$中为$f'$,那么光的波长在$K_1$中为$\lambda=1/f$,在$K_2$中为$\lambda'=1/f'$.

取$K_1$中光的两个相邻的波峰\footnote{也可以是相邻波谷,或者任何相邻的两个同相位的点.},作为两个以光速运动的点.为简化讨论,不妨设其中一个点的世界线\footnote{见\textbf{时空的四维表示}\upref{SR4Rep}.}表示为
\begin{equation}
\pmat{t\\t\\0\\0}
\end{equation}
而另一个点为
\begin{equation}
\pmat{t\\t+\lambda\\0\\0}
\end{equation}

代入
