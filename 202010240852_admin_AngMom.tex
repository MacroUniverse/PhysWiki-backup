% 角动量
% 动量|角动量|角动量分析|坐标系变换|质心系

\pentry{质点的角动量\upref{AMLaw1}}

\subsection{系统的角动量}
角动量是矢量,若把系统看做质点系,则系统的角动量等于所有质点的角动量矢量相加\upref{GVecOp}.
\begin{equation}\label{AngMom_eq1}
\bvec L = \sum_i \bvec L_i = \sum_i \bvec r_i \cross\bvec p_i = \sum_i m_i \bvec r_i \cross\bvec v_i
\end{equation}
\begin{example}{旋转圆环}\label{AngMom_ex1}
在\autoref{AMLaw1_ex2}~\upref{AMLaw1}中, 我们知道单个质点做圆周运动的角动量为 $\bvec L = m r^2 \omega \bvec z$. 现在考虑一个半径为 $r$ 质量为 $M$ 的细圆环绕处于原点的圆心做逆时针圆周运动, 每个质点的轨迹就是圆环本身. 如果我把圆环看成由许多质点 $m_i$ 组成, 每个质点都做上述圆周运动, 则总角动量为
\begin{equation}
\bvec L = \sum_i m_i \omega r^2 \uvec z = M \omega r^2 \uvec z
\end{equation}
\end{example}

\subsection{角动量的坐标系变换}
可类比力矩的坐标系变换(\autoref{Torque_eq5}),坐标系 $A$ 中总角动量为
\begin{equation}
\bvec L_A = \sum_i \bvec r_{Ai} \cross \bvec p_i 
\end{equation}
变换到坐标系 $B$ 中, 假设 $A, B$ 相对静止, 从 $B$ 原点指向 $A$ 原点的矢量为 $\bvec r_{BA}$, 那么总角动量为
\begin{equation}\label{AngMom_eq4}
\bvec L_B = \sum_i (\bvec r_{BA} + \bvec r_{Ai})\cross \bvec p_i = \bvec r_{BA}\cross \sum_i \bvec p_i + \bvec L_A
\end{equation}
可以发现, 当系统总动量为零时, 两参考系中计算系统角动量结果相同.

\begin{example}{旋转圆环 2}
我们把\autoref{AngMom_ex1} 中选取的参考系称为 $A$, 如果我们切换到另一个与 $A$ 相对静止, 但原点不相同的参考系 $B$,  使圆环中心不处于坐标原点, 那么角动量会如何变化呢? 
\end{example}

\subsection{角动量的分解}
质心系中的角动量为
\begin{equation}\label{AngMom_eq5}
\bvec L_0 = \sum_i \bvec r_{ci} \cross \bvec p_i
\end{equation}
定义\textbf{质心角动量}为“ 质心处具有系统总质量 $M$ 的质点的角动量” (类比质心动量的定义, \autoref{SysMom_eq2}~\upref{SysMom})
\begin{equation}\label{AngMom_eq6}
\bvec L_c  = \bvec r_c \cross (M \bvec v_c) = \bvec r_c \cross \bvec p_c
\end{equation}

现在我们变换到任意坐标系中,令总角动量为 $\bvec L$,由\autoref{AngMom_eq4} 得
\begin{equation}
\bvec L = \bvec r_{c} \cross \sum_i \bvec p_i + \bvec L_0
\end{equation}
由于系统总动量 $\sum_i \bvec p_i$ 等于质心动量 $\bvec p_c$,右边第一项等于质心角动量\autoref{AngMom_eq6}.最后得到
\begin{equation}
\bvec L = \bvec L_c + \bvec L_0
\end{equation}
所以\textbf{任何坐标系中,系统的总角动量等于其质心的角动量加上相其相对质心的角动量}.
