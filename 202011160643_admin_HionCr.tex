% 氢原子电离截面
% 一阶微扰理论|氢原子|偶极子

\begin{issues}
\issueDraft
\end{issues}

\pentry{类氢原子的波函数\upref{HWF}}

本文使用原子单位制\upref{AU}.一阶微扰理论就是单光子电离.

基态与平面波的积分

\begin{equation}
\bvec d_{\bvec k} =  \mel{\bvec k}{\bvec r}{0} =  \frac{ \uvec k}{\sqrt2\pi} \int_0^{+\infty} \int_0^\pi \E^{-r} \E^{-\I k r \cos\theta} r \cos\theta \cdot r^2 \sin\theta \dd{\theta} \dd{r}
\end{equation}
换元, 令 $u = \cos\theta$, 得
\begin{equation}\ali{% 已检查多次
\bvec d_{\bvec k} &= \frac{\uvec k}{\sqrt{2}\pi}  \int_0^{+\infty} r^3 \E^{-r} \int_{-1}^1 \E^{-\I k r u} u  \dd{u} \cdot \dd{r}\\
&=  \I\frac{\sqrt2 \uvec k}{\pi k}  \int_0^{+\infty} r^2 \E^{-r} \qty[\cos(kr) - \frac{1}{kr}\sin(kr)] \dd{r}\\
&= -\I \frac{8\sqrt2}{\pi} \frac{\bvec k}{(k^2+1)^3}
}\end{equation}

严格来说, 需要把平面波替换为库仑函数.

\subsection{速度规范}
含时微扰理论(\autoref{TDPT_eq10}~\upref{TDPT}) 为
\begin{equation}\label{HionCr_eq1}
c_i(t) = -\I \int_0^t \mel{i}{H'(t)}{j} \E^{\I\omega_{ij} t} \dd{t}
\end{equation}
速度规范\upref{LVgaug}中,
\begin{equation}
H'(t) = -\frac{q}{m}\bvec A \vdot \bvec p = \frac{\I q}{m}\bvec A \vdot \grad
\end{equation}
$\bvec A$ 只是 $t$ 的函数, 可以分离
\begin{equation}
\mel{i}{H'(t)}{j} = -\frac{q}{m}\bvec A(t) \vdot \mel{i}{\bvec p}{j}
\end{equation}
代入\autoref{HionCr_eq1} 得
\begin{equation}
c_i(t) = \frac{\I q}{m} \mel{i}{\bvec p}{j} \int_0^t  \bvec A(t) \E^{\I\omega_{ij} t} \dd{t}
\end{equation}