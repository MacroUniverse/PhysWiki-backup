% 标量扰动
把爱因斯坦方程进行扰动,我们可以得到
\begin{equation}\label{ScaPT_eq1}
\delta G^\mu_\nu = 8 \pi G\delta T^\mu_\nu ~.
\end{equation}
我们主要关心标量的扰动,可以得到
\begin{equation}
\begin{aligned}
\delta G^0_0 & = 2 a^{-2} [\nabla^2 \Phi- 3 \mathcal H(\Phi' - \mathcal H \Psi)] ~,  \\
\delta G^i_0 & = - 2 a^{-2} \partial^i(\Phi' - \mathcal H \Psi)~, \\
\delta G^i_j &= 2 a^{-2} \bigg[ (\mathcal H^2 + 2 \mathcal H')\Psi +\mathcal H \Psi' - \Phi'' - 2 \mathcal H \Phi' + \frac{1}{3} \nabla^2 (\Phi+\Psi)  \bigg] \delta^i_j \\
& - a^{-2} \bigg( \partial^i\partial_j - \frac{1}{3} \delta^i_j \nabla^2 \bigg) (\Phi+\Psi) ~.
\end{aligned}
\end{equation}
对能量动量张量同样进行扰动,我们有
\begin{equation}
\begin{aligned}
\delta T^0_0 & = - \delta \rho~, \\
\delta T^i_0 & = - (\bar \rho +\bar p) \partial^i v~, \\
\delta T^i_j & = \delta p \delta^i_j + \bigg( \partial^i\partial_j - \frac{1}{3} \nabla^2 \bigg) \sigma ~,
\end{aligned}
\end{equation}
使用上述方程,\autoref{ScaPT_eq1} 的00分量的方程可以写成
\begin{equation}\label{ScaPT_eq2}
\nabla^2 \Phi - 3 \mathcal H (\Phi' - \mathcal H\Psi) = - 4 \pi G a^2 \delta \rho~. 
\end{equation}
上述方程可以视作平坦空间的泊松方程在弯曲空间中的推广.定义物理距离$\mathbf r = a \mathbf x$和物理时间$t$,我们可以把上面方程右边的$a^2$吸收掉.最终我们可以得到如下方程
\begin{equation}
\nabla^2_{\mathbf r} \Phi - 3 H (\dot\Phi - H \Psi) = - 4 \pi G \delta \rho
\end{equation}
现在我们来看$(i,0)$方程
\begin{equation}\label{ScaPT_eq3}
\Phi' - \mathcal H \Psi = 4 \pi G a^2 (\bar \rho + \bar p) v ~.
\end{equation}
综合\autoref{ScaPT_eq2} 和\autoref{ScaPT_eq3} 我们可得
\begin{equation}
\nabla^2\phi = - 4 \pi G a^2 [\delta \rho - 3 \mathcal H(\bar\rho+ \bar p) v]~.
\end{equation}
等式的右边实际上是规范不变的密度扰动$(\delta\rho)_*$.我们可以把上式写成
\begin{equation}
\nabla^2\Phi = - 4 \pi G a^2 \bar \rho \delta_* ~,
\end{equation}
其中$\delta_* = (\delta \rho)_*/\bar \rho$是Gauge-invariant density contrast.

