% 联络(向量丛)
\pentry{向量丛\upref{VecBun}}

本节采用爱因斯坦求和约定.

设 $M$ 是 $n$ 维微分流形, $E$ 是其上秩为 $k$ 的光滑向量丛. 

\subsection{定义与例子}
向量丛 $E$ 上的一个\textbf{联络(connection)}是指一个映射 $D:\Gamma(E)\otimes \mathfrak{X}(M)\to\Gamma(E)$, 
%或者等价地 $D:\Gamma(E)\to T^*M\otimes\Gamma(E)$, 
满足如下条件:

\begin{enumerate}
\item 对于截面 $\xi\in\Gamma(E)$ 和任何切向量场 $X$, $X\to D_X\xi$ 是 $C^\infty(M)$-线性的, 即
$$
D_{fX_1+fX_2}\xi=fD_{X_1}\xi+gD_{X_2}\xi,\,f,g\in C^\infty(M).
$$
\item 对于截面 $\xi\in\Gamma(E)$ 和任何切向量场 $X$, $\xi\to D_X\xi$ 是实线性的, 即对于实数 $c_1,c_2$ 有
$$
D_X(c_1\xi_1+c_2\xi_2)=c_1D_X\xi_1+c_2D_X\xi_2.
$$
\item 对于任何 $f\in C^\infty(M)$, 都有莱布尼兹律:
$$
D_X(f\xi)=X(f)\xi+fD_X\xi.
$$
\end{enumerate}

直观上说, 在向量丛上给定联络, 就是给定一个"符合张量规律的导数运算". 

如果取 $M=\mathbb{R}^n$, $E$ 为平凡向量丛 $\mathbb{R}^n\times\mathbb{R}^k$, 则截面 $\xi$ 就是通常的 $k$-维向量值函数, 通常的对笛卡尔坐标的微分运算
$$
D\xi=(\partial_i\xi^\alpha)_{1\leq i\leq n}^{1\leq\alpha\leq k}
$$
就是一个联络.

不平凡的例子在黎曼几何中多有出现.

\subsection{联络形式}
设 $D$ 是向量丛 $E$ 上的联络. 设 $\{e_i\}_{i=1}^n,\{\theta^i\}_{i=1}^n,\{s_\alpha\}_{\alpha=1}^k$ 分别是 $TM,T^*M,\Gamma(E)$ 的局部光滑标架, 其中 $\{e_i\}$ 和 $\{\theta^i\}$ 是对偶标架. 在这些局部光滑标架之下, 按照联络的定义, 有
$$
D_X\xi=X^i\cdot\left[e_i(\xi^\alpha)s_\alpha+\xi^\alpha D_{e_i}s_\alpha\right].
$$
若命 $D_{e_i}s_\alpha=\Gamma_{\alpha i}^\beta s_\beta$, 则有
$$
D_X\xi=X^i\cdot\left[e_i(\xi^\alpha)+\Gamma_{\beta i}^\alpha\xi^\beta \right]s_\alpha.
$$
称系数 $\Gamma^\alpha_{\beta i}$ 为联络系数(coefficients of connection)或克氏符(Christoffel symbol). 于是就得到1-形式的 $k\times k$ 矩阵
$$
\omega=(\omega_\beta^\alpha)=(\Gamma_{\beta i}^\alpha\theta^i).
$$
它们称为联络 $D$ 在局部的\textbf{联络1-形式矩阵(matrix of connection 1-forms)}. 从而可写
$$
D\xi=(d\xi^\beta+\omega_\alpha^\beta\xi^\alpha)\otimes s_\beta.
$$

注意, 克氏符和联络1-形式都只能局部定义. 若 $\{s'_\beta\}$ 是 $E$ 的另一个局部标架, 同 $\{s_\alpha\}$ 之间的转换公式为 $s_\beta=b_\beta^\alpha s_\alpha$, 则
$$
b_\alpha^\gamma{\omega'}_{\beta}^{\alpha}\otimes s_\gamma={\omega'}_{\beta}^{\alpha}s'_\alpha=Ds'_\beta=(db_\beta^\gamma+b_\beta^\alpha\omega_\alpha^\gamma)\otimes s_\gamma.
$$
从而新标架下的联络1-形式矩阵 $\omega'$ 同原标架下的 $\omega$ 之间的转换关系是
\begin{equation}\label{VecCon_eq1}
\omega'=db\cdot b^{-1}+b\cdot\omega\cdot b^{-1}.
\end{equation}
这是联络1-形式在向量丛的局部标架变换下的转换公式, 由此可见联络1-形式矩阵的定义依赖标架的选取, 从而并非全局定义的张量场.

在 $\mathbb{R}^n\times\mathbb{R}^k$ 上,笛卡尔坐标系下的微分
$$
D\xi=(\partial_i\xi^\alpha)_{1\leq i\leq n}^{1\leq\alpha\leq k}
$$
在经过坐标变换之后也要改变形式.这就是联络形式定义的最初来源.

\subsection{存在定理}
由转换公式\autoref{VecCon_eq1} 此可得联络的存在定理, 或者说, 只要给出了满足转换公式\autoref{VecCon_eq1} 的1-形式矩阵, 就等于给出了联络:
\begin{theorem}{}
设 $\mathfrak{U}$ 是流形 $M$ 的开覆盖, 使得在每一个开集 $U\in\mathfrak{U}$ 上都给定了切丛的局部标架 $\{e^U_i\}$, $E$ 的局部标架 $\{s^U_\alpha\}$ 以及1-形式的 $k\times k$ 方阵 $\omega^U$. 若对于任意两个相交的 $U,V\in\mathfrak{U}$, 在 $U\cap V$ 总成立
$$
\omega^V=db^V_U\cdot \{b_V^U\}^{-1}+b^V_U\cdot\omega^U\cdot \{b^V_U\}^{-1},
$$
其中 $b^V_U$ 是从 $\{s^U_\alpha\}$ 到 $\{s^V_\alpha\}$ 的转换矩阵, 则存在唯一一个 $E$ 上的联络, 使得其在局部标架 $\{s^U_\alpha\}$ 之下的联络1-形式矩阵是 $\{\omega^U\}$.
\end{theorem}

还可以适当选取 $E$ 的标架, 使得联络1-形式矩阵在指定的点处有简单的形式:
\begin{theorem}{}
给定任何一点 $p\in M$ 以及一个 $E$ 上的联络 $D$, 都存在 $E$ 在 $p$ 附近的局部标架, 使得这标架下的联络1-形式矩阵 $\omega(p)=0$.
\end{theorem}
\textbf{证明大意} 先取一个 $E$ 的标架 $\{s_\alpha\}$, 并设
$$
\omega^\beta_\alpha=\Gamma_{\alpha i}^\beta dx^i.
$$
命 $\{x^i\}$ 为 $p$ 附近的坐标系, 使得 $x^i(p)=0$. 则命
$$
b_\alpha^\beta=\delta_\alpha^\beta-\Gamma_{\alpha i}^\beta x^i,
$$
并定义 $E$ 的新的局部标架 $s'_\alpha=b_\alpha^\beta s_\beta$. 则从 $db(p)=-\omega(p)$ 和转换公式\autoref{VecCon_eq1} 就得到新标架下 $\omega'(p)=0$.

注意,这不表示可以取到局部标架使得联络1-形式矩阵在 $p$ 的邻域内都等于零. 某个局部标架下的联络系数局部为零是非常特殊的性质.详见平行性(向量丛)\upref{VecPar}.

\subsection{对偶联络; 联络的和与积}
$E$ 上的联络 $D$ 自然诱导出对偶丛 $E^*$ 上的联络 $D^*$: 如果 $\xi\in\Gamma(E),\eta^*\in\Gamma(E^*)$, 则
$$
d\langle \eta^*,\xi\rangle=\langle D^*\eta^*,\xi\rangle+\langle \eta^*,D\xi\rangle.
$$
如果 $\{s_\alpha\}$ 是 $E$ 的局部标架而 $\{s^{*\gamma}\}$ 是对偶标架, 那么 $D^*$ 的联络1-形式矩阵在此标架下为
$$
\omega^{*\beta}_\alpha
=\langle D^*s^{*\beta},s_\alpha\rangle
=-\langle s^{*\beta},Ds_\alpha\rangle
=-\omega^\beta_\alpha,
$$
或者可简写为 $D^*s^{*\beta}=-\omega^\beta_\alpha s^{*\alpha}$. 若截面 $\eta^*\in\Gamma(E^*)$ 有局部表达式 $\eta^*=\eta_\beta s^{*\beta}$, 则
$$
D^*\eta^*=(d\eta_\beta-\omega_\beta^\alpha\eta_\alpha)\otimes s^{*\beta}.
$$

如果 $D_1,D_2$ 分别是向量丛 $E_1,E_2$ 上的联络, 则在直和 $E_1\oplus E_2$ 和张量积 $E_1\otimes E_2$ 上的和与积分别为
$$
D(\xi_1\oplus \xi_2)=D\xi_1\oplus D\xi_2,
\,
D(\xi_1\otimes \xi_2)=D\xi_1\otimes \xi_2+\xi_1\otimes D\xi_2.
$$