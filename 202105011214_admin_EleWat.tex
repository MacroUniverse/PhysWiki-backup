% 电路和水路的形象对比

\pentry{电压和电动势\upref{Voltag}}

\addTODO{图}

形象来说, 我们可以把电路中的导线类比为水平面上的水管, 管道中充满某种不可压缩的非粘稠流体, 即该流体不会因自身的摩擦损耗能量. 导线中自由电子的电荷可以类比流体的质量, 电流则类比液体单位时间流经管道横截面的质量. 没有电阻的理想导线类比为内壁无摩擦的管道.

在该模型中, 电势可以理解为管道内某点处的绝对水压. 事实上把整个管道系统的绝对水压处处增加一个常数并不会影响

某处离水平地面的高度, 而电压/电势差可以理解为管道不同两点的高度只差.
