% 物理单位前缀

在物理中, 我们会在一些单位符号前面加上一个表示数量级的前缀(prefix)以方便书写. 如长度单位 $\Si{m}$ (米) 可以加不同的前缀拓展 $\Si{cm}$(厘米), $\Si{mm}$(毫米). 另外我们也会有一些前缀用来表示更大的数量级, 如频率单位 $\Si{Hz}$ (赫兹)可以拓展为 $\Si{MHz}$ (兆赫兹), $\Si{GHz}$ (千兆赫兹), 常见于无线电术语中. 每一个这样的前缀表示一个 $10^N$ 的整数, 下面我们来看常用前缀代表的数量级.

\begin{table}[ht]
\centering
\caption{请输入表格标题}\label{UniPre_tab1}
\begin{tabular}{|c|c|c|c|}
\hline
* & * & * & * \\
\hline
$m$ & $10^{-3}$ & $k$ & $10^3$ \\
\hline
$\mu$ & $10^{-6}$ & $M$ & $10^6$ \\
\hline
* & * & * & * \\
\hline
* & * & * & * \\
\hline
* & * & * & * \\
\hline
* & * & * & * \\
\hline
* & * & * & * \\
\hline
* & * & * & * \\
\hline
* & * & * & * \\
\hline
\end{tabular}
\end{table}
