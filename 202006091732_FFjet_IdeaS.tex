% 理想气体的熵:纯微观分析

设某理想气体由$N $个原子组成,体积为$V$,能量为$U$,现在,我们利用玻尔兹曼熵公式计算它的熵(可以加上一个常量).假设$N $为定值,但是$U $和$V $可以变化.我们要求的量是$S(U,V)$.

理想气体的能量全部为动能,与粒子的位置无关.我们必须求与$U$和$V$相对应的状态数$\Omega(U,V)$的对数.

我们已经知道,
\begin{equation}
\Omega \left( U,V \right) =\left( \frac{V}{a^3} \right) ^N\times \Omega _p\left( U \right) 
\end{equation}
式中$V/a^3$是每个原子可占据的位置的数量;$\Omega_p(U)$是内能为$U$的气体中,动量微观分布的数目.(计算自由膨胀的熵变