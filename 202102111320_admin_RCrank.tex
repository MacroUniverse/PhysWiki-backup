% 厄米算符的映射结构
% 线性无关|列秩|行秩|零空间|解空间|正交归一|基底

\begin{issues}
\issueDraft
\end{issues}

\pentry{矩阵与矢量空间\upref{MatLS}, 正交子空间\upref{OrthSp}, 线性映射的结构\upref{MatLS2}, 厄米共轭\upref{HerMat}}

\begin{figure}[ht]
\centering
\includegraphics[width=10cm]{./figures/RCrank_1.pdf}
\caption{请添加图片描述} \label{RCrank_fig1}
\end{figure}

\begin{theorem}{}\label{RCrank_the1}
令 $X, Y$ 为有限维线性空间, 线性算符 $A:X \to Y$ 的零空间为 $X_0$, 其厄米共轭\upref{HerMat}(自伴算符) $A\Her: Y \to X$ 的零空间为 $Y_0$; 令 $Y_1 = A(X)$, $X_1 = A\Her(Y)$. 那么 $X_0, X_1$ 是 $X$ 中的正交补(\autoref{OrthSp_def1}~\upref{OrthSp}), $Y_0, Y_1$ 是 $Y$ 中的正交补. 即
\begin{equation}
X = X_0 \oplus X_1 \qquad X_0 \perp X_1
\end{equation}
\begin{equation}
Y = Y_0 \oplus Y_1 \qquad Y_0 \perp Y_1
\end{equation}
\end{theorem}
其中 $\oplus$ 表示直和\upref{DirSum}, $\perp$ 表示两空间正交\upref{OrthSp}.

\begin{corollary}{}
矩阵的行秩等于列秩.
\end{corollary}
证明: 令矩阵 $\mat A$ 的算符为 $A$, 该矩阵的行秩就是 $Y_1$ 的维数. 而矩阵的列秩等于 $\mat A\Her$ 的行秩, 也就是 $X_1$ 的维数. \autoref{RCrank_the1} 中 $X_1, Y_1$ 有一一对应关系, 所以维数相同. 证毕.

\subsection{证明}
以下证明\autoref{RCrank_the1} 中 $Y_1$ 的正交补就是 $Y_0$. 令 $Y_1$ 的正交补为 $Y_2$, 那么 $y \in Y_2$ 的充分必要条件是和 $Y_1$ 中所有矢量都垂直.
\begin{equation}
\mel{y}{A}{x} = 0 \qquad (\forall x \in X)
\end{equation}



============= 回收 =============
, $X_0, X_1$ 同理. 令 $\mat A$ 为 $N_Y$ 行 $N_X$ 列的矩阵, 对应的线性算符为 $A: X\to Y$. $X, Y$ 空间的维数分别为 $N_X, N_Y$. $\mat A$ 的厄米共轭\upref{HerMat}记为 $\mat A\Her$, 对应算符记为 $A\Her$.











我们先在 $Y_1$ 中找到一套 $N_{Y1}$ 个正交归一基底 ${y_{1i}}$, 再在 $Y$ 空间中找到剩下 $N_Y - N_{Y1}$ 个正交归一基底 ${y_i}$. 对任意 ${x} \in X$, 有 $A {x} \in Y_1$, 所以
\begin{equation}
\mel{y_i}{A}{x} = 0
\end{equation}
对两边做厄米共轭得% 未完成: 狄拉克符号里面需要介绍
\begin{equation}
\mel{x}{A\Her}{y_i} = 0
\end{equation}
对任意 ${x}\in X$ 都成立, 所以
\begin{equation}
A\Her{y_i} = 0
\end{equation}
所以 ${y_i} \in Y_0$. 然而对于任意 ${y_1} \in Y_1$ 且 $y_1 \ne 0$, 必存在一些 ${x}$ 使
\begin{equation}
\mel{y_1}{A}{x} \ne 0
\iff
\mel{x}{A\Her}{y_1} \ne 0
\end{equation}
所以
\begin{equation}
A\Her{y_1} \ne 0
\end{equation}
即 ${y_1} \notin Y_0$. 所以 ${y_i}$ 就是 $Y_0$ 的(完备)正交归一基底, 且与 ${y_{1i}}$ 组成 $Y$ 的完备正交归一基底. 证毕.
