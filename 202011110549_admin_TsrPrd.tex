% 张量积
% 张量|张量积|张量积空间
\pentry{张量\upref{Tensor},爱因斯坦求和约定\upref{EinSum}}

在张量\upref{Tensor}词条中我们知道,一个张量可以看成是将若干个线性空间$V$映射到域$\mathbb{F}$上的映射.

域是可以进行加减乘除的集合,因此我们可以在域上定义映射的乘积.比如如果有实数域上的实函数$f, g:\mathbb{R}\rightarrow\mathbb{R}$,那么我们可以定义两个函数的乘积$f\cdot g$如下:$\forall x\in\mathbb{R}, f\cdot g(x)=f(x)\cdot g(x)$.就是说,把各点的函数值相乘,得到的结果还是一个函数,它就是两个函数的乘积.

同样的方法可以应用到张量上,这样所得到映射乘积,就是所谓的张量积.

\begin{definition}{张量积}

给定线性空间$V$,在线性空间$V$上有$n-$线性函数$f$和$m-$线性函数$g$,则可以定义$(n+m)-$线性函数$f\cdot g$,方式为:对于任意$x_i, y_j\in V$,$f\cdot g(x_1, \cdots, x_n, y_1, \cdots, y_m)=f(x_1, \cdots, x_n)\cdot g(y_1, \cdots, y_m)$.

\end{definition}

用矩阵可以直观地看出张量积的“升维”性质,但是这个方法很难推广到任意阶的张量.我们举一个一阶张量相乘得到二阶张量的例子:

\begin{example}{一阶张量的张量积}
在一个$3$维线性空间$V$里默认取某个基时,有
\end{example}




