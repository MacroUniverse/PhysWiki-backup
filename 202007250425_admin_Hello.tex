% Hello
\documentclass {article}
\usepackage {ctex}
\usepackage {amsmath}
\usepackage {amsthm}
\usepackage {bm}
\usepackage {graphicx}
\usepackage {float} 
\usepackage {multicol}
\usepackage {url}
\usepackage {fancyhdr}
\usepackage {listing}
\pagestyle {fancy}
\rhead {19336035 陈梓乐}
\renewcommand{\headrulewidth}{0.5pt}


\begin {document}
\title {浅述简单图中的单源最短路算法}
\author {中山大学数学学院 19336035 陈梓乐}
\maketitle
\begin {abstract}
最短路径问题是图论中的一个经典算法问题,旨在寻找图中两结点的最短路径.本文将介绍单源最短路问题的一些算法,并对这些算法进行对比和应用.
\par\textbf{关键词:最短路算法,Dijkstra算法,Bellman-Ford算法,SPFA}
\end {abstract}
\tableofcontents
\newtheorem {define} {定义}
\newtheorem {theorm} {算法}
\newtheorem {extra} {定理}
\newtheorem {ex} {例}

\begin {center}
\section {前言}
\end {center}

最短路径问题是图论中的一个经典算法问题,旨在寻找图中两结点的最短路径.
根据不同情况的需要,算法可被大致分为以下几种形式:\begin {itemize} \item确定起点的最短路问题.即给出图中一结点为起点,求任意一点为终点所需要的最低代价.\item已知终点的最短路问题.与确定起点的问题相反,该问题是已知终结结点,求最短路径的问题.在无向图中该问题与确定起点的问题完全等同,在有向图中该问题等同于把所有路径方向反转的确定起点的问题.\item 确定起点终点的最短路径问题.即已知起点和终点,求两结点之间的最短路径.\item全局最短路径问题. 即求图中任意两个结点的最短路径.\cite {ref1} \end {itemize} 

根据题目的不同,选取不同的解决方法,解决这类问题的方法叫做“最短路径算法”.例如,深度优先搜索(DFS)就是一种解决这类问题的方法,它的方法实现起来很简单:\begin {theorm} 给定简单图G(V, E),从给定的起点出发,访问它的第一个未访问过的邻接点,并对该邻接点做同样的处理,直到访问到某个点的所有邻接点均被访问过,记下该路径,将该点标记为未访问过,并返回上一个点,对上一个点访问它的第二个未访问过的邻接点......以此类推,直到遍历完整个图.这样的方法成为深度优先搜索\cite {ref2}\end {theorm}

用DFS的方法,可以找出从起点出发的所有可能路径,只要将这些路径按照一定标准排序,这样,就可以找出两点间的最短距离.但是,因为其复杂度非常高,以至于现代计算机都需要花费很长的时间去运行才能得出结果,因此这种想法是不合适的,在接下来的环节中,我们将介绍解决这类问题的一些常见方法.由于篇幅有限,本文仅介绍给定起点的最短路问题,我们把这类问题叫做单源最短路问题.

\begin {center}
\section {最短路径算法}
\end {center}
\subsection {广度优先搜索(BFS)}
对于给定的简单图,使用BFS搜索最短路是一种不错的选择,它的复杂度低至O(E),是一种值得考虑的算法.\begin {theorm} 给定简单图G(V, E),将起点$v_0$加入到待搜索队列中.接下来每一次操作,都弹出队列中的首元素I,并且对I的所有邻接点执行以下操作:如果它的邻接点没有被访问过,则将它的邻接点加入到待搜索队列中,并且称这个邻接点的父结点为I.当这个点被加入到待搜索队列中时,我们称找到了从起点到该结点的最短路径,该最短路径就是该结点的父结点的最短路径加上该结点.这种搜索方法称之为广度优先搜索(BFS).\end {theorm}
我们来证明用这种方法找到的路径为最短路径.
\paragraph {证明:}
假定对于v而言存在从$v_0$到v更短的路径,设v的父结点为v',v通过BFS得到的父结点为v'',则v'的最短路径比v''小,因此在待搜索队列中,v'在v''前,即在BFS搜索中,v应该先作为被v'的邻接点被访问.于是通过广度优先搜索得到的的父结点为v'或最短路径小于等于v’'的点.\qed

对于简单图,我们找到了一种简单快速的方法去严格找出某个结点的最短路径.但是,在实际情况中,我们面对的问题往往比简单图要复杂.例如,在一张地图上,A和B、C两个城市都有路直接相连,但是从A到B、C所需要的时间(或者是代价)是不一样的.BFS的适用范围仅仅是简单图,它把所有的路看成是平等的路.为了解决更复杂的实际问题,我们引入有边权的简单图的概念.\begin {define} 在简单图G(V, E)中,假定对于任给的$e\in E$,定义e.length是一个实数,表示从起点到终点的代价.这样的图称为有边权的简单图.约定e.length为e的权,e.from为e的起点,e.to为e的终点.\end {define}

我们不妨先假定该图没有负边,即,$\forall e\in G(V, E), e.length\ge0$.在这种情况下,我们来介绍一种算法.
\subsection {Dijkstra 算法}
\begin {define} 对于给定的有边权的简单图,称从给定起点到某一点v所需要的代价(或花费)成为该点的权,用v.cost表示.\end {define}
有了这个概念之后,我们来给出找最短路的一种算法.
\begin {theorm} 对于给定的没有负边的有边权的简单图,初始时将到达起点的代价(或者是花费)定义为0,其他的结点所需代价定义为无穷.等Dijkstra算法执行结束后,v.cost就是从起点到v所需要的最小代价.Dijkstra算法维护两个集合,分别是S和Q,集合S保留所有已知实际最短路径值的顶点,而集合Q则保留其他所有顶点.集合S初始状态为空,而后每一步都有一个顶点从Q移动到S.这个被选择的顶点是Q中拥有最小权的顶点.当一个顶点u从Q中转移到了S中,算法对u的每条外接边e(u, v)进行如下操作:如果$$ u.cost+e.length<v.cost $$那么:$$ v.cost=u.cost+e.length $$ \cite {ref2, ref3}\end {theorm}

《算法导论》给出了该问题的伪代码:
\begin {figure} [H]
    \centering
    \includegraphics[width=1\textwidth]{1.jpg}
    \caption {Dijkstra伪代码}
\end {figure}

下面我们来证明这个方法是正确的.实际上,Dijkstra本人在他的论文中给出了一个证明.\cite {ref3}而《算法导论》中也给出了另一个证明.\cite {ref2}下面给出一些其他证明方式:
\paragraph {证明:}要证明Dijkstra算法的正确性,实际上只要保证每次向集合S中加入的元素u是集合Q中的最优元素即可.由Dijkstra算法的执行过程可知:$\forall u\in S, v\in Q$, 都有:$$ u.cost\le v.cost $$因此Q中的元素不可能作为起点更新S中的元素,即S中的元素的花费自加入集合S以后就不会被更改.也就是说S中的元素为最优.这样,在执行了|V|次操作后,所有结点都在S集合中,于是找到了所有结点的最优解.\qed

这样,我们就证明了Dijkstra算法的正确性,而实际上,我们需要对所有的结点、所有的边进行遍历,在进行所有的结点的遍历时,每遍历一个结点,我们都需要去寻找集合Q中权最小的结点,因此,每确定一个结点,我们都需要对Q中所有结点进行遍历.也即是说,Dijkstra算法的时间复杂度是$O(|V|^2+|E|)$,实际上,相比于DFS算法,该复杂度已经低了很多,但是相比于BFS,复杂度仍然不低,但是我们可以通过一些方法降低Dijkstra的复杂度.

\subsection {使用二叉堆优化的Dijkstra算法}
带二叉堆优化的Dijkstra算法实际上优化的是从Q中选取最小权点的过程,Dijkstra算法是对Q中所有结点进行遍历,通过对比找最小权点,所需要的代价为$O(|V|)$,使用二叉堆优化,能把这个过程优化到$O(log|V|)$.\cite {ref5}
\begin {define} 二叉堆是一个完全二叉树,满足性质:父结点的键值总是保持固定的序关系于任何一个子结点的键值,且每个结点的左子树和右子树都是一个二叉堆.\end {define}
例如,小根堆满足二叉树的根结点比子结点小,并且左右左子树和右子树都是小根堆.这样,堆顶元素就是最小值.而二叉堆的增加、删除所需要的时间复杂度均为O(log n).关于这点,在《二叉堆的定义、性质、操作和使用》一文中给出了详细证明,\cite {ref6}这里就不再赘述.经过这一优化,Dijkstra算法的时间复杂度为$O((|E|+|V|)log|V|)$.\cite {ref4, ref5}

\subsection {Bellman-Ford 算法}
前文我们讨论了,在给定的无负边的有边权的简单图中,给定起点的最短路算法,即Dijkstra算法.使用最小堆优化之后,该算法的时间复杂度低至$O((|E|+|V|)log|V|)$.下面我们把问题推广到所有的有边权的简单图,即,我们去除无负边这一限制条件,来讨论这样的图上的算法.在开始之前,我们给出一个定理,说明我们会讨论什么样的图:
\begin {extra} 有负环的连通图没有最短路. \end {extra}
实际上,我们今天所有的讨论都是在连通图上进行的.因为如果不连通,我们可以在每个连通分支上讨论最短路问题,而且各个连通分支互不干扰.若一个连通图有负环,则不可能找到一条从起点到终点的最短路.证明如下:
\paragraph {证明:} 若找到一条路从起点到终点,那么一定存在更短的路,这条路经过负环数次,并连接起点和终点.\qed

于是我们约定,我们接下来讨论的最短路算法,是建立在无负环、有边权的连通简单图上.
\begin {theorm} 在给定起点的图G(V, E)中,设起点的权为0,其余各点的权为无穷大,设置信号灯为1.接下来,如果信号灯为1,则执行以下操作:信号灯变为0,$\forall e\in E, $若:$$ e.from.cost+e.length>e.to.cost $$那么$$ e.to.cost=e.length+e.from.cost $$并且信号灯改为1.重复上述操作直到信号灯为0.\cite {ref2} \end {theorm}

容易发现,Bellman-Ford算法是Dijkstra算法的类似,只是,Dijkstra算法以贪心法选取未被处理的具有最小权值的结点,然后对其的出边进行松弛操作;而Bellman-Ford算法简单地对所有边进行松弛操作.\cite {ref7}也正因如此,Bellman-Ford算法可以处理有负边的图.既然是对Dijkstra算法的加强,那么算法正确性的证明和Dijkstra类似,这里就不再赘述.为了得到Bellman-Ford算法的复杂度,我们来证明下面这个定理:
\begin {extra} Bellman-Ford算法的执行过程中,信号灯至多在第$|V|$次操作后变为0. \end {extra}
\paragraph {证明:} 假定从$v_0$到$v_k$的最短路是$$v_0\rightarrow v_1\rightarrow\cdots\rightarrow v_k$$当k=0时,结论显然成立;假定经过k-1轮操作,所有长度为k-1的最短路都已经被找到,在进行第k次松弛时,$v_k.cost$肯定会被$v_{k-1}\rightarrow v_k$松弛,因此$$ v_k.cost\le v_{k-1}.cost+e(v_{k-1}, v_k).length $$因此找到了该最短路径.\qed

因此,Bellman-Ford最多只需要操作$|V|$次即可.时间复杂度为\\$O(|V||E|)$.\cite {ref7}

此外,从上面的讨论中,我们还能得到一个结论:\begin {extra} 如果对图G(V, E)进行$|V|$次Bellman-Ford算法操作后,信号灯仍然为1,则图G中存在负环.\end {extra}

根据上述的讨论,这个结论是显然的,该方法可用于判断图中是否有负环.

我们在上面的讨论中可以发现,虽然Bellman-Ford算法可以处理有负边的图,但是复杂度偏高,能不能找到方法把时间复杂度降低呢?实际上是可以的.注意到Bellman-Ford算法中,每次操作都需要对全部的边进行松弛,其中包含大量无用松弛,比如,刚开始的时候,未被访问的结点也作为起点参与了松弛,这是没有意义的.再比如,在上一次松弛中,权没有改变的结点也没有必要作为起点进行松弛.我们对这些点进行了去除,就有了快速最短路径算法(Shortest Path Faster Algorithm (SPFA)).

\subsection {Shortest Path Faster Algorithm (SPFA)}

快速最短路径算法SPFA是Bellman-Ford算法的优化,使用了队列优化,首先在1959年由Edward F. Moore作为广度优先搜索的扩展发表.\cite {ref8}相同算法在1994年由段凡丁重新发现.\cite {ref9}快速最短路径算法的操作如下:
\begin {theorm} 在给定起点的图G(V, E)中,设起点的权为0,其余各点的权为无穷大.维护一个队列,开始时队列中只有起点,若队列非空,则取出队首元素u,并执行如下操作:依次访问u的邻接结点,对于任意邻接结点v,如果:$$ u.cost+e(u, v).length>v.cost $$则:$$ v.cost=u.cost+e(u, v).cost $$并且若v不在队列中,则将v加入队尾.\end {theorm}
SPFA本质上是Bellman-Ford的优化,它避免了无用的松弛判断,正确性由Bellman-Ford算法给出.Pacific给出SPFA的复杂度计算,SPFA的数学期望复杂度为$O(|V|\log|V|\log(|E|/|V|)+|E|)$.\cite {ref10}是一种不错的算法.

\begin {center}
\section {各最短路算法的对比及最短路算法的应用}
\end {center}

在前文中,我们介绍了寻找图G单源最短路问题的几种算法,我们将图中的情况按照有无边权、有无负边分成了三种情况.各情况有不同的算法适用.下表对比了各个算法的使用情况和复杂度:
$$\resizebox{\textwidth}{!}{\begin {array} {|l|c|c|}
\hline
\text{算法} & \text{适用情况} & \text{复杂度}\\\hline
DFS & \text{全部} & \text{起点出发的所有的路}\\
BFS & \text{无边权、无负边} & O(|E|) \\
Dijkstra \text{算法} & \text{无负边} & O(|V|^2+|E|)\\
\text{二叉堆优化的Dijkstra 算法} & \text{无负边} & O((|E|+|V|)log|V|)\\
Bellman-Ford \text{算法} & \text{无负环} & O(|V||E|)\\
SPFA & \text{无负环} & O(|V|\log|V|\log(|E|/|V|)+|E|)\\\hline
\end {array}}$$
我们给出下面的例子:
\begin {ex}
给定路径表,路径表为csv格式,共有n行,代表n条路径.每一行的格式为:
$$\begin {array} {|l|l|l|l|l|l|}
\hline
from & to & way & time & cost & \text{extra info}\\\hline
\end {array}$$
给定起点和终点,分别给出时间最短和花费最少的出行方式. 
\end {ex}

本题无负边,可用Dijkstra算法或SPFA算法算出,这里给出一个SPFA算法的C++代码供参考.
\begin {figure} [H]
    \centering
    \includegraphics[width=1\textwidth]{2.jpg}
\end {figure}
\begin {figure} [H]
    \centering
    \includegraphics[width=1\textwidth]{3.jpg}
    \includegraphics[width=1\textwidth]{4.jpg}
    \includegraphics[width=1\textwidth]{5.jpg}
\end {figure}
\begin {figure} [H]
    \centering
    \includegraphics[width=1\textwidth]{6.jpg}
    \includegraphics[width=1\textwidth]{7.jpg}
    \includegraphics[width=1\textwidth]{8.jpg}
    \includegraphics[width=1\textwidth]{9.jpg}
\end {figure}

最短路问题的提出和最短路算法的实现让导航和地图成为了可能,实际上,为了提高效率,在实际应用中,我们通常会根据不同的情况选择不同的方法.在一些情况下,我们不一定非要找到最短路径,有的时候我们愿意找到一条近似的最短路,从而减少计算的时间,这就是贪心算法.一个好的启发函数是贪心算法的保证,因此,有人将最短路算法和贪心算法结合,通过自学习来获取一个合适的函数,从而提高算法的实用性.最短路算法还有广阔的市场,许多都对最短路算法有需求,这也必将会成为最短路算法被不断研究的动力.


\begin {multicols} {2}
\begin {thebibliography}{99} 
\bibitem {ref1} Wikipedia, 最短路问题.\url {https://en.wikipedia.org/wiki/Shortest_path_problem}
\bibitem {ref2} Introduction to Algorithms [算法导论]. ISBN 978-7-111-40701-0.
\bibitem {ref3} Dijkstra, E. W. A note on two problems in connexion with graphs (PDF). Numerische Mathematik. 1959, 1: 269–271 [2020-01-27]. doi:10.1007/BF01386390.
\bibitem {ref4} Wikipedia, Dijkstra's algorithm. \url {https://en.wikipedia.org/wiki/Dijkstra%27s_algorithm}
\bibitem {ref5} Johnson, Donald B. Efficient algorithms for shortest paths in sparse networks. Journal of the ACM. 1977, 24 (1): 1–13. doi:10.1145/321992.321993.
\bibitem {ref6} Purelight. 二叉堆的定义、性质、操作和使用. \url {https://mp.weixin.qq.com/s/5gTQhLl6r53mTCMDkYn4RA}
\bibitem {ref7} Wikipedia, Bellman–Ford algorithm. \url {https://en.wikipedia.org/wiki/Bellman%E2%80%93Ford_algorithm}
\bibitem {ref8} Moore, Edward F. The shortest path through a maze. Proceedings of the International Symposium on the Theory of Switching. Harvard University Press: 285–292. 1959. SPFA is Moore's “Algorithm D”.
\bibitem {ref9} Duan, Fanding, 关于最短路径的SPFA快速算法, 西南交通大学学报 [Journal of Southwest Jiaotong University], 1994, 29 (2): 207–212
\bibitem {ref10} Pacific. 如何卡spfa? - Pacific的回答 - 知乎. 2018-03-09 [2019-07-23].
\bibitem {ref11} \url {https://github.com/Czile-create/SPFA}
\bibitem {ref12} \url {https://www.cnblogs.com/tianjifa/p/10443113.html}
\end {thebibliography}
\end {multicols}

\end {document}
