% test3

% Helium Atom Numerical Solution TDSE Notes
% Helium atom|Schrodinger equation|wave function|momentum spectrum|photoelectron|ionization|FEDVR|numerical solution

\pentry{Generalized spherical harmonic function\upref{GenYlm}}% already contains "tensor product space"

\subsection{Wave function expansion}
The angular basis is the eigenstate \upref{AMAdd} of the total orbital angular momentum of the two electrons and the $z$ component, which is the generalized spherical harmonic function $\mathcal{Y}_{l_1,l_2}^{L,M} (\uvec r_1, \uvec r_2)$.

The space where the total wave function is located can be regarded as the tensor product space of angular space and radial space. Now that we have a base in angular space, the total wave function can be expanded on this base
\begin{equation}
\ket{\Psi} = \sum_\lambda \ket{R_\lambda}\ket{\mathcal{Y_\lambda}}
\end{equation}
Among them, $\lambda$ is the sequence number after all the $(l_1,l_2,L,M)$ combinations are sorted. $\ket{R_\lambda}$ is the two-dimensional radial wave function
\begin{equation}
\ket{R_\lambda} = \frac{1}{r_1 r_2} \psi_\lambda(r_1, r_2)
\end{equation}
That is, the total wave function is
\begin{equation}\label{test3_eq2}
\Psi(\bvec r_1, \bvec r_2) = \sum_{L, M, l_1, l_2} \frac{1}{r_1 r_2} \psi_{l_1, l_2}^{L, M}(r_1, r_2) \mathcal{Y}_{l_1, l_2}^{L, M}(\uvec r_1, \uvec r_2)
\end{equation}

From the conclusion in the tensor product space \upref{DirPro} , we can find the "matrix element" $H_{\lambda, \lambda'}$ of the Hamiltonian $H$ on the angular basis, and each matrix element is An operator in radial space. List the "matrix form" of the Schrodinger equation, and you get a set of Schrodinger equations for the radial wave function of couple
\begin{equation}
\sum_{\lambda'} H_{\lambda, \lambda'} \ket{R_{\lambda'}} = \I \pdv{t} \ket{R_{\lambda}}
\end{equation}

The total Hamiltonian can be expressed as
\begin{equation}
H = H_1 + H_2 + V_{12} + V_{F1} + V_{F2}
\end{equation}
Where $H_i \ \ (i = 1, 2)$ is the Hamiltonian of a single electron, opposite to the angular matrix
\begin{equation}
H_i = K_{ri} + \frac{L_i^2}{2m r_i^2}-\frac{2}{r_i} \qquad (i = 1,2)
\end{equation}
The second term is a diagonal matrix because the basis of $\mathcal Y$ is the eigenfunction of $L_i^2$. $V_{12}$ is the Coulomb potential energy between two electrons, and $V_{Fi}$ is the potential energy of the electric field on each electron.

\subsection{Symmetry of wave function}
We generally assume that the electron spin state is always singlet (antisymmetric), so the total wave function satisfies particle exchange symmetry $\Psi(\bvec r_1, \bvec r_2) = \Psi(\bvec r_2, \bvec r_1)$. The symmetry of the partial wave can be derived from \autoref{GenYlm_eq5}~\upref{GenYlm}
\begin{equation}\label{test3_eq1}
\psi_{l_1, l_2}^{L, M}(r_1, r_2) = (-1)^{l_1 + l_2-L} \psi_{l_2, l_1}^{L, M}(r_2, r_1)
\end{equation}
Theoretically, using this formula can reduce the repetitive partial wave by half in the program, and the coupling evolution of the partial wave can also be simplified, but it has not been realized for the time being due to the trouble of programming.

In addition, it can be proved (actually found by numerical calculations) that $l_1 + l_2-L$ is an odd partial wave (odd partial wave) and $l_1 + l_2-L$ is an even partial wave (even partial wave) in the online polarization electric field It evolves independently, that is, $V_{12}$ and $z_1, z_2$ will not couple the partial waves with different parities. So if the helium atom evolves from the ground state (including only the even partial wave: $l_1 = l_2, L = 0$) in the linear polarization electric field, then the odd partial wave term will always be zero and can be removed in the program.

\subsection{Electronic Interaction Item}
\begin{equation}
\ali{
V_{12} = \frac{1}{\abs{\bvec r_2-\bvec r_1}} &= 4\pi \sum_{l=0}^{\infty} \frac{1}{2l+1} \frac{r_<^l}{r_>^{l+1}} \sum_{m = -l}^l Y_{l,m}^*(\uvec r_1) Y_{l,m}(\uvec r_2)\\
&= 4\pi \sum_{l=0}^{\infty} \frac{(-1)^l}{\sqrt{2l + 1}} \frac{r_<^l}{r_>^{l +1}} \mathcal{Y}_{l,l}^{0,0} (\uvec r_1, \uvec r_2)
}\end{equation}
This is equivalent to decomposing the linear operator in the tensor product space into the sum of the tensor product of radial and angular operators. The second step uses \autoref{SphCup_eq9}~\upref{SphCup} , that is
\begin{equation}
\mathcal{Y}_{l,l}^{0,0} (\uvec r_1, \uvec r_2) = \frac{(-1)^l}{\sqrt{2l+1}} \sum_{m = -l}^l Y_{lm}^*(\uvec r_1) Y_{lm}(\uvec r_2)
\end{equation}

To calculate the matrix element $\mel{\mathcal Y_\lambda}{V_{12}}{\mathcal Y_{\lambda'}}$, the linear combination of the product of 6 spherical harmonic functions must be made twice angular integral. Using \autoref{SphCup_eq1}~\upref{SphCup} , the above formula can be expressed as a linear combination of six CG coefficients (double summation).
\begin{equation}\label{test3_eq10}
\ali{
&\quad\mel{\mathcal{Y}_{l_1 l_2}^{LM}}{Y_{lm}^*(\uvec r_1)Y_{lm}(\uvec r_2)}{\mathcal{Y}_ {l_1' l_2'}^{L'M'}}\\
&= \sum_{m_1',m_2'}\sum_{m_1,m_2} \bmat{l_1 & l_2 & L\\ m_1 & m_2 & M}\bmat{l_1' & l_2' & L'\\ m_1' & m_2' & M'} \times\\
&\qquad \iint Y_{l_1m_1}^*(\uvec r_1)Y_{l_2m_2}^*(\uvec r_2)Y_{lm}^*(\uvec r_1)Y_{lm}(\uvec r_2)Y_{l_1 'm_1'}(\uvec r_1)Y_{l_2' m_2'}(\uvec r_2) \dd{\Omega_1}\dd{\Omega_2}\\
&=\sum_{m_1',m_2'}\sum_{m_1,m_2} (-1)^{m_1 + m_2 + m} \bmat{l_1 & l_2 & L\\ m_1 & m_2 & M}\bmat{l_1 '& l_2' & L'\\ m_1' & m_2' & M'} \times\\
&\qquad \int Y_{l_1, -m_1}(\uvec r_1)Y_{l,-m}(\uvec r_1)Y_{l_1' m_1'}(\uvec r_1) \dd{\Omega_1} \int Y_ {l_2, -m_2}(\uvec r_2)Y_{lm}(\uvec r_2)Y_{l_2' m_2'}(\uvec r_2) \dd{\Omega_2}\\
&=\frac{2l+1}{4\pi}\sqrt{(2l_1+1)(2l_1'+1)(2l_2+1)(2l_2'+1)} \sum_{m_1',m_2'}\ sum_{m_1,m_2} (-1)^{m_1 + m_2 + m} \times\\
&\qquad\pmat{l_1 & l & l_1'\\ 0 & 0 & 0}\pmat{l_1 & l & l_1'\\ -m_1 & -m & m_1'}\pmat{l_2 & l & l_2'\ \ 0 & 0 & 0}\pmat{l_2 & l & l_2'\\ -m_2 & m & m_2'}\times\\
&\qquad\bmat{l_1 & l_2 & L\\ m_1 & m_2 & M}\bmat{l_1' & l_2' & L'\\ m_1' & m_2' & M'}
}\end{equation}
Now try to sum $m$ and use the choice theorem (the sum of the second line of the 3j symbol is 0), (note the range of $m$) at most only one item in the sum is not 0, so
\begin{equation}\label{test3_eq9}
\ali{
& \mel{\mathcal{Y}_{l_1 l_2}^{LM}}{\mathcal{Y}_{l,l}^{0,0}}{\mathcal{Y}_{l_1' l_2' }^{L'M'}}
=\\
& \delta_{M,M'} (-1)^l \frac{\sqrt{2l+1}}{4\pi} \sqrt{(2l_1+1)(2l_1'+1)(2l_2+1) (2l_2'+1)}\times\\
&\sum_{m_1',m_2'}\sum_{m_1,m_2} (-1)^{m_1' + m_2} \bmat{l_1 & l_2 & L\\ m_1 & m_2 & M}\bmat{l_1' & l_2' & L'\\ m_1' & m_2' & M} \pmat{l_1 & l & l_1'\\ 0 & 0 & 0} \times\\
&\pmat{l_1 & l & l_1'\\ -m_1 & m_1-m_1' & m_1'}\pmat{l_2 & l & l_2'\\ 0 & 0 & 0}\pmat{l_2 & l & l_2'\ \ -m_2 & m_2-m_2' & m_2'}
}\end{equation}
Among them, the summation requirements of $m_1, m_2$
\begin{equation}
m_1 + m_2 = m_1' + m_2' = M \qquad
\abs{m_1'-m_1} \leqslant l \qquad
\abs{m_2'-m_2} \leqslant l
\end{equation}

Since the CG coefficients are all real numbers, this must be a real number matrix. Selection rules are as follows
\begin{itemize}
\item According to the conservation of total angular momentum ($1/r_{12}$ operator and $L, L_z$ operator are commutative), when $L \ne L'$ or $M \ne M'$, the matrix element is zero .
\item According to parity CG coefficients, when $l_1' + l + l_1$ or $l_2' + l + l_2$ are odd numbers, the above equation is equal to zero.
\item According to $\Pi \mathcal{Y} = (-1)^{l_1 + l_2} \mathcal{Y}$, considering that $1/r_{12}$ has an even parity, so $l_1 + l_2$ and The parity of $l_1' + l_2'$ must be different.
\end{itemize}

Aihua’s thesis Lieutenant General \autoref{test3_eq9} is represented by the 9j symbol, but only limited to $M \ne M'$. However, after numerical verification, I found that it should be possible to extend Aihua’s formula to
\begin{equation}
\ali{
\mel{\mathcal{Y}_{l_1 l_2}^{LM}}{\mathcal{Y}_{ll}^{00}}{\mathcal{Y}_{l_1' l_2'}^{L' M'}}
= &\frac{2l+1}{4\pi} \delta_{L,L'}\delta_{M,M'} \sqrt{(2l_1'+1)(2l_2'+1)(2L+1) }\times\\
&\bmat{l & l_1' & l_1\\ 0 & 0 & 0}
\bmat{l & l_2' & l_2\\ 0 & 0 & 0}
\Bmat{l & l_1' & l_1\\ l & l_2' & l_2\\ 0 & L & L}
}\end{equation}
Note that when $M = M'$, the result has nothing to do with $M$.

\subsubsection{Poisson integral}
According to the thinking of the FEDVR base, we will naturally want to expand the $V_{12}$ operator to $\mel{r_i r_j}{r_<^l/r_>^l}{r_{i'} r_{j '}}$ is calculated by Gaussian Lobatto\upref{GLquad} integral, but this is not accurate, because the expansion of $r_<^l/r_>^l$ with finite-order polynomials will cause errors. Using Poisson's integral method can reduce the calculation error and still maintain the form of the diagonal matrix, that is
\begin{equation}
\mel*{r_i r_j}{\frac{r_<^l}{r_>^{l+1}}}{r_{i'} r_{j'}} \propto \delta_{i,i'} \ delta_{j,j'}
\end{equation}
According to my guess (check it if you have time!), the result of Poisson integral integration should be the same as the result of direct numerical integration.
\begin{equation}\label{test3_eq3}
\mel*{r_i r_j}{\frac{r_<^l}{r_>^{l+1}}}{r_{i'} r_{j'}} = \delta_{i,i'} \delta_ {j,j'} \qty[\frac{2l+1}{r_i r_j \sqrt{\omega_i \omega_j}} (2\mat T_l)^{-1}_{i,j} + \frac{r_i ^l r_j^l}{r_{max}^{2l+1}}]
\end{equation}
Derivation refer to McCurdy, Solving the three-body Coulomb breakup problem using exterior complex scaling formula 68. Where $\mat T_l$ is the matrix of single electron kinetic energy operators
\begin{equation}
\begin{aligned}
(T_l)_{ij} &= \mel*{\phi_i}{\qty[-\frac{1}{2m}\dv[2]{r} + \frac{l(l+1)}{2mr }]}{\phi_j}\\
&= -\frac{1}{2m} \mel*{\phi_i}{\dv[2]{r}}{\phi_j} + \delta_{i,j}\frac{l(l+1)} {2mr_i^2}
\end{aligned}
\end{equation}
Among them, $\phi_i$ is the orthogonal normalized base \upref{FEDVR} of the one-dimensional FEDVR grid point at $r_i$, and $r_{max}$ is the boundary of the last FE when calculating $T_l$. Although \autoref{test3_eq3} contains $r_{max}$, the result has nothing to do with $r_{max}$. The reason is that $T_l$ is also related to $r_{max}$. In the two items, $r_{max }$ offset.

\subsection{Electric field action term}
For each electron,
\begin{equation}\label{test3_eq12}
V_F = \bvec E \vdot \bvec r = E_x x + E_y y + E_z z
\end{equation}
Same as in the case of hydrogen atoms, there are
\begin{equation}
x = \sqrt{\frac{2\pi}{3}} r (Y_{1,-1}-Y_{1,1} ) \qquad
y = \I \sqrt{\frac{2\pi}{3}} r (Y_{1,-1}+Y_{1,1} ) \qquad
z =2\sqrt{\frac{\pi}{3}} rY_{1,0}
\end{equation}
For the linearly polarized electric field, we only need the last item, taking the first electron as an example,
% Verified: M = M'= 0 is the same as Aihua's program
\begin{equation}\ali{
&\mel{\mathcal{Y}_{l_1 l_2}^{LM}}{z_1}{\mathcal{Y}_{l_1' l_2'}^{L'M'}}
= 2\sqrt{\frac{\pi}{3}} r \sum_{m_1',m_2'}\sum_{m_1,m_2} \bmat{l_1 & l_2 & L\\ m_1 & m_2 & M}\bmat {l_1' & l_2 & L'\\ m_1' & m_2' & M'} \times\\
&\qquad \iint Y_{l_1m_1}^*(\uvec r_1)Y_{l_2m_2}^*(\uvec r_2)Y_{10}(\uvec r_1)Y_{l_1' m_1'}(\uvec r_1)Y_{ l_2' m_2'}(\uvec r_2) \dd{\Omega_1}\dd{\Omega_2}\\
&= 2 \sqrt{\frac{\pi}{3}} \delta_{l_2, l_2'} r \sum_{m_1,m_2, m_1'} \bmat{l_1 & l_2 & L\\ m_1 & m_2 & M }\bmat{l_1' & l_2' & L'\\ m_1' & m_2 & M'} \times\\
&\qquad \iint Y_{l_1m_1}^*(\uvec r_1) Y_{10}(\uvec r_1)Y_{l_1' m_1'}(\uvec r_1)\dd{\Omega_1}\\
&= \delta_{l_2, l_2'} \sqrt{(2l_1 + 1)(2l_1'+1)} r \sum_{m_1,m_2, m_1'} (-1)^{m_1} \times\\
&\qquad \bmat{l_1 & l_2 & L\\ m_1 & m_2 & M} \bmat{l_1' & l_2 & L'\\ m_1' & m_2 & M'} \pmat{l_1 & 1 & l_1'\\ 0 & 0 & 0} \pmat{l_1 & 1 & l_1'\\ -m_1 & 0 & m_1'}\\
&= \delta_{l_2, l_2'} \delta_{M, M'} \sqrt{(2l_1 + 1)(2l_1'+1)} r \sum_{m_1,m_2} (-1)^{m_1} \ times\\
&\qquad \bmat{l_1 & l_2 & L\\ m_1 & m_2 & M} \bmat{l_1' & l_2 & L'\\ m_1 & m_2 & M} \pmat{l_1 & 1 & l_1'\\ 0 & 0 & 0} \pmat{l_1 & 1 & l_1'\\ -m_1 & 0 & m_1}
}\end{equation}
The last step uses the properties of $m_1 + m_2 + m_3 = 0$ in the 3j symbol and $m_1 + m_2 = M$ in the CG coefficient. The two delta functions verify that the $z$ direction electric field acting on the first electron will not affect the second electron’s angular momentum and the $z$ component of the total angular momentum. In the derivation, the conclusion that $m_1' = m_1, m_2' = m_2$ is successively drawn, that is, the electric field also does not affect the angular momentum $z$ component of each electron (so the calculation of $m_1', m_2'$ And eliminated). In the final summation of $m_1,m_2$, it is required that $m_1+m_2=M$, and $\abs{m_1} <\min\qty{l_1, l'_1}$.

In the same way, for the second electron
% Verified: M = M'= 0 is the same as Aihua's program
\begin{equation}\ali{
&\mel{\mathcal{Y}_{l_1 l_2}^{LM}}{z_2}{\mathcal{Y}_{l_1' l_2'}^{L'M'}}
= \delta_{l_1, l_1'} \delta_{M, M'} \sqrt{(2l_2 + 1)(2l_2'+1)} r \sum_{m_1,m_2} (-1)^{m_2} \times \\
&\qquad \bmat{l_1 & l_2 & L\\ m_1 & m_2 & M} \bmat{l_1 & l_2' & L'\\ m_1 & m_2 & M} \pmat{l_2 & 1 & l_2'\\ 0 & 0 & 0} \pmat{l_2 & 1 & l_2'\\ -m_2 & 0 & m_2}
}\end{equation}

In Aihua’s paper, $M = M'= 0$ is represented by the 9j symbol as (the numerical verification is the same)
\begin{equation}
\begin{aligned}
&\mel{\mathcal{Y}_{l_1 l_2}^{LM}}{z_1}{\mathcal{Y}_{l_1' l_2'}^{L'M'}}=
r\sqrt{9(2l'_1+1)(2l'_2+1)(2L'+1)/(4\pi)^2}\\
&\times\bmat{1 & l'_1 & l_1\\ 0 & 0 & 0} \bmat{0 & l'_2 & l_2\\ 0 & 0 & 0} \bmat{1 & L'& L\\ 0 & 0 & 0}\Bmat{1 & l'_1 & l_1\\ 0 & l'_2 & l_2\\ 1 & L'& L}
\end{aligned}
\end{equation}
\begin{equation}
\begin{aligned}
&\mel{\mathcal{Y}_{l_1 l_2}^{LM}}{z_2}{\mathcal{Y}_{l_1' l_2'}^{L'M'}}=
r\sqrt{9(2l'_1+1)(2l'_2+1)(2L'+1)/(4\pi)^2}\\
&\times\bmat{0 & l'_1 & l_1\\ 0 & 0 & 0} \bmat{1 & l'_2 & l_2\\ 0 & 0 & 0} \bmat{1 & L'& L\\ 0 & 0 & 0}\Bmat{0 & l'_1 & l_1\\ 1 & l'_2 & l_2\\ 1 & L'& L}
\end{aligned}
\end{equation}

\subsection{Nonlinear polarization electric field}
For the nonlinear polarization electric field, the first two items of \autoref{test3_eq12} are not zero, but only the same as the result of the $z$ component (only the last 3j coefficient becomes two items, and then divide by $\sqrt{2}$ To normalize)
% Not completed: None of the following four items have been verified
\begin{equation}\ali{
&\mel{\mathcal{Y}_{l_1 l_2}^{LM}}{x_1}{\mathcal{Y}_{l_1' l_2'}^{L'M'}}
= \frac{\delta_{l_2, l_2'} \delta_{M, M'}}{\sqrt2} \sqrt{(2l_1 + 1)(2l_1'+1)} r \times\\
&\qquad \sum_{m_1,m_2} (-1)^{m_1} \bmat{l_1 & l_2 & L\\ m_1 & m_2 & M} \bmat{l_1' & l_2 & L'\\ m_1 &m_2 & M} \times\\
&\qquad \pmat{l_1 & 1 & l_1'\\ 0 & 0 & 0}\qty[\pmat{l_1 & 1 & l_1'\\ -m_1 & -1 & m_1}-\pmat{l_1 & 1 & l_1'\\ -m_1 & 1 & m_1}]
}\end{equation}

\begin{equation}\ali{
&\mel{\mathcal{Y}_{l_1 l_2}^{LM}}{y_1}{\mathcal{Y}_{l_1' l_2'}^{L'M'}}
= \frac{\I \delta_{l_2, l_2'} \delta_{M, M'}}{\sqrt2} \sqrt{(2l_1 + 1)(2l_1'+1)} r \times \\
& \qquad \sum_{m_1,m_2} (-1)^{m_1} \bmat{l_1 & l_2 & L\\ m_1 & m_2 & M} \bmat{l_1' & l_2 & L'\\ m_1 & m_2 & M} \times\\
&\qquad \pmat{l_1 & 1 & l_1'\\ 0 & 0 & 0}\qty[\pmat{l_1 & 1 & l_1'\\ -m_1 & -1 & m_1} + \pmat{l_1 & 1 & l_1'\\ -m_1 & 1 & m_1}]
}\end{equation}

\begin{equation}\ali{
&\mel{\mathcal{Y}_{l_1 l_2}^{LM}}{x_2}{\mathcal{Y}_{l_1' l_2'}^{L'M'}}
= \frac{\delta_{l_1, l_1'} \delta_{M, M'}}{\sqrt2} \sqrt{(2l_2 + 1)(2l_2'+1)} r\\
&\qquad \sum_{m_1,m_2} (-1)^{m_2} \bmat{l_1 & l_2 & L\\ m_1 & m_2 & M} \bmat{l_1 & l_2' & L'\\ m_1 & m_2 & M} \times\\
&\qquad \pmat{l_2 & 1 & l_2'\\ 0 & 0 & 0} \qty[\pmat{l_2 & 1 & l_2'\\ -m_2 & -1 & m_2}-\pmat{l_2 & 1 & l_2'\\ -m_2 & 0 & m_1}]
}\end{equation}

\begin{equation}\ali{
&\mel{\mathcal{Y}_{l_1 l_2}^{LM}}{y_2}{\mathcal{Y}_{l_1' l_2'}^{L'M'}}
= \frac{\I \delta_{l_1, l_1'} \delta_{M, M'}}{\sqrt2} \sqrt{(2l_2 + 1)(2l_2'+1)} r\\
&\qquad \sum_{m_1,m_2} (-1)^{m_2} \bmat{l_1 & l_2 & L\\ m_1 & m_2 & M} \bmat{l_1 & l_2' & L'\\ m_1 & m_2 & M} \times\\
&\qquad \pmat{l_2 & 1 & l_2'\\ 0 & 0 & 0} \qty[\pmat{l_2 & 1 & l_2'\\ -m_2 & -1 & m_2} + \pmat{l_2 & 1 & l_2'\\ -m_2 & 0 & m_1}]
}\end{equation}

\subsection{Binding State}
Bound states with lower energy can be easily obtained by imaginary time evolution, that is, replacing $\exp(-\I H \Delta t)$ with $\exp(- H \Delta t)$, which is equivalent to changing $\Delta t $ Becomes an imaginary number $-\I \Delta t$, hence the name. The ground state of helium atom only contains the partial waves of $L = 0$, which requires $l_1 = l_2 = 0, 1, 2, \dots$ (refer to \autoref{AMAdd_fig2}~\upref{AMAdd} ).

\subsection{wave function evolution}

Evolution (propagation) also uses split operator
\begin{equation}
\exp(-\I H\Delta t) = \exp(-\I H_0\frac{\Delta t}{2} )\exp(-\I H_{int}\Delta t) \exp(-\I H_0 \frac{\Delta t}{2} ) + \order{\Delta t^3}
\end{equation}
It’s just that $H_{int} = V_{12} + V_F$ is all $\mel{\mathcal{Y}}{\ }{\mathcal{Y}}$ after the effect is a non-diagonal (operator) matrix The term is the interaction of electrons and the effect of the electric field on each electron.

$H_0 = H_1 + H_2$ Since $H_1$ and $H_2$ are exchangeable, we can evolve them independently
\begin{equation}
\exp(-\I H_0 \frac{\Delta t}{2} ) = \exp(-\I H_1 \frac{\Delta t}{2} ) \exp(-\I H_2 \frac{\Delta t }{2} )
\end{equation}
That is, for the two-dimensional grid of each partial wave, use a method similar to hydrogen atoms to evolve each column and then evolve each row.

According to the above derivation, it can be seen that after $H_{int}$ is expanded into a linear combination of the tensor product of the radial operator and the angular operator, the radial operators are all functions of $r$. The bases are all diagonal matrices (even if Poisson integrals are used). So when evolving $\exp(-\I H_{int}\Delta t)$, each two-dimensional FEDVR grid point can be performed independently, which provides the possibility for a large number of parallelism. In summary, the \textbf{evolution in each direction of the three-dimensional wavefunction matrix can be performed independently for each column} in the program.

When evolving $\exp(-\I H_{int}\Delta t)$, Aihua's approach is to further split into three operators
\begin{equation}\label{test3_eq4}
\exp(-\I H_{int}\Delta t) = \exp(-\I V_F\frac{\Delta t}{2}) \exp(-\I V_{12} \Delta t) \exp( -\I V_F\frac{\Delta t}{2})
\end{equation}
This is because $V_F = V_{F1} + V_{F2}$ is time dependent and proportional to the electric field component $E_z$.

The reason for Aihua's further division is that the eigenvalue and eigenvector matrix after the diagonalization of $V_F$ and $V_{12}$ are pre-stored in his code. If you directly diagonalize $H_{int}$ , Then you need to re-diagonalize each step. The memory required for pre-storage is proportional to the square of the number of subwaves, but when we need a lot of subwaves, the memory is far from enough. In order to solve the memory problem, I use Crank-Nicolson evolution both radially and angularly, and regenerate the $V_F + V_{12}$ matrix for each grid point at each time step, and one evolution does not need to be divided into three times. For the grid points whose absolute value of the wave function is lower than a certain threshold, we can even ignore it and not evolve.

Another point, if you only care about single ionization, you can use a rectangular grid (see Ossiander 2017 Nat. Phys.). But pay attention to adding absorber on the short side to prevent reflection. )

\subsection{Partial wave convergence analysis}
Using $l_{1,max} , l_{2,max} , L_{max}$ to generate the partial wave list is sometimes wasteful, and there will be many partial waves with extremely low probability. The best way is to draw the probability of each partial wave as a graph. From the figure, it is easy to see whether the partial wave has converged and whether there is waste.

\subsection{Analysis of partial wave error (original)}
When we intercept a finite number of partial waves, new errors will be introduced during the evolution of the partial waves (\autoref{test3_eq4} ). After interception, $V_F, V_{12}$ are still Hermitian matrices, so their propagators are also unitary, so the modulus length of the wave function cannot be used to estimate the error. So another method can be used to estimate the error. The error can be estimated every few steps. Let $\mat V'_F, \mat V'_{12}$ be a matrix with twice the number of bases, then you can calculate $\mat (\I V'_F\Delta t) \bvec v$ Several more partial waves than the base, sum the absolute value of each partial wave.


\subsection{Binding State}
Theoretically, the eigenvalues ​​of the total Hamiltonian matrix can be solved numerically, but in practice this matrix is ​​so large that it is almost impossible to solve it (in fact, we never store the total Hamiltonian matrix directly, but store the split operator). We use imaginary time propagation \upref{ImgT} .

For the initial state of virtual-time evolution, I use the multiplication of the ground state of two He+ (real wave function), which is slightly different from Aihua, but this has little effect. Since our split operators are all real-numbered matrices, the exponential function is also a real-numbered matrix after using imaginary time, so the ground state wave function obtained by the imaginary time evolution must also be a pure real number.
We only need to use all partial waves with $L = 0, M = 0$.

%%eng
