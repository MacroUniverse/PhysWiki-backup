% 映射空间

%待完成
\pentry{拓扑空间\upref{Topol}}
\subsection{紧开拓扑}

给定拓扑空间$X$和$Y$,$Y^X$表示从$X$到$Y$的所有\textbf{连续映射}的集合.我们在此连续映射集合上定义一个拓扑.

\begin{definition}{紧开拓扑}

给定拓扑空间$X$和$Y$,在$X$中取一个紧子集$K$,在$Y$中取一个开集$U$,那么所有使得$K$的像包含在$U$中的连续映射构成的集合$\{f\in X^Y|f(K)\subseteq U\}$,定义为$X^Y$中的一个开集.由于这个开集是由紧子集$K$和开集$U$决定的,我们可以记为$M_{K, U}$.全体$M_{K, U}$构成的集合,作为\textbf{子基}生成一个拓扑$\mathcal{T}$\footnote{注意,这里是子基,而非拓扑基.},那么$\mathcal{T}$称为一个\textbf{紧开拓扑}.$(X^Y, \mathcal{T})$称为$X$到$Y$的\textbf{(连续)映射空间}.

\end{definition}

紧开拓扑的定义初看并不直观,实际上它和度量空间有紧密的联系.在特定条件下,紧开拓扑和度量拓扑是等价的,因此紧开拓扑可以看成度量拓扑更一般的形式.

\begin{theorem}{紧开拓扑与度量拓扑的等价性}

设拓扑空间$X$为紧空间,$Y$为度量空间,其距离函数记为$\opn{d}$,在$Y^X$上定义度量$\opn{\rho}$如下:对于$f, g\in Y^X$,令$\opn{\rho}(f, g)$等于集合$\{\opn{d}(f(x), g(x))\}_{x\in X}$中的上确界.则这个度量导出的拓扑,等价于紧开拓扑.

\end{theorem}

