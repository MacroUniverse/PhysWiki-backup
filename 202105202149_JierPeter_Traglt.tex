% 单纯剖分(三角剖分)
% triangulation|triagulate|多面体|剖分|代数拓扑|拓扑|同调|homology|topology

\pentry{单纯形与复形\upref{SimCom}}

\begin{definition}{多面体}
给定复形$K$,则$\bigcup\limits_{A\in K}A$称为$K$的\textbf{多面体(polyhedron)},记为$\abs{K}$.此时称$K$为$\abs{K}$的一个\textbf{三角剖分(triangulation)},也称\textbf{单纯剖分}.
\end{definition}

该定义是先有了复形的概念,再在此基础上得到多面体的概念.反过来,我们也可以认为多面体是欧几里得空间中一种可以表示为若干单纯形之并的子集,而由于单纯形就是各个维度的“三角形”,这种表示自然被称为三角剖分.

许多常见的拓扑空间都可以进行三角剖分.

\begin{example}{有界圆柱面}

考虑一个圆柱体的表面,即集合$\{(x, y, z)\in\mathbb{R}^3|x^2+y^2=1, z\in [0, 1]\}$,它可以表示为$6$个\textbf{规则相处}的单纯形的并集,\autoref{Traglt_fig1} 展示的是其展开的样子.

\end{example}

\begin{figure}[ht]
\centering
\includegraphics[width=8cm]{./figures/Traglt_1.pdf}
\caption{有界圆柱面的单纯剖分示意图.图中展示的是圆柱面沿着单形$[1, 2]$剪开并铺平后的样子,方便观察.圆柱面被} \label{Traglt_fig1}
\end{figure}




