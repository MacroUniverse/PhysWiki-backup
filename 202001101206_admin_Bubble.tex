% 冒泡法
% 算法|排序|冒泡|Matlab

\pentry{Matlab 的函数\upref{MatFun}}
我们先来看 Matlab 自带的排序函数 \lstinline|sort|. 假设数列 \lstinline|age| 是几个小朋友的年龄, \lstinline|name| 是这几个小朋友对应的名字, 现在按年龄从小到大排序如下
\begin{lstlisting}[language=MatlabCom]
>> age = [1, 6, 2, 5, 3];
>> name = ['a', 'b', 'c', 'd', 'e'];
>> [age1, order] = sort(x);
age1 = 1 2 3 5 6
order = 1 3 5 4 2
>> name1 = name(order);
name1 = 'acedb'
\end{lstlisting}
其中输出变量 \lstinline|order| 是排序后每个数字在原来数列中的位置索引, 即 \lstinline|name1| 等于 \lstinline|name(order)|. 现在我们用冒泡法实现 \lstinline|sort| 的功能. \lstinline|sort| 函数默认把数列升序排列, 即第二个输入变量默认为 \lstinline|'ascend'|. 若要降序排列, 可以用 \lstinline|'descend'| 作为第二个输入变量.

冒泡法是一种简单的排序算法, 但由于效率较低, 一般仅作为教学演示用. 在实际使用时还是建议用 \lstinline|sort|. 冒泡法的算法为: 以升序排列为例, 给出一个数列, 先把第一个数与第二个进行比较, 若第一个数较大, 就置换二者的位置(具体操作是, 把第一个数的值赋给一个临时变量, 再把第二个数的值赋给第一个, 最后把临时变量的值赋给第二个), 再把第二个数与第三个进行比较, 若第二个较大, 就置换二者的位置, 这样一直进行到最后两个数, 完成第一轮. 然后再进行第二轮, 第三轮, 直到某一轮没有出现置换操作, 即可确定排序完成. 至于输出变量 \lstinline|order|, 我们可以先令 \lstinline|order = 1:N|, 每置换数列的两个数, 就把 \lstinline|order| 中对应的两个数也置换即可. 这样, 数列与其原来的索引将始终一一对应. 

\Code{bubble}

第 6 行的循环每循环一次, 数列将从头到尾被扫描一遍. 每个循环开始时 \lstinline|changed| 的值被设为 0, 若有任何置换, \lstinline|changed| 则变为 1(第 11 行), 使 \lstinline|while| 的判断条件成立, 循环继续. 为了使第一个循环发生, 在循环前必须把 \lstinline|changed| 设为 1. 再来看第 9-18 行的判断结构. 如果前一个数大于后一个数, 则置换发生. 注意要置换数列中的两个数, 必须要设一个临时变量(\lstinline|temp|). 函数的最后, 判断输入变量的个数, 如果只有一个变量, 则默认按照前面的代码升序排列, 若第二个变量为 \lstinline|'descend'|, 则把 \lstinline|x| 和 \lstinline|order| 翻转一下即可.
