% SLISC 的 matt 和 matb 文件格式

\pentry{SLISC 简介}

SLISC 库提供了两种特殊的文件格式用于储存各种变量和矩阵, 保存时可以保存变量名, 读取时可以指定要读取的变量名. 两种文件分别使用拓展名 \verb|.matt| 和 \verb|.matb|. 其中 mat 表示 matrix, t 表示 text, t 表示 binary. 顾名思义, 第一种是文本文件, 第二种是二进制文件.

以下

\begin{lstlisting}[language=cpp]
#include "SLISC/matb.h"
using namespace slisc;

int main()
{
    // 随便初始化一些变量, 目前支持 SLISC 库中的绝大部分标量和密矩阵
    Int i = 1; Doub d = 3.1; Comp c(1, 2);
    VecInt v(3); linspace(3, 1, 3); CmatInt a(2,2); linspace(a);

    Matb matb("test.matb", 'w'); // 打开 matb 文件, 使用写入模式
    // 相当于 Matb matb; matb.open("test.matb", 'w');
    save(i, "i", matb); // 保存变量, 指定变量名, 可以是任意字符串
    save(d, "d", matb);
    save(c, "c", matb);
    save(v, "c", matb);
    save(a, "a", matb);
    matb.close(); // 关闭文件.
    // 在 destructor 中会自动调用, open() 时也会自动调用

    Int i1; Doub d1; Comp c1; VecInt v1(3); CmatInt a1;
    matb.open("test.matb", 'w'); // 重新用读取模式打开文件
    // 读取变量, 可以按照任何顺序, 不需要全部读取
    // 矩阵会被自动 resize()
    load(i1, "i", matb); 
    load(d1, "d", matb);
    load(c1, "c", matb);
    load(v1, "c", matb);
    load(a1, "a", matb);
    matb.close();

    // 此时可以对比读取的变量是否和保存的变量一致
}
\end{lstlisting}
