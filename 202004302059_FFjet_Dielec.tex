% 电介质

首先来了解一下何为电介质.
\begin{definition}{电介质}
\textbf{电介质(dielectric)}是电阻率很大、导电能力很差的物质,其主要特征在于它的原子或分子中的电子与原子核的结合力很强,电子处于束缚状态.当电介质处在电场中时,在电介质中,不论是原子中的电子、还是分子中的离子或是晶体点阵上的带电粒子,在电场的作用下都会在原子大小的范刚内移动,当达到静电平衡时,在电介质表面层或在体内会出现极化电荷,这个现象称作电介质的\textbf{极化(polarization)}.
\end{definition}
下面就研究电场与电介质间的相互作用,从而说明电介质的一些性质.

\subsection{电介质的电结构}

电介质中每个分子都由正、负电荷组成,是一个复杂的带电系统一般来说,正、负电荷在分子中都不集中在一点,但在考虑这些电荷在离分子较远处所产生的电场时,或是考虑一个分子受外电场作用时,可以认为分子中的全部正电荷可用一等效正电荷来代替,全部负电荷用一等效负电荷来代替等效正、负电荷在分子中所处的位置,分别称为该分子的\textbf{正负电荷“中心”}.

按照电介质分子内部的电结构不同,可以把电介质分为两大类:有极分子和无极分子.有一类电介质,如水蒸气($\rm H_2O$)、氯化氢($\rm HCl$)、一氧化碳($\rm CO$) 、氨($\rm NH_3$)等,分子内正、负电荷中心不相重合,这类分子称为\textbf{有极分子(polar molecules)}.设有极分子的正电荷中心和负电荷中心之间的距离为$l$,分子中全部正电荷或负电荷的总电荷量为$q$,则有极分子的等效电偶极矩$\mathbf p=q\mathbf l$.整块电介质可以看成是无数个电偶极子的聚集体,虽然每一个分子的等效电偶极矩不为零,但由于分子的无规则热运动,各个分子的电偶极矩的方向是杂乱无章地排列的,所以不论从电介质的整体来看,还是从电介质中的某一小体积(其中包含有大量的分子)来看,其中各个分子电偶极矩的矢量和$\sum \mathbf p$平均说来等于零,电介质是呈电中性的.
\begin{figure}[ht]
\centering
\includegraphics[width=10cm]{./figures/Dielec_1.pdf}
\caption{有极分子的电偶极矩示例} \label{Dielec_fig1}
\end{figure}