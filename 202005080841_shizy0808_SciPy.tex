% 数值计算库SciPy(一):微分与积分
\pentry{Python 入门\upref{Python}}
\verb|SciPy|函数库在\verb|NumPy|库\upref{numpy}的基础上增加了众多的数学、科学以及工程计算中常用的库函数.例如线性代数\upref{Vector}、常微分方程数值求解\upref{OdeNum}、信号处理、图像处理、稀疏矩阵\upref{SprMat}等等.
使用\verb|SciPy|库之前需要先导入库,
\begin{lstlisting}[language=python]
import  scipy#导入scipy中所有模块
from scipy import integrate#导入其中积分模块
\end{lstlisting}
\subsection{数值积分}
数值积分是对定积分的数值求解.例如可以利用数值积分计算某个形状的面积.下面让我们来考虑一下如何计算半径为1的半圆的面积,根据圆的面积公式,其面积应该等于$\pi/2$.单位半圆曲线可以用函数\verb|half_circle()|表示, 对应计算积分的代码如下:
\begin{lstlisting}[language=python]
def half_circle(x):
    return (1-x**2)**0.5

from scipy import integrate
res, err = integrate.quad(half_circle, -1, 1)
print(res,err)
输出:
1.5707963267948983 1.0002354500215915e-09
\end{lstlisting}
此处调用了积分模块下的\verb|quad|函数, 这里不仅返回了积分结果,还给出来计算误差.多重定积分的求值可以通过多次调用\verb|quad|函数实现, 为了调用方便, \verb|integrate|库提供了\verb|dblquad|函数
进行二重定积分, \verb|tplquad|函数进行三重定积分.

\verb|dblquad|函数的调用方式为:
\begin{lstlisting}[language=python]
dblquad(func2d, a, b, gfun, hfun)
\end{lstlisting}
对于\verb|func2d(x,y)|函数进行二重积分, 其中\verb|a,b|为变量$x$的积分区间,而\verb|gfun(x)|到\verb|hfun(x)|为变量$y$的积分区间.

\verb|tblquad|函数的调用方式与上面一样.

\subsection{数值解微分方程组}
\verb|scipy.integrate|库提供了数值积分和常微分方程组求解算法\verb|odeint|.下面让我们来看看如何用\verb|odeint|计算十分经典的洛仑兹吸引子的轨迹.

洛仑兹吸引子由下面的三个微分方程定义:
\begin{equation}
\dot{x}
\end{equation}
