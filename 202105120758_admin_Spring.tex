% 弹簧的串联和并联

\begin{issues}
\issueDraft
\end{issues}

\pentry{胡克定律}

理想弹簧切割后, 每段的弹性系数和弹簧长度成反比.

串联:
\begin{equation}
\frac{1}{k} = \frac{1}{k_1} + \frac{1}{k_2}
\quad \text{或} \qquad
k = \frac{k_1 k_2}{k_1 + k_2}
\end{equation}

并联:
\begin{equation}
k = k_1 + k_2
\end{equation}

可见弹簧的串联和并联分别类似于电阻的并联和串联.

\addTODO{推导}

\begin{example}{弹簧分割}
一根弹性系数为 $k$ 的弹簧划分为比例为 $x$ 和 $1-x$ 的部分, 求两部分分别的弹性系数 $k_1, k_2$.

解: 列方程
\begin{equation}
\leftgroup{
&\frac{k_1}{k_2} = \frac{1-x}{x}\\
&\frac{1}{k} = \frac{1}{k_1} + \frac{1}{k_2}
}
\end{equation}
解得
\begin{equation}\label{Spring_eq1}
k_1 = \frac{k}{x} \qquad k_2 = \frac{k}{1-x}
\end{equation}
\end{example}

\begin{example}{}
一根弹性绳劲度系数为 $k$, 固定在水平相距为 $L$ 的两点之间, 绳子原长远小于 $L$. 在绳上某处固定一个质点, 质点受重力下沉后使其平衡静止, 此时质点距离一端距离为 $x$, 求下沉的高度.
\addTODO{图;讲解}
\begin{equation}
\leftgroup{
&\frac{k_1}{k_2} = \frac{1-y}{y}\\
&\frac{1}{k} = \frac{1}{k_1} + \frac{1}{k_2}
}
\end{equation}
根据\autoref{Spring_eq1}
\begin{equation}
k_1 = \frac{k}{y} \qquad
k_2 = \frac{k}{1-y}
\end{equation}
受力分析
\begin{equation}
\leftgroup{
&T_1\sin\theta_1 + T_2\sin\theta_2 = mg\\
&T_1\cos\theta_1 = T_2\cos\theta_2
}
\end{equation}
其中
\begin{equation}
\tan\theta_1 = \frac{h}{x}
\qquad
\tan\theta_2 = \frac{h}{L-x}
\end{equation}
\begin{equation}
T_1 = \frac{x}{\cos\theta_1} k_1 \qquad
T_2 = \frac{L-x}{\cos\theta_2} k_2
\end{equation}
解得
\begin{equation}
h = \frac{mg}{Lk\qty(\frac{1}{x} + \frac{1}{L-x})}
\end{equation}
\end{example}
