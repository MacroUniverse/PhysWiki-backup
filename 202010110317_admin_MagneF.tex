% 磁场
% 磁感应强度|磁场|磁感线

\begin{issues}
\issueDraft
\end{issues}

\pentry{电场\upref{Efield}}

\textbf{磁感应强度(magnetic inductance)} $\bvec B$ 可以用广义洛伦兹力\autoref{Lorenz_eq2}~\upref{Lorenz}来定义: 磁场使得一切在电磁场中运动的粒子所受的电磁力为
\begin{equation}
\bvec F = q(\bvec E + \bvec v \cross \bvec B)
\end{equation}
由于历史原因, “磁场强度” 这个而名字已经被占用% 链接未完成
, 所以 $\bvec B$ 只好叫做磁感应强度. 磁场也可以使用安培力\upref{FAmp}来定义, 但安培力在微观本质上也是洛伦兹力. 磁场是三维空间中的矢量场\upref{Vfield}, 即空间中的每一点 $\bvec r$ 都对应一个磁场矢量 $\bvec B(\bvec r)$.




磁感应强度单位为\textbf{特斯拉(Tesla)}, 即 $\Si{kg\cdot C^{-1}s^{-1}}$.

与电场线一样, 我们可以在空间中画出许多有方向的磁感线, 使得磁感线上任意一点的方向等于该点处磁场的方向, 磁感线在单位横截面的条数与截面处的磁场大小成正比.
