% 一维阶梯势能
% 阶梯势能|薛定谔方程|散射

(未完成: 本来想借这个问题来讨论波函数的完备性, 但周期边界条件不好弄, 所以还是用对称的方势垒吧.)

使用原子单位
\begin{equation}
V(x) = \leftgroup{
&+\infty & \quad & (x \leqslant -l)\\
&0 & \quad & (-l \leqslant x < 0)\\
&V_0 & \quad & (0 \leqslant x \leqslant l)\\
&+\infty & \quad & (l \leqslant x)\\
}
\end{equation}

阶梯两侧得波数满足
\begin{equation}
k^2 = 2V_0 + k_1^2
\end{equation}

令波函数的一个解为
\begin{equation}
\psi_1(x) = \leftgroup{
&A_1 \exp(\I k x) & \quad & (-l \leqslant x < 0)\\
&C_1 \exp(\I k x) + D_1 \exp(-\I k x) && (0 \leqslant x \leqslant l)
}
\end{equation}
在原点匹配函数值和导数, 得
\begin{equation}
\leftgroup{
    A_1 &= C_1 + D_1\\
    kA_1 &= k_1 C_1 - k_1 D_1
}
\end{equation}
解得
\begin{equation}
\leftgroup{
    C_1 = \frac{k_1 + k}{2k_1} A_1\\
    D_1 = \frac{k_1 - k}{2k_1} A_1
}
\end{equation}

令波函数的一个解为
\begin{equation}
\psi_2(x) = \leftgroup{
&B_2 \exp(-\I k x) & \quad & (-l \leqslant x < 0)\\
&C_2 \exp(\I k x) + D_2 \exp(-\I k x) && (0 \leqslant x \leqslant l)
}
\end{equation}
在原点匹配函数值和导数, 得
\begin{equation}
\leftgroup{
    A_1 &= C_1 + D_1\\
    -kA_1 &= k_1 C_1 - k_1 D_1
}
\end{equation}



