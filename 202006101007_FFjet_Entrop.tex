% 熵
% 热力学|玻尔兹曼|熵|统计力学

\pentry{理想气体状态方程\upref{PVnRT}}
在热力学和统计力学中, \textbf{熵(entropy)}用于描述系统的无序程度, 是一个状态量, 通常记为 $S$. 例如若已知理想气体的 $P, V, n, T$ 等状态量, 就可以确定它的熵. %链接未完成

\subsection{宏观定义}

熵是一个系统的状态参量,它的增量为
\begin{equation}
\mathrm{d} S = \left . \frac{\Delta Q}{T}\right |_{\text{可逆}}
\end{equation}
其中,$\Delta Q$是可逆吸热,也就是说系统应无限地接近平衡态.

根据这个定义,经过任一卡诺循环,墒均回到初始值:
\begin{equation} \label{Entrop_eq1}
\sum{\text{d}S_i=0}
\end{equation}
然而,要成为真正意义上的「状态参量」的桂冠,\eqref{Entrop_eq1}应该对任何循环都成立,而不是只对卡诺循环成立.换言之,我们想要
\begin{equation}
\oint S \mathrm d S =0
\end{equation}
对于所有循环均成立.这恰恰是正确的.这个证明基于这样的事实: 热不能全部变为功而不产生其他效果.因为它意味着$\eta=1$,违背了卡诺的结论.

现在,我们证明,对于任意循环都有$\oint \mathrm d S =0 $.设系统经过如图所示的循环,图中有个温度为$T_0$的辅助热库.在循环上取一小段过程$i$.若此过程中的吸热为$\Delta Q_i$,则设想它是由一个卡诺制冷机供给的,这个制冷机工作于热库$T_0$和系统此时的温度$T_i$之间.工作过程中,制冷机从热库$T_0$吸收的热量为$\Delta Q_{0i}$,所需的功为$\Delta W_i$.卡诺制冷机满足


\subsection{微观定义}
\begin{equation}
S = k_B \ln \Omega
\end{equation}

% 举例: 高温物体热传导给低温物体, 损失多少?
 
% 未完成: 热机中熵如何变化? 等压等温绝热过程中熵如何变化?
