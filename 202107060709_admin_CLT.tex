% 中心极限定理
% 统计|随机变量|方差|高斯分布|中心极限定理

\begin{issues}
\issueDraft
\end{issues}

\pentry{高斯分布\upref{GausPD}}

若一个连续随机变量 $x$ 的平均值为零, 方差为 $\ev{x^2}$, 令 $X = \sum_i^N x_i$, 当 $N \gg 1$ 时, $X$ 满足高斯分布, 且方差为
\begin{equation}
\ev{X^2} = N\ev{x^2}
\end{equation}
证明略.