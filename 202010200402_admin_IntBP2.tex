% 分部积分(高维)
% 分部积分|散度|体积分|面积分

\pentry{矢量算符常用公式\upref{VopEq}}

\subsection{标量版}
这是最接近一元函数分部积分的高维拓展
\begin{equation}\label{IntBP2_eq2}
\int (\grad f) g \dd{\tau} = \oint fg \dd{\bvec s} - \int f (\grad g) \dd{\tau}
\end{equation}
对于二维情况, 面积分变为线积分, 方向沿逆时针. 一维情况就一元函数分部积分.

\subsubsection{证明}
把\autoref{VopEq_eq2}~\upref{VopEq} 两边积分, 移项得
\begin{equation}
\int (\grad f) g \dd{\tau} = \int \grad (fg) \dd{\tau} - \int f (\grad g) \dd{\tau}
\end{equation}
现在只需证明
\begin{equation}
\int \grad (fg) \dd{\tau} = \oint fg \dd{\bvec s}
\end{equation}
我们可以对每个分量依次证明. 两边乘以第 $i$ 个分量的单位矢量 $\uvec x_i$ 得
\begin{equation}
\int \pdv{x_i}(fg) \dd{\tau} = \oint (fg\uvec x_i) \dd{\bvec s}
\end{equation}
由散度定理得
\begin{equation}
\oint (fg\uvec x_i) \dd{\bvec s} = \int \div (fg\uvec x_i) \dd{\tau} = \int \pdv{x_i}(fg) \dd{\tau}
\end{equation}
证毕.

\subsection{矢量版}
\begin{equation}\label{IntBP2_eq1}
\int f (\div \bvec A)\dd{\tau} =  \oint f \bvec A \vdot \dd{\bvec s} - \int\bvec A \vdot (\grad f)\dd{\tau}
\end{equation}
参考\cite{GriffE}, 在电动力学中有应用(见“电场的能量\upref{EEng}”).

\subsubsection{证明}
把\autoref{VopEq_eq1}~\upref{VopEq} 两边做体积分
\begin{equation}
\int\div (f \bvec A) \dd{\tau} = \int f (\div \bvec A)\dd{\tau} + \int\bvec A \vdot (\grad f)\dd{\tau}
\end{equation}
由散度定理, 左边的体积分变为面积分 $\oint f \bvec A \vdot \dd{\bvec s}$, 移项得\autoref{IntBP2_eq1} . 证毕.
