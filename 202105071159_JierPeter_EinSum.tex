% 爱因斯坦求和约定
% 爱因斯坦求和|线性代数|求和符号|下标|上标

\pentry{对偶空间\upref{DualSp}}

\subsection{引入的动机}

在线性代数中,我们通常研究的是向量、对偶向量和线性映射等对象,也就是最多涉及二阶及以下的张量.在这种情况下,纸面上可以很方便地写出低阶张量的矩阵形式,毕竟零阶张量的坐标就是一个 $1\times 1$ 矩阵、一阶张量的坐标就是一个 $1\times n$ 或 $n\times 1$ 矩阵,二阶张量的坐标就是一个 $n\times n$ 矩阵.

在张量代数和一切需要大量应用张量的理论,如微分几何中,我们不可避免地要经常涉及高阶的张量,它们的坐标就没法用矩阵表示.我们当然可以把矩阵拓展为立体阵等概念,但随着阶数上升,这种表示法的复杂程度几何级增加;我们也可以使用\textbf{张量}\upref{Tensor}词条中所提过的向量矩阵的方法,比起立体阵要清楚一些,但套娃式的表达方式也对理解一个张量的性质造成了障碍.

爱因斯坦求和约定正是为了简洁地表达高阶张量的坐标运算而存在的.

\subsection{定义和例子}

在爱因斯坦求和约定中,各维度矩阵的元素用上下标来表示.具体来说,对于二维的矩阵,上标表示行数,下标表示列数,于是 $a^i_j$ 就表示“第 $i$ 行第 $j$ 列的元素”.进一步,我们也可以直接用 $a^i_j$ 来表示一个二维矩阵.在 “张量\upref{Tensor}” 词条的\autoref{Tensor_eq3} 中我们引入了“行矩阵的行矩阵”这一类\textbf{套娃}概念来描述二阶张量——比如说,一个“列矩阵的列矩阵”就可以表示为 $a^{ij}$,各 $a^{ij}$ 就表示这个矩阵中“第 $i$\textbf{列}的矩阵的第 $j$\textbf{列}的元素”.

由此可见,张量\upref{Tensor}词条中想方设法进行的复杂表示,都可以简洁地用上下指标来描述.“行矩阵的行矩阵”这样套娃的表达,用指标来表示其实就是 $m_{ij}$,两个下标.在这里,$m_{ij}$ 中的 $i$ 表示第 $i$ 个行矩阵,而 $j$ 表示这个行矩阵里的第 $j$ 个元素——由此可见,两个下标的地位是不一样的,先后次序很重要.因此在现代数学和物理的文献中,为了减少歧义,也通常明确地写出两个坐标的次序,比如 $m^{i\phantom{1}k}_{\phantom{1}j}$ 就表示三个指标的优先顺序分别是 $i$、$j$、$k$.

为了把矩阵乘法的规则推广到上下指标的描述法,我们有了爱因斯坦求和约定.

用一句话来总结爱因斯坦求和约定,就是:\textbf{当式子中任何一个指标出现了两次,并且一次是上标、一次是下标时,那么该式表示的实际上是对这个角标一切可能值的求和}.换言之,如果角标 $i$ 作为上标和下标各出现了一次,那么式子相当于添加了一个关于 $i$ 的求和符号 $\sum_i$.

我们举例来说明:

\begin{example}{线性函数}
从张量\upref{Tensor}中我们知道,一个 $1$-线性函数可以表示为一个向量,这样的向量常被称为\textbf{余向量}、\textbf{补向量}或者\textbf{$1$-形式}.通常,我们用下标来表示一个余向量的各分量:$\bvec{\alpha}=(\alpha_1, \alpha_2, \alpha_3)$;而用上标来表示一个通常的几何向量:$\bvec{v}=(v^1, v^2, v^3)$.注意,上标不是乘方.

$\bvec\alpha$ 和 $\bvec v$ 的内积是$$\sum\limits_{i=1, 2, 3}\alpha_i v^i$$
用爱因斯坦求和约定,$\bvec\alpha$ 和 $\bvec v$ 的内积就可以写为 $\alpha_i v^i$.
\end{example}

\begin{example}{矩阵运算}
对于矩阵 $\bvec{A}$,我们把其第 $i$ 行第 $j$ 列的元素表示为 $A^i_j$.
\begin{itemize}
\item 矩阵乘法表示为:如果 $\bvec{A}=\bvec{B}\bvec{C}$,那么 $A^i_j=B^i_k C^k_j$.
\item 矩阵 $\bvec{A}$ 的迹为 $A^i_i$.

\end{itemize}
\end{example}

由于重复出现而实际上应该是求和的指标,被称为\textbf{赝指标}或者\textbf{哑指标(dummy index)},因为它们不是真正的指标,而是可以用任意字母代替的.没有求和的指标是固定的,是真正的指标.比如说,$B^i_k C^k_j$ 中 $k$ 可以是任何字母,但是 $i$ 和 $j$ 是不可以替换成别的字母的,因为它们由 $A^i_j$ 决定了.在这里,哑指标实际上是表示遍历全部可能的真指标.

求和的哑指标不一定是遍历 $1, 2, 3$,也可能是对更多或者更少的指标求和,甚至是无穷集合.指标的范围要根据具体情境来决定.

爱因斯坦求和约定的表示方法脱胎于矩阵乘法的要求,但是却不依赖于矩阵行和列的形式,转而关注指标间的配合,相比传统的矩阵表达,能更方便地推广到高阶张量的情形中.


\subsection{若干约定俗成的表示}
\begin{itemize}
\item 向量的坐标分量用上标表示,对偶向量的坐标分量用下标表示,这对应于对偶空间\upref{DualSp}词条中说的“向量坐标表示为列矩阵,对偶向量坐标表示为行矩阵”.用物理学中更常用的术语来说,逆变向量分量用上标表示,协变向量分量用下标表示.

\item 在闵可夫斯基时空中,$x^0$ 表示时间坐标 $t$,而 $x^1$,$x^2$ 和 $x^3$ 分别表示空间坐标 $x$,$y$ 和 $z$.有时候还会使用 $x^4$ 来表示虚时间分量 $\I t$.注意这里取了光速 $c=1$.




\end{itemize}







\subsection{抽象指标}

在涉及微分几何时我们常使用抽象指标,这是爱因斯坦求和约定的延伸,请参见\textbf{抽象指标}\upref{AbsInd}词条.



\begin{issues}
如有约定俗成的补充,请继续添加.
\end{issues}


