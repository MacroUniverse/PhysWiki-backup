% 编辑器事项
% BUG|编辑器|LaTeX

\subsection{符号替换模式下的 bug}
“符号替换” 功能是指, 把一些 latex 命令(如 \verb|\alpha|)显示成符号以增加可读性, 并用一个紫色框表明. 以下按优先级排序
\begin{itemize}
\item 公式调换位置后, \verb|\autoref| 显示的序号不会自动更新

\item 自动引用同页面公式插入 label 的时候, 有可能会插入到 \verb|\begin{equation}| 的中间(有待复现).

\item 【完成】复制一个公式以后, 被替换的符号不会变为代码

\item 【完成】为什么\href{https://docs.julialang.org/en/v1/manual/unicode-input/}{这个 Julia 文档页面}中的 \lstinline|\varphi| 和 \lstinline|\phi| 粘贴到浏览器(包括符号替换规则窗口)中之后显示出来是调换的?

\item 【完成】搜索框中的 \lstinline|\phi| 和 \lstinline|\varphi| 的符号替换调换了, 但搜索的内容是正确的. 应该也按照 \lstinline|config.json| 中的规则替换(这应该和上一个 bug 有关)

\item 【完成】搜索框中不能搜索 \lstinline|\subseteq|, 因为打了 \lstinline|\subset| 以后就会直接替换成符号. 任何两个前面部分相同的命令都有这个问题

\item 【完成】自动引用公式后再连续撤销, 符号替换的紫色下划线会全部消失, 如果这时保存, tex 文件中的代码也会替换成符号

\item 【完成】有点担心符号替换会出现循环 bug, 就是修好了 A 出现 B 修好了 B 出现 C, 修好了 C 出现 A. 是否要考虑一下系统的解决方法(但愿是我多虑了).

\item 【完成】选中一段公式代码, 按 【对齐】按钮, 公式中带紫色框的符号消失

\item 【完成】删除一个带紫色框的符号再撤销, 紫色框消失

\item 【完成】代码中空心句号被替换为实心句号后, 预览中仍然是空心句号(如 “.”)

\item 【完成】使用替换功能时, 若替换的内容中的命令被替换成符号, 替换后不会出现紫色框
\end{itemize}

\subsection{Editor BUG}
按优先级排序
\begin{itemize}
\item 到了自动备份的时间, 如果编辑器中有修改但没保存的内容, 则应该自动保存并自动备份, 并将 tab 左上角的小红点去掉.

\item 有时候打开词条列表中显示 “没有记录”, 需要刷新页面才可以. 估计是因为登录异常

\item 有时候笔记页面登录后会弹出百科的词条列表, 甚至可以打开词条, 这比较严重

\item 有时候笔记页面登录后编辑器窗口不显示任何内容

\item 引用其他词条内容后, 点击预览中的链接有误. 应该使用 scan 输出的链接.

\item 编辑器可以读取网站用户的编辑权限(等健叔更新 api)

\item 引用文献的按钮图标改成 \verb|[1]|

\item 如果文章带有脚注(例如 crypto.tex), 导出 md 文件到知乎并发布后, 脚注编号会错误, 且返回链接会直接以文本的形式显示. 注意为了鼓励用户查看原文, 脚注的链接仍然跳转到 wuli.wiki 的相关页面.

\item 【完成】实时预览模式下, 重命名后预览中标题不会更新.

\item 【完成】实时预览关闭时, 打开词条图片显示不出

\item 【完成】section label 的 autoref 显示不出来

\item 【完成】html 分段后公式序号显示错误

\item 【完成】隐藏 “保存后自动刷新预览” 开关

\item 【完成】如果当前词条有未保存的内容, 切换词条 tab 或者打开新词条的时候, 预览会跳回之前词条的预览. 这应该是因为离开一个 tab 的时候的自动保存引起的, 所以离开 tab 只保存就好不要刷新预览(还能节约流量提高速度).

\item 【完成】自动补全括号 \verb|()[]{}| 的算法需要和 vscode 的行为一样, 以 \verb|{| 为例: 1. 插入 \verb|{| 时, 只有右边是空格、换行符、end of file 或者 \verb|)]}| 之一时才补全右括号. 2.  在插入 \verb|}| 时, 如果右边是自动配对的 \verb|}|, 则不插入并将光标向右移动一个(或者理解为覆盖). 3. 选中几个字符并插入 \verb|{| 时, 自动在选中字符的左边插入 \verb|{|, 右边插入\verb|}|.

\item 【完成】\verb|$$| 的补全算法和 \verb|()[]{}| 不一样: 1. 只有插入 \verb|$| 时左边是空格、回车或 start of file, 且右边是空格、回车或者 end of file 时补全另一个 \verb|$|. 2. 例外: 当在被补全的 \verb|$| 左边插入 \verb|$|, 则覆盖. 3. 选中几个字符并插入 \verb|$| 时, 自动在选中字符的两边分别插入 \verb|$|.

\item 【完成】用论坛账号登录后应该自动处理 “登录异常,请重新登录” 而无需提示

\item 本文件中还是有一些 verb 和 lstinline 高亮错误

\item 【完成】同一个用户名无法同时登录百科和笔记(提示 “没有编辑权限”)

\item 【完成】用户笔记预览里面的 title.png 的 url 更新到用户文件夹, 例如 https://wuli.wiki/user/admin/title.png

\item 【完成】在一对大括号 \verb|{}| 中打左大括号, 不会自动补全右大括号

\item 【完成】切换或者重连网络后, 显示 “由于登录状态异常, 操作终止, 请重新登录后操作”. 能否直接提示重新输入账号密码登录?

\item 【完成】有时候网速慢时保存以后代码中的光标会卡住, 直到预览刷新了以后还要过几秒种才能恢复

\item 【完成】有时候切换 tab 时左边的预览不会切换, 需要 save 才会更新

\item 【完成】初始页面下, 帮助按钮按不动

\item 【完成】重新打开编辑器的时候, \verb|是否打开上次的 xxx.tex?| 即使选了 cancel 仍然会打开

\item 【完成】如果 \verb|main.tex| 中词条标题含有反斜杠, 则高亮不正常

\item 【完成】两个 “重新编译” 按钮, 如果程序出现 segmentation fault 等致命错误, 不会显示在输出窗口, 而且仍然显示 “编译成功”.

\item 【完成】“导出 md 文件” 按钮中, 要给公式编号的话, 直接在后面添加例如 \verb|\qquad (1)|  就可以了, 不要套一个 \verb|align| 环境

\item 【完成】“导出 md 文件” 按钮中, 每一个 \verb|<h2>| 和 \verb|<h3>| 上方加一个空行

\item 【完成】“导出 md 文件” 按钮中,支持对行间公式 \verb|$$...$$|、\verb|\[...\]| 和行内公式 \verb|\(...\)| 的导出

\item 【完成】\verb|verb| 命令和 \verb|lstinline| 命令中的内容不可以影响高亮. 例如 \verb|%| 后面的内容就都成了绿色. delimiter 可以是两个任意的相同符号, 例如 \verb*123* 或者 \verb+|||+ 也是合法的. 

\item 【完成】wuli.wiki/note 中创建新词条显示 \verb|读取词条文件 /user/xxx/changed/xxx.html 失败,服务器找不到该文件|.

\item 【完成】wuli.wiki/note 中保存或删除 test 用户的词条显示 \verb|j is not defined|.

\item 【完成】“重新编译并发布” 按钮没有成功调用 scan 程序, 也没有显示命令行的结果(应该指定 online 路径重新运行 \verb|./PhysWikiScan .|)

\item 【完成】上传图片等待的时候提示 “PhysWikiScan 没有响应, 请稍后再试”, 过一段时间后, 图片上传成功.

\item 【完成】在使用替换图片功能时, 输入的编号是图片的显示编号(也就是按照代码中 figure 环境出现的顺序编号)而不是图片文件名中的编号, 二者可能会不同.

\item 【完成】上传图片时如果进行其他操作会出问题(移动光标或者再次上传图片). 点击上传图片按钮后, 应该先在当前光标处插入代码然后再开始上传(现在是传完再插入). 上传图片时在左下角显示转圈或进度条, 可以点 X 终止. 上传过程中禁止再次上传图片, 如果终止或其他原因上传失败, 则提示 \verb|上传图片失败, 请手动删除图片环境|.

\item 【完成】普通用户(如 test5)使用 “下载变更文件” 按钮时出错: \verb|下载已更改文件失败:Error in api "/editor/api/download-files": findLines is not defined|

\item 【完成】预览窗口有时候会把词条显示两次, 关闭后再打开可解决

\item 【完成】自动备份对 \verb|FnalNt.tex| 停止了.

\item 【完成】在 “打开文件” 窗口输入 \lstinline|t.tex|, 不会出现在第一个, 是不是排序算法有问题? 其他词条也有类似的情况.

\item 【完成】自动引用一条没有 label 的公式时候只会添加 label 不会添加 \lstinline|\autoref| 命令(是否表格,图片等也有类似问题?)(在关闭符号替换模式发现的, 不确定是否和这个有关)

\item 【完成】打开词条有一定概率提示 “PhysWikiScan” 无响应

\item 【完成】已经注册的用户(如 admin)使用编辑器时不应该出现 “修改密码” 按钮

\item 【完成】自动补全设置里面的顺序不能决定候选的顺序

\item 【完成】符号替换的目标不能为 "*" (还有什么符号不能作为目标?)

\item 【完成】自动引用会弹出 “插入内部引用出错:xxxxxx” (不确定是谁的问题)

\item 【完成】自动引用(例如引用本文公式)时会出现 “PhysWikiScan 正在处理, 请稍后……” 会一直转圈.

\item 【完成】打第一个美元符号不会出现一对美元符号 $abc^2$

\item 【完成】手动打第二个美元符号时会出现两个 $1 + x (x \in X)$

\item 【完成】 “原子单位\upref{AU}” 的表格中和表格结束以后所有字体都变成了橙色
\end{itemize}

\subsection{Editor TODO}
按优先级排序
\begin{itemize}
\item 备份 tex 时, 若与上一个备份文件相同(无论是否同一作者), 则不备份(或者说把新文件删除).

\item 把 “修改词条标题” 按钮的 tip 改成 “重命名”; “插入词条引用” 改成 “引用其他词条”; “插入外部引用” 改成 “引用其他词条内容”; “插入内部引用” 改成 “引用本词条内容”;

\item 按钮放到 “引用其他词条” 后面依次应该是:“引用文献”、 “插入链接”

\item “插入脚注” 放到 “注释” 后面; “插入代码” 放到 “插入多个公式” 后面; “插入预备知识” 放到 “插入小标题” 后面; “插入推论” 后面加一条竖线.

\item 【完成】鼠标右键菜单需要添加 “粘贴”. 剪切复制粘贴需要放到右键菜单顶部

\item 右键菜单的复制粘贴后面需要提示快捷方式

\item 右键菜单中提示 “查找” 和 “替换” 的快捷键 (\verb|Ctrl+F|, \verb|Ctrl+H|).

\item 编辑器中的数字的高亮应该和字母一样, 可以参考 HeAnal.tex

\item 应该允许删除标题后面的换行符

\item 菜单栏增加一个红色的按钮,图标类似于 font-awesome 的 crosshairs, tip 是 “窗口定位”. 如果在代码窗口选中一段文字并按下该按钮, 那么在预览窗口中搜索并高亮该文字, 并滚动到窗口正中, 如果有多于一个 match, 那么再次按下按钮跳到下一个 match. 反之如果在预览窗口中选中一段文字并按下该按钮, 定位到代码窗口中. 如果按下时没有选中文字, 则提示 “请在代码或预览中选择一段文字用于定位(不能包括 LaTex 命令)”
\item 编辑百科按保存时, \verb|PhysWikiScan| 命令使用 \verb|PhysWikiScan --entry xxx --path 1|, 发布时使用 \verb|PhysWikiScan --entry xxx|

\item 中文双引号不需要自动补全, 但如果选中一段文字键入双引号中的一个, 则在文字两边插入中文双引号.

\item 用户笔记按保存时, \verb|PhysWikiScan| 命令中的 url 设为空, 发布时 url 不变

\item 在代码按钮旁边添加一个按钮, 插入 \verb|\verb{}| 命令

\item 删除一个文件时, 图片和源文件也要一并删除(包括 figures 中所有格式为 \verb|词条名_数字.任意拓展名| 的文件, online, changed 中对应的 svg 和 png 文件.

\item 删除词条时同时删除 \verb|changed/| 和 \verb|online/| 目录中对应的 html 和图片文件 (图片名格式为 \verb|词条名_编号.拓展名|)

\item 每次保存时确认 figures 文件夹中该词条相关的 pdf 和 png 图片都有被使用(图片名格式为 \verb|本词条标签_序号.拓展名|). 如果没有被使用则自动删除(先完成新要求文档中的图片备份功能再做这个).

\item 保存时, 检查所有 \verb|\label{词条名_类型序号}| (如 \verb|\label{Sample_eq1}|)是否正确, 如果不正确, 提示 \verb|检查到 xx 个 label 与词条名不匹配, 是否批量替换?:xxx, xxx, xxx| 如果选是, 则将 contents 目录中的所有词条中 \verb|\label{}| 和 \verb|\autoref{}| 中出现的该标签都替换成符合要求的. 注意每个 label 在词条中都必须是独一无二的.

\item 【先不处理】上一条中批量替换时会产生一个和空心引号按钮类似的问题, 就是如果有其他人正在编辑需要添加或修改 label 的词条怎么办? 是否考虑采用新提出的 “介入” 机制? 还是可以做到实时刷新这些人打开的词条?

\item 作为一个内容冲突的检查机制, 每次以任何形式保存词条后, 记录下后台词条文件的 hash, 下次保存前检查该文件的 hash. 如果和记录的不同, 那么提示 \verb|服务器上该词条文件发生了改变, 和这里的内容冲突, 这是一个 bug, 请马上停止编辑, 手动保存这里的内容到您的电脑, 并联系管理员|.

\item 上传图片的时候允许额外上传一个其他格式的文件(例如源码文件), 重命名为和图片文件相同. 允许 \verb|.m|, \verb|.py|, \verb|.agx|.

\item 每五分钟的备份也要包括图片和源文件(即 figures 文件夹中所有格式为 \verb|词条名_数字.任意拓展名| 的文件), 命名规则和 tex 文档一样. 只有图片文件内容被改变时才备份(可以通过 hash 判断或者逐字节对比).

\item 【完成】引用 subsection 按钮的 tip 改成 “子节”

\item 【完成】即使光标选中预览窗口, \verb|Ctrl+S| 也要和在代码窗口功能一样.

\item 【完成】如果某个 tab 中有未保存到服务器的内容, tab 的左上角加一个小红点.

\item 【完成】“改进意见” 按钮的 tip 改为 “标记词条: xxx”

\item 【完成】百分号图标有点太大了跟周围的图标不协调, 适当调小一点.

\item 【完成】红色和蓝色按钮的尺寸适当调小(保持里面的图标尺寸不变)

\item 【完成】“查看备份列表” 按钮改为 “查看历史版本”, 包括弹出窗口的标题.

\item 【完成】增加一个 “引用文献” 按钮, 图标为 \verb|[1]|, 点击以后弹出文献列表(从 \verb|data/bib_label.txt| 和 \verb|bib_detail.txt| 中每行分别读取 \verb|标签| 和 \verb|内容|), 每一行的格式为 \verb|[行号 标签] 内容| 支持搜索, 点击以后在光标处插入 \verb|\cite{标签}|

\item 【完成】设置每个词条第一行为只读(除了 \verb|main.tex| 以及 \verb|bibliography.tex|), 如果试图直接修改, 就提示 “请使用重命名按钮修改标题”.

\item 【先不做】当编辑器提示 PhysWikiScan 输出的 cerr 时, 文字需要可以复制, 如果 chrome 本身的弹窗不支持, 就用自己的弹窗.

\item “切换布局” 按钮增加一种模式, 即编辑窗口在左边预览在右边. 分隔栏的位置同样需要保留记忆.

\item 【完成】要求:如果在论坛上面已经登录了, 编辑器页面应该也可以自动登录(应该是用 cookie 实现的吧?)

\item 【完成】每个 \verb|\subsection| 需要可以折叠, 像环境一样
\end{itemize}

\subsubsection{性能优化}
\begin{itemize}
\item 【完成】当以上两点做完以后, 如果性能允许, 可以将手动保存改为自动保存(在设置里面放一个实验开关, 先默认关): 当编辑器有任何修改时触发自动保存. 如果连续快速打字, 那么每次保存完后应该马上开始下一次保存.

\item 【完成】 把 Scan 程序放到前端, 以减小网络需求, 速度更快. 当然后端也仍然需要 Scan, 例如 autoref 添加 label 的任务.

\item 【不做】 传输 tex 和 html 预览的时候, 只传上次传输后改变的部分.

\item 【不做】 每次刷新预览时 MathJax 只渲染 html 中改变的部分.
\end{itemize}

\subsubsection{历史版本}
\begin{itemize}
\item 做一个历史版本页面, 网址使用 \verb|wuli.wiki/history/|, 可以通过参数查看任意词条的任意历史(html 通过 scan 程序临时生成).

\item 在 “历史版本” 界面上方添加说明: \verb|请选中要恢复的历史版本, 或选中两个历史版本以进行对比|.

\item 把 “查看历史版本” 改为 “历史版本”

\item “历史版本” 界面上方加入一个选项(默认不选) “只显示每次编辑的最后一个备份”. 规则:任意一串作者相同且相邻间隔小于 30 分钟的备份, 只在列表中显示最后一个.

\item 在 “历史版本” 界面的每个版本右边显示 \verb|git diff| 返回的增减数, 如 \verb|+6 -7|. 如果选中上一条中的按钮, 就把隐藏的增减数合并到最后显示的那条中.

\item 在 “历史版本” 界面每个版本左边显示 radio button, 当选中一个时, 可以通过界面下方的 “恢复” 按钮恢复, 当选中两个时, 可以通过下方的 “版本对比” 跳转到对比页面, 使用 \verb|diff2html| 对比两个版本.

\item 关于对比页面: 网址使用 \verb|wuli.wiki/diff/|, 注意页面上需要可以选择 \verb|diff2html| 程序的 \verb|diffStyle| (char, word, line 等, 其中 char 非常重要, 否则连续的中文字符都会被看成一个 word).
\end{itemize}

\subsubsection{先不做}
\begin{itemize}
\item 做一个 Windows 的离线版的编辑器, 使用 localhost, 相当于把服务器端在用户电脑中运行.

\item 考虑用 Electron 做一个离线版的 wuli.wiki/note. 不需要网络连接(但 mathtype 应该需要网络才能用), 功能完全一样, 除了: 添加一个按钮可以设置用户名, 如果没有用户名, 则弹出提示框输入用户名 “请输入任意用户名, 以后可以在菜单栏修改”. 不需要任何密码. 另外增加一个按钮设置输入输出路径, 相当于 \verb|littleshi.cn/root/user/|. 默认路径为 \verb|Desktop/wuli.wiki-note/|
\end{itemize}

\subsubsection{已完成}
\begin{itemize}
\item 【完成】在粗体按钮右边增加一个斜体按钮, 图标是斜体的 \verb|I|, 插入命令 \verb|\textsl{}|, 功能和粗体按钮一模一样(包括选中文字以后再按下的效果)

\item 【完成】增加一个 “改进意见” 按钮, 图标用一个包含感叹号的三角形, 放到 “对齐” 图标的右边. 图标可以下拉选择 “草稿”(\verb|\issueDraft|), “缺引用”(\verb|\issueNeedCite|), “未完成”(\verb|\issueTODO|) , “缺说明”(\verb|\issueAbstract|), “缺预备知识”(\verb|\issueMissDepend|). 如果文章没有 \verb|\begin{issues}...\end{issues}| 环境, 那么点击任何一项时, 在文章开始(例子见 Sample)插入该环境, 以及在该环境中插入选中的命令. 如果已经存在 issues 环境, 那么点击某项时在该环境内最底部插入选中的命令.

\item 【完成】编辑器预览窗口中 MathJax 的渲染方式用 CHTML 而不是 SVG 会更漂亮 (既然大家都用 Chrome, 不存在兼容问题)

\item 【完成】弃用 \verb|\begin{...}| 内部自动补全功能, 全部使用普通的自动补全, 也就是把 \verb|config.json| 里面的 \verb|"LaTeX 环境"| 对应的功能弃用

\item 【完成】“切换布局” 按钮移动到 “重命名” 的左边, 上面可以有更多空间显示 tab

\item 【完成】增加一个注释按钮, 图标为 \verb|%|, 下拉菜单有两个按钮, 一个是 “注释选中内容”(默认), 另一个 “取消注释选中内容”. 分别相当于按下快捷键 \verb|ctrl + k, c| 和 \verb|ctrl + k, u|.

\item 【完成】“导出 md 文件” 按钮中, 不需要包括标题图和标题

\item 【完成】创建词条的窗口在文件名下方添加一个选填栏 “关键词”,在上方的说明改为 \lstinline+请输入标题(如“角动量守恒”),文件名(如"AngMom")和关键词.文件名不能超过 6 个字符,只能含有字母或数字,不能以数字开头; 关键词格式如 “关键词1|关键词2|关键词3”, 建议填写以方便检索.+

\item 【完成】创建词条后, 在文件第二行注释关键词, 如 \lstinline+% 关键词1|关键词2|关键词3+, 光标停留在第四行. 如果创建时没有填写关键词, 则第二行为空, 光标停在第三行

\item 【完成】菜单上的 “上传图片并创建新标签” 按钮应该始终保持为默认, 而不是保持上次的选择.

\item 【完成】在编辑百科时, 禁止关闭 “自动替换空心句号为实心” 功能(把开关锁定)

\item 【完成】在 “打开词条” 列表的第二行插入 \verb|bibliography.tex|, 编译该词条时, 使用 \verb|PhysWikiScan --bib| 而不是 \verb|PhysWikiScan --entry bibliography|, 后面的 \verb|--path-in-out-data-url| 选项不变

\item 【完成】插入美元符号时, 如果后面有字符(非空格,tab,或回车等), 那么只插入一个, 小中大括号同理.

\item 【完成】紫色符号如 \verb|{} ()|  等的颜色还要再浅一点以增加对比度

\item 【完成】保存按钮的 tip 改为 \verb|保存并编译(Ctrl+S)|

\item 【完成】定义,定理,itemize 等环境中(不包括 equation) 高亮效果应该和正文一样(现在全都是绿的).

\item 【完成】插入 \verb|\autoref{}| 的时候在后面多加一个空格(前面不需要).

\item 【完成】code 按钮支持新语言: python, pythonC(python 命令行), julia, markdown, bash, makefile, json, git, cmake, regex, latex

\item 【完成】符号替换设置打开时, json 编辑窗自动排版格式太慢, 是否考虑关闭时自动排版并保存排版后的文件?

\item 【完成】在编辑器加载页面(有转圈动画的那个)用小字显示 “建议使用 Chrome 或 Chromium 内核的浏览器”

\item 【完成】替换模式下上传图片后不需要再提示 “该图片已存在, 是否替换?”

\item 【完成】历史版本列表中应该显示所有用户的备份以及用户名, 目前好像仅显示当前用户的, 在 i 图标后面的文字改为 \verb|请选择要恢复的版本(使用北京时间)|, “search backup name”  改为 “搜索”

\item 【完成】预览窗口中的 css 使用 PhysWikiScan 输出文件夹中的 css (不同用户可能不一样)

\item 【完成】“重新编译所有内容及目录树” 按钮的输出框不能全选复制, 增加一个 “复制” 按钮.

\item 【完成】“重新编译所有内容及目录树” 按钮右边加一个类似公式的下拉选项, 可以选择 “编译并发布” 或者 “编译但不发布”. 前者输出到 online 文件夹, 后者输出到 changed 文件夹

\item 【完成】在管理员和用户笔记的界面中添加 “重新编译所有内容及目录树” 按钮, 按下后调用 \lstinline|PhysWikiScan . --path...|  命令.

\item 【完成】删除 \verb|main.tex| 时提示 “该文件不能被删除”

\item 【完成】禁止创建文件名为 \lstinline|index.tex| 的文件

\item 【完成】打开词条页面的 “i” 图标和后面的提示可以删掉, 然后把搜索框里面的英文改成 “搜索标题或文件名……”

\item 【完成】设置菜单中 “恢复默认设置” 时, 恢复到 note-template 中的设置

\item 【完成】增加一个脚注按钮, 插入 \lstinline|\footnote{脚注}|, 并自动选中 “脚注”

\item 【完成】增加一个代码按钮, 图标为 \lstinline|</>|, 按下以后弹出输入框 “请指定语言(不指定则没有高亮)” \verb|"\\begin{lstlisting}[language=输入的语言]\n${1}\n\\end{lstlisting}"|

\item 【完成】“插入词条引用按钮” 同时也插入词条的中文名, 且自动选中中文名以便修改. 例如 “词条示例\upref{Sample}”

\item 【完成】保存缓慢时区分是因为网络缓慢还是 PhysWikiScan 无响应. 如果是前者, 就一直尝试连接直到手动关闭提示框.

\item 【完成】上传的图片保存文件名的格式为 \verb|词条名_序号.后缀名|

\item 【完成】选中一段文字后点击链接按钮, 插入 \lstinline|\href{http://www.example.com}{被选中的文字}|, 自动选中网址

\item 【不要做】iOS 的 Safari 中拖动文字导致拖动整个屏幕(只有从论坛登录才会,不是编辑器本身的问题)

\item 【不要做】iOS 的 Safari 中选中文字光标位置错误

\item 【不要做】iOS 的 Safari 中键盘有时候无法弹出(即使外接键盘)
\end{itemize}

\subsubsection{公式编辑器相关}
\begin{itemize}
\item 已使用 \href{http://www.wiris.net/client/editor/resources/help.html?v=7.9.0.6564}{MathType Web} 作为公式编辑器插件(比上面那个 codecogs 颜值高多了).初次使用该功能时将动态加载插件,国内测试加载时间约为 10-12s,此后使用可直接打开无需再加载.
目前公式编辑器仅在 Chrome 上测试正常,其他平台未知.已知的 bug 如下:

\item 由于站点问题,编辑器加载有时会失败(提示 Connection Reset),需要支持自动重新加载

\item 因为是基于 CSS 排版且用的不是数学专用字体(Times New Roman),编辑中的公式显示会和 MathJax 显示的存在样式和位置上的出入(TODO:字体看看有没有办法引用 MathJax 的字体)

\item Edge 下点击 “确定” 按钮不能正常插入公式

\item 移动端下不能显示 “确定” 和 “取消” 按钮

\item TODO:对公式编辑器支持插入的符号进行完善,已经按照百科词条加了一些常用符号,看还有什么需求(是否需要一个复制 LaTeX 而不关闭编辑器的按钮?)

\item 【暂时不需要】TODO:因该插件不支持 physics 宏包,选中大多数词条中现有公式的 LaTeX 打开编辑器,是无法正确显示的(如 $I = \int\bvec j \vdot \dd{\bvec S}$ 选中这段打开编辑器),并且由编辑器生成的 LaTeX 也不符合百科的命令规范(但显示效果基本一致),需编写 processor 对命令进一步转换,如将$\frac{\partial^{n}{\cdot}}{\partial{\cdot}^{n}}$ 转换成 $\pdv[n]{\cdot}{\cdot}$,$\left\langle{\cdot}\vert{\cdot}\right\rangle$ 转换成 $\braket{\cdot}{\cdot}$ etc.
\end{itemize}

\subsection{PhysWikiScan BUG}

\subsection{changed}
\begin{itemize}
\item 【完成】changed.txt 中只有一个词条的时候, 发布词条会提示错误. 为空时也会错误?
\end{itemize}

\subsubsection{表格标签多次定义}
【完成】删除某个表格再重新插入一个具有同样标签的表格就会出现 “标签多次定义” 的错误.

\subsubsection{表格标题}
【完成】第二个表格会具有第一个表格的标题

\subsection{PhysWikiScan TODO}
\begin{itemize}
\item 深度兼容 \lstinline|\(\)|, 和 \lstinline|$$| 完全一样对待

\item 点击链接跳转到公式, 表格等 tag 以后, 短时间高亮目标(方法看聊天记录).

\item 中英文标点符号都不能出现在行首.

\item 支持在正文中改变颜色

\item 【完成】在每个词条底部加上目录树的链接
\end{itemize}

\subsubsection{改用 MathJax2}
【完成】MathJax3 在 iOS 的 Safari 上显示有时候公式上半部分消失. 注意当前的 MathJax3 文件夹复制一个备份

\subsubsection{不要使用 MathJax 的 newcommand}
\begin{itemize}
\item 【完成】而是直接进行命令替换, 从而增加公式代码在其他网站的兼容性
\end{itemize}

\subsubsection{脚注加上返回链接}
【完成】参考维基百科和知乎. 或者是否可以做成点了以后直接弹出一个子窗口而不是跳到底部? 如何实现?
