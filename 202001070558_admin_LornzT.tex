% 洛伦兹变换

\pentry{相对论简介, 伽利略变换} % 未完成

\subsection{相对论的基本假设}
\begin{enumerate}
\item 相对性原理——任何惯性系中,物理定律及相同实验的结果都相同
\item 光速不变原理——任何惯性系中,光速都不改变
\end{enumerate}

设 $t = 0$ 时空间直角坐标系 $S$ 与 $S'$ 系重合. $S'$ 系相对 $S$ 系沿 $x$ 轴方向以速度 $v$ 匀速运动.

洛伦兹变换的假设是,在 $S$ 系中的坐标 $(x, y, z, t)$ 一一对应到 $S'$ 系中的坐标 $(x', y', z', t')$. 变换一定是线性变换,因为时间和空间是均匀的, 这也叫物理定律的平移对称及时间对称.

先考虑 $x$ 轴的一维情况, 令线性变换为
\begin{equation}
\leftgroup{
x' &= ax + bt\\
t' &= mx + nt
}
\end{equation}
逆变换为
\begin{equation}
\leftgroup{
x = \frac{nx' - bt'}{an - bm}
t = \frac{mx' - at'}{bm - an}
}
\end{equation}
由相对性原理, 正变换和逆变换必须完全相同, 对比系数并化简得
\begin{equation}
\leftgroup{
&a = -n\\
&bm - an = 1
}
\end{equation}
由光速不变原理,(为方便起见,设长度单位为秒,1秒=299792458米,则光速为1,速度无单位)
