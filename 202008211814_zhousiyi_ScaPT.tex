% 标量扰动
把爱因斯坦方程进行扰动,我们可以得到
\begin{equation}\label{ScaPT_eq1}
\delta G^\mu_\nu = 8 \pi G\delta T^\mu_\nu ~.
\end{equation}
我们主要关心标量的扰动,可以得到
\begin{equation}
\begin{aligned}
\delta G^0_0 & = 2 a^{-2} [\nabla^2 \Phi- 3 \mathcal H(\Phi' - \mathcal H \Psi)] ~,  \\
\delta G^i_0 & = - 2 a^{-2} \partial^i(\Phi' - \mathcal H \Psi)~, \\
\delta G^i_j &= 2 a^{-2} \bigg[ (\mathcal H^2 + 2 \mathcal H')\Psi +\mathcal H \Psi' - \Phi'' - 2 \mathcal H \Phi' + \frac{1}{3} \nabla^2 (\Phi+\Psi)  \bigg] \delta^i_j \\
& - a^{-2} \bigg( \partial^i\partial_j - \frac{1}{3} \delta^i_j \nabla^2 \bigg) (\Phi+\Psi) ~.
\end{aligned}
\end{equation}
对能量动量张量同样进行扰动,我们有
\begin{equation}
\begin{aligned}
\delta T^0_0 & = - \delta \rho~, \\
\delta T^i_0 & = - (\bar \rho +\bar p) \partial^i v~, \\
\delta T^i_j & = \delta p \delta^i_j + \bigg( \partial^i\partial_j - \frac{1}{3} \nabla^2 \bigg) \sigma ~,
\end{aligned}
\end{equation}
使用上述方程,\autoref{ScaPT_eq1} 的00分量的方程可以写成
\begin{equation}
\nabla^2 \Phi - 3 \mathcal H (\Phi' - \mathcal H\Psi) = - 4 \pi G a^2 \delta \rho~. 
\end{equation}

