% 类氢原子的约化质量
% 约化质量|氢原子|类氢原子|原子核

\pentry{玻尔原子模型}%未完成
玻尔氢原子模型假设原子核的质量远大于电子质量. 得出的结论是
\begin{equation}
E_n = -\frac{mZ^2 e^4}{8\epsilon_0^2 h^2}\frac{1}{n^2} \approx  - 13.6\Si{eV} \frac{Z^2}{n^2}
\end{equation}
当考虑原子核运动的时候(但仍然忽略万有引力), 上式变为
\begin{equation}
E_n = -\frac{\mu Z^2 e^4}{8\epsilon_0^2 h^2} \frac{1}{n^2}
\end{equation}
其中 $\mu  = m_1 m_2/(m_1 + m_2)$ 叫做\textbf{约化质量}. $m_1$ 和 $m_2$ 分别是原子核质量和核外电子质量. 要注意这时的 $E_n$ 包括了原子核的动能.
\subsection{推导}
\begin{enumerate}
\item 经典力学的条件
\begin{equation}
m_1 \frac{v_1^2}{r_1} = m_2 \frac{v_2^2}{r_2} = \frac{1}{4\pi\epsilon_0} \frac{Z e^2}{(r_1 + r_2)^2}
\end{equation}
\item 角动量量子化条件(注意角动量是总角动量!)
\begin{equation}
m_1 v_1 r_1 + m_2 v_2 r_2 = n\hbar 
\end{equation}
\item 质心系条件
\begin{equation}
m_1 r_1 = m_2 r_2 \qquad m_1 v_1 = m_2 v_2
\end{equation}
要求的能量为
\begin{equation}
E_n = \frac12 m_1 v_1^2 + \frac12 m_2 v_2^2 - \frac{1}{4\pi\epsilon_0} \frac{Z e^2}{r_1 + r_2}
\end{equation}
\end{enumerate}
联立这几条式子, 就可以解出考虑原子核运动的能级公式. 推荐的步骤是, 把所有的 $r_2$ 和 $v_2$ 写成 $m_1 r_1/m_2$ 和 $m_1 v_1/m_2$ . 然后用和玻尔原子模型%链接未完成
中一样的方法解出来即可.
