% 拉普拉斯方程与调和函数
% 梯度|散度|拉普拉斯方程|偏微分方程|分离变量法

\begin{issues}
\issueDraft
\end{issues}

\pentry{梯度\upref{Grad}, 散度\upref{Divgnc}, 偏微分方程(未完成)}

若令 $\mathbb R^N$ 上某区域的实函数 $u(\bvec r)$ 的拉普拉斯等于零, 那么我们就得到了一个偏微分方程, 即\textbf{拉普拉斯方程(laplacian equation)}
\begin{equation}
\laplacian u = 0
\end{equation}
从物理上, 二元函数的拉普拉斯方程可以理解为一个静止的, 不受外力的薄膜\upref{Wv2D}所满足的方程. 要得到方程的解, 我们需要规定一些边界条件. 常见的条件是给定一个区域, 然后给出 $u(\bvec r)$ 在边界上的函数值.

\begin{theorem}{刘维尔定理}\label{LapEq_the1}
$\mathbb R^N$ 上的调和函数有界当且仅当它是常数.
\end{theorem}
\addTODO{证明}

\begin{theorem}{最大值定理}
$\mathbb R^N$ 上一个区域内的调和函数的最大值和最小值总出现在该区域的边界处.
\end{theorem}
\addTODO{证明}

\begin{theorem}{光滑}
调和函数是光滑的, 即在定义域处处无穷阶可导.
\end{theorem}
\addTODO{证明}
