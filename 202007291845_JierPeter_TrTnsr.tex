% 张量的坐标变换

\pentry{张量\upref{Tensor}, 过渡矩阵\upref{TransM}}

在\textbf{张量}\upref{Tensor}词条中我们看到,张量表示为矩阵依赖于相关的各线性空间中基的选择.本节将讨论基的变换是如何影响张量的矩阵表示的.

\subsection{一阶张量的坐标变换}
一阶张量是将一个向量映射为一个数,因此只涉及一个线性空间,最为简单.

给定$k$维线性空间$V$,及其上一个张量$f:V\rightarrow\mathbb{R}$.如果$V$的基是$\{\bvec{e}_1, \cdots\bvec{e}_k\}$,那么坐标为$\bvec{c}=(x_1\cdots x_k)\Tr$的向量$\bvec{v}$被映射为:
\begin{equation}
f(\bvec{v})=\sum\limits_{i=1}^k x_if(\bvec{e}_i)
\end{equation}

因此,$f$可以表示为$V$中的一个行向量$\bvec{F}$,坐标为$(f(\bvec{e}_1), \cdots, f(\bvec{e}_k))$.对于$V$中任何向量$\bvec{v}$,都有$f(\bvec{v})=\bvec{F}\bvec{c}$(按矩阵乘法).

若取另一个基$\{\bvec{e}_1', \cdots, \\bvec{e}_k'\}$,其中过渡矩阵为$\bvec{Q}$.如果在新的基下$\bvec{v}$的坐标变为$\bvec{c}'$,那么$\bvec{Q}\bvec{c}'=\bvec{c}$\footnote{见过渡矩阵\upref{TransM}.}.

设在新的基下,$f$表示为行向量$\bvec{F}'$,那么应有$f(\bvec{v})=\bvec{F}\bvec{c}=\bvec{F}'\bvec{c}'$.考虑到$\bvec{Q}\bvec{c}'=\bvec{c}$,我们可知对于任何坐标$\bvec{c}, \bvec{c}'$都有$\bvec{F}\bvec{Q}\bvec{c}'=\bvec{F}'\bvec{c}'$,因此

\begin{equation}
\bvec{F}\bvec{Q}=\bvec{F}'
\end{equation}

这就是一阶张量的坐标变换.
\subsection{二阶张量的坐标变换}

二阶张量涉及两个同构的线性空间$V$,而且是在物理学中最为常见的张量形式,因此我们重点讨论该情况.在这里,我们把二阶张量$f$理解为从$V_1$到$V_2$的一个线性映射,其中$V_1$和$V_2$同构.

如果给$V_1$指定一组基$\{\bvec{a}_1\}$








