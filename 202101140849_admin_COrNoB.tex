% 傅里叶变换与连续正交归一基底
% 正交归一|delta 函数|内积

\pentry{傅里叶变换(指数)\upref{FTExp}, 矢量空间\upref{LSpace}}

\subsection{离散的函数基底}
本文使用狄拉克符号\upref{braket}. 在 “傅里叶级数(三角)\upref{FSTri}” 中, 我们介绍了正交归一函数基底的概念, 即把满足一定条件的一元函数的集合看作一个矢量空间\upref{LSpace}, 两个函数(矢量)的内积\upref{InerPd}定义为
\begin{equation}
\braket{f}{g} = \int_{-\infty}^{+\infty} f(x)^* g(x) \dd{x}
\end{equation}
其中 $*$ 表示复共轭, 如果空间中的函数都是实函数则可忽略.

该空间中的一组正交归一基底用狄拉克符号\upref{braket}表示为 $\ket{x_i}$ ($i = 1, 2,\dots$), 基底的个数可以是有限个或无限个, 空间的维数就是基底的个数.

基底满足正交归一条件(\autoref{OrNrB_eq3}~\upref{OrNrB})
\begin{equation}\label{COrNoB_eq2}
\braket{x_i}{x_j} = \delta_{i,j}
\end{equation}
若这组正交归一基底是完备的, 那么任何函数都可以分解为它们的线性组合: 令
\begin{equation}\label{COrNoB_eq5}
c_i = \braket{x_i}{f}
\end{equation}

\begin{equation}\label{COrNoB_eq6}
\ket{f} = \sum_i c_i\ket{x_i} = \sum_i \ket{x_i}\braket{x_i}{f}
\end{equation}


\subsection{连续的函数基底}
我们接下来用类似的方法来理解傅里叶变换(\autoref{FTExp_eq6}~\upref{FTExp}).
\begin{align}\label{COrNoB_eq4}
g(k) &= \frac{1}{\sqrt{2\pi }} \int_{-\infty }^{+\infty } f(x)\E^{-\I kx} \dd{x} \\
\label{COrNoB_eq3}f(x) &= \frac{1}{\sqrt{2\pi }} \int_{-\infty }^{+\infty } g(k)\E^{\I kx} \dd{k}
\end{align}

我们令所有可以做傅里叶变换的函数构成的空间为 $X$, 从傅里叶变换的公式, 我们猜想该空间的正交归一 “基底” 为
\begin{equation}\label{COrNoB_eq1}
\ket{k} = \frac{1}{\sqrt{2\pi}} \E^{\I kx} \qquad (k \in \mathbb R)
\end{equation}
严格来说, $X$ 空间的函数必须要满足 $\braket{x}{x}$ 为有限值, 而\autoref{COrNoB_eq1} 中的函数显然不满足这点, 所以它们并不属于 $X$ 空间, 而是一个包含 $X$ 的更大的空间, 所以这个 “基底” 只是一个形象的说法, 需要加上引号.

显然, \autoref{COrNoB_eq1} 中的任意两个 “基底” 的内积都不收敛, 而且 $k$ 的取值是连续的, 所以我们不可能用\autoref{COrNoB_eq2} 表示它们的正交归一关系. 通过(\autoref{Delta_eq8}~\upref{Delta})
\begin{equation}
\int_{-\infty}^{+\infty} \E^{\I kx}\dd{x} = 2\pi \delta(k)
\end{equation}
可以得到一个和\autoref{COrNoB_eq1} 类似的关系
\begin{equation}
\braket{k'}{k} = \int_{-\infty}^{+\infty} \frac{\E^{-\I k'x}}{\sqrt{2\pi}} \frac{\E^{\I kx}}{\sqrt{2\pi}}\dd{x}
= \frac{1}{2\pi}\int_{-\infty}^{+\infty} \E^{\I (k'-k)x}\dd{x}
= \delta(k' - k)
\end{equation}
即
\begin{equation}
\braket{k'}{k} = \delta(k' - k)
\end{equation}
这可以看作是\textbf{连续基底的正交归一条件}.

现在, \autoref{COrNoB_eq4} 和\autoref{COrNoB_eq3} 可以分别对应到\autoref{COrNoB_eq5} 和\autoref{COrNoB_eq6}, 用狄拉克符号记为
\begin{align}
&g(k) = \braket{k}{f}\\
\label{COrNoB_eq7}&f(x) = \int_{-\infty}^{+\infty} g(k) \ket{k} \dd{k} = \int_{-\infty}^{+\infty} \ket{k}\braket{k}{f} \dd{k}
\end{align}
要证明\autoref{COrNoB_eq7}, 只需要使用狄拉克 $\delta$ 的性质(\autoref{Delta_eq7}~\upref{Delta}) $\int \delta(x - x_0) f(x) \dd{x} = f(x_0)$, 以及多重积分可以交换积分顺序:
\begin{equation}
\begin{aligned}
\int \ket{k}\braket{k}{f} \dd{k} &= \int \frac{\E^{\I kx}}{\sqrt{2\pi }} \qty(\int \frac{\E^{-\I k x'}}{\sqrt{2\pi }} f(x') \dd{x'}) \dd{k}\\
&= \int f(x') \qty(\int \frac{\E^{-\I x' k}}{\sqrt{2\pi }}\frac{\E^{\I x k}}{\sqrt{2\pi }}\dd{k}) \dd{x'}\\
&= \int f(x') \delta (x - x') \dd{x'}\\
&= f(x)
\end{aligned}
\end{equation}
这样就证明了傅里叶变换.
