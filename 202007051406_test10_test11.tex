% test11

0



1
$\begin{array}{l}AB=\begin{pmatrix}-3&2&5\\-2&0&2\\1&1&0\end{pmatrix},\left|AB\right|=\begin{vmatrix}-5&2&5\\-2&0&2\\0&1&0\end{vmatrix}=0\\\end{array}$


2
$\begin{array}{l}\mathrm{相似矩阵有相同的特征多项式},即\left|A-\lambda E\right|=\left|B-\lambda E\right|,\mathrm{而不是}A-\lambda E=B-\lambda E;\\\mathrm{由于}A与B\mathrm{相似},\mathrm{则存在可逆矩阵}P,\mathrm{使得}P^{-1}AP=B,则\\\left|B-tE\right|=\left|P^{-1}AP-P^{-1}tEP\right|=\left|P^{-1}(A-tE)P\right|=\left|P^{-1}\right|\left|A-tE\right|\left|P\right|=\left|A-tE\right|\\A与B\mathrm{不一定与对角矩阵相似},\mathrm{且相似矩阵有相同的特征值},\mathrm{但不一定有相同的特征向量}.\end{array}$


3
$\begin{array}{l}Ax=0\mathrm{有无穷多解}\Rightarrow\mathrm r\left({\mathrm A}_{\mathrm m\times\mathrm n}\right)<\;\mathrm n,\mathrm{若方程组}Ax=b\mathrm{有解},则r\left(A\right)=r\left(A\;B\right)<\;n,\mathrm{即有无穷多解};\\\mathrm{但直接由r}\left({\mathrm A}_{\mathrm m\times\mathrm n}\right)<\;\mathrm{n无法判断Ax}=\mathrm{b是否有解}.\end{array}$


4
$A_{34}=\left(-1\right)^{3+4}\begin{vmatrix}a_{11}&a_{12}&a_{13}&a_{15}\\a_{21}&a_{22}&a_{23}&a_{25}\\a_{41}&a_{42}&a_{43}&a_{45}\\a_{51}&a_{52}&a_{53}&a_{55}\end{vmatrix}.$


5
$\begin{array}{l}\begin{array}{l}\mathrm{方程组的系数行列式}\\\left|A\right|\;=\begin{vmatrix}\lambda&1&1&1\\1&\lambda&1&1\\1&1&\lambda&1\\1&1&1&\lambda\end{vmatrix}=\left(\lambda+3\right)\left(\lambda-1\right)^3.\;\;\\\mathrm{故当}\lambda\neq1且\lambda\neq-3时,\mathrm{有唯一解};\;\;\\当\lambda=1时,\\\overline A=\begin{vmatrix}1&1&1&1&1\\0&0&0&0&0\\0&0&0&0&0\\0&0&0&0&0\end{vmatrix},\;\;\mathrm{有无穷多解},\mathrm{易求得解为}\;\;x=k_1\begin{pmatrix}-1\\1\\0\\0\end{pmatrix}+k_2\begin{pmatrix}-1\\0\\1\\0\end{pmatrix}+k_3\begin{pmatrix}-1\\0\\0\\1\end{pmatrix}+\begin{pmatrix}1\\0\\0\\0\end{pmatrix},\mathrm{其中}k_1,k_2,k_3\in R\end{array}\\\end{array}$


6
$a_{14}a_{23}a_{32}a_{41}\mathrm{的列标的逆序数为}1+2+3=6,\mathrm{故此项前边带有正号}.$


7
$\begin{array}{l}\mathrm{由逆矩阵的运算性质可知}:\\(A^2)^{-1}=(AA)^{-1}=A^{-1}A^{-1}=(A^{-1})^2;\;\;(kA)^{-1}=\frac1kA^{-1}((k\neq0);\\\mathrm{由于}(A+B)(A^{-1}+B^{-1})=2E+BA^{-1}+AB^{-1}\neq E,\mathrm{所以}(A+B)^{-1}\neq A^{-1}+B^{-1};\\而(A+B)(A-B)=A^2-AB+BA-B^2,若A,B\mathrm{不可交换},则AB\neq BA\end{array}$


8
$\begin{array}{l}\mathrm{二次型矩阵为}A=\begin{pmatrix}5&-1&3\\-1&5&-3\\3&-3&c\end{pmatrix},由r(A)=2\mathrm{可知}\left|A\right|=0\Rightarrow c=3;由\left|A-\lambda E\right|=0\mathrm{可求得矩阵}A\mathrm{的特征值为}\lambda_1=0,\lambda_2=4,\lambda_3=9\\故f\mathrm{可通过正交变换}X=PY\mathrm{化为}f=4y_2^2+9y_3^2,\mathrm{即方程}4y_2^2+9y_3^2=1,\mathrm{故方程表示的图形为椭圆柱面}.\end{array}$


9
$\begin{array}{l}\mathrm{矩阵}\begin{pmatrix}1&2&3\\2&5&5\end{pmatrix}\mathrm{是由}\begin{pmatrix}1&2&3\\0&1&-1\end{pmatrix}\mathrm{的第一行乘以}2\mathrm{加到第二行得到的},\mathrm{由初等变换和初等矩阵的知识},\mathrm{可知}\\A=\begin{pmatrix}1&0\\2&1\end{pmatrix}.\end{array}$


10
$\text{化为}-\frac{x^2}8-\frac{y^2}{16}+\frac{z^2}{64}=1\text{,是双叶双曲面}$


11
$\left|A^\ast\right|=\left|A\right|^{n-1}=\left|A\right|^4$


12
$\begin{array}{l}\mathrm{向量组中有零向量},\mathrm{则向量组线性相关},\mathrm{但没有零向量也不一定线性无关},\\如:\alpha_1=(1,0,0),\alpha_2=(0.1,0),\alpha_3=(1,1,0),\mathrm{因此向量组不含零向量与向量组线性无关不等价};\\\mathrm{由向量组线性无关的定义可知其余选项都与向量组线性无关等价}.\end{array}$


13
$\begin{array}{l}\;\;\mathrm{齐次线性方程组只有零解},即r(A)=\;3\Leftrightarrow\left|A\right|\neq0,则\\\;\;\;\;\;\;\;\;\;\;\;\;\;\;\;\;\;\;\;\;\;\;\begin{vmatrix}2&1&1\\k&1&1\\1&-1&1\end{vmatrix}=2(2-k)\Rightarrow k\neq2.\end{array}$


14
$n\mathrm{阶矩阵}A\mathrm{与对角矩阵相似的充分必要条件是}A有n\mathrm{个线性无关的特征向量}$


15
$\left|A-\lambda E\right|=\begin{vmatrix}4-\lambda&-1&0\\6&-\lambda-1&-2\\0&0&3-\lambda\end{vmatrix}=-(\lambda-1)(\lambda-2)(\lambda-3)=0,故\lambda_1=1,\lambda_2=2,\lambda_3=3$


16
$\mathrm{原式}=1\times2\times3....\times n=n!.$


17
$\mathrm{相似矩阵具有相同的迹},\mathrm{所以}1+4=3+x,\mathrm{解之得}x=2$


18
$\begin{array}{l}\mathrm{由距离公式},得\\d=\frac{\left|1\times1+2\times1+1\times1+2\right|}{\sqrt{1^2+1^2+1^2}}=2\sqrt3.\end{array}$


19
$\begin{array}{l}\begin{array}{l}\mathrm{原方程组即为}x_n=-nx_1-(n-1)x_2-\cdots-2x_{n-1},\\取x_1=1,x_2=x_3=\cdots=x_{n-1}=0,得x_n=-n\\取x_2=1,x_1=x_3=x_4=\cdots=x_{n-1}=0,得x_n=-(n-1)=-n+1;\cdots\;\cdots;\end{array}\\取x_{n-1}=1,x_1=x_2=\cdots=x_{n-2}=0,得x_n=-2\end{array}$


20
$\begin{array}{l}由AX+E=A^2+X,\left(A-E\right)X=A^2-E,而A-E=\begin{pmatrix}1&0&0\\0&1&0\\1&6&1\end{pmatrix}\mathrm{可逆},\\故X=A+E=\begin{pmatrix}3&0&0\\0&3&0\\1&6&3\end{pmatrix}\end{array}$


21
$\begin{array}{l}由\;\mathrm{题意}(\beta_1,\beta_2,\beta_3)=(\alpha_1,\alpha_2,\alpha_3)\begin{pmatrix}1&0&0\\1&1&0\\1&1&1\end{pmatrix},\\记B=AK,因\vert K\vert=1\neq0,由\alpha_1,\alpha_2,\alpha_3\mathrm{线性无关}\;是\beta_1,\beta_2,\beta_3\mathrm{线性无关的充要条件}.\end{array}$


22
$\begin{array}{l}\mathrm{根据矩阵秩的定义可知},\mathrm{原矩阵的所有}3\mathrm{阶子式全为零},\mathrm{且存在一个}2\mathrm{阶子式不为零};\mathrm{且矩阵的行}、\mathrm{列向量}\\\mathrm{组的秩都为}2,\mathrm{因此都线性相关}.\end{array}$


23
$\begin{array}{l}设A\mathrm{的某个}r_1\mathrm{阶子式}D_r\neq0.\;\mathrm{矩阵}B\mathrm{是由矩阵}A\mathrm{增加一列得到的},\mathrm{所以在}B\mathrm{中能找到与}D_r\mathrm{相同的}r_1\mathrm{阶子式}\;\overline{D_r},\\\mathrm{由于}\\\;\;\;\;\;\;\;\;\;\;\;\;\;\;\;\;\;\;\;\;\;\;\;\;\;\;\;\;\;\;\;\;\;\;\;\;\;\;\;\;\;\;\;\;\;\;\;\;\;\;\;\;\;\;\;\;\;\;\;\;\;\;\;\;\;\;\;\;\;\;\;\;\overline{D_r}=D_r\neq0,\\\mathrm{故而}\;R(B)\;\geq R(A),即r_1\leq r_2.\\\end{array}$


24
$\begin{array}{l}\begin{array}{l}设x_2为A\mathrm{对应于}\lambda_2=\lambda_3=3\mathrm{的特征向量},\mathrm{实对称矩阵不同特征值对应的特征向量正交},则x_2x_1^T=0,\mathrm{可建立方程组直接求解};\;\\则x_2={(-1,0,1)}^T,x_3={(-1,1,0)}^T.将x_1\mathrm{单位化},x_2,x_3\mathrm{正交单位化后组成矩阵为}\end{array}\\P=\begin{pmatrix}\frac1{\sqrt3}&-\frac1{\sqrt2}&-\frac1{\sqrt6}\\\frac1{\sqrt3}&0&\frac2{\sqrt6}\\\frac1{\sqrt3}&\frac1{\sqrt2}&-\frac1{\sqrt6}\end{pmatrix}\\由P^{-1}AP=\begin{pmatrix}6&0&0\\0&3&0\\0&0&3\end{pmatrix},且P^{-1}=P^T\mathrm{则有}\\A=P\begin{pmatrix}6&0&0\\0&3&0\\0&0&3\end{pmatrix}P^T=\begin{pmatrix}4&1&1\\1&4&1\\1&1&4\end{pmatrix}\\\end{array}$


25
$\mathrm{向量组与部分组的关系为}:\mathrm{若向量组线性无关},\mathrm{则部分组扔线性无关};\mathrm{若部分组线性相关},\mathrm{则向量组也线性相关}.$


26
$A=\begin{bmatrix}2&3\\4&5\end{bmatrix},则A_{11}=5,\;\;A_{12}=-4,\;\;A_{21}=-3,\;\;A_{22}=2,A^\ast=\begin{bmatrix}5&-3\\-4&2\end{bmatrix}$


27
$\begin{array}{l}\begin{array}{l}由\;\mathrm{题意}(\beta_1,\beta_2,\beta_3,\beta_4)=(\alpha_1,\alpha_2,\alpha_3,\alpha_4)\begin{pmatrix}-1&1&0&0\\0&-1&1&0\\0&0&-1&1\\0&0&0&-1\end{pmatrix},\\记B=AK,因\vert K\vert=1\neq0,由\alpha_1,\alpha_2,\alpha_3,\alpha_4\mathrm{线性无关知}\beta_1,\beta_2,\beta_3,\beta_4\mathrm{线性无关}\end{array}\end{array}$


28
$\begin{array}{l}令\;A=\begin{pmatrix}\frac12&0&0\\0&1&5\\0&1&6\end{pmatrix},B=\begin{pmatrix}1&1&2\\0&0&-6\end{pmatrix},\mathrm{因为}\left|A\right|=\frac12\neq0,故X=BA^{-1},\mathrm{由初等变换得}\\A^{-1}=\begin{pmatrix}2&0&0\\0&6&-5\\0&-1&1\end{pmatrix},X=BA^{-1}=\begin{pmatrix}2&4&-3\\0&6&-6\end{pmatrix}\end{array}$


29
$\begin{vmatrix}5a_{11}&4a_{11}-a_{12}&a_{13}\\5a_{21}&4a_{21}-a_{22}&a_{23}\\5a_{31}&4a_{31}-a_{32}&a_{33}\end{vmatrix}=\begin{vmatrix}5a_{11}&4a_{11}&a_{13}\\5a_{21}&4a_{21}&a_{23}\\5a_{31}&4a_{31}&a_{33}\end{vmatrix}-\begin{vmatrix}5a_{11}&a_{12}&a_{13}\\5a_{21}&a_{22}&a_{23}\\5a_{31}&a_{32}&a_{33}\end{vmatrix}=0-5=-5.$


30
$\mathrm{由于二次型矩阵是对称矩阵},\mathrm{所以}a_{ij}=a_{ji},\mathrm{即二次型可表示成矩阵形式如下}:(x_1,x_2)\begin{pmatrix}1&3\\3&3\end{pmatrix}\begin{pmatrix}x_1\\x_2\end{pmatrix}$


31
$\begin{array}{l}\beta_1,\;\beta_2\mathrm{是非齐次线性方程组的}Ax=b\mathrm{两个不同的解},\\则\frac12\left(\beta_1+\beta_2\right)为Ax=b\mathrm{的一个解},且\frac12\left(\beta_1-\beta_2\right)为Ax=0\mathrm{的解};\;\;\\a_1,\;a_2是Ax=0\mathrm{的基础解系},则a_1,\;a_2\mathrm{线性无关},又a_1,\;a_1-\;a_2\mathrm{也线性无关},\\\mathrm{因此也构成}Ax=0\mathrm{的基础解系},故Ax=b\mathrm{的通解可表示为}\\k_1a_1+k_2\left(a_1-a_2\right)+\frac12\left(\beta_1+\beta_2\right)\;.\;\;\\附:a_1,\;\beta_1-\beta_2\mathrm{由于不一定线性无关},\mathrm{因此不能构成基础解系}.\end{array}$


32
$\mathrm{二次型矩阵}A=\begin{pmatrix}a&4&-2\\4&a&2\\-2&2&6\end{pmatrix},A\mathrm{的特征值为}7,7,-2.由\left|A\right|=7\cdot7\cdot(-2)或A\mathrm{的迹即可得}a=3$


33
$\begin{array}{l}A^2+A-2E+3E=O\Rightarrow\left(A-E\right)\left(A+2E\right)=-3E,即\\\;\;\;\;\;\;\;\;\;\;\;\;\;\;\;\;\;\;\;\;\;\;\;\;\;\;\;\;\;\left(A-E\right)\left[-\frac13\left(A+2E\right)\right]=E,故A^{-1}=-\frac{A+2E}3\end{array}$


34
$\mathrm{该直线的方向向量为}\{0,1,2\},x\mathrm{轴的方向向量为}\{1,0,0\},\mathrm{向量的数量积为零},\mathrm{所以该直线于}x\mathrm{轴垂直}$


35
$\mathrm{由初等矩阵的性质可知初等矩阵的行列式等于}\pm1\mathrm{或非零常数}k,\mathrm{因此两个初等矩阵乘积的行列式也非零},\mathrm{故可逆}.\mathrm{其它选项不能确定}.$


36
$\begin{vmatrix}-1&2&1&1\\0&2&1&11\\0&3&6&6\\-1&0&0&0\end{vmatrix}=\begin{vmatrix}2&1&1\\2&1&11\\3&6&6\end{vmatrix}=\begin{vmatrix}2&1&1\\0&0&10\\3&6&6\end{vmatrix}=-10\begin{vmatrix}2&1\\3&6\end{vmatrix}=-90.$


37
$\mathrm{矩阵}a_{ij}=a_{ji},且f=x^TAx=a_{11}x_1^2+a_{22}x_2^2+a_{33}x_3^2+2a_{12}x_1x_2+2a_{13}x_1x_3+2a_{23}x_2x_3=\sqrt2x_1x_2+3x_2^2+2\;x_1x_3-3x_2x_3$


38
$\begin{array}{l}(\alpha+\beta,\beta+\gamma,\gamma+\alpha)=(\alpha,\beta,\gamma)\begin{pmatrix}1&0&1\\1&1&0\\0&1&1\end{pmatrix},记B=AK,因\begin{vmatrix}K\end{vmatrix}=2\neq0,知K\mathrm{可逆},\mathrm{由矩阵的秩的性质知}\\r(B)=r(A)=3.\\故\alpha+\beta,\beta+\gamma,\gamma+\alpha\mathrm{也线性无关}\end{array}$


39
$\begin{array}{l}记\left(A\;b\right)=B,则\;r\left(A\right)=r\left(B\right)=n\Rightarrow Ax=b\mathrm{有唯一解};\\\;r\left(A\right)=r\left(B\right)<\;n\Rightarrow Ax=b\mathrm{有无穷多解};\;\;\\r\left(A\right)<\;n\Rightarrow Ax=0\mathrm{有非零解};\\r\left(A\right)=\;n\Rightarrow Ax=0\mathrm{只有零解}.\;\;\\若A有n\mathrm{阶子式不为零},又r\left(A\right)=min\left\{m,n\right\},故r\left(A\right)=\;n,则Ax=O\mathrm{仅有零解}.\end{array}$


40
$\begin{array}{l}\mathrm{因为}A\mathrm{能对角化},A\mathrm{必有三个线性无关的特征向量},\mathrm{由于}\left|A-\lambda E\right|=\begin{vmatrix}-\lambda-1&-2&0\\1&2-\lambda&0\\x&-2&1-\lambda\end{vmatrix}=-(\lambda-1)(\lambda^2-\lambda)\\\lambda=1\mathrm{是二重特征值},\mathrm{必有两个线性无关的特征向量},\mathrm{因此}r(A-E)=1,得x=-2.\end{array}$


41
$xOz\mathrm{平面为}y=0,\mathrm{代入曲面方程得}x^2=z,\mathrm{所以题目所求方程为}\left\{\begin{array}{l}x^2=z\\y=0\end{array}\right..$


42
$A_{14}=\left(-1\right)^{1+4}\begin{vmatrix}0&-1&2\\4&7&2\\0&2&1\end{vmatrix}=-20.$


43
$\begin{array}{l}AX=3C-B=\begin{pmatrix}2&-1\\2&3\end{pmatrix},\\X=A^{-1}\begin{pmatrix}2&-1\\2&3\end{pmatrix}=\begin{pmatrix}1&-2\\0&1\end{pmatrix}\begin{pmatrix}2&-1\\2&3\end{pmatrix}=\begin{pmatrix}-2&-7\\2&3\end{pmatrix}\end{array}$


44
$将y=4\mathrm{代入到双曲面方程得}x^2-z^2/9=5,\mathrm{所以相交的曲线为双曲线}.$


