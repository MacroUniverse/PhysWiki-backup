% 机器学习数据类型
\subsection{基本数据类型}
按照采用的计量尺度不同,可以将统计数据分为分类数据、顺序数据和数值型数据.
\subsubsection{分类数据}
分类数据是只能归于某一类别的非数字型数据,他是对事物进行分类的结果,数据表现为类别,是用文字来表述的.
\subsubsection{顺序数据}
顺序数据是只能归于某一有序类别的非数字型数据.
\subsubsection{数值型数据}
数值型数据是按照数字尺度测量的观测值,其结果表现为具体的数值.
分类数据和顺序数据说明的是事物的品质特征,通常是文字来表述,其结果表现为类别,因而可以统称为定型数据;数值型数据说明的是现象的数量特征,通常是数值来表现的,因此也成为定量数据.
\subsection{变量}
变量是说明现象某种特征的概念,其特点是从一次观察到下一次观察结果会呈现差别或者变化.变量可以分为以下几种类型:
\subsubsection{定类变量(nominal)}
变量的不同取值仅仅代表了不同类的事物,这样的变量叫定类变量.问卷的人口特征中最常使用的问题,而调查被访对象的“性别”,就是 定类变量.
\subsubsection{定序变量(ordinal)}
变量的值不仅能够代表事物的分类,还能代表事物按某种特性的排序,这样的变量叫定序变量.问卷的人口特征中最常使用的问题“教育程度“,以及态度量表题目等都是定序变量,定序变量的值之间可以比较大小,或者有强弱顺序.
\subsubsection{定距变量(interval)}
变量的值之间可以比较大小,两个值的差有实际意义,这样的变量叫定距变量.
\subsubsection{定比变量(ratio variable)}:有绝对0点,如质量,高度.定比变量与定距变量在市场调查中一般不加以区分,它们的差别在于,定距变量取值为“0”时,不表示“没有”,仅仅是取值为0.
\subsection{四种变量的比较}
\begin{table}[ht]
\centering
\caption{四种变量的比较}\label{DatTyp_tab1}
\begin{tabular}{|c|c|c|c|c|}
\hline
支持计算 & 定类变量 & 定序变量 & 定距变量 & 定比变量 \\
\hline
计数、分布 & 是 & 是 & 是 & 是 \\
\hline
最大、最小 &   & 是 & 是 & 是 \\
\hline
范围 &   & 是 & 是 & 是 \\
\hline
百分比 &   & 是 & 是 & 是 \\
\hline
方差、标准差 &   &   & 是 & 是 \\
\hline
众数 & 是 & 是 & 是 & 是 \\
\hline
中位数 &   & 是 & 是 & 是 \\
\hline
平均数 &   &   & 是 & 是 \\
\hline
可计数 & 是 & 是 & 是 & 是 \\
\hline
可定义顺序 &   & 是 & 是 & 是 \\
\hline
可定义差异(加减计算) &   &   & 是 & 是 \\
\hline
可定义零(乘除计算) &   &   &   & 是 \\
\hline
\end{tabular}
\end{table}
