% 力的分解合成 合力

\pentry{几何矢量\upref{GVec}}

在经典力学中, 力的分解与合成可以看作一个基本假设. 这个假设是牛顿运动定律\upref{New3}的基础.

\begin{theorem}{力的合成}
当若干个力 $\bvec F_i$ ($i = 1, 2, \dots, N$)作用在同一个质点上时, 等效于一个力
\begin{equation}\label{Fdecom_eq1}
\bvec F = \sum_i \bvec F_i
\end{equation}
作用在同一个质点上.
\end{theorem}
我们把 $\bvec F$ 叫做 $N$ 个 $\bvec F_i$ 的\textbf{合力}, 这个过程叫做\textbf{力的合成}. 如果把 $\bvec F$ 写成任意\autoref{Fdecom_eq1} 的形式, 就叫做\textbf{力的分解}, 每个 $\bvec F_i$ 就是一个\textbf{分力}.

这里所说的 “效果” 可以指这个质点受力后的运动情况, 或者例如它固定在弹簧上, 弹簧的形变. 所以一般而言, 如果我们能找到任何两组力, 使得
\begin{equation}
\sum_{i=1}^N \bvec F_i = \sum_{i=1}^{N'} \bvec F'_i
\end{equation}
那么所有的 $\bvec F_i$ 同时作用在质点上的效果和所有的 $\bvec F'_i$ 同时作用在该质点是等效的.

注意在