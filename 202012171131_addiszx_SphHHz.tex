% 球坐标系中的亥姆霍兹方程
% 球坐标系|亥姆霍兹方程|球贝塞尔方程|球贝塞尔函数

\pentry{球坐标系中的拉普拉斯方程\upref{SphLap}}

亥姆霍兹方程为\footnote{一般地, $k$ 可以是复数, 所以 $k^2$ 也可以是负实数.}
\begin{equation}
\laplacian f + k^2 f = 0
\end{equation}
球坐标系中, 可以将拉普拉斯算子(\autoref{SphNab_eq4}~\upref{SphNab})分解为径向和角向两部分
\begin{equation}
\laplacian = \laplacian_r + \frac{1}{r^2}\laplacian_{\Omega}
\end{equation}
其中
\begin{equation}
\laplacian_r = \frac{1}{r^2} \pdv{r} \qty(r^2 \pdv{u}{r})
\end{equation}
\begin{equation}
\laplacian_\Omega = \frac{1}{\sin\theta}\pdv{\theta} \qty(\sin \theta \pdv{u}{\theta}) + \frac{1}{\sin^2 \theta} \pdv[2]{u}{\phi}
\end{equation}


与拉普拉斯方程\upref{SphLap} 中的过程类似, 用分离变量法, 令 $f(\bvec r) = R(r) Y(\uvec r)$, 则分离后的径向方程和角向方程分别
\begin{equation}
r^2\laplacian_r R(r) + \qty[k^2 r^2 - l(l+1)] R(r) = 0
\end{equation}
\begin{equation}
\laplacian_{\Omega} Y(\uvec r) = -l(l+1) Y(\uvec r)
\end{equation}
我们已知角向的解为球谐函数 $Y_{l,m}(\uvec r)$. 而径向方程比普拉斯方程中的多出了含 $k$ 的项, 使用变量代换 $\rho = kr$ 得
\begin{equation}
\rho^2\dv[2]{R}{\rho} + [\rho^2 - l(l+1)] R = 0
\end{equation}
该方程被称为\textbf{球贝塞尔方程(spherical Bessel equation)}, 两个线性无关解分别是第一和第二类\textbf{球贝塞尔函数(spherical Bessel function)}\upref{SphBsl} $j_l(\rho)$ 和 $y_l(\rho)$.

综上, 方程的通解为
\begin{equation}
f(\bvec r) = \sum_{l,m} [A_l j_l(kr) + B_l y_l(kr)] Y_{l,m}(\uvec r)
\end{equation}
