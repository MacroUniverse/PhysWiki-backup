% 9

\subsection{股票基本指标}
我们以从CSMAR数据库上下载的各支股票的周回报率作为股票周收益率.

\subsubsection{1、期望收益率}
运用Excel计算十支股票的期望收益率,计算结果如下表所示.

\begin{table}[ht]
\centering
\caption{十支股票期望收益率}\label{inv9_tab1}
\begin{tabular}{|c|c|c|c|c|c|}
\hline
股票代码 &600016 & 600028 & 600050 & 600104 &	600588 \\
\hline
期望收益率 &-0.000802&0.001282&0.001586	&0.002030&0.008136\\
\hline
股票代码 &601138&601166&601288&601319&601668\\
\hline
期望收益率&-0.0018197&0.002843&0.001452&0.004942	&0.002205\\
\hline
\end{tabular}
\end{table}

\subsubsection{2、标准差}
运用Excel计算十支股票的标准差,计算结果如下表所示.

\begin{table}[ht]
\centering
\caption{十支股票标准差}\label{inv9_tab2}
\begin{tabular}{|c|c|c|c|c|c|}
\hline
股票代码 &600016 & 600028 & 600050 & 600104 &	600588 \\
\hline
标准差 &0.021491&0.029004&0.045676&0.040468&0.065377\\
\hline
股票代码 &601138&601166&601288&601319&601668\\
\hline
标准差&0.053231&0.033032&0.021184&0.082794&0.036732\\
\hline
\end{tabular}
\end{table}
\subsubsection{3、协方差矩阵}
运用MATLAB计算十支股票的协方差矩阵,代码片段如下所示.

图1:计算协方差矩阵代码

计算结果如下图所示.

图2:十支股票协方差矩阵

\subsubsection{4、相关系数矩阵}
运用MATLAB计算十支股票的相关系数矩阵,代码片段如下所示.

图3:计算相关系数矩阵代码

计算结果如下图所示.

图4:十支股票相关系数矩阵
\subsubsection{5、股票相关性分析}
以上相关系数矩阵表中:绿色部分相关系数小于0.3,表示股票之间不存在相关关系;黄色部分相关系数在0.3~0.5,表示股票之间存在低度相关关系;橙色部分相关系数在0.5~0.7,表示股票之间存在显著相关关系.

对十支股票的相关系数矩阵进行分析.从总体上看,所选的十支股票中大部分股票之间不存在相关关系或只存在低度相关关系,只有极少数几支股票之间存在较为显著的相关关系.基于以上分析,我们可以得出结论:我们从上证50中选出的十支股票分散化程度较高,组合的方差较低,投资风险也较低.

\subsection{单支股票的单因素模型}
我们选取上证综指的回报率作为市场指数回报率进行计算,得出十支股票的单指数模型.
\subsubsection{1、单指数模型的回归方程}

回归方程是:
$R_{i}(t)=\alpha_{i}+\beta_{i}R_{M}(t)+e_{i}(t)$.
其中,$\alpha$是当上证综指的超额收益率为0时该股票的期望超额收益率,斜率$\beta_{i}$是股票对指数的敏感性,$e_{i}$的均值为0.

运用Excel对数据进行回归处理,回归结果如下所示,十支股票的单指数模型回归方程分别为:

股票代码600016:$R_{1}=-0.001159 + 0.995744R_{M}$

股票代码600028:$R_{2}=0.000920 + 1.066996R_{M}$

股票代码600050:$R_{3}=0.001007 + 1.404773R_{M}$

股票代码600104:$R_{4}=0.001679 + 0.913641R_{M}$

股票代码600588:$R_{5}=0.007769 + 1.144919R_{M}$

股票代码601138:$R_{6}=-0.002407 + 1.144003R_{M}$

股票代码601166:$R_{7}=0.002491 + 0.93768R_{M}$

股票代码601288:$R_{8}=0.001094 + 1.018268R_{M}$

股票代码601319:$R_{9}=0.004582+0.638417R_{M}$

股票代码601668:$R_{10}=0.001851+0.957975R_{M}$


\subsubsection{2、单指数模型回归结果检验与分析}
以农业银行(股票代码:601288)为例,对单指数模型的回归结果进行检验与分析,检验分析步骤如下:

(1)回归统计

回归统计部分体现了模型的拟合性.表中的R为相关系数,$R^{2}$为可决系数,Adjusted$R^{2}$为校正后的可决系数.这三个指标统计意义相似,我们以$R^{2}$作为检验对象.农业银行的$R^{2}$为0.5064,表示上证综指超额收益率可以解释农业银行超额收益率50.64的变化,拟合效果较好.

(2)方差分析

方差分析部分主要对F值和p值进行检验.

对总回归方程进行F检验.给出零假设$H_{0}: \beta=0$,备择假设$H_{1}\ne0$.
根据上表,该模型的p值为8.00849E-41 < 0.05,假设检验有效.在检验水平为0.05的情况下,利用excel表进行查询,输入公式=FINV(0.05,1,254),得到$F_{\alpha}(1,254)=3.878329734$.该模型的F值为260.5895>3.878329734,因此拒绝零假设$H_{0}$,接受备择假设$H_{1}$,说明该模型的拟合具有显著性.

(3)参数检验

对于参数检验表,我们主要关注非常数项的p值.在该参数检验表中,$\beta$的概率p值为8.00849E-41 < 0.05,表面该偏回归系数统计有效.

对剩余九支股票进行以上检验分析,我们对检验分析结果作出以下汇总:

\begin{table}[ht]
\centering
\caption{十支股票回归结果检验分析}\label{inv9_tab3}
\begin{tabular}{|c|c|c|c|c|c|}
\hline
股票代码 & $R^{2}$ & F & FINV& Sign.F & p-value of $\beta$\\
\hline
600016 & 0.5137 & 268.2634 & 3.878329734& 1.21358E-41 & 1.21358E-41\\
\hline
600028 & 0.3692 & 148.6465 & 3.878329734 & 3.17083E-27 & 3.17083E-27\\
\hline
600050 & 0.2909 & 95.98651 &3.881505376& 3.27217E-19 & 3.27217E-19\\
\hline
600104 & 0.1746 & 53.74051 &3.878329734& 3.05341E-12 & 3.05341E-12\\
\hline
600588 & 0.1066 & 30.29159 &3.878329734& 9.07712E-08 & 9.07712E-08\\
\hline
600138 & 0.1535 & 27.37966 &3.903781387& 5.5007E-07 & 5.5007E-07\\
\hline
600166 & 0.2352 & 78.11948 &3.878329734& 1.6501E-16 & 1.6501E-16\\
\hline
601288 & 0.5064 & 260.5895 & 3.878329734&8.00849E-41 & 8.00849E-41\\
\hline
601319 & 0.0224 & 2.952182 &3.914559163& 0.088160927 & 0.088160927\\
\hline
601668 & 0.2233 & 73.00991 &3.878329734 & 1.21E-15 & 1.21232E-15\\
\hline
\end{tabular}
\end{table}

从表3我们可以看出,大部分股票的单指数模型拟合效果部分股票的线性回归拟合效果一般,中国人保(股票代码:601319)的拟合效果不好,可能是中国人保上市较晚,数据样本较小导致的.










