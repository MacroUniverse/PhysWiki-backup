% 正交子空间
% 内积|子空间|线性代数|正交

\subsection{正交子空间}
\pentry{直和\upref{DirSum}, 内积\upref{InerPd}}

\begin{definition}{正交子空间}
一个内积空间 $V$ 中, 如果两个子空间 $V_1$ 和 $V_2$ 任意各选一个矢量 $\ket{v_1}$ 和 $\ket{v_2}$ 都有 $\braket{v_1}{v_2} = 0$, 那么我们就说者两个子空间是\textbf{正交}的.
\end{definition}

\begin{theorem}{}
从基底的角度来看, 两个空间正交的充分必要条件是: 如果从两空间各选一组基底 $\ket{\alpha_i}$ $(i = 1, \dots, N_1)$ 和 $\ket{\beta_i}$ $(i = 1, \dots, N_2)$, 有对任意 $i, j$ 都有 $\braket{\alpha_i}{\beta_j} = 0$.
\end{theorem}

\subsection{正交子空间的直和}
特殊地, 如果 $V_1$ 和 $V_2$ 的两组基底合并后仍然正交归一, 那么合并后就得到了直和空间 $V_1 \oplus V_2$ 中的一组正交归一基底. 但注意直和空间中的任意一组正交归一基底未必可以划分为 $V_1$, $V_2$ 空间中的两组基底.

\begin{exercise}{}
请将\autoref{DirSum_ex1}\upref{DirSum} 和\autoref{DirSum_ex2}\upref{DirSum} 稍作修改以适用于本文.
\end{exercise}
