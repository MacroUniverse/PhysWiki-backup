% 理想气体(微正则系综法)
% 理想气体|微正则系综|相空间

\begin{issues}
\issueDraft
\end{issues}

\pentry{理想气体的状态密度(相空间)\upref{IdSDp}}
\subsection{$N$ 粒子相空间}

由 “理想气体的状态密度(相空间)\upref{IdSDp}” 中的结论, $\Delta E$ 能量内的状态数为
\begin{equation}
\Omega (E,V,N; \dd{E}) = g(E)\dd{E} = \frac{V^N}{N! h^3}\frac{(2\pi m)^{3N/2}}{(3N/2-1)!}E^{3N/2 - 1} \dd{E}
\end{equation}
定义
\begin{equation}
\lambda  = \frac{h}{\sqrt{2\pi mkT}}
\end{equation}
根据熵的定义%未完成链接
\begin{equation}
S = k\ln \Omega  = Nk \qty(\ln \frac{V}{N\lambda^3} + \frac52)
\end{equation}
其中用到了Stirling近似%未完成链接
$\ln N! = N\ln N - N$. 根据熵的微分关系
\begin{equation}
\dd{S} = \frac{1}{T} \dd{E} + \frac{P}{T} \dd{V} - \frac{\mu}{T} \dd{N}
\end{equation}
可求出温度, 压强, 化学势和能量之间的关系.
