% 反开普勒问题
% 斥力|平方反比|轨道方程

\pentry{开普勒问题\upref{CelBd}}

在中心力场问题\upref{CenFrc} 中, 若 $F(r)$ 是平方反比的斥力, 即
\begin{equation}
F(r) = \frac{k}{r^2}  \qquad V(r) = \frac{k}{r}
\end{equation}
(其中 $k$ 为大于零的常数) 则该问题被称为\textbf{反开普勒问题}. 在我们学过的各种力中, 只有两个同种点电荷间的库仑力满足这一要求.

在反开普勒问题中, 力心为双曲线的一个焦点, 质点的轨迹为双曲线离力心较远的一支. 与开普勒问题相同, 反开普勒问题中质点的能量 $E$ (质点的动能加势能, $E>0$) 和角动量 $L$ 可以唯一地确定轨道的形状和大小, 且 \autoref{CelBd_eq2}~\upref{CelBd} 到 \autoref{CelBd_eq8}~\upref{CelBd} 仍然成立
\begin{align}
a &= \frac{k}{2E}\\
b &= \frac{L}{\sqrt{2mE}}
\end{align}

推导的过程也和开普勒问题中的类似, 我们只需要将\autoref{CelBd_eq11}~\upref{CelBd} 到\autoref{CelBd_eq13}~\upref{CelBd} 过程中的所有负号变为正号即可.

\subsection{轨道方程推导}
\pentry{比耐公式\upref{Binet}}

将平方反比斥力 $F(r) = k/r^2$ 即 $F(1/u) = ku^2$ 代入比耐公式
\begin{equation}
\dv[2]{u}{\theta} + u = -\frac{m}{L^2 u^2} F\qty(\frac 1u)
\end{equation}
通解\upref{Ode2N}为
\begin{equation}\label{InvKep_eq3}
u(\theta) = -\frac{1}{p} \qty[1 + e\cos(\theta  + \phi_0)]
\end{equation}
其中
\begin{equation}
p = \frac{L^2}{mk}
\end{equation}
与开普勒问题中的双曲线轨道(\autoref{Keple1_eq7}~\upref{Keple1})相比, \autoref{InvKep_eq3} 中的常数项由正号变为负号, 这使得极坐标的双曲线方程表达双曲线离焦点较远的一支(见\autoref{Cone_eq6}~\upref{Cone}).