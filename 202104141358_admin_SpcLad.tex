% 太空电梯

\pentry{万有引力\upref{Gravty}, 离心力\upref{Centri}, 积分方程}% 未完成

\footnote{关于太空电梯的概念参考 Wikipedia \href{https://en.wikipedia.org/wiki/Space_elevator}{相关页面}.}太空电梯通常是从地球同步轨道外的一个空间站悬挂一根缆绳到地球表面, 使得我们可以通过该缆绳将一定质量的物体直接运送到太空中. 缆绳的张力由空间站的离心力提供. 碳纳米管材料的发现使得太空电梯从的可行性大大提高, 本文中我们来计算理想情况下太空电梯所需缆绳的质量. 根据已有的碳纳米管材料数据, 下面的计算发现所需缆绳的质量仅仅是最大载重(包括动力系统)的 2.7 到 13.5 倍. 可见碳纳米管材料的大规模生产将有可能开启全新的航天时代.

以下的计算是风速为零的理想情况, 为了避免极端天气, 可在空间站安装伸缩机, 只有在适合的天气将缆绳下放到地面. 太空电梯的另一个问题是需要避免低轨道卫星和太空碎片的碰撞, 这些不在本文讨论范围内.

\subsection{理想情况计算}
在地球所在的旋转参考系中, 令缆绳上某点的张力为 $F(r)$, $r$ 为该点到地心的距离. 令缆绳底部悬挂重物的重力为 $F_0$, 则缆绳底部所需张力为 $F(r_0) = F_0$, $r_0$ 是地球赤道半径. 显然在近地处缆绳张力 $F(r)$ 随着高度 $r$ 的增加, 因为 $F(r)$ 等于底部重物的重力加上 $r$ 下方所有缆绳的重力减离心力. 当 $r$ 大于同步轨道半径时, $F(r)$ 慢慢减小, 这是因为同步轨道以上缆绳受到的离心力已经大于万有引力. 为了节约材料, 缆绳的线密度(单位长度的质量)也可以设计为高度的函数 $\lambda(r)$. 于是缆绳的张力可以通过以下积分计算, 其中括号的第一项是引力\upref{Gravty}, 第二项是离心力\upref{Centri}.

\begin{equation}\label{SpcLad_eq1}
F(r) = F_0 + \int_{r_0}^{r} \qty(\frac{GM}{r'^2} - \omega^2 r') \lambda(r') \dd{r'}
\end{equation}
其中 $\omega$ 是地球自转角速度.

令缆绳材料单位截面可承受的最大张力为 $p$, 那么截面面积至少为
\begin{equation}
A(r) = F(r)/p
\end{equation}
绳的密度为常数 $\rho$, 那么线密度为
\begin{equation}
\lambda(r) = \rho A(r) = \rho F(r)/p
\end{equation}
带回\autoref{SpcLad_eq1} 有积分方程
\begin{equation}
F(r) = \frac{GM\rho}{p} \int_{r_0}^{r} \frac{F(r')}{r'^2} \dd{r'} - \frac{\omega^2\rho}{p}\int_{r_0}^r F(r') r' \dd{r'}
\end{equation}
令两个积分前面的常数为 $\alpha, \beta$, 两边求导(右边使用\autoref{NLeib_eq10}~\upref{NLeib})得微分方程
\begin{equation}
\dv{F}{r} = \alpha \frac{F(r)}{r^2} - \beta r F(r)
\end{equation}
分离变量, 解得
\begin{equation}
F = C\exp(-\frac{\alpha}{r} + \frac{\beta}{2} r^2)
\end{equation}
令初始条件为 $F(r_0) = F_0$, $F_0$ 是载重. 代入得到方程的解为
\begin{equation}
F(r) = F_0 \exp[\frac{GM\rho}{p} \qty(\frac{1}{r_0} - \frac{1}{r}) - \frac{\omega^2\rho}{2p}(r^2 - r_0^2)]
\end{equation}

目前最强的材料碳纳米管保守估计\footnote{来源: Nature 2019 \href{https://www.nature.com/articles/s41467-019-10959-7}{相关文章}.} $p = 2.5$-$6.6\times 10^{10} \Si{Pa}$, 密度使用 $\rho = 1.34\times 10^{3} \Si{kg/m^3}$. 地球半径取 $6.370\times 10^6\Si{m}$, 同步轨道半径为 $4.2164\times 10^7 \Si{m}$. 地球质量 $M = 5.972 \times 10^{24} \Si{kg}$.

代入上式得 $F = 2.7$-$13.5 F_0$.

\subsubsection{附: Matlab 代码}
\begin{lstlisting}[language=matlab]
G = 6.6743e-11; % 引力常数
M = 5.972e24; % 地球质量
r0 = 6.370e6; % 地球半径
r = 4.2164e7; % 同步轨道半径
rho = 1.34e3; % 碳纳米管密度 0.037-1.34
p = 2.5e10; % 碳纳米管截面压强 2.5-6.6
w = 2*pi/(24*3600); % 地球角速度

ratio = exp(G*M*rho/p*(1/r0 - 1/r) - w^2*rho/(2*p)*(r^2 - r0^2));
\end{lstlisting}
