% 三维简谐振子(球坐标)
% 量子力学|薛定谔方程|球坐标|球谐函数

\begin{issues}
\issueDraft
\end{issues}

\pentry{球坐标系中的径向方程\upref{RadSE}}

势能为 $V(r) = m\omega^2 r^2/2$,  总波函数和能级为
\begin{equation}
\psi_{n,l,m} = R_{n,l}(r) Y_{l,m}(\Omega)
\qquad
E_{n,l} = \qty(2n + l + \frac32) \hbar \omega
\end{equation}    
由于角动量量子数 $l$ 只决定离心势能 $\frac{\hbar^2}{2m} \frac{l(l + 1)}{r^2}$  的大小, 所以 $l$ 可以取任意非负整数. 径向波函数为. 令 $x = r/\beta $
\begin{equation}
R_{n,l}(r) = \frac{1}{\beta^{3/2} \pi^{1/4}} \sqrt{\frac{2^{n+l+2} n!}{(2n + 2l + 1)!!}} x^l L_n^{l+1/2}(x^2) \E^{-x^2/2}
\end{equation}
其中 $L_n^{l+1/2}$ 是广义拉盖尔多项式(链接未完成), 氢原子中的拉盖尔函数是 $L_n(x) = L_n^0(x)$. 

递推关系
\begin{equation}
L_{n+1}^\alpha (x) = [(2n + 1 + \alpha  - x)L_n^\alpha (x) - (n + \alpha )L_{n - 1}^\alpha (x)]/(n + 1)
\end{equation}
\begin{equation}
L_0^\alpha (x) = 1
\qquad
L_1^\alpha (x) = 1 + \alpha  - x
\end{equation}
    
罗德里格斯公式
\begin{equation}
L_n^\alpha (x) = \frac{x^{-\alpha} \E^x}{n!} \dv[n]{x} (\E^{-x} x^{n+\alpha})
\end{equation}

前几个束缚态为 (简并 $\deg  = \sum (2l + 1)$ )
\begin{itemize}
\item $E = 3\hbar \omega /2$ ( $\deg  = 1$ )
\begin{equation}
R_{0,0}(r) = \frac{1}{\beta^{3/2} \pi^{1/4}} 2\E^{-\frac12 x^2}
\end{equation}

\item $E = 5\hbar \omega /2$ ( $\deg  = 3$ )
\begin{equation}
R_{0,1}(r) = \frac{1}{\beta^{3/2} \pi^{1/4}} \frac{2\sqrt 6}{3} x \E^{-\frac12 x^2}
\end{equation}

\item $E = 7\hbar \omega /2$ ( $\deg  = 6$ )
\begin{equation}
R_{0,2}(r) = \frac{1}{\beta^{3/2} \pi^{1/4}} \frac{4}{\sqrt{15}} x^2 \E^{- x^2/2}
\end{equation}
\begin{equation}
R_{1,0}(r) = \frac{1}{\beta^{3/2} \pi^{1/4}} \frac{2\sqrt 6}{3} \qty(\frac32 - x^2) \E^{-x^2/2}
\end{equation}

\item $E = 9\hbar\omega /2$ ( $\deg  = 8$ )
\begin{equation}
R_{0,3}(r) = \frac{1}{\beta^{3/2} \pi^{1/4}} 4\sqrt{\frac{2}{105}} x^3 \E^{-x^2/2}
\end{equation}
\begin{equation}
R_{1,1}(r) = \frac{1}{\beta^{3/2} \pi^{1/4}} \frac{4}{\sqrt{15}} \qty(\frac52 - x^2) x \E^{-x^2/2}
\end{equation}
\end{itemize}
