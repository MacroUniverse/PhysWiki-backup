% 海伦公式
\pentry{余弦定理\upref{CosThe}}
\begin{figure}[ht]
\centering
\includegraphics[width=5cm]{./figures/Heron_1.pdf}
\caption{三角形} \label{Heron_fig1}
\end{figure}
\footnote{参考 Wikipedia \href{https://en.wikipedia.org/wiki/Heron's_formula}{相关页面}.}若已知三角形的边长(\autoref{Heron_fig1} ), 其面积可以用海伦公式计算


\begin{equation}
A = \sqrt{s(s-a)(s-b)(s-c)}
\end{equation}
其中 $s = (a+b+c)/2$.

\addTODO{例题, 选一组方便计算的边长, 且不容易用其他方法计算面积}

\subsection{推导}
设a、b、c是三角形中角$\alpha$、$\beta$、$\gamma$相应的对边,那么有
\begin{equation}
A=\frac{1}{2}\text{底}\times\text{高}
=\frac{1}{2}ab\sin\gamma
=\frac{1}{2}ab\sqrt{1-\cos^2\gamma}\\
\end{equation}
由余弦定理\autoref{CosThe_eq1}~\upref{CosThe},
则上式可写为
\begin{equation}
\begin{aligned}
A&=\frac{1}{2}ab\sqrt{1-\frac{1}{4a^2b^2}(a^2+b^2-c^2)^2}\\
&=\frac{1}{4}\sqrt{4a^2b^2-(a^2+b^2-c^2)^2}\\
&=\frac{1}{4}\sqrt{(2ab+a^2+b^2-c^2)(2ab-a^2-b^2+c^2)}\\
&=\frac{1}{4}\sqrt{[(a+b)^2-c^2][c^2-(a-b)^2]}\\
&=\sqrt{\frac{(a+b+c)}{2}\frac{(a+b-c)}{2}\frac{(c+a-b)}{2}\frac{(c-a+b)}{2}}\\
&=\sqrt{s(s-a)(s-b)(s-c)}
\end{aligned}
\end{equation}
\begin{example}{海伦公式的应用}
某农民有一块形状为四边形的土地,现要种植玉米,为了节约成本,农民要计算土地面积以购买适应的种子量,而农民家里只有测距用的米尺,幸而该农民日常有看数学知识的习惯,知道海伦公式可以计算三角形的面积,现用米尺测得以下各顶点之间的距离:
\begin{equation}
\overline{AB}=50m,\overline{AC}=50m,\overline{BC}=80m,\overline{BD}=150m,\overline{CD}=120m
\end{equation}



\end{example}
