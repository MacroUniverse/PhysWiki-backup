% 张量扰动
对于在视界内部的模式来说,张量扰动对应了在FRW背景下传播的引力波.

\subsection{宇宙演化}
在共形牛顿规范下,我们只保留$h_{ij}^{TT}$,可得
\begin{equation}
ds^2 = a^2[-d  \eta^2+(\delta_{ij}+h_{ij}^{TT})dx^i dx^j ]~.
\end{equation}
对于爱因斯坦张量,我们有$\delta G^0_0 = 0, \delta G^i_0 = 0$以及
\begin{equation}
\delta G^i_j = \frac{1}{2 a^2} [ (h_{ij}^{TT}  )'' + 2\mathcal H (h_{ij}^{TT})' - \nabla^2 h_{ij}^{TT}  ]~.
\end{equation}
于是,扰动的爱因斯坦方程可以写成如下形式
\begin{equation}
(h_{ij}^{TT})'' + 2 \mathcal H (h_{ij}^{TT})' - \nabla^2 h_{ij}^{TT} = 16 \pi G a^2 \sigma_{ij}^{TT} ~.
\end{equation}
我们可以换到动量空间然后以极化张量为基进行如下展开
\begin{equation}
\tilde h_{ij}^{TT} (\eta,\mathbf k) = \sum_{A = +,\times} e^A_{ij} (\hat{\mathbf k}) \tilde h_A (\eta,\mathbf k)~, 
\end{equation}
类似地,我们有
\begin{equation}
\tilde \sigma_{ij}^{TT} (\eta,\mathbf k) = \sum_{A = +,\times} e^A_{ij} (\hat{\mathbf k}) \tilde \sigma_A (\eta,\mathbf k) ~.
\end{equation}





