% 电场波动方程
% 麦克斯韦方程组|波动方程|电场|介质|电位移矢量

\begin{issues}
\issueDraft
\end{issues}

\pentry{麦克斯韦方程组(介质)\upref{MWEq1}, 矢量算符运算法则\upref{VopEq}}

真空中, 由麦克斯韦方程组\upref{MWEq} 得
\begin{equation}
\curl(\curl\bvec E) = -\pdv{t} (\curl\bvec B) = -\epsilon_0\mu_0 \pdv[2]{t} \bvec E
\end{equation}
根据\autoref{VopEq_eq7}~\upref{VopEq} 化简得
\begin{equation}
\laplacian \bvec E - \frac{1}{c^2} \pdv[2]{\bvec E}{t} = 0
\end{equation}
这就是电场的波动方程. 所以电场的各个分量分别满足三维波动方程(链接未完成). 它的解为平面波
\begin{equation}
\bvec E(\bvec r, t) = \bvec E_0 \cos(\bvec k \bvec r - \omega t)
\end{equation}
其中 $\omega = c\abs{\bvec k} = ck$. 而通解是这些平面波的任意线性组合.

注意 $\bvec E_0$ 和 $\bvec k$ 未必垂直. 但是如何说明垂直? 必须要垂直的应该.

求出磁场的表达式.

\subsection{介质中}

非线性光学中一般认为介质具有 $\mu = \mu_0$,且假设 $\div\bvec E = 0$ 仍然成立

介质中没有自由电荷或自由电流.

类似真空情况的推导过程,有
\begin{equation}
\curl(\curl\bvec E) = -\pdv{t} (\curl\bvec B) = -\mu_0 \pdv{t} (\curl\bvec H)
= -\mu_0 \pdv[2]{t} \bvec D
\end{equation}

把电位移矢量的定义 $\bvec D = \epsilon_0\bvec E + \bvec P$ 代入上式,化简为

\begin{equation}
\curl(\curl\bvec E) = -\pdv{t} (\curl\bvec B) = -\mu_0 \pdv{t} (\curl\bvec H)
= -\mu_0 \pdv[2]{t} \bvec D
\end{equation}