% 力电类比

从上面的讨论可以知道,电磁振荡和机械振动的规律非常相似,所以运用力电类比就可以把电磁振荡和机械振动对应起来,只要知道一种振动的解,就可以用类比方法得到另一种振动的解.虽然机械振动比较直观,但由于电学的迅速发展,人们对交变电路规律的熟悉程度已经超过机械振动.因此,在工程上,常常把复杂的机械振动问题用力电类比方法化成交变电路问题,然后通过计算或实验测定,找出它们的解机械振动和电磁振荡对应的物理量列在下表中.

\begin{table}[ht]
\centering
\caption{机械振动和电磁振荡对应的物理量}\label{MeElec_tab1}
\begin{tabular}{|c|c|c|c|}
\hline  机械振动  & &  电磁振荡(  串联电路 ) \\ \hline  位移  & x &  电荷  & q \\ \hline  速度  & v &  电流  & i \\ \hline  质量  & m &  电感  & L \\ \hline  劲度系数  & k &  电容的倒数  & \frac1C \\ \hline  阻力系数  & \gamma &  电阻  & R  \\ \hline  驱动力  F &  电动势  & \mathscrE & \\ \hline  弹性势能  & \frac12 k x^2 &  电场能量  & \frac12 \fracq^2C \\ \hline  动能  & \frac12 m v^2 &  ?场能量  & \frac12 L i^2 \\ \hline
\end{tabular}
\end{table}