% 椭圆坐标系

\begin{issues}
\issueDraft
\end{issues}

\pentry{椭圆的三种定义\upref{Elips3}、正交曲线坐标系\upref{CurCor}}

\subsection{椭圆坐标系}
\footnote{参考 Wikipedia \href{https://en.wikipedia.org/wiki/Elliptic_coordinate_system}{相关页面}.}
\textbf{椭圆坐标系(elliptic coordinate system)}是一个二维正交曲线坐标系,它是三种三维正交曲线坐标系定义的基础,这三种正交曲线坐标系为:\textbf{椭圆柱坐标系(elliptic cylindrical coordinate system)}、\textbf{长椭球坐标系(ellipsoidal coordinate system)}和\textbf{扁椭球坐标系(oblate spheroidal coordinate system)}.椭圆坐标系的坐标线为共焦的椭圆和双曲线,椭圆柱坐标系由$xOy$面上的椭圆坐标系向$z$轴方向平移得到;长(短)椭球坐标系是将椭圆坐标系绕椭圆长(短)轴方向旋转得到.

椭圆坐标系上的点的位置由 $(\xi,\eta)$ 这2个有序实数表示.$\xi$的等值曲线为一组共焦椭圆族,焦距为 $2c$;$\eta$ 的等值曲线为一组共焦的双曲线族,其焦点与椭圆族焦点相重.$\xi$、$\eta$由直角坐标定义
\begin{equation}\label{EliCor_eq3}
\left\{\begin{aligned}
&x=c\cosh\xi\cdot\cos\eta\\
&y=c\sinh\xi\cdot\sin\eta
\end{aligned}\right.
\end{equation}

容易看出
\begin{equation}
\frac{x^2}{c^2\cosh^2\xi}+\frac{y^2}{c^2\sinh^2\xi}=1
\end{equation}
\begin{equation}
\frac{x^2}{c^2\cos^2\eta}-\frac{y^2}{c^2\sin^2\eta}=1
\end{equation}
这便是刚刚说的:$\xi$的等值曲线为一组共焦椭圆族,而$\eta$ 的等值曲线为一组共焦的双曲线族.

容易证明椭圆坐标系是一个正交曲线坐标系.在某点 $\bvec{r}$ 处,坐标轴$\xi$ 和 $\eta$ 的方向分别为$\pdv*{\bvec{r}}{\xi}$ 和 $\pdv*{\bvec{r}}{\eta}$.
由\autoref{EliCor_eq3} 
\begin{equation}
\left\{
    \begin{aligned}
    &\dd x=c\sinh\xi\cdot\cos\eta\dd\xi-c\cosh\xi\sin\eta\dd\eta\\
    &\dd y=c\cosh\xi\cdot\sin\eta\dd\xi+c\sinh\xi\cdot\cos\eta\dd\eta\\
    \end{aligned}\right.
\end{equation}
令椭圆坐标 $\xi$、 $\eta$对应的单位矢量为 $\uvec{\xi}$、 $\uvec{\eta}$,由\autoref{CurCor_eq8}~\upref{CurCor}
\begin{equation}
\left\{
    \begin{aligned}
    &\uvec{\xi}=\frac{\pdv*{\bvec{r}}{\xi}}{|\pdv*{\bvec{r}}{\xi}|}=\frac{1}{\sqrt{\sinh^2\xi+\sin^2\eta}}\qty(\sinh\xi\cdot\cos\eta\uvec{x}+\cosh\xi\cdot\sin\eta\uvec{y})\\
    &\uvec{\eta }=\frac{\pdv*{\bvec{r}}{\eta}}{|\pdv*{\bvec{r}}{\eta}}=\frac{1}{\sqrt{\sinh^2\xi+\sin^2\eta}}\qty(-\cosh\xi\sin\eta\uvec{x}+\sinh\xi\cdot\cos\eta\uvec{y})\\
    &\uvec{z}=\frac{\pdv*{\bvec{r}}{z}}{|\pdv*{\bvec{r}}{z}|}=\uvec{z}
    \end{aligned}\right.
\end{equation}
\subsection{椭圆柱坐标系}
椭圆柱坐标系中点的位置可由 $(\xi,\eta,z)$ 这3个有序实数表示.$\xi$的等值曲面为一组共焦椭圆柱面族,焦距为 $2c$;$\eta$ 的等值曲面为一组共焦的双曲柱面族,其焦点与椭圆柱面族焦点相重.而 $z$轴一般与直角坐标系 $z$ 轴重合.

椭圆柱坐标与直角坐标的关系为
\begin{equation}\label{EliCor_eq1}
\left\{\begin{aligned}
&x=c\cosh\xi\cdot\cos\eta\\
&y=c\sinh\xi\cdot\sin\eta\\
&z=z
\end{aligned}\right.
\end{equation}
其中 $\xi\geq0,0\leq\eta<2\pi,-\infty<z<+\infty$.

容易看出
\begin{equation}
\frac{x^2}{c^2\cosh^2\xi}+\frac{y^2}{c^2\sinh^2\xi}=1
\end{equation}
\begin{equation}
\frac{x^2}{c^2\cos^2\eta}-\frac{y^2}{c^2\sin^2\eta}=1
\end{equation}
这便是刚刚说的:$\xi$的等值曲面为一组共焦椭圆柱面族,而$\eta$ 的等值曲面为一组共焦的双曲柱面族.

\subsection{长椭球坐标系}
设二维椭圆坐标系定义在 $xOz$ 平面上,椭圆长轴与 $z$ 轴重合.将椭圆坐标系绕着 $z$ 轴旋转,便可得到长椭球坐标系(而绕 $x$ 轴旋转则得到扁椭球坐标系).

\addTODO{图}
\subsection{椭圆坐标系中的单位矢量}
容易证明椭圆坐标系是一个正交曲线坐标系.在某点 $\bvec{r}$ 处,坐标轴$\xi$、$\eta$、$z$的方向分别为$\pdv*{\bvec{r}}{\xi}$、$\pdv*{\bvec{r}}{\eta}$、$\pdv*{\bvec{r}}{z}$.由\autoref{EliCor_eq1} 
\begin{equation}\label{EliCor_eq2}
\left\{
    \begin{aligned}
    &\dd x=c\sinh\xi\cdot\cos\eta\dd\xi-c\cosh\xi\sin\eta\dd\eta\\
    &\dd y=c\cosh\xi\cdot\sin\eta\dd\xi+c\sinh\xi\cdot\cos\eta\dd\eta\\
    &\dd z=\dd z
    \end{aligned}\right.
\end{equation}
则
\begin{equation}
\left\{
    \begin{aligned}
&\pdv*{\bvec{r}}{\xi}=c\sinh\xi\cdot\cos\eta\uvec{x}+c\cosh\xi\cdot\sin\eta\uvec{y}\\
&\pdv*{\bvec{r}}{\eta}=-c\cosh\xi\sin\eta\uvec{x}+c\sinh\xi\cdot\cos\eta\uvec{y}\\
&\pdv*{\bvec{r}}{z}=\uvec{z}
    \end{aligned}\right.
\end{equation}

令3个椭圆坐标 $\xi$、 $\eta$、 $z$ 对应的单位矢量为 $\uvec{\xi}$、 $\uvec{\eta}$、 $\uvec{z}$,由\autoref{CurCor_eq8}~\upref{CurCor}
\begin{equation}
\left\{
    \begin{aligned}
    &\uvec{\xi}=\frac{\pdv*{\bvec{r}}{\xi}}{|\pdv*{\bvec{r}}{\xi}|}=\frac{1}{\sqrt{\sinh^2\xi+\sin^2\eta}}\qty(\sinh\xi\cdot\cos\eta\uvec{x}+\cosh\xi\cdot\sin\eta\uvec{y})\\
    &\uvec{\eta }=\frac{\pdv*{\bvec{r}}{\eta}}{|\pdv*{\bvec{r}}{\eta}}=\frac{1}{\sqrt{\sinh^2\xi+\sin^2\eta}}\qty(-\cosh\xi\sin\eta\uvec{x}+\sinh\xi\cdot\cos\eta\uvec{y})\\
    &\uvec{z}=\frac{\pdv*{\bvec{r}}{z}}{|\pdv*{\bvec{r}}{z}|}=\uvec{z}
    \end{aligned}\right.
\end{equation}

显然,3个坐标轴 $\xi$、$\eta$、$z$相互垂直,即椭圆柱坐标系$(\xi,\eta,z)$为正交曲线坐标系.而按$(\xi,\eta,z)$排序是由于$\uvec{\xi}\cross\uvec{\eta}=\uvec{z}$(类比直角坐标系$(x,y,z)$中$\uvec x\cross\uvec y=\uvec z$)

\addTODO{写出梯度散度旋度, 引用 “正交曲线坐标系中的矢量算符\upref{CVecOp}”}
