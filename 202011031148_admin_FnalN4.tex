% 泛函分析笔记4
% 泛函分析|数学分析|空间|Banach 空间|希尔伯特空间

\subsection{4.1 Symmetric Operators}
\begin{itemize}
\item Hilbert 空间 $X$ 上的线性算符 $A:D(A)\subseteq X\to X$ 是对称的当且仅当 $D(A)$ 在 $X$ 中稠密且 $(Au|u) = (u|Av)$ 对所有 $u,v\in D(A)$ 成立

\item Hilbert 空间 $X$ 上的线性对称算符 $A:D(A)\subseteq X\to X$ 满足: (1) $(Au|u)$ 是实数, (2) 所有本征值为实数, (3) 不同本征值对应的本征矢正交, (4) 本征矢构成的至多可数的完备正交归一系对应的本征值包含 $A$ 的所有本征值
\end{itemize}


\subsection{4.2 The Hilbert-Schmidt Theory}
\begin{itemize}
\item 令 $A:X\to X$ 为可分 Hilbert 空间中的一个非空线性对称紧算符, 那么 (1) $A$ 存在本征矢构成的完备正交归一系

\item 所有本征值 $\lambda$ 为实数, 且每个 $\lambda\ne0$ 存在有限简并

\item 不同本征值对应的本征矢正交

\item 如果 $A$ 有可数个本征值(注意 $\lambda=0$ 不是本征值), 那么本征值序列 $\lambda_n \to 0$
\end{itemize}
