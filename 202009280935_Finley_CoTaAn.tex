% 列联表独立性检验和拟合优度检验
\pentry{机器学习数据类型\upref{DatTyp}}
分类数据是对事物进行分类的结果,它虽然是用数值表示,但是数值仅仅反映对象的不同特征,其大小没有意义.分类数据的结果是频数,对其进行统计分析主要利用$\chi^2$分布.
\subsection{$\chi^2$统计量}
$\chi^2$统计量可用于测定2个分类变量之间的相关程度.用$f_o$表示观察值频数,$f_e$表示期望值频数,则:
\begin{equation}
\chi^2 =  \sum \frac {(f_o-f_e)^2}{f_e}
\end{equation}
利用$\chi^2$统计量,可以对分类数据进行拟合优度检验和独立性检验.
\subsection{拟合优度实验}
拟合优度实验依据总体分布,计算出各类别的期望频数,与观察频数进行对比,判断两者是否有显著差异,从而对分类变量进行分析.
\subsubsection{原假设和备择假设}
$H0$:观察频数与期望频数一致

$H1$:观察频数与期望频数不一致
\subsubsection{检验统计量}
检验统计量为$\chi^2$统计量.自由度为$d_f = R-1$,$R$为分类变量的类型的个数.
对于总体比例,同样可以使用拟合优度检验(比例可视为2个类别的分类变量).$z$检验只能针对二项分布问题,而$\chi^2$检验既可以分析二项分布,也可以分析多项分布(对总体的多个比例的假设进行检验).
\addTODO{例题}
\begin{example}{}
列联表是由2个以上的变量交叉分类的频数分布表.将行变量视为R(3类),列变量视为C(3类),可以把每一个列联表称为R×C列联表.下表为3×3列联表:
\begin{figure}[ht]
\centering
\includegraphics[width=12cm]{./figures/CoTaAn_1.png}
\caption{列联表数据} \label{CoTaAn_fig1}
\end{figure}
分析列联表中行变量和列变量是否独立.

(1)原假设和备择假设

$H0$:行变量和列变量不存在依赖关系

$H1$:行变量和列变量存在依赖关系

(2)计算个单元期望频数值
\begin{equation}
f_e=\frac {R_T}{n}\times \frac {C_T}{n}\times n =\frac{R_T \times C_T}{n}
\end{equation}
其中$f_e$是给定单元中的期望频数,$R_T$是单元所在行的合计,$C_T$是单元所在列的合计,$n$是样本量.
\begin{figure}[ht]
\centering
\includegraphics[width=12cm]{./figures/CoTaAn_2.png}
\caption{计算过程} \label{CoTaAn_fig2}
\end{figure}
自由度为$d_f=(R-1)(C-1)=2\times2=4$.

由于$\chi^2>\chi^2_{0.05}=9.488$,故拒绝$H0$,接受$H1$,地区与等级之间存在依赖关系.
\end{example}
\subsection{列联表独立性检验}
拟合优度检验是针对一个分类变量的检验,对于两个分类变量,我们会关心它们是否有关联,称为独立性检验,通过列联表的方式呈现.
\subsection{列联表}
列联表(contingency table)是观测数据按两个或更多属性(定性变量)分类时所列出的频数表.它是由两个以上的变量进行交叉分类的频数分布表.