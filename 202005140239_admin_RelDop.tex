% 光的多普勒效应
% 相对论|光速|多普勒|频率|波长

\pentry{多普勒效应(一维匀速)\upref{Dople1}, 洛伦兹变换\upref{LornzT}}

我们只讨论真空中光源和接收者在同一直线做匀速运动的情况. 和机械波的最大的区别在于, 光的传播不需要介质, 任何参考系中真空中的光速都是常数 $c$. % 链接未完成
\begin{equation}
\frac{f'}{f} = \sqrt{\frac{c - u}{c + u}}
\end{equation}
其中 $u$ 是两点间的相对速度, 远离为正, 靠近为负.

\subsection{推导}
令光源所在参考系为 $S$, 接收者所在参考系为 $S$. 光源频率为 $f$, 接收者得到的频率为 $f'$

我们现在 $S$ 系中分析, 此时光源不动, 接收者以速度 $u$ 运动(可以取负值). 根据经典多普勒效应, 有
\begin{equation}
f'_0 = \frac{c \mp v_2}{c \mp v_1}f_1 = \qty(1 \mp \frac{u}{c})f
\end{equation}
当接收者在波源右边($x$ 坐标比观察者大)时该式取负号, 在波源左边时取正号. $f'_0$ 是在 $S$ 参考系中接收者与波峰相遇的频率, 但在接收者的参考系, 即 $S'$ 系中, 我们还要考虑钟慢效应. 所以有
\begin{equation}
f' = \frac{f'_0}{\sqrt{1 - u^2/c^2}} = \sqrt{\frac{c \mp u}{c \pm u}} f
\end{equation}
当接收者在波源右边时符号上负下正, 在波源左边时符号上正下负.
