% 1

% SIM

\subsection{股票基本指标}
我们以从CSMAR数据库上下载的各支股票的周回报率作为股票周收益率.

\subsubsection{1、期望收益率}
运用Excel计算十支股票的期望收益率,计算结果如下表所示.

\begin{table}[ht]
\centering
\caption{十支股票期望收益率}\label{inv1_tab1}
\begin{tabular}{|c|c|c|c|c|c|}
\hline
股票代码 &600016 & 600028 & 600050 & 600104 &	600588 \\
\hline
期望收益率 &-0.000802&0.001282&0.001586&0.002030&0.008136\\
\hline
股票代码 &601138&601166&601288&601139&601668\\
\hline
期望收益率&-0.001808&0.002843&0.001452&0.004904&0.002205\\
\hline
\end{tabular}
\end{table}

\subsubsection{2、标准差}
运用Excel计算十支股票的标准差,计算结果如下表所示.

\begin{table}[ht]
\centering
\caption{十支股票标准差}\label{inv1_tab2}
\begin{tabular}{|c|c|c|c|c|c|}
\hline
股票代码 &600016 & 600028 & 600050 & 600104 &	600588 \\
\hline
标准差 &0.021491&0.029022&0.045676&0.040544	&0.065387\\
\hline
股票代码 &601138&601166&601288&601139&601668\\
\hline
标准差&0.053056&0.032673&0.021215&	0.082476&0.036732\\
\hline
\end{tabular}
\end{table}
\subsubsection{3、协方差矩阵}
运用MATLAB计算十支股票的协方差矩阵,代码片段如下所示.

图1:计算协方差矩阵代码

计算结果如下图所示.
图2:十支股票协方差矩阵

\subsubsection{4、相关系数矩阵}
运用MATLAB计算十支股票的相关系数矩阵,代码片段如下所示.

图3:计算相关系数矩阵代码

计算结果如下图所示.
图4:十支股票相关系数矩阵

\subsection{单支股票的单因素模型}
