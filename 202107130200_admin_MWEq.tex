% 麦克斯韦方程组
% 麦克斯韦方程组|电磁感应|高斯定律|电位移|安培定律

\begin{issues}
\issueDraft
\end{issues}

\pentry{法拉第电磁感应定律\upref{FaraEB},磁场的高斯定律\upref{MagGau}, 位移电流\upref{DisCur}}

\footnote{参考 Wikipedia \href{https://en.wikipedia.org/wiki/Maxwell's_equations}{相关页面}.}麦克斯韦方程组描述了经典电磁理论中电荷如何影响电磁场, 以及电磁场变化的规律.它有多种不同的表示方法,本词条中讨论的是多数学科工作中常用的形式.

\subsection{微分形式}
麦克斯韦方程组共有四条方程
\begin{align}\label{MWEq_eq1}
&\div \bvec E = \frac{\rho}{\epsilon_0}\\
\label{MWEq_eq2}
&\curl \bvec E = -\pdv{\bvec B}{t}\\
\label{MWEq_eq3}
&\div \bvec B = 0\\
\label{MWEq_eq4}
&\curl \bvec B = \mu_0 \bvec j + \mu_0\epsilon_0 \pdv{\bvec E}{t}
\end{align}
其中\autoref{MWEq_eq1} 到\autoref{MWEq_eq4} 分别是电场的高斯定律\upref{EGauss},法拉第电磁感应定律\upref{FaraEB},磁场的高斯定律\upref{MagGau}, 安培环路定理\upref{AmpLaw}(加位移电流\upref{DisCur}).%链接未完成

高斯单位制\upref{GaussU}中的麦克斯韦方程组更为对称(\autoref{GaussU_eq4}~\upref{GaussU})
\begin{equation}
\begin{aligned}
&\div \bvec E = 4\pi\rho\\
&\curl \bvec E = -\frac{1}{c}\pdv{\bvec B}{t}\\
&\div \bvec B = 0 \\
&\curl \bvec B = \frac{4\pi}{c} \bvec j + \frac{1}{c}\pdv{\bvec E}{t}
\end{aligned}
\quad\text{(麦克斯韦方程组)}
\end{equation}
注意电场和磁场的公式仍然不是完全对称的. 可以通过引入磁单极子使它们完全对称(\autoref{BMono_eq1}~\upref{BMono}).

麦克斯韦方程组完整地描述了经典电磁场的变化规律, 那么一个自然的问题是:已知一个矢量场的散度和旋度, 是否能唯一确定该矢量场? 一般答案是不能, 因为还可以叠加一个任意调和场\upref{HarmF}(见 “亥姆霍兹分解\upref{HelmTh}”). 但如果加上边界条件(如该矢量场在无穷远处趋于零), 那么就可以唯一确定.

\subsection{积分形式}
麦克斯韦方程组的积分和微分形式是完全等价的, 可以通过散度定理\upref{Divgnc}和斯托克斯定理\upref{Stokes}互相转换.
\begin{align}
\oint \bvec E \vdot \dd{\bvec s} &= \frac{1}{\epsilon_0}\int \rho \dd{V}\\
\oint \bvec E \vdot \dd{\bvec l} &= -\int \pdv{\bvec B}{t} \vdot \dd{\bvec s}\\
\oint \bvec B \vdot \dd{\bvec s} &= 0\\
\oint \bvec B \vdot \dd{\bvec l} &= \mu_0 \int \bvec j \vdot \dd{\bvec s} + \mu_0 \epsilon_0 \int \pdv{\bvec E}{t} \vdot \dd{\bvec s}
\end{align}

\subsection{外微分形式}
目前仅在部分理论物理领域讨论的\textbf{外微分形式},见\textbf{麦克斯韦方程组(外微分形式)}\upref{MWEq2}词条;由于其预备知识较多,不收录在本节里.
