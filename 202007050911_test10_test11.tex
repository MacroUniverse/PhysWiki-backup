% test11

$\begin{array}{l}\mathrm{由分布律的性质知}a=1-0.5-0.1=0.4.\\E(X^2)=1^2\cdot a+2^2\cdot0.1=0.8,\;\;\;E(X)=a+2\cdot0.1=0.6.\\D(X)=E(X^2)-\;E^2(X)=0.44.\end{array}$


$\mathrm{设总体}X\sim N\left(\mathrm\mu,\mathrm\sigma^2\right),X_1,X_2,\cdots,X_n\mathrm{是来自总体}X\mathrm{的简单随机样本},\overline X\mathrm{是样本均值},S^2\mathrm{是样本方差},\mathrm 则\frac{\overline{\mathrm X}-\mathrm\mu}{\mathrm S/\sqrt{\mathrm n}}\;\mathrm{服从的分布为}\;\left(\;\;\;\;\right)$"


"$若X\sim N(\mu,\;\sigma^2),\;则\frac{X-\mu}\sigma\sim N(0,\;1)$"


"$\begin{array}{l}P\{Y\leq\frac12\vert X\leq\frac12\}=\frac{P\{Y\leq\frac12,\;\;X\leq\frac12\}}{P\{X\leq\frac12\}}=\frac{\int_{-\infty}^{\displaystyle\frac12}\;\int_{-\infty}^\frac12f(x,y)dxdy}{\int_{-\infty}^\frac12\;\int_{-\infty}^{+\infty}f(x,y)dxdy}=\frac{\int_0^{\displaystyle\frac12}dy\;\int_
{\displaystyle\frac y2}^\frac121dx}{\int_0^\frac12\;dx\int_0^{2x}1dy}=\frac{\displaystyle\frac3{16}}{\displaystyle\frac14}=\frac34.\\\end{array}$"


"$\begin{array}{l}P(B\vert A)+\;P(\overline B\vert\overline A)=\frac{P(AB)}{P(A)}+\frac{P(\overline A\;\overline B)}{P(\overline A)}=\frac{P(AB)\lbrack1-P(A)\rbrack+\lbrack1-P(A\cup B)\rbrack P(A)}{P(A)\lbrack1-P(A)\rbrack}=1\\\mathrm{整理},得\\P(AB)=P(A)P(B)\end{array}$"


"$\begin{array}{l}\mathrm{已知某炼铁厂的铁水含碳量}X\mathrm{在正常情况下服从正态分布}N\left(4.55,0.108^2\right),\mathrm{一天测了}6\mathrm{炉铁水},\mathrm{其含碳量为}:4.48,4.40,4.46,4.50,4.44,4.43,\\\mathrm{假设方差不会改变},\mathrm{检验}H_0:\mu=\mu_0,H_1:\mu>\mu_0,\mathrm{则在显著性水平}\alpha\mathrm{下的拒绝域为}\;\left(\;\;\;\;\;\right)\end{array}$"


"$F_X(\;x\;)\;=P\{X\leq x\}=P\{X\leq x,Y<+\infty\}=F(\;x,+\infty\;)$"i


"$E\left(X\right)=\sum_kx_kp_k=3-2\theta,\begin{array}{l}令E\left(X\right)=\overline X,\mathrm 则\widehat\theta=\frac{3-\overline X}2,由\overline x=\frac43,\mathrm 得\widehat\theta=\frac56.\end{array}$"


"$X\sim B(10,0.4),则EX=4,DX=2.4.\;E(X^2)=DX+E^2(X)=18.4.$"


"$\begin{array}{l}设A_1=“\mathrm{产品是甲工厂生产的}”,A_2=“\mathrm{产品是乙工厂生产的}”,\;\;\;B=\;\;\;\;“\mathrm{次品}”.则\;\;\mathrm{由贝叶斯公式得}\\P(A_1\vert B)=\frac{P(A_1)P(B\vert A_1)}{P(A_1)P(B\vert A_1)+P(A_2)P(B\vert A_2)}=\frac{0.6\cdot0.01}{0.6\cdot0.01+0.4\cdot0.02}=\frac37\end{array}$"


"$\begin{array}{l}\mathrm{因为}\int_{-\infty}^{+\infty}\int_{-\infty}^{+\infty}f(x,y)dxdy=1,\mathrm{可得}1=\int_{-\infty}^{+\infty}\int_{-\infty}^{+\infty}f(x,y)dxdy=\int_0^{+\infty}dx\int_0^{+\infty}ke^{-2x-3y}dy=\frac16k,\\\mathrm{所以}k=6.\end{array}$"


"$\mathrm{正态总体方差未知时},\mu\mathrm{的置信区间为}\left[\overline x-t_{1-\alpha/2}s/\sqrt n,\overline x+t_{1-\alpha/2}s/\sqrt n\right].\mathrm{代入即得}$"


"$若X\mathrm{是连续型随机变量},\mathrm{则对任意实数}a,\;有P\{X=a\}=0.$"


"$\begin{array}{l}F(x,\;y)=P\{X\leq x,Y\leq y\}\\0\leq P\{X\leq x,Y\leq y\}\leq1\end{array}$"


"$\begin{array}{l}P\{X=1\}=P\{X\leq1\}-P\{X\;<\;1\}=F(1)-\lim_{x\rightarrow1^-}F(x)=0.25;\\P\{1\;<\;X\;<\;2\}=\lim_{x\rightarrow2^-}F(x)-F(1)=0;\\P\{2\;<\;X\;\}=1-F(2)=0.55;\\P\{4\;<\;X\;\}=1-F(4)=0.\\\end{array}$"


"$P\{\vert X\vert\leq2016\}=P\{-2016\leq X\leq2016\}=P\{-2016\;<\;X\leq2016\}=F(2016)-F(-2016)=F(2016)-(1-F(2016))=2F(2016)-1$"v


"$\begin{array}{l}P\{X\geq1\}=1-P\{X=0\}=1-C_2^0p^{{}^0}(1-p)^2=1-(1-p)^2=\frac59,\;\mathrm{解得}p=\frac13.\\P\{Y\geq1\}=1-P\{Y=0\}=1-C_3^0p^{{}^0}(1-p)^3=\frac{19}{27}.\end{array}$"


"$\begin{array}{l}P\{X=1\}=\frac16+\frac12=\frac23,P\{X=1,Y=1\}=P\{X=1\}P\{Y=1\},得P\{Y=1\}=\frac14,\\P\{X=2,Y=1\}=P\{Y=1\}-P\{X=1,Y=1\}=\frac14-\frac16=\frac1{12}.\end{array}$"


"$P\{X-Y=1\}=P\{X=1,Y=0\}+P\{X=2,Y=1\}=P\{X=1\}P\{Y=0\}+P\{X=2\}P\{Y=1\}=\frac38\times\frac13+\frac18\times\frac23=\frac5{24}.$"


"$\begin{array}{l}\int_0^1(ax+b)\operatorname dx=1,\;\mathrm{所以}\frac a2+b=1,EX=\int_0^1x(ax+b)\operatorname dx=\frac a3+\frac b2=\frac13,\\\mathrm{解得}a=-2,b=2.\end{array}$"


"$\mathrm{利用性质公式}\;1-P(\overline A)-P(\overline B)+P(\overline A\;\overline B)=1-P(\overline A\cup\;\overline B)=1-P(\overline{AB})=P(AB)$"


"$\begin{array}{l}\mathrm{似然函数}L\left(\theta\right)=P\left\{X_1=x_1,X_2=x_2,\cdots,X_8=x_8\right\}=4\theta^6\cdot\left(1-\theta\right)^2\cdot\left(1-2\theta\right)^4,\mathrm{对数似然函数}\ln L\left(\theta\right)=\ln4+6\ln\theta+2\ln\left(1-\theta\right)+4\ln\left(1-2\theta\right),\\\frac{d\ln L\left(\theta\right)}{d\theta}=\frac6\theta-\frac2{1-\theta}-\frac8{1-2\theta}=0,\mathrm 则\widehat\theta=\frac{7\pm\sqrt{13}}{12},\mathrm{由于}0<\theta<1/2,\mathrm 故\widehat\theta=\frac{7-\sqrt{13}}{12}.\end{array}$"


"$\sigma^2\mathrm{已知},\mathrm{置信区间应选}\left[\overline x-\frac\sigma{\sqrt n}u_{\alpha/2},\overline x+\frac\sigma{\sqrt n}u_{\alpha/2}\right],\mathrm{这里}n=16,\sigma=2,\alpha=0.1,\mathrm{代入即得}.$"


"$\begin{array}{l}P\{0\;<\;X\;<\;1,\;0\;<\;Y\;<\;2\}=\underset{0\;<\;x\;<\;1,0\;<\;y\;<\;2}{\int\int}f(x,y)dxdy=\int_0^1dx\int_0^212e^{-3x-4y}dy=(1-e^{-3})(1-e^{-8})=1-e^{-3}-e^{-8}+e^{-11}.\\\end{array}$"


"$P\{X=2,Y=1\}=P\{X=2\}P\{Y=1\vert X=2\}=\frac13\times\frac12=\frac16$"


