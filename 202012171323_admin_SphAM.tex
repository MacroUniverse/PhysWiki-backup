% 球坐标系中的角动量算符

\pentry{角动量(量子)\upref{QOrbAM}}

本文使用原子单位制\upref{AU}. 在量子力学中, 我们一般把角动量算符放在球坐标中表示. 把轨道角动量算符在直角坐标系中的定义(\autoref{QOrbAM_eq2}~\upref{QOrbAM})通过链式法则\upref{PChain}用球坐标表示(留作习题).
\begin{equation}
L_x = \I \qty(\sin\phi\pdv{\theta} + \cot\theta\cos\phi\pdv{\phi})
\end{equation}
\begin{equation}
L_y = \I \qty(-\cos\phi\pdv{\theta} + \cot\theta \sin\phi \pdv{\phi})
\end{equation}
\begin{equation}
L_z = -\I\pdv{\phi}
\end{equation}
\begin{equation}
L^2 = L_x^2 + L_y^2 + L_z^2 = -\frac{1}{\sin\theta}\pdv{\theta} \qty(\sin \theta \pdv{u}{\theta}) - \frac{1}{\sin^2 \theta} \pdv[2]{u}{\phi}
\end{equation}
注意 $L^2$ 恰好是球坐标系中拉普拉斯算符的角向部分(\autoref{SphNab_eq3}~\upref{SphNab}) $\laplacian_\Omega$ 乘以 $-1$.
\begin{equation}\label{SphAM_eq1}
L^2 = -\laplacian_\Omega
\end{equation}
这并不奇怪, 经典力学中球坐的哈密顿量可以记为(\autoref{HamCan_eq3}~\upref{HamCan})
\begin{equation}
H = \frac{p_r^2}{2m} + \frac{L^2}{2mr^2} + V
\end{equation}
其中 $p_r = m\dot r$, $L = mr^2\dot\theta$. 而量子化后, 得到哈密顿算符为
\begin{equation}
H = -\frac{1}{2m}\laplacian + V = \frac{-\laplacian_r}{2m} +\frac{-\laplacian_\Omega}{2mr^2} + V
\end{equation}
这让我们很容易猜出 $p_r^2 = -\laplacian_r$ 和\autoref{SphAM_eq1}.

另外类比动量算符 $\bvec p = -\I \grad$, 我们可以定义 $\grad_\Omega$
\begin{equation}
\bvec L = -\I \grad_\Omega
\end{equation}
于是 $\laplacian_\Omega$ 可以看作是两个 $\grad_\Omega$ 相乘而得.

\subsection{角动量算符的本征函数}
\pentry{球谐函数\upref{SphHar}}
我们已经知道 $L^2, L_z$ 对易且具有共同本征矢 $\ket{l, m}$, 现在我们在球坐标中求解它的波函数. 来看本征方程
\begin{equation}
L_z \ket{l, m} = m\ket{l, m}
\end{equation}
\begin{equation}
L^2 \ket{l, m} = l(l+1)\ket{l, m}
\end{equation}
它的解就是球谐函数 $Y_{l,m}(\theta,\phi)$. 但本征波函数应该是三维的, 所以任意波函数 $R(r)Y_{l,m}(\theta, \phi)$ 都是 $L^2$ 和 $L_z$ 的本征波函数.
