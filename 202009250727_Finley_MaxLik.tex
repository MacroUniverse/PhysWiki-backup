% 最大似然估计
\begin{issues}
\issueTODO
\end{issues}
\pentry{随机变量 概率分布函数\upref{RandF},定积分\upref{DefInt}}
在统计学中,最大似然估计(maximum likelihood estimation,缩写为MLE),也称极大似然估计,是用来估计一个概率模型的参数的一种方法.

下边的讨论要求读者熟悉概率论中的基本定义,如概率分布、概率密度函数、随机变量、数学期望等.读者还须先熟悉连续实函数的基本技巧,比如使用微分来求一个函数的极值(即极大值或极小值).
最大似然估计的原理
给定一个概率分布$D $,已知其概率密度函数(连续分布)或概率质量函数(离散分布)为$f_D $,以及一个分布参数$\theta $ ,我们可以从这个分布中抽出一个具有$n$个值的采样$X_{1},X_{2},... ,X_{n} $,利用$f_D$计算出其似然函数:
$$
L(\theta|x_1,...x_n ) = f_{\theta }(x_1,...x_n )
$$
若$D$是离散分布,$f_{\theta }$即是在参数为$\theta$ 时观测到这一采样的概率.若其是连续分布,$f_{\theta }$则为$X_{1},X_{2},... ,X_{n} $联合分布的概率密度函数在观测值处的取值.一旦我们获得$X_{1},X_{2},... ,X_{n} $,我们就能求得一个关于$\theta $的估计.最大似然估计会寻找关于$\theta$ 的最可能的值(即,在所有可能的$\theta $取值中,寻找一个值使这个采样的“可能性”最大化).从数学上来说,我们可以在$\theta $的所有可能取值中寻找一个值使得似然函数取到最大值.这个使可能性最大的$\widehat{\theta}$值即称为$\theta $的最大似然估计.由定义,最大似然估计是样本的函数.
\begin{itemize}
\item 这里的似然函数是指$x_1,x_2,\ldots,x_n$不变时,关于$\theta $的一个函数.
\item 最大似然估计不一定存在,也不一定唯一.
\end{itemize}