% 线性方程组

\subsection{低维简单例子}
\subsubsection{二元一次方程组}
在许多实际生活中,我们往往需要解决类似的问题:
%大体完成,还差插入图片与表达的细节优化
%littlefeng打算以线性方程组,以及高斯消元法作为线性代数的引入.后续的矩阵初等变换,向量组以及矩阵的秩都和这个息息相关

小明拿着家长给的$100$元去超市买饮料.超市里有$3$元钱一瓶的可口可乐和$5$元钱一瓶的阿萨姆奶茶,最后他带着$28$瓶饮料回家过暑假了.求问小明买了多少瓶可口可乐,多少瓶阿萨姆奶茶?

这个问题实际上是一个二元一次方程组的问题,我们设小明买了$x$瓶可口可乐,$y$瓶阿萨姆奶茶,可以列出一个方程组:\begin{equation}
\leftgroup{
100 &= 3x + 5y (a)\\
28 &= x + y (b)\\
}\end{equation}
一个简单的解决方法是计算$(a)-3*(b)$,得到$16 = 0x + 2y$即$y = 8$,进一步就知道$x = 20$.

在这个简单的例子中,方程组$(1)$可以在二维坐标平面上找到两条直线作为对应,而它们的交点正是方程组的解,如图所示:
%图片后加
\subsubsection{三元一次方程组}
类似的在三维坐标系里面,考察三个平面:$S_1:x - 3y-2z=3$,$S_2:-2x+y-4z=-9$与$S_3:-x+3y-z=-7$的交点.
那么这个问题等价于解决如下三元一次方程组:\begin{equation}
\leftgroup{
x - 3y - 2z &= 3\\
- 2x + y - 4z &= -9\\
- x + 3y - z &= -7\\
}\end{equation}
解得\begin{equation}
\leftgroup{
x &= 2\\
y &= -1\\
z &= 1\\
}\end{equation}
那么$(2,-1,1)$就是三维坐标系中平面$S_1$,$S_2$与$S_3$的交点.
\subsection{一般情况下的定义}
一般的,形如:
\begin{equation}
\leftgroup{
&a_{1,1}x_1 + a_{1,2}x_2 + … + a_{1,n}x_n&=\quad &y_1\\
&a_{2,1}x_1 + a_{2,2}x_2 + … + a_{2,n}x_n&=\quad &y_2\\
&\qquad \qquad …  \qquad … \qquad  …\\
&a_{m,1}x_1 + a_{m,2}x_2 + … + a_{m,n}x_n&=\quad &y_m}\\
\end{equation}
的等式组统称为线性方程组,也可以根据其未知数的个数称为$n$元一次方程组.

其中$x_1… x_n$ 为 $n$ 个未知量,$y_1… y_m$ 与$a_{1,1} ,a_{1,2}… a_{1,n},a_{2,1} …… a_{n,m}$ 为给定的参数.(形如$a_{i.j}$的参数表示它是方程组中第$i$个方程的$x_j$对应的系数,也即第$i$行第$j$列系数)
线性方程组拥有一个判定是否有解,甚至是计算出具体解的普适方法,称为高斯(Gauss)——若尔当(Jordan)算法\upref{GAUSS}
\subsubsection{与几何的关系}
类似在低维例子中的表现,我们可以自然视方程组$(4)$的解表示$n$维空间中一列“超平面”
\begin{equation}
\leftgroup{
S_1 : &a_{1,1}x_1 + a_{1,2}x_2 + … + a_{1,n}x_n&=\quad &y_1\\
S_2 : &a_{2,1}x_1 + a_{2,2}x_2 + … + a_{2,n}x_n&=\quad &y_2\\
&\qquad \qquad …  \qquad … \qquad  …\\
S_m : &a_{m,1}x_1 + a_{m,2}x_2 + … + a_{m,n}x_n&=\quad &y_m}\\
\end{equation}的交集.

而一般情况下线性方程组解不唯一,相对应的这个集合不仅仅是一个点,而是$n$维空间的一个子空间.

事实上在高维空间中一个“超平面”难以凭借空间想象力得到其“映像”,必须借助代数表达研究其性质,这种情况下,可以认为形如$S : a_1x_1 + a_2x_2 + … + a_nx_n=y$($a_1… a_n$,不全为零)的等式唯一定义了一个$n$维空间中的“超平面”.