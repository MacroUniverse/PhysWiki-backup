% LaTeX 简介(除公式)

考虑到许多网友只用过 LaTeX 编辑公式而没有写过正文, 这里来结合本编辑器做一个简要的介绍.

LaTeX 是一种所见非所得的排版语言, 即用户编辑的是代码, 需要经过编译过程才能获得最终的显示效果. 小时物理百科的 PDF 编译使用 TeXlive2019, 而在线编辑器则是我们自行开发的.

一个简单完整的 LaTeX 文档如下
% 未完成:高亮
\begin{lstlisting}
\documentclass{article}
\usepackage{amsmath}

\begin{document}
\title{My Title}
\author{My Name}
\maketitle

\section{Introduction}
Some introduction.

\begin{equation}
a^2 + b^2 = c^2
\end{equation}

\subsection{Subtitle}
Subsection text.

\end{document}
\end{lstlisting}

编译后效果如\autoref{latxIn_fig1}.
\begin{figure}[ht]
\centering
\includegraphics[width=13cm]{./figures/LatxIn1.png}
\caption{排版效果} \label{latxIn_fig1}
\end{figure}
下面我们来解释本例中的代码.

\subsection{环境}
一个完整的 LaTeX 文档是由许多环境构成的, 环境的格式如下
\begin{lstlisting}
\begin{环境名}[可选设置]
...
...
\end{环境名}
\end{lstlisting}
其中 \lstinline|[可选设置]| 不一定会出现. 在一个完整的 LaTeX 文档中, 最大的环境是 \lstinline|document| 环境, 文档的所有内容(包括其他环境)都在 \lstinline|document| 环境中. 在 \lstinline|document| 环境之前通常会有一些设置, 例如规定文档的类别(第 1 行中的 \lstinline|article|), 使用一些语言拓展包(第 2 行的 \lstinline|amsmath|)等. 这些设置过于复杂, 这里不进行介绍. 一般建议直接直接使用模板.

在 \lstinline|document| 环境中, 我们可以用 \lstinline|\section| 和 \lstinline|\subsection| 等命令把文章划分成不同的章节和子章节. 编译器还可以根据这些命令自动生成目录. 在小时物理百科中, 我们使用四级标题, 分别是 \lstinline|\part|(部分), \lstinline|\section| (词条), \lstinline|\subsection|(节), \lstinline|\subsection|(子节).
