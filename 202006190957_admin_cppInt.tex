% c++ 的整数

本文的\textbf{模(modulo)}运算 \verb|m % n| 都是指 $m$ 加或减整数个 $n$ 后使结果范围在 $0$ 到 $n-1$ 之间.

\subsection{无符号整数}
\begin{itemize}
\item 取值范围为 $0$ 到 $2^n-1$
\item 如果溢出就把二进制的高位截去, 也就是\textbf{模(modulo)} $2^n$
\end{itemize}

\subsection{有符号整数}
\begin{itemize}
\item 取值范围为 $-2^{n-1}$ 到 $2^{n-1}-1$
\item overflow 的结果无定义
\end{itemize}

\subsection{2 的补}
有符号整数类型的负数在内存中的二进制表示常采用 \textbf{2 的补(2's complement)}.
\begin{itemize}
\item 若采用 2 的补, 一个整数和它的相反数相加等于 $2^n$
\item 若采用 2 的补, 范围内最大的整数加上 1 等于最小的整数, 加上 2 等于第二小的整数
\end{itemize}

\subsection{转换规则}
\begin{itemize}
\item 其他类型转换为 \verb|bool|: 0 变为 \verb|false|, 否则变为 \verb|true|
\item \verb|bool| 转换为其他类型: \verb|true| 变为 1, \verb|false| 变为 0
\item 浮点类型转换为整数类型: 向 0 取整
\item 整型转为浮点: 如果位数太多会不精确
\item 超出范围的值转换为无符号整型, 二进制的高位被截去, 即模 $2^n$ ($n$ 是目标类型的比特数), 例如 -1 变为 8bit 整数的 255
\item 超出范围的值转换为有符号整型, 结果无定义
\end{itemize}

\subsection{literal}
按照 c++11 标准
\begin{itemize}
\item 没有后缀: int, long int, long long int
\item 后缀 \verb|u| 或 \verb|U|:unsigned int, unsigned long int, unsigned long long int
\item 后缀 \verb|l| 或 \verb|L|: long int, long long int
\item 后缀 \verb|l/L| 以及 \verb|u/U|: unsigned long int, unsigned long long int
\item 后缀 \verb|ll| 或 \verb|LL|: long long int
\item 后缀 \verb|ll/LL| 以及 \verb|u/U|: unsigned long long int
\end{itemize}

总结:
\begin{itemize}
\item 整数 literal 最小是 int, 最大是 long long int
\item 有 \verb|u| 必定是 \verb|unsigned|
\item 没有 \verb|u| 的十进制必定是 signed
\item 没有 \verb|u| 的其他进制可以是 signed 或 unsigned (首选 signed)
\item \verb|l| 和 \verb|ll| 只规定最小类
\end{itemize}
