% 可缩空间
\pentry{同伦\upref{HomT1}}
\subsection{收缩核映射}

首先我们要引入三个依次增强的概念.第一个是\textbf{保核收缩映射},就是将拓扑空间映射到自身的子集(称为核)上,同时还保证核的元素都是映射的不动点;第二个是\textbf{形变收缩映射},它是一种保核收缩映射,但是多了一个条件,要求它还和恒等映射同伦;第三个是\textbf{强形变收缩映射},它是一种形变收缩映射,同样多了一个条件,要求同伦时时刻刻保持核中的元素是不动点.

下面,我们严格描述这三个概念.

\begin{definition}{保核收缩映射}
设拓扑空间$X$及其子集$A$.如果存在一个映射$f:X\rightarrow X$,使得$f(X)=A$,并且$\forall a\in A, f(a)=a$,那么称$f$是$X$的一个\textbf{保核收缩映射},同时称$A$是$f$的收缩核.保核收缩映射也可以简称\textbf{保核收缩}.
\end{definition}

\begin{definition}{形变收缩映射}
设拓扑空间$X$及其子集$A$.如果$f$是$X$的一个保核收缩,且$f$
\end{definition}
