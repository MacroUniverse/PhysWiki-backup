% 双生子佯谬
% 双生子悖论|孪生子悖论
\pentry{斜坐标系表示洛伦兹变换\upref{SROb}}

双生子佯谬,又称孪生子佯谬,是一个著名的相对论问题.“佯谬”一词的意思是“看起来像是错误但实际上不是”,其中“谬”指“错误”,而“佯”指“假的”.它曾经被认为是一个悖论,但在今天已经被完美解决了,所以成了一个佯谬.和\textbf{时间的变换与钟慢效应}\upref{SRtime}词条的结尾所指出的一样,我们将从双生子佯谬入手,尝试讨论非惯性系眼中的时空.

\subsection{问题描述}

假设地球在某个惯性系$K_1$中静止,忽略一切引力等作用,把地球考虑成一个质点.为方便理解,也可以说$K_1$是“地球系”.在地球上有一对完全同龄的双胞胎,其中弟弟始终留在地球上,而哥哥则乘坐飞船离开地球.称飞船的参考系$K_2$是“飞船系”,同样看成一个质点.一段时间以哥哥返回并降落在地球上.称飞船的参考系$K_2$是“飞船系”.当飞船降落后,兄弟俩的年龄是否有差异?差异是什么?

\subsection{双生子佯谬的解答}

我们现有的工具只有狭义相对论,而弟弟所在的地球系是一个惯性系,因此我们可以先从地球系开始讨论.假设兄弟俩各戴着一只完美的手表,走时绝无误差,并且双方都把飞船出发的一刻设为时间零点.这样,我们只需要比较哥哥返回地球时两人手表的读数即可.

问题的最简形式,就是哥哥以匀速直线运动离开地球,某一时刻瞬间反向,沿着原道路以相反速度回到地球.这样,我们甚至不需要关心飞船降落的过程,因为当飞船和地球重合的时候,它们的时空坐标完全相同,被认为是同一个事件,无论在哪个参考系看来都是同时发生的,不会出现同时性的相对性.这样,无论飞船有没有和地球保持静止,我们都可以比较哥哥和弟弟各自手上的手表读数.

事实上并不会存在这样瞬间反向的运动,这个模型有物理意义吗?答案是有的,我们只需要耍一个小小的花招.我们考虑两艘飞船,都在哥哥的航道上,一艘自地球出发做匀速直线运动,另一艘飞船自远方飞向地球,速度和第一艘相反.这样,两艘飞船都是惯性系,都可以用我们已有的狭义相对论方法去讨论它们眼中的时空.这两艘飞船的运动可以看成是,飞船$1$始终离开地球,飞船$2$先靠近地球,然后穿过地球,即两艘飞船一直匀速,只是中途重合了一瞬间;但是也可以看成,飞船$1$离开地球后,到达重合点突然变成飞船$2$,飞向地球;飞船$2$也在重合点处突然变成飞船$1$,飞离地球.为了把这个双飞船的模型等价转化为问题中单飞船的情况,我们只需要让两艘飞船对表,使得它们重合时手表读数相同,然后只观察“飞船$1$飞到重合点后变成飞船$2$再回到地球”这一分支就可以,将“飞船$2$飞到重合点后变成飞船$1$再飞向无穷远”这一过程视为\textbf{不存在}.

\begin{figure}[ht]
\centering
\includegraphics[width=14cm]{./figures/Twins_1.pdf}
\caption{在\textbf{地球系}中所看到的三个惯性系的网格划分.} \label{Twins_fig1}
\end{figure}

我们将地球、飞船$1$和飞船$2$的坐标网格分别表示如\autoref{Twins_fig1} ,三个网格都是在\textbf{地球系}中划分的.图中的点$P$表示“飞船$1$和飞船$2$相互转换”这一事件.显然,地球系自身的等时线都和$x$轴平行,等距线都和$t$轴平行,因此它的网格看起来是一个矩形划分,如图中左边的黑色网格所示;飞船$1$的坐标网格如中间的蓝色划分,其中橙色线代表空间坐标始终为$0$的点,即飞船$1$本身的轨迹,而灰色线代表忽略掉的等时线,因为它们出现在$P$点之后,是不存在的;飞船$2$的坐标网格如右边的红色划分,其中绿色线代表空间坐标始终为$0$的点,即飞船$2$本身的轨迹,而灰色线同样代表忽略掉的等时线,因为出现在$P$之前,不存在.

\autoref{Twins_fig1} 没有明显画出来的是,各等距线被灰色等时线所覆盖的部分也要忽略掉.

现在,把两个飞船的坐标网格放到一起,擦掉各自不存在



