% 映射
% 单射|满射|函数|一一对应|定义域|值域|逆映射

\pentry{集合\upref{Set}}

给定集合 $A$ 和 $B$,我们可以假想从 $A$ 中\textbf{每一个}元素上拉\textbf{一根}有方向的线连接到 $B$ 中的一个元素,这些线的连接方式就被称为一个从 $A$ 到 $B$ 的\textbf{映射(mapping)}, 也叫\textbf{算符(operator)}.将这个映射记为 $f$, $A$ 叫做 $f$ 的\textbf{定义域(domain)}, $B$ 叫做\textbf{到达域(codomain)}\footnote{也叫陪域、上域、目标集(target set)}, $B$ 中被线连接到的元素的集合叫做 $f$ 的\textbf{值域(range)}\footnote{值域在一些文献中指的是到达域.}或\textbf{像(image)}, 记为 $f(A)$.

我们一般将 “$f$ 是从 $A$ 到 $B$ 的映射” 记为
\begin{equation}
f:A\to B
\end{equation}
也就是说从 $A$ 的元素上拉线到 $B$ 的元素上. 有时候, 为了表示映射的定义域 $A$ 是另一个集合的 $S$ 的子集, 我们也会将映射记为
\begin{equation}
f: A\subseteq S \to B
\end{equation}
注意映射是有方向区分的,比如在上面的例子中, $A$ 中每个元素都有且只有一根线拉出去, 但是 $B$ 中的元素可以同时被一根或多根线连接, 也可以没有连接(即不在值域中). 有时候为了把值域 $R$ 和到达域 $B$ 同时写进映射, 也可以将其记为
\begin{equation}
f: A \to R \subseteq B
\end{equation}

\subsection{映射的类型}
\begin{definition}{}\label{map_def1}
如果映射 $f:A \to B$ 中每个 $B$ 中元素只被1根或者0根线连接,那么称$f$是一个\textbf{单射(injection)}. 如果$f:A\to B$中每个$B$中元素都被至少1根线连接, 那么称$f$是一个\textbf{满射(surjection)}. 如果 $f$ 既是单射又是满射,那么称它为一个\textbf{双射(bijection)}, 或者叫\textbf{一一对应(one-to-one correspondence)}.
\end{definition}

\begin{figure}[ht]
\centering
\includegraphics[width=8cm]{./figures/map_1.pdf}
\caption{映射的分类} \label{map_fig1}
\end{figure}

如果 $f:A\to B$ 是一个双射,那么 $A$ 中每一个元素都唯一地连接到 $B$ 中某一个元素,并且 $B$ 中每一个元素也都唯一被 $A$ 中某一个元素所连接,因此很明显可以将这个过程反过来,从 $B$ 中向 $A$ 中拉连接线.另外,如果 $A$ 和 $B$ 存在双射,意味着 $A$ 和 $B$ 的元素数量应该一致\footnote{本书中统一使用这种定义. 一些其他教材中也把我们的 “单射” 称为 “一一映射”, 把 “满射” 称为 “到上”, 把 “双射” 称为 “一一到上”, 需要特别小心.}.

函数是一种常见的映射, 例如 $f(x) = 2x$ 可以看作映射 $f: \mathbb R \to \mathbb R$. 但是映射可以从任意集合到任意集合. 例如将整数映射到正多边形, 将函数的映射到函数或实数(一般把这种映射称为\textbf{算符})等.

注意当一个集合中有无限个元素时, 我们有可能在它的子集和它本身之间建立一一映射, 例如函数 $\tan(x)$ 可以从实轴的开区间 $(-\pi/2, \pi/2)$ 一一映射到整个实轴 $\mathbb R$, 又例如我们可以将全体整数 $\mathbb Z$ 乘以二后一一映射到全体偶数 $2\mathbb Z$ 上. 这时我们仍然认为这两个集合的元素一样多, 虽然直觉上可能不容易接受.

\textbf{Cantor-Bernstein 定理}显示,如果集合$A$到集合$B$上存在一个单射$f$和一个满射$g$,那么总可以利用$f$和$g$来构造出一个双射.

\subsection{多元运算}\label{map_sub1}
有时候我们需要将两个集合 $A, B$ 中任意各取一个元素, 然后映射另一个集合 $C$ 中的元素, 称为\textbf{二元运算(binary operation)}. 我们可以使用笛卡尔积(\autoref{Set_eq1}~\upref{Set})将这个映射表示为
\begin{equation}\label{map_eq1}
A \times B \to C
\end{equation}
一个简单的例子就是两个实数的的加法减法或乘法可以表示为 $\mathbb R \times \mathbb R \to \mathbb R$ (或简记为 $\mathbb R^2 \to \mathbb R$), 但除法不可以, 因为除数的集合不是 $\mathbb R$ 而是 $\mathbb R$ 去掉 $0$.

同理, \textbf{多元运算}可以用多个卡氏积表示为
\begin{equation}
A_1 \times \dots \times A_N \to C
\end{equation}
例如含有 $N$ 个自变量的函数就是一个 $N$ 元运算. 特殊地, $N$ 个相同集合 $A$ 做卡氏积可以简单表示为 $A^N$, 例如 $N$ 个有序复数的集合为 $\mathbb C^N$.

\subsection{相等和拓展}
当映射(算符) $f:A\to B$ 和 $g:C\to D$ 的定义域相等($A = C$)且对任意 $x\in A$ 都有 $fx = gx$, 那么我们就说两个映射(算符)\textbf{相等}, 记为 $f = g$, 否则它们就不相等.

若 $A$ 是 $C$ 的子集($A\subseteq C$), 我们就说 $g$ 是 $f$ 的\textbf{拓展(extension)}, 记为 $f \subseteq g$. 特殊地, 当 $A$ 是 $C$ 的真子集($A\subset C$), 就记为 $f \subset g$.

\subsection{复合映射}
给定两个映射 $f:A\to B$ 和 $g:C\to D$, 如果 $f$ 的值域 $R$ 是 $g$ 的定义域 $C$ 的一个子集, 则可以定义\textbf{符合映射} $g\circ f: A\to D$, 即先将 $A$ 中的元素通过 $f$ 映射到 $R \subseteq C$, 再通过 $g$ 映射到 $D$ 的元素. 在没有歧义的情况下也可以将 $\circ$ 省略, 尤其是将映射称为\textbf{算符}时.

复合映射常见的例子是复合函数, 令 $\mathbb R$ 上的函数 $f(x) = \sin x$, $g(x) = x^2$, 则复合函数 $g\circ f: \mathbb R \to [0, 1]$ 为 $(g\circ f)(x) = g(f(x)) = \sin^2 x$.

根据定义, 复合映射满足分配律, 令 $f, g, h$ 为映射, 则
\begin{equation}
h \circ (g \circ f) = (h \circ g) \circ f
\end{equation}

\subsection{恒等映射}
若一个集合到它子集的映射 $f:X\to X$ 把任意 $x\in X$ 映射到 $x$ 本身, 我们就叫它\textbf{恒等映射(identity map)}或者\textbf{单位算符(unit operator)}, 通常用 $I$ 或 $E$ 表示. 注意对不同集合 $X$, 它们的单位算符定义域并不相等, 所以它们的单位算符也不相等.



\subsection{逆映射}

给定集合$A, B$,定义$B^A$为“从$A$到$B$的所有可能的映射所构成的集合”.如果$B$是一个二元集合,即它只有两个元素,不妨记为$B=\{0,1\}$,那么$B^A$可以用来表示$A$的\textbf{幂集},即由$A$的所有子集所构成的集合.这是因为对于任意的$f\in B^A$,我们可以把这个$f$对应到$A$的子集$S$,其中$S$的元素全都被$f$映射到1上,$A-S$的元素全都被$f$映射到0上.当然,0和1的地位反过来也可以.由于这个特点,我们简单地把$A$的幂集记为$2^A$\footnote{当$A$,$B$都是有限集的时候,$|B^A|=|B|^{|A|}$.特别地,$|2^A|=2^{|A|}$.}. 

对于映射$f:A\to B$,可以定义其逆映射$f^{-1}:2^B \to 2^A$.对于$B$的任意子集$C$,有:

\begin{equation}
f^{-1}(C) = \qty{x | x \in A, f(x) = C}
\end{equation}

此时,$f^{-1}(C)$称为$C$在$f$下的\textbf{逆像(inverse image)}或\textbf{原像(preimage)}.

特别地,和$f$的值域中不相交的$C$被$f^{-1}$映射到空集上,而空集也是$A$的一个子集.如果$f$是一个双射,那么对于任意$y\in B$,单元素子集$\{y\}$都被$f^{-1}$映射在$A$的某个单元素子集上,那么我们也可以认为此时$f^{-1}$实际上是单个元素映射在单个元素上,也就是从$B$到$A$的映射.

对于任意单射 $f:A\to B$, 令其值域为 $R \subseteq B$, 那么 $f:A\to R$ 就是一一映射, 所以我们总可以唯一地定义其逆映射 $f^{-1}:R \subseteq B \to A$.

对于双射 $f:A\to B$, 显然 $f\circ f^{-1}$ 和 $f^{-1}\circ f$ 都是单位算符(恒等映射).注意两者的定义域分别为 $A$ 和 $B$, 当 $A \ne B$ 时不能写成 $f\circ f^{-1} = f^{-1}\circ f$.如果把$A$,$B$到自身的恒等映射分别记为$I_A$和$I_B$,那么$f\circ f^{-1}=I_B$,$f^{-1}\circ f=I_A$.

\begin{example}{}
如果取正弦函数 $y = \sin x$ 的值域为 $R = [-1, 1]$ 如果取定义域为 $\mathbb R$,  那么它不是一个单射, 因为每一个 $y \in R$ 都对应无穷个 $x$, 所以不存在反函数. 但如果取定义域为 $[-\pi/2, \pi/2]$, 那么它是一个单射, 存在反三角函数 $\sin^{-1}: [-1, 1] \to [-\pi/2, \pi]$.

根据以上定义, $\sin^{-1} (\sin(x))$ 是定义在 $[-\pi/2, \pi/2]$ 上的恒等函数, 而 $\sin (\sin^{-1}(x))$ 是定义在 $[-1, 1]$ 上的恒等函数, 所以有 $\sin \circ \sin^{-1} \subseteq \sin^{-1} \circ \sin$.
\end{example}
