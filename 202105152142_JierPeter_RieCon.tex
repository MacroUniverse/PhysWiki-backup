% 黎曼联络
% 仿射联络|Riemannian connection|联络|挠率|度量|黎曼度量|Riemannian metric|流形|manifold

\pentry{仿射联络\upref{affcon},线性代数}

\subsection{黎曼流形}

\textbf{黎曼度量与伪黎曼度量}\upref{RiMetr}词条中包含了黎曼度量的概念,但其中使用了高度凝练的术语,对初学者并不友好,因此我们在这里顺便给出更通俗的定义.

\begin{definition}{黎曼度量}
给定一个实流形$M$,定义其上一个映射$g$,它将$M$上任意点处的两个切向量映射为一个数字.如果$g$满足对于任意向量$x, y$,都有:
\begin{enumerate}
\item \textbf{对称性}:$g(x, y)=g(y, x)$;
\item \textbf{正定性}:$g(x, y)\geq 0$,且仅在$x=0$时有$g(x, x)=0$.
\end{enumerate}
则称$g$是$M$上的一个\textbf{黎曼度量(Riemannian metric)}.
\end{definition}

从定义可知,黎曼度量实际上就是规定了“如何做内积”,进而得到“切向量的长度”、“切向量之间的角度”等概念.注意,黎曼度量只对同一个切空间中的向量有用,不同切空间的向量之间无法定义黎曼度量.

如果在某个切点附近给定一个图(chart),那么黎曼度量可以表达为这个图中的一个矩阵$g_{ab}$,而该切点处两个切向量$x^a, y^b$的内积就是$g_{ab}x^ay^b$.

\textbf{对称性}意味着$g_{ab}\equiv g_{ba}$.如果将$g_{ab}$写成方阵\footnote{注意,按照我们的规范表达,$g_{ab}$应为“行矩阵的行矩阵”,所以这里说的方阵实际上是指把第$a$行$b$列的元素定义为$g_{ab}$.当然,反过来把第$b$行$a$列的元素定义为$g_{ab}$也行.},那么它就是线性代数中讨论的\textbf{正定矩阵}.这就是\textbf{黎曼度量与伪黎曼度量}\upref{RiMetr}词条中“对称”与“正定”二词的含义.“截面”实际上就是指这是一个映射.


\begin{definition}{黎曼流形}
给定实流形$M$.若在$M$上处处定义一个黎曼度量$g$,且在任意图中$g$的坐标都是光滑函数,则称$(M, g)$为一个\textbf{黎曼流形(Riemannian manifold)}.
\end{definition}





\subsection{黎曼联络}

\begin{definition}{黎曼联络}
给定黎曼流形$(M, g)$,如果$M$上有一个仿射联络$\nabla$,满足
\end{definition}





