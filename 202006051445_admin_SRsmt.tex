% 事件与尺缩效应

\pentry{狭义相对论的基本假设\upref{SpeRel}}

\subsection{事件}

在和时空相关的理论中,当我们描述一件事的时候,我们并不关心这件事具体是什么,只关心它发生在何时何地.因此为了将来的讨论,我们首先需要定义“事件”的概念.

一个\textbf{事件(event)}是指在时空坐标系中的一个点.事件所发生的时间、地点,就是事件作为一个点的坐标.

\subsection{对事件的观测}

狭义相对论的核心是光.在任何参考系中,光速不变.光的其它性质并不能保证一定不变,如光强的分布,偏振的角度等.

除了光速以外,事件本身也是不随惯性系变化的.这就是说,在任何惯性系$K_1$中同时同地发生的事情,在任何惯性系$K_2$中也是同时同地发生的.鉴于我们已经为了简便,将事件简单表达为它所发生的时间和地点,那么同时同地发生的事件都应看成同一个事件.

光速和事件的不变性,是目前我们观测事件的最基本工具.

\subsection{同时性的相对性}

考虑一根的铁轨,向左向右都无限延伸.在这铁轨上取一个点作为原点,向右作为正方向,可以画一个$x_1$轴,用来测量和铁轨静止的参考系中的事件位置,这个参考系称作\textbf{铁轨系}记为$K_1$.

现在,铁轨上从左到右开过一辆火车.和火车静止的参考系也可以沿着铁轨画一个$x_2$轴,只不过它是用来描述火车参考系中事件位置的,称作\textbf{火车系},记为$K_2$.

在铁轨系中,如果某时刻看到火车的两个不重叠的轮子同时发光,那么这两道光会在铁轨上的两个发光点的中点相遇,而“相遇”也是一个事件.从发光到相遇,两束光通过了相同的路程,由于光速不变,它们经过了相同的时间,由此反推可知发光的时间是一样的.但是同样的三个事件在火车系看来是不一样的:在火车系中,“相遇”发生在更靠近后轮的位置,也就是说,在火车看来前轮所发的光走过了更长的路程,花了更长的时间,从“相遇”的时间反推回去,可知在火车眼里是前轮先发光.

\begin{figure}[ht]
\centering
\includegraphics[width=8cm]{./figures/SRsmt_1.png}
\caption{在铁路系中所看到的三个事件,分别用三个点表示.上图是两个轮子同时发光的两个事件;下图是一段时间以后,火车运动了一段距离,而两束光相遇的事件.} \label{SRsmt_fig1}
\end{figure}

事实上,在$K_1$中同时但不同地发生的事情,在$K_2$中必然不同时发生.“同时”这一概念并非绝对,两个事件是否同时,取决于从什么参考系来观察它们.

\begin{exercise}{火车系中的事件}

\autoref{SRsmt_fig1}中是以铁轨的视角,选取了两个时刻,描述了“前轮发光”、“后轮发光”和“光束相遇”这三个事件.请你尝试画出火车的视角下三个事件的先后关系.提示:你需要三个关键时刻,依次是“前轮发光”时,“后轮发光”时和“光束相遇”时.

\end{exercise}

\subsection{尺缩效应\footnote{本节所用的符号和技巧均为笔者高中第一次推导狭义相对论时所用的.}}

我们还是使用上一节定义的火车系$K_2$和铁轨系$K_1$.如果说,在铁轨上标记了两个点$A$和$B$,使得前轮通过$B$时发光,后轮通过$A$时发光.在$K_1$中,前轮和后轮分别同时通过这两个点,也就是说,在$K_1$中,火车两轮的间距和$A$、$B$的间距一样;但是在$K_2$中来看,前轮先发光,后轮后发光,这就意味着火车两轮的间距比$A$、$B$的间距要长.

这说明,运动的物体应该比静止时看起来要短.由于没有任何点是特殊的,所以这种运动造成的收缩在每一个地方都是一样的,或者说,运动造成的尺缩效应是均匀的.那么运动造成的收缩的比例应该怎么计算呢?

\begin{figure}[ht]
\centering
\includegraphics[width=15cm]{./figures/SRsmt_2.png}
\caption{尺缩效应配图} \label{SRsmt_fig2}
\end{figure}

把$A$、$B$的间距看成$AB$的长度,车轮间距看成火车的长度.设$AB$的静止长度(在$K_1$中的长度)为$2S$,而火车的静止长度(在$K_2$中的长度)为$2L$.

记火车相对铁轨的运动速度为(沿着$x$正方向)$v$,考虑到两个参考系中没有哪个更特殊,则铁轨相对火车的运动速度为$-v$;同样,火车在铁轨系中的收缩比例,也应该和铁轨在火车系中的收缩比例相等.记这个收缩比例是$m\in(0,1)$,那么“在\textbf{铁轨系}中火车和$AB$长度一样”意味着:

\begin{equation}\label{SRsmt_eq1}
2mL=2S
\end{equation}

在\textbf{火车系}中,火车的长度是$2L$,大于$AB$的长度$2mS$,所以前轮先碾过$B$点发光,然后才轮到后轮碾过$A$点发光.前后轮发光各自是一个独立的事件,所以它们是否同时取决于参考系的选择;但是有一个东西在两个参考系中都是一样的,那就是两束光相遇的位置,因为两束光的相遇是一个单独的事件.

我们来考察一下,在两个参考系中,光相遇的位置对应于车上的哪个地方.如下图所示,在$K_1$中,设相遇点到火车后轮的距离是$a_1$,到火车前轮的距离是$b_1$;在$K_2$中,设相遇点到火车后轮的距离是$a_2$,到火车前轮的距离是$b_2$.由于收缩是均匀的,相遇点在两个参考系中都是同一个点,因此

\begin{equation}\label{SRsmt_eq5}
\frac{a_1}{b_1}=\frac{a_2}{b_2}
\end{equation}

根据两个场景的不同,具体计算一下$a_1$,$b_1$,$a_2$,$b_2$,得到\footnote{第四个等式两端都是$K_2$中两个轮子发光的时间间隔}:

\begin{equation}\label{SRsmt_eq2}
a_1=S-v\cdot\frac{S}{c},b_1=S+v\cdot\frac{S}{c},a_2+b_2=2L,\frac{b_2-a_2}{c}=\frac{2L-2mS}{v}
\end{equation}

整理\autoref{SRsmt_eq2}并代入\autoref{SRsmt_eq1}得:

\begin{equation}\label{SRsmt_eq3}
\frac{a_1}{b_1}=\frac{c-v}{c+v}
\end{equation}

\begin{equation}\label{SRsmt_eq4}
\frac{a_2}{b_2}=\frac{v-c(1-m^2)}{v+c(1-m^2)}
\end{equation}

把\autoref{SRsmt_eq5}代入\autoref{SRsmt_eq3}和\autoref{SRsmt_eq4}中得:

\begin{equation}
\frac{c-v}{c+v}=\frac{v-c(1-m^2)}{v+c(1-m^2)}
\end{equation}

在等式右边上下同乘以$c/v$得:

\begin{equation}
\frac{c-v}{c+v}=\frac{c-\frac{c^2}{v}(1-m^2)}{c+\frac{c^2}{v}(1-m^2)}
\end{equation}

这样就可以直接得到:

\begin{equation}
v=\frac{c^2}{v}(1-m^2)
\end{equation}

解得

\begin{equation}
m=\sqrt{1-\frac{v^2}{c^2}}
\end{equation}

结论就是,如果一个物体静止时的长度为$L$,那么在某一惯性系中若它沿着自身长度的方向运动速度为$v$,则在此参考系中它的长度是$\sqrt{1-\frac{v^2}{c^2}}\cdot L<L$.






