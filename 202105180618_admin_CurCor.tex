% 正交曲线坐标系
% 多元微积分|坐标系|柱坐标系|球坐标系|矢量|内积|内积|导数|偏导数|曲线坐标系|正交曲线坐标系

\begin{issues}
\issueTODO
\end{issues}

\pentry{球坐标系\upref{Sph}, 柱坐标系\upref{Cylin}}

\footnote{本文参考 Wikipedia \href{https://en.wikipedia.org/wiki/Curvilinear_coordinates}{相关页面}.}如果 $u, v, w$ 是三维空间中某曲线坐标系的三个坐标, 空间任意一点的位置矢量\upref{Disp} $\bvec r$ 都是它们的函数 $\bvec r(u, v, w)$. 那么定义任意一点处三个单位矢量为
\begin{equation}\label{CurCor_eq8}
\uvec u = \frac{\pdv*{\bvec r}{u}}{\abs{\pdv*{\bvec r}{u}}}\qquad
\uvec v = \frac{\pdv*{\bvec r}{v}}{\abs{\pdv*{\bvec r}{v}}}\qquad
\uvec w = \frac{\pdv*{\bvec r}{w}}{\abs{\pdv*{\bvec r}{w}}}
\end{equation}
注意一般来说, 这三个矢量会随着 $\bvec r$ 改变. 形象地说: 当我们分别只把 $u, v, w$ 增加一点时, $\bvec r$ 会分别沿 $\uvec u, \uvec v, \uvec w$ 方向移动(请以球坐标系和柱坐标系为例思考).

若对空间中任意一点, \autoref{CurCor_eq8} 中的三个矢量都两两正交, 那么这个曲线坐标系就是\textbf{正交曲线坐标系(orthogonal curvilinear coordinate system)}. 常见的例子除了球坐标系\upref{Sph}, 柱坐标系\upref{Cylin} 还有抛物线坐标系\upref{ParaCr}.
\begin{exercise}{}
练习:试着用\autoref{CurCor_eq8} 计算球坐标和柱坐标中的单位矢量, 例如球坐标中
\begin{equation}
\bvec r = r\sin\theta\cos\phi\,\uvec x + r\sin\theta\sin\phi\,\uvec y + r\cos\theta\,\uvec z
\end{equation}
然后用点乘\upref{Dot}证明它们的单位矢量总是两两垂直. 即球坐标系和柱坐标系都是正交曲线坐标系.
\end{exercise}
