% 科里奥利力
% 惯性系|惯性力|非惯性系|旋转参考系|离心力|科里奥利力

\pentry{离心力\upref{Centri},平面旋转矩阵\upref{Rot2D}}%未完成

\textbf{科里奥利力(Coriolis Force)}是旋转的参考系中由质点运动产生的惯性力.
\begin{equation}
\bvec F_c = 2m \bvec v_{S'} \cross \bvec \omega
\end{equation}
其中 $m$ 是质点的质量, $\bvec v_{S'}$ 是质点相对于旋转参考系 $S'$ 的瞬时速度, $\bvec\omega$ 是旋转系相对于某惯性系 $S$ 转动的角速度矢量.%未完成: 考虑使用脚注或链接
式中的乘法是叉乘\upref{Cross}.
在匀速转动参考系(属于非惯性系)中,若质点保持相对静止,则惯性力只有离心力.然而当质点与转动参考系有相对速度时,惯性力中还会增加一项与速度垂直的力,这就是科里奥利力.地理中的地转偏向力就是科里奥利力,可用上式计算(见“地球表面的科里奥利力\upref{ErthCf}”).

\subsection{推导(矢量法)}
\pentry{连续叉乘的化简\upref{TriCro}, 速度的参考系变换\upref{Vtrans}}

我们先令质点的位矢关于时间变化的函数为 $\bvec r(t)$, 某时刻相对于 $S$ 和 $S'$ 系的速度分别为 $\bvec v_{S}$ 和 $\bvec v_{S'}$, 根据\autoref{Vtrans_ex1}\upref{Vtrans} 中的结论, 任意时刻都有
\begin{equation}\label{Corio_eq5}
\bvec v_{S} = \bvec v_{S'} + \bvec\omega\cross\bvec r
\end{equation}
两边在 $S$ 系中对时间求导得(我们先假设 $\bvec \omega$ 是常矢量\footnote{如果角速度 $\bvec \omega$ 随时间变化, 那么惯性力中会多出一项, 但这项并不属于科里奥利力, 所以科里奥利力的表达式并不要求角速度不变.})
\begin{equation}\label{Corio_eq6}
\bvec a_{S} = \qty(\dv{\bvec v_{S'}}{t})_{S} + \bvec\omega\cross\bvec v_{S}
\end{equation}
注意 $S'$ 系中的加速度 $\bvec a_{S'}$ 并不是上式右边第一项, 而是 $(\dv*{\bvec v_{S'}}{t})_{S'}$. 令\autoref{Corio_eq4} 中的 $\bvec A = \bvec v_{S'}$, 得
\begin{equation}\label{Corio_eq7}
\qty(\dv{\bvec v_{S'}}{t})_{S} = \bvec a_{S'} + \bvec\omega\cross\bvec v_{S'}
\end{equation}
将\autoref{Corio_eq5} 和\autoref{Corio_eq7} 代入\autoref{Corio_eq6}, 得
\begin{equation}
\bvec a_{S} = \bvec a_{S'} + 2\bvec\omega\cross\bvec v_{S'} + \bvec\omega\cross(\bvec\omega\cross\bvec r)
\end{equation}
所以旋转参考系中的总惯性力(\autoref{Iner_eq1}\upref{Iner})为
\begin{equation}
\bvec f = m(\bvec a_{S'} - \bvec a_{S}) = -2m\bvec\omega\cross\bvec v_{S'} - m\bvec\omega\cross(\bvec\omega\cross\bvec r)
\end{equation}
其中第二项为离心力(\autoref{Centri_eq5}\upref{Centri}), 而第一项被称为科里奥利力.

% 未完成:需要引用矩阵相乘的求导法则
\subsection{推导(旋转矩阵法)}

