% shenry
1. 轴对称,即当横坐标到对称轴的距离相等时,函数值也相等
\begin{equation}
f\left( {a + x} \right) = f\left( {a - x} \right)
\end{equation}
2. 中心对称,即当横坐标到对称轴的距离相等时,函数值互为相反数
\begin{equation}
f\left( {a + x} \right) =  - f\left( {a - x} \right)
\end{equation}
3. 周期性,即横坐标经过一定长度后的函数值相等,
\begin{equation}
f\left( {a + x} \right) = f\left( x \right)
\end{equation}
其中,$a$分别为对称轴、对称中心和周期$T$,现在来看同时含有参数$a$和参数$b$的情况
\begin{equation}
f\left( {a + x} \right) = f\left( {b - x} \right)
\end{equation}
\begin{equation}
f\left( {a + x} \right) =  - f\left( {b + x} \right)
\end{equation}
\begin{equation}
f\left( {a + x} \right) = f\left( {b + x} \right)
\end{equation}
其对称轴、对称中心、周期分别为:$x = \frac{{a + b}}{2}$、$\left( {\frac{{a + b}}{2},0} \right)$和$T = \left| {b - a} \right|$