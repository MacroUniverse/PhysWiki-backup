% 标量场

\pentry{洛伦兹群\upref{qed1}欧拉-拉格朗日方程\upref{qed1}}
\begin{definition}{}
标量场即洛伦兹变换下不变的场.设在两个惯性系中,标量场分别为$\phi(x)$和$\phi'(x')$,那么有
\begin{equation}
\phi^{\prime}\left(x^{\prime}\right)=\phi(x)
\end{equation}
图景是这样的,对于一个固定点P而言,洛伦兹变换前后对应的坐标不同,场函数形式不同,结果是P的数值不变.
\end{definition}
\begin{corollary}{}
做一个无穷小变换
\begin{equation}
x^{\rho} \rightarrow x^{\prime \rho}=x^{\rho}+\delta x^{\rho}
\end{equation}
并且
\begin{equation}\label{qed2_eq1}
\delta x^{\rho}=\omega_{\sigma}^{\rho} x^{\sigma}=-\frac{i}{2} \omega_{\mu \nu}\left(J^{\mu \nu}\right)_{\sigma}^{\rho} x^{\sigma}
\end{equation}
其中,$\left(J^{\mu \nu}\right)_{\sigma}^{\rho}=i\left(\eta^{\mu \rho} \delta_{\sigma}^{\nu}-\eta^{\nu \rho} \delta_{\sigma}^{\mu}\right)$
显然,令$J^{\mu \nu}=0$,该标量表示(0,0,)能满足标量场的定义.这是个平庸的结果,在场表示里,我们能做到更多.
对于固定点x,我们做场的无穷小变换
\begin{equation}
\delta_{0} \phi \equiv \phi^{\prime}(x)-\phi(x)
\end{equation}
为了找到该表示下的生成元,我们做一阶泰勒展开
\begin{equation}
\delta_{0} \phi=\phi^{\prime}\left(x^{\prime}-\delta x\right)-\phi(x)=-\delta x^{\rho} \partial_{\rho} \phi(x)
\end{equation}
代入\autoref{qed2_eq1} ,我们有
\begin{equation}
\delta_{0} \phi=\frac{i}{2} \omega_{\mu \nu}\left(J^{\mu \nu}\right)_{\sigma}^{\rho} x^{\sigma} \partial_{\rho} \phi \equiv-\frac{i}{2} \omega_{\mu \nu} L^{\mu \nu} \phi
\end{equation}
在这里,我们定义
\begin{equation}
L^{\mu \nu}=-\left(J^{\mu \nu}\right)_{\sigma}^{\rho} x^{\sigma} \partial_{\rho}=i\left(x^{\mu} \partial^{\nu}-x^{\nu} \partial^{\mu}\right)
\end{equation}
我们可以验证,$L^{\mu \nu}$满足洛伦兹群的李代数,从而确实是洛伦兹变换的生成元.所以在这里我们给出了无限维的洛伦兹群的标量场表示.(由于场函数随着坐标不同而不同,所以是无限维的)
实际上,如果定义$p^{\mu}=+i \partial^{\mu}$,会有$L^{\mu \nu}=x^{\mu} p^{\nu}-x^{\nu} p^{\mu}$,若上标取空间坐标i,j,我们会发现这就是轨道角动量.由于$J^{\mu \nu}=L^{\mu \nu}+S^{\mu \nu}$,所以标量场表示是自旋为0的.
\end{corollary}
\subsection{举例}
克莱因-戈登场即为标量场.克莱因-戈登方程为
\begin{equation}
\frac{1}{c^{2}} \frac{\partial^{2}}{\partial t^{2}} \psi-\nabla^{2} \psi+\frac{m^{2} c^{2}}{\hbar^{2}} \psi=0
\end{equation}
取自然单位,$c=\bar{h}=1$,得到协变形式
\begin{equation}
\left(\square^{2}+m^{2}\right) \psi=0
\end{equation}
达朗贝尔算符$\square^{2}=\frac{1}{c^{2}} \frac{\partial^{2}}{\partial t^{2}}-\nabla^{2}$
在经典场论,它的拉格朗日量为$L=\frac{1}{2} \partial_{\mu} \phi \partial^{\mu} \phi-\frac{1}{2} m^{2} \phi^{2}$,由欧拉-拉格朗日方程可解得克莱因-戈尔登方程方程.根据能量本征值解,我们知道这是一个相对论性的自旋为0的标量场.
\begin{corollary}{}
以上我们得到了经典场论下的标量场,接着我们将标量场进行正则量子化.\\
对于量子力学,我们有坐标-动量基本对易关系
$\left[\hat{q^{i}}, \hat{p^{j}}\right]=i \delta^{i j}$
若在海森堡绘景,该对易子则是等时性的.
同样的,在量子场论里,我们也可以有类似的操作.场$\hat{\phi}(t, \mathbf{x})$对应于坐标算符$\hat{q^{i}}$,场的共轭动量定义为\begin{equation}
\Pi_{\phi}=\frac{\partial \mathcal{L}}{\partial\left(\partial_{0} \phi\right)}=\partial_{0} \phi
\end{equation},其中$\mathcal{L}$为拉氏量密度,\textbf{Action} $S=\int d t L=\int d^{4} x \mathcal{L}\left(\phi, \partial_{\mu} \phi\right)$\\
$\hat{\Pi}(t, \mathbf{x})$对应于动量算符$\hat{q^{i}}$,则有
\begin{equation}
[\hat{\phi}(t, \mathbf{x}), \hat{\Pi}(t, \mathbf{y})]=i \delta^{(3)}(\mathbf{x}-\mathbf{y})
\end{equation}
等时下
\begin{equation}
[\hat{\phi}(t, \mathbf{x}), \hat{\phi}(t, \mathbf{y})]=[\hat{\Pi}(t, \mathbf{x}), \hat{\Pi}(t, \mathbf{y})]=0
\end{equation}
实标量场是厄米算符.由此我们可得到克莱因-戈登方程的算符解,即:
\begin{equation}
\hat{\phi}(x)=\int \frac{d^{3} p}{(2 \pi)^{3} \sqrt{2 E_{\mathbf{p}}}}\left(\hat{a}_{\mathbf{p}} e^{-i p x}+\hat{a}_{\mathbf{p}}^{\dagger} e^{i p x}\right)
\end{equation}
可以验证算符的厄米性.\\
其中$ipx=ip_\mu x^\mu=iEt-i\vec{p}\vec{x}$,可以证明,场算符的对易关系等价于
\begin{equation}
\left[\hat{a}_{\mathbf{p}}, \hat{a}_{\mathbf{q}}^{\dagger}\right]=(2 \pi)^{3} \delta^{(3)}(\mathbf{p}-\mathbf{q})
\end{equation}
\begin{equation}
\left[\hat{a}_{\mathbf{p}}, a_{\mathbf{q}}\right]=0, \quad\left[\hat{a}_{\mathbf{p}}^{\dagger}, a_{\mathbf{q}}^{\dagger}\right]=0
\end{equation}
\end{corollary}
\subsection{能量}
我们需要知道量子化标量场后的哈密顿密度形式.由于
\begin{equation}
\mathcal{H}=\Pi_{\phi} \partial_{0} \phi-\mathcal{L}=\frac{1}{2}\left[\Pi_{\phi}^{2}+(\nabla \phi)^{2}+m^{2} \phi^{2}\right]
\end{equation}
\begin{equation}
H=\int d^{3} x \mathcal{H}\int \frac{d^{3} p}{(2 \pi)^{3}} E_{p} \frac{1}{2}\left(\hat{a}_{\mathbf{p}}^{\dagger} \hat{a}_{\mathbf{p}}+\hat{a}_{\mathbf{p}} \hat{a}_{\mathbf{p}}^{\dagger}\right)=\int \frac{d^{3} p}{(2 \pi)^{3}} E_{p}\left(\hat{a}_{\mathbf{p}}^{\dagger} \hat{a}_{\mathbf{p}}+\frac{1}{2}\left[\hat{a}_{\mathbf{p}}, \hat{a}_{\mathbf{p}}^{\dagger}\right]\right)
\end{equation}
\subsection{因果性}
\subsection{传播子}