% kkk

### 题解报告$[$$QkOI$#$R1$$]$ $Quark and Equations$

题意:给出两个数$n$,$m$,求有多少对$(i,j)$,使得
$$\begin{cases}
i+j=n\\
\lfloor \frac{i}{j} \rfloor+\lceil \frac{j}{i} \rceil=m\\
\end{cases}
$$

分析:

首先,我们很容易的想到暴力,复杂度$O(tn)$

你会愉快的拿到一半的分数.

看一看数据范围,恩,$TLE$没错了...

然后,我们看到:$Sbstack3,4,5$都只有$5pts$,而且每个都有限制条件,于是我们运用套路的思维,知道这三个$Substack$,一定是用特判可以拿到的!

于是,运用我们的暴力代码,发现:

$1.m=1$时$ans=0$

$2.m>n$时$ans$还是$0$

$3.m∈[n,n-1]$时$ans=1$($m$不等于$1$)

于是,特判完成

你会拿到$65pts$~~

最后,来一波数学推导,正解来了!

我们再来看看:
$$\begin{cases}
i+j=n\\
\lfloor \frac{i}{j} \rfloor+\lceil \frac{j}{i} \rceil=m\\
\end{cases}
$$
我们发现,要么$i < j$要么$i < j$要么$i=j$(废话)

但是把这个式子代入到第二个式子中,我们发现:

若$i < j$,则$\lfloor \frac{i}{j} \rfloor=0$

若$i = j$,则$\lfloor \frac{i}{j} \rfloor+\lceil \frac{j}{i} \rceil=1+1=2$

若$i > j$,则$\lceil \frac{j}{i} \rceil=1$

所以,要么$m=2$且$n$%$2==0$,第二个式子里总有一个玩意是$0$或$1$

于是,分类讨论咯:

### $1.i<j$

此时$\lfloor \frac{i}{j} \rfloor=0$

所以就要保证$\lceil \frac{j}{i} \rceil=m$

我们又根据上取整的性质,得到

$$m-1<\frac{j}{i} \leq m$$

同时乘$i$:

$$im-i<j \leq im$$

根据第一个式子,又可以得到$j=n-i$,所以

$$im-i<n-i \leq im$$
$$im<n \leq i(m+1)$$

拆成两个不等式:
$$\begin{cases}
im<n\\
n \leq i(m+1)\\
\end{cases}
$$
所以
$$\begin{cases}
i<\frac{n}{m}\\
i \geq \frac{n}{m+1}\\
\end{cases}
$$
于是
$$\frac{n}{m+1} \leq i<\frac{n}{m}$$

### $2.i>j$

同上面的推导,$\lceil \frac{j}{i} \rceil=1$,所以$\lfloor \frac{i}{j} \rfloor=m-1$

此处省略推理过程,有兴趣的读者可以自行尝试.

结论:$\frac{(m-1)n}{m} \leq i < \frac{mn}{m+1}$

于是,综合一下:
$$\begin{cases}
\frac{n}{m+1} \leq i<\frac{n}{m}\\
\frac{(m-1)n}{m} \leq i < \frac{mn}{m+1}\\
\end{cases}
$$

我们就可以用这玩意求有多少个$i$,就是答案啦!

$code$:
```
#include<bits/stdc++.h>
using namespace std;
#define ll long long
#define f(i,a,b) for(int i=a;i<=b;i++)
inline ll read(){
   int s=0,w=1;
   char ch=getchar();
   while(ch<'0'||ch>'9'){if(ch=='-')w=-1;ch=getchar();}
   while(ch>='0'&&ch<='9') s=s*10+ch-'0',ch=getchar();
   return s*w;
}
#define d read()
ll t;
ll n,m;
ll a,b,c,dd;
int main(){
    t=d;
    while(t--){
        n=d,m=d;
        if(m!=1&&n!=1&&(m==n||m==n-1)){puts("1");continue;}
        if(m>n||m==1||n==1){puts("0");continue;}
        if(n%m==0)a=n/m-1,dd=(m-1)*n/m;else a=n/m,dd=(m-1)*n/m+1;
        if(n%(m+1)==0)b=n/(m+1),c=n*m/(m+1)-1;else b=n/(m+1)+1,c=n*m/(m+1);
        printf("%lld\n",a-b+1+c-dd+1-max((ll)0,min(b,dd)-max(a,c)+1));
    }
    return 0;
}

```
