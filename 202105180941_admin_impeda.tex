% 阻抗

\begin{issues}
\issueDraft
\end{issues}

\pentry{振动的指数形式\upref{VbExp}, 交流电, 交流电的复数形式, 电阻 欧姆定律\upref{Resist}} % 链接未完成

\footnote{参考 Wikipedia \href{https://en.wikipedia.org/wiki/Electrical_impedance}{相关页面}.}\textbf{阻抗(electrical impedance)} 的定义和电阻\upref{Resist} 类似, 都使用电压 $V$ 除以电流 $I$, 不同的是, 这 $V, I$ 都是复数表示的交流电, 即
\begin{equation}
V = \abs{U} \E^{\I \phi_V - \I \omega t}
\qquad
I = \abs{U} \E^{\I \phi_I - \I \omega t}
\end{equation}
那么阻抗的定义为(通常用大写 $Z$ 表示)
\begin{equation}
Z = \frac{V}{I} = \frac{\abs{V}}{\abs{I}} \E^{\I (\phi_V - \phi_I)}
\end{equation}
这里的除法是复数相除\upref{CplxNo}, 几何意义上就是把两个复数的模长相除, 幅角相减.

那么欧姆定律(\autoref{Resist_eq5}~\upref{Resist})的复数拓展就是
\begin{equation}
V = IZ
\end{equation}

\begin{example}{电阻}
根据定义, 普通电阻的阻抗就是其电阻值 $R$. 这是因为在交流电作用下, 电阻两端的电压和电流相位相同.
\end{example}

\begin{example}{电容}
电容的阻抗是纯虚数, 相位差相差 $\pi/2$
\addTODO{正还是负? 推导?}
\end{example}

