% 线性映射

\begin{issues}
\issueDraft
\issueOther{本词条需要重新创作和整合,融入章节逻辑体系.}
\end{issues}



\pentry{对偶空间\upref{DualSp}}

对于线性空间$V$,其对偶空间$V^*$中的一个元素就是$V$上的一个线性函数,即把一个向量映射为一个数字.如果取两个线性空间$V$和$U$,取它们之间的一个映射$f:V\rightarrow U$,使得$f$也满足线性性,那么我们称$f$是$V$到$U$的一个\textbf{线性映射}.在矢量空间的表示\upref{VecRep}中我们会看到,选定了各自的基以后,两个线性空间中的向量都可以看成是数字的组合,因此$f$也可以看成是$m$个线性函数$f_i$的组合,其中$m=\opn{dim}U$.

\begin{definition}{线性映射}\label{LinMap_def1}
给定域$\mathbb{F}$上的线性空间$V$和$U$.如果有映射$f:V\rightarrow U$满足,对于任意的向量$\bvec{v}_1, \bvec{v}_2\in V$和标量$a_1, a_2\in\mathbb{F}$,都有$f(a_1\bvec{v}_1+a_2\bvec{v}_2)=a_1f(\bvec{v}_1)+a_2f(\bvec{v}_2)$,那么称$f$是$V$到$U$的一个\textbf{线性映射(linear map)}.
\end{definition}


\autoref{LinMap_def1} 的内涵比看上去广一些.对于任意的一组有限个向量$\{\bvec{v}_i\}\subseteq V$和一组对应的标量$\{a_i\}\subseteq\mathbb{F}$,都有$f(\sum_i a_i\bvec{v}_i)=\sum_i a_if(\bvec{v}_i)$





