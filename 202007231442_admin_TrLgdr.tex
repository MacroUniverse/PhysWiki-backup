% 勒让德变换

\pentry{分部积分\upref{IntBP}}

勒让德变换的定义很多,有的定义虽然严谨却不易理解.本节采用其最方便理解的方式来定义.

\subsection{定义}

\subsubsection{单自变量勒让德变换}
\begin{definition}{}
给定一个实单自变量函数$f(t)$,假设它是一个凸函数,即其导函数$f'(t)$严格单调.将$f'(t)$重命名为一个新的实变量$s$,如果存在一个实函数$g(s)$,使得$g'(s)=t$,那么称$g(s)$是$f(t)$的\textbf{勒让德变换(Legendre transformation)}.
\end{definition}

简单来说,勒让德变换就是有两个函数,$f(t)$和$g(s)$,它们使用不同的自变量,但是各自求导以后却能得到对方的自变量:$f'(t)=s$,$g'(s)=t$.这个关系是对称的,因此$f(t)$也是$g(s)$的勒让德变换.使用勒让德变换,可以把凸函数$f(t)$所表示的某些性质改用$g(s)$来表示,相当于用新自变量描述同样的信息.

\subsubsection{与分部积分的关系}

假设$f(t)$和$g(s)$互为彼此的勒让德变换.由分部积分可得,
\begin{equation}
\begin{aligned}
\dd{ts}&=s\dd{t}+t\dd{s}\\&=f'(t)\dd{t}+g'(s)\dd{s}\quad\text{这一步由勒让德变换得}\\&=\dd{f}+\dd{g}
\end{aligned}
\end{equation}

这个结果说明,$\dd{g}=\dd{ts}-\dd{f}$,由此得
\begin{equation}
g=ts-f
\end{equation}

简单总结一下,把$f(t)$变换成$g(s)$的过程,就是用相互交换的两个元素$t$和$s$作乘积,再减去$f$.注意$t$和$s$的关系是:$f'(t)=s$,$g'(s)=t$.

\subsubsection{多自变量勒让德变换}

把微分拓展到偏微分,我们可以自然地推广出多自变量的勒让德变换.

\begin{definition}{}\label{TrLgdr_def1}
给定一个实函数$f(t_1, t_2, \cdots, t_n)$,假设对于任意$i=1, 2, \cdots, n$,都有$\pdv{t_i}f$是关于$t_i$的凸函数.将$\pdv{t_i}f$重命名为$s_i$,如果存在$g(s_1, s_2, \cdots, s_k, t_{k+1},\cdots, t_n)$,使得对于任意$i=1,2,\cdots k$,都有$\pdv{s_i}g=t_i$,那么称$g$是$f$替换了自变量$t_1, \cdots, t_k$后的勒让德变换.
\end{definition}

同样地,多自变量勒让德变换也可以用分部积分来求得.如果$g$是\autoref{TrLgdr_def1}中所说的替换了前$k$个$t_i$的$f$的勒让德变换,那么有:\begin{equation}\label{TrLgdr_eq1}
g=\sum\limits_{i=1,2,\cdots, k}t_is_i-f
\end{equation}

\subsection{勒让德变换的性质}

以下描述中,假设$f_i(t)$的勒让德变换是$g_i(t)$.

\begin{exercise}{标度性质}
证明:
\begin{itemize}
\item 如果$f_2(t)=af_1(t)$,那么$g_2(s)=ag_1(s/a)$,
\item 如果$f_2(t)=f_1(at)$,那么$g_2(s)=g_1(s/a)$.
\end{itemize}
\end{exercise}

\begin{corollary}{标度性质的推论}
如果$f_1(t)$是一个$m$次齐次函数\upref{Homeul},那么$g_1(s)$就是一个$n$齐次函数,其中$1/m+1/n=1$.
\end{corollary}

\begin{exercise}{平移性质}
证明:
\begin{itemize}
\item $f_2(t)=f_1(t)+c\Rightarrow g_2(s)=g_1(s)-c$,其中$c$是任意常数,下同;
\item $f_2(t)=f_1(t+c)\Rightarrow g_2(s)=g_1(s)-cs$
\end{itemize}
\end{exercise}

\begin{exercise}{反函数性质}
证明:

$f_2(t)=f_1^{-1}(t)\Rightarrow g_2(s)=-sg_1(1/p)$
\end{exercise}





如何求$f(t)$勒让德变换呢?如果已知$f(t)$的表达式,那么就可以计算出$f'(t)=s$,由定义,$f'(t)$是一个单调函数,因此存在反函数$F(s)=t$.由于$g'(s)=t=F(s)$,我们就可以写出:
\begin{equation}
g(s)=\int t \dd{s}=\int F(s) \dd{s}
\end{equation}

许多情形下,我们并不知道$f(t)$的显式表达,这个方法就可能并不适用.


