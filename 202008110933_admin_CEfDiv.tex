% 点电荷电场的散度

\pentry{球坐标系中的梯度散度旋度及拉普拉斯算符\upref{SphNab}, 狄拉克 delta 函数\upref{Delta}}

我们先看点电荷产生的电场
\begin{equation}\label{CEfDiv_eq1}
\bvec E(\bvec r) = \frac{q}{4\pi\epsilon_0}\frac{\uvec r}{r^2}
\end{equation}
要计算某点的散度, 最方便的做法是使用球坐标公式(\autoref{SphNab_eq2}~\upref{SphNab})得
\begin{equation}
\div \frac{\uvec r}{r^2} = \frac{1}{r^2} \pdv{r} \qty(r^2 \frac{1}{r^2}) = 0
\end{equation}
注意由于\autoref{CEfDiv_eq1} 在原点处无定义, 该结论不适用于 $\bvec r = \bvec 0$.

根据电场的叠加原理, 即使空间中有许多点电荷, 如果除去它们所在的位置, 它们产生的电场分布的散度处处为零.
\begin{equation}\label{CEfDiv_eq3}
\div \bvec E = 0
\end{equation}
我们还是假设只有一个点电荷位于坐标原点, 现在我们以原点为球心做一个半径为任意 $R > 0$ 的球面 $\mathcal S$, 并计算电场在球面上的通量为
\begin{equation}\label{CEfDiv_eq2}
\Phi = \oint_{\mathcal S} \bvec E \vdot \dd{\bvec s} = \frac{q}{4\pi\epsilon_0 R^2} \cdot 4\pi R^2 = \frac{q}{\epsilon_0}
\end{equation}
然而根据散度定理\autoref{Divgnc_eq1}~\upref{Divgnc}, 如果 $\div \bvec E$ 在球内处处为零, 应该有 $\Phi = 0$ 才对. 所以问题应该出在原点, 由于散度定理要求矢量场处处可偏导, 所以\autoref{CEfDiv_eq2} 的结果并不严格适用于散度定理.

结合\autoref{CEfDiv_eq3} 和\autoref{CEfDiv_eq2}, 我们却可以得到另一个结论: 如果任意另取一个包含上述球面的闭合曲面 $\mathcal S'$, 那么电场在该曲面上的通量应该也是相同的. 这是因为, 我们可以把这两个曲面共同看成是它们之间一块空间的边界.

可以推出原点处有无穷大的散度, 使得积分后恰好等于 $q/\epsilon_0$. 这让我们马上想到(三维)狄拉克 $\delta$ 函数\upref{Delta}, 在原点处为无穷大, 但体积分为有限值
\begin{equation}
\int \delta(\bvec r)\dd{V} = 1
\end{equation}

所以如果我们假设电场在原点处的散度为
\begin{equation}
\div \bvec E = \frac{q}{\epsilon_0}\delta(\bvec r)
\end{equation}
那么恰好可使电场满足散度定理
\begin{equation}
\int \div \bvec E \dd{V} = \frac{q}{\epsilon_0}
\end{equation}
