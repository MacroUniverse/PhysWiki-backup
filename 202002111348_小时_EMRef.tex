% 电磁场参考系变换
% 参考系变换|洛伦兹变换|电磁场

% 让我比较自豪的是,我高中的时候曾经独立推导出了洛伦兹变换,再通过这里给出的例子独立推导出了电磁场的参考系变换. 上大学时我才在新概念电磁学上看到一模一样的变换公式.

我们考虑一个电流为 $I$ 的无限长直导线, 令电流方向为正方向. 离导线距离为 $r_0$ 处有一个电荷为 $q$ 的粒子沿正方向以速度 $v$ 运动. 如果观察者的参考系沿正方向相对导线运动,速度为 $u$, 在该参考系中求粒子所受的力. 这个力和 $u$ 有关吗?

% 图未完成

\subsection{错误的分析}
当 $u = 0$ 时, 电流在导线周围产生的磁场使粒子受到垂直于导线洛伦兹力. 而当 $u = v$ 时, 导线同样产生磁场,而粒子却是静止的所以不受力.这两个结论互相矛盾.当 $u$ 取其他不同值时,得到的受力也各不相同.

\subsection{正确的分析}
上面 $u = 0$ 时的结论是正确的,空间中只有磁场没有电场. 然而当 $u \ne 0$ 时,由于导线中的异号电荷运动快慢不同,相对论尺缩效应使两种电荷的线密度产生区别,从而产生垂直导线的电场,与洛伦兹力共同作用在粒子上, 使粒子受力与 $u$ 无关.

\subsection{计算}
作为一个简单的导线模型,我们假设导线中的所有正电荷保持相对静止,其线电荷密度为 $\lambda$,以速度 $v_0$ 相对导线延正方向运动,导线中的所有负电荷同样保持相对静止,线电荷密度为 $-\lambda$, 以速度 $-v_0$ 相对导线延负方向运动. 这样,导线的电流为
\begin{equation}
I = 2\lambda v_0
\end{equation}

(未完成)
