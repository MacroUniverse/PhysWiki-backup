% 联络形式与结构定理
% 微分形式|联络|仿射联络|曲率|曲率形式|联络形式|connection form|differential form|1-form|curvature form|affine connection|1-形式

\pentry{仿射联络(流形)\upref{affcon}, 爱因斯坦求和约定\upref{EinSum}}

仿射联络的定义是高度抽象的,并不涉及具体的运算.我们在本节所介绍的是将来进行计算时非常关键的理论基础.计算的实例请参见\textbf{庞加莱半平面(微分几何计算实例)}\upref{PoiHP}.

本词条中默认$(M, \nabla)$为一个带仿射联络的实流形.

\subsection{联络形式}

对于任意$p\in M$,取$p$的一个邻域$U\subseteq M$,使得存在一组光滑向量场$\{\uvec{e}_i\}$构成$C^{\infty}(U)$上的一组基.这就是说,$U$上的每个光滑向量场都可以表示为$f^i\uvec{e}_i$的形式,其中各$f^i$是$U$上的光滑函数.

对于任意$X\in\mathfrak{X}(U)$,我们知道$\nabla_X\uvec{e}_i$也是一个光滑向量场,因此存在一组\textbf{光滑函数}$\omega^j_i(X)$,使得$\nabla_x\uvec{e}_i=\omega^j_i(X)\uvec{e}_j$.

每个光滑函数$\omega^j_i(X)$都由$X$唯一确定,而且由$\nabla$的性质知,对于任意光滑函数$f$和$g$,光滑向量场$X, Y$,都有$\omega^j_i(fX+gY)=f\omega^j_i(X)+g\omega^j_i(Y)$.也就是说,$\omega^j_i$本身是$X$的\textbf{线性函数},也就是$U$上的一个$1$-形式.

我们将以上讨论所得出的$\omega^j_i$称为$\nabla$的\textbf{联络形式(connection form)},$\omega^j_i$构成的方阵称为$\nabla$的\textbf{联络形式矩阵(metric of connection forms)}.

\subsubsection{联络形式作为坐标分量}

在线性代数中我们知道,一个向量(或者任何非零阶的张量)不能简单地理解为一组坐标数字,因为它的坐标具体取值取决于基的选择.而以上讨论的$\omega^j_i$虽然不是数字,却也有类似的性质,即“$\omega^j_i$具体是哪个$1$-形式,取决于选择哪一组$\{\uvec{e}_i\}$作为$\mathfrak{X}(M)$的基”.换句话说,$\omega^j_i(X)$具体是哪个函数,不仅取决于$X$,也取决于$\{\uvec{e}_i\}$的选择.

因此,尽管$\omega_i^j$是微分形式,我们也把它看成一种坐标分量,即联络$\nabla$在基$\{\uvec{e}_i\}$下的\textbf{局部}坐标分量.之所以强调“局部”,是因为我们只能保证在$\mathfrak{X}(U)$中能找到一组基,而在整个$\mathfrak{X}(M)$中则不一定存在基\footnote{比如考虑$M=S^2$,即球面,那么球面上任何一个连续向量场总存在零点,因此对于任意\textbf{两个}光滑向量场$\{\uvec{e}_i\}$,在$\uvec{e}_1$的零点$p$处,仅靠$\uvec{e}_2$是无法张成整个切空间$T_pM$的,因此只要一个光滑向量场在$p$点的值和$\uvec{e}_2$不平行,这个场就没法被$\{\uvec{e}_i\}$表示出来.}.

\subsubsection{联络的计算}

如果我们知道了一个基下具体的联络形式,就可以计算出联络了.

\begin{theorem}{}
设$\{\uvec{e}_i\}$是$(M, \nabla)$上,某邻域$U$上的一组基.对于任意$S=a^i\uvec{e}_i, X\in\mathfrak{X}(U)$,可以计算出:
\begin{equation}
\nabla_XS=(Xa^j+a^i\omega^j_i(X))\uvec{e}_j
\end{equation}
\end{theorem}

\textbf{证明}:

\begin{equation}
\begin{aligned}
\nabla_XS&=\nabla_X(a^i\uvec{e}_i)\\
&=(Xa^i)\uvec{e}_i+a^i\nabla_X\uvec{e}_i\\
&=(Xa^j)\uvec{e}_j+a^i\nabla_X\uvec{e}_i\\
&=(Xa^j)\uvec{e}_j+a^i\omega^j_i(X)\uvec{e}_j\\
&=(Xa^j+a^i\omega^j_i(X))\uvec{e}_j
\end{aligned}
\end{equation}

\textbf{证毕}.

实际上,联络和联络形式并不一定是只能定义在流形的切丛上,它也可以定义在流形上的任何向量丛上.具体拓展请参见\textbf{联络(向量丛)}\upref{VecCon}词条.

\subsection{曲率形式}

曲率算子的定义为$R(X, Y)=\nabla_X\nabla_Y-\nabla_Y\nabla_X-\nabla_{[X, Y]}$.同样地,我们在$U\subseteq M$局部取$\{\uvec{e}_i\}$作为光滑向量场的基,定义出一组光滑函数$\Omega^j_i(X, Y)$,满足:
\begin{equation}
R(X, Y)\uvec{e}_i=\Omega^j_i(X, Y)\uvec{e}_j
\end{equation}

类似地,我们也可以证明$\Omega^j_i(X, Y)$是关于$X$和$Y$的一个线性函数,因此$\Omega^j_i$本身是一个$U$上的$2$-形式,称为$\nabla$的\textbf{曲率形式(curvature form)}.$\Omega^j_i$构成的方阵称为$\nabla$的\textbf{曲率形式矩阵(matrix of curvature forms)}.

\begin{exercise}{}
证明对于$U$上的任意光滑函数$a^i, b$和光滑向量场$X_i, Y$,有$R(a^iX_i, bY)=a^ibR(X_i, Y)$,再由$R$的反对称性,推论出$\Omega^j_i$本身是一个$2$-形式.
\end{exercise}

\subsection{挠率形式}


类似联络形式和曲率形式,我们有挠率形式的定义:

\begin{equation}
T(X, Y)=\nabla_XY-\nabla_YX-[X, Y]:=\tau^i(X, Y)\uvec{e}_i
\end{equation}

此处各$\tau^i(X, Y)$依然是光滑函数,$\tau^i$则是$2$-形式,故同样可得\textbf{挠率形式(torsion form)}和\textbf{挠率形式矩阵(matrix of torsion forms)}的概念.





\subsection{结构定理}

结构定理是我们进行和联络相关的计算时的关键工具.在介绍这个定理前,我们还有最后一点铺垫.


\begin{definition}{对偶基}
设$\{\uvec{e}_i\}$是流形$M$上某邻域$U$上光滑向量场的基.定义一组$1$-形式$\{\uvec{\theta}^i\}$,使得$\uvec{e}_j\uvec{\theta}^i=\delta^i_j$处处成立,那么$\{\uvec{\theta}^i\}$构成$U$上全体$1$-形式的一组基,称为$\uvec{e}_i$的\textbf{对偶基(duel basis)}.
\end{definition}



\begin{theorem}{Schmidt标准化}
$U$上的任何基都可以Schmidt标准化.
\end{theorem}

标准化的意思就是说,$<\uvec{e}_i, \uvec{e}_j>=\delta_{ij}$处处成立,也就是说在每一个点处$\{\uvec{e}_i\}$都构成一个标准正交基.Schmidt标准化的方法是线性空间中Schmidt标准化的直接推广,若不熟悉请见\textbf{施密特正交归一化}\upref{SmdtOt}词条.

\begin{theorem}{反对称定理}
若$\{\uvec{e}_i\}$是$\mathfrak{X}(U)$上的\textbf{标准正交基},那么联络形式矩阵$\omega^j_i$必\textbf{反对称},即$\omega^j_i=-\omega^i_j$.
\end{theorem}

\textbf{证明}:

由于$<\uvec{e}_i, \uvec{e}_j>=\delta_{ij}$,且$\nabla_X\uvec{e}_i=\omega^j_i(X)\uvec{e}_j$,故

\begin{equation}
\begin{aligned}
0=X<\uvec{e}_i, \uvec{e}_j>&=<\omega^k_i(X)\uvec{e}_k, \uvec{e}_j>+<\uvec{e}_i, \omega^k_j(X)\uvec{e}_k>\\
&=\omega^k_i(X)\delta_{kj}+\delta_{ik}\omega^k_j(X)\\
&=\omega^k_i(X)\delta_{k}^j+\delta_{i}^k\omega^k_j(X)\\
&=\omega^j_i(X)+\omega^i_j(X)
\end{aligned}
\end{equation}

\textbf{证毕}.

接下来,我们就可以摆出最为关键的结构定理了.

\begin{theorem}{联络形式的结构定理}
\begin{enumerate}
\item 曲率形式满足:$\Omega^i_j=\dd \omega^i_j+\omega^i_k\wedge\omega^k_j$;
\item 挠率形式满足:$\tau^i=\dd\uvec{\theta}^i+\omega^i_j\wedge\uvec{\theta}^j$.
\end{enumerate}

其中对于$1-$形式$f, g$,有
\begin{equation}
f\wedge g(X, Y)=f(X)g(Y)-f(Y)g(X)
\end{equation}
和
\begin{equation}
\dd f(X, Y)=Xf(Y)-Yf(X)-f([X, Y])
\end{equation}

如果将微分形式局部表示为指标形式,比如在$U\subseteq M$上给定基$$
\end{theorem}













