% test
% test|测试|编辑器

参考 \href{https://docs.julialang.org/en/v1/manual/unicode-input/}{Julia 符号表}和 \href{https://oeis.org/wiki/List_of_LaTeX_mathematical_symbols}{LaTeX 符号表}, 以及 \href{http://www.onemathematicalcat.org/MathJaxDocumentation/TeXSyntax.htm#U}{MathJax 符号表}.

\begin{equation}
\cap, \bigcap, \cup, \bigcup, \vee, \wedge, \int, \iint, \iiint, \oint
\end{equation}
\begin{equation}
\diamond, \ominus, \triangleleft, \triangleright, \Longleftarrow, \Longrightarrow, \iff, \leftrightarrow, \updownarrow, \cdots
\end{equation}
\begin{equation}
\ddots, \top, \bot, \measuredangle
\end{equation}

\subsection{化学式}
编辑器预览的 MathJax 3 开始支持化学式了. 但是网站用的 2 还不支持. 也不确定 LaTeX 是否支持.
\begin{equation}
\ce{SO4^2- + Ba^2+ -> BaSO4 v}
\end{equation}

令 $x$ 轴竖直向下, $y$ 轴水平, 轨道的函数曲线为 $y(x)$, 其导函数为 $\dot y(x)$ 质点从原点开始沿曲线运动, 初速度为零, 根据动能定理, 质点在轨道任意一点处速度为 $v = \sqrt{2gx}$. 则质点从原点下落高度 $h$ 所需的时间为
\begin{equation}
t = \int_{0}^{h} \frac{\dd{x}}{v\cos\theta} = \int_{0}^{h} \sqrt{\frac{1 + \dot y^2}{2gx}}\dd{x}
\end{equation}
