% 极化电流

(未完成)
首先证明磁介质产生的电流 $\curl \bvec M = \bvec j_M$, 即证 $\oint \bvec j_M \vdot \dd{\bvec s} = \oint \bvec M \vdot \dd{\bvec l}$. 假设磁偶极子都是小线圈组成, 曲面内部净电流为零, 曲面边界只有穿过小线圈才能在曲面上产生净电流.
\begin{equation}
I = I_1 (\pi R^2 \uvec n \vdot \dd{\bvec l}) n = \bvec M \vdot \dd{\bvec l}
\end{equation}
证毕.

再来看电介质的极化电流
\begin{equation}
\bvec j_p = \dv{\bvec P}{t}
\end{equation}
所以根据
\begin{equation}
\curl {\bvec B} = \mu_0 (\bvec j_f + \bvec j_M + \bvec j_p + \bvec j_E) = \mu_0 \bvec j_f + \mu_0 \curl \bvec M + \mu_0 \pdv{\bvec P}{t} + \mu_0 \epsilon_0 \pdv{\bvec E}{t}
\end{equation}
定义\textbf{磁场强度}为
\begin{equation}
\bvec H = \frac{\bvec B}{\mu_0} - \bvec M
\end{equation}

则
 
另外, 假设磁介质为线性,
 ,  ,  ,  
这就是说,  的磁介质会使电流加强.
