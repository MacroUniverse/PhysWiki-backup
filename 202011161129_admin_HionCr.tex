% 氢原子电离截面
% 一阶微扰理论|氢原子|偶极子

\begin{issues}
\issueDraft
\end{issues}

\pentry{类氢原子的波函数\upref{HWF}}

本文使用原子单位制\upref{AU}.一阶微扰理论就是单光子电离.

基态与平面波的积分

\begin{equation}
\bvec d_{\bvec k} =  \mel{\bvec k}{\bvec r}{0} =  \frac{ \uvec k}{\sqrt2\pi} \int_0^{+\infty} \int_0^\pi \E^{-r} \E^{-\I k r \cos\theta} r \cos\theta \cdot r^2 \sin\theta \dd{\theta} \dd{r}
\end{equation}
换元, 令 $u = \cos\theta$, 得
\begin{equation}\ali{% 已检查多次
\bvec d_{\bvec k} &= \frac{\uvec k}{\sqrt{2}\pi}  \int_0^{+\infty} r^3 \E^{-r} \int_{-1}^1 \E^{-\I k r u} u  \dd{u} \cdot \dd{r}\\
&=  \I\frac{\sqrt2 \uvec k}{\pi k}  \int_0^{+\infty} r^2 \E^{-r} \qty[\cos(kr) - \frac{1}{kr}\sin(kr)] \dd{r}\\
&= -\I \frac{8\sqrt2}{\pi} \frac{\bvec k}{(k^2+1)^3}
}\end{equation}

严格来说, 需要把平面波替换为库仑函数.

\subsection{长度规范下的微扰跃迁理论}
含时微扰理论(\autoref{TDPT_eq10}~\upref{TDPT}) 为
\begin{equation}\label{HionCr_eq1}
c_i(t) = -\I \int_0^t \mel{i}{H'(t)}{j} \E^{\I\omega_{ij} t} \dd{t}
\end{equation}
长度规范\upref{LenGau}中,
\begin{equation}
H'(t) = -q\bvec {\mathcal E} \vdot \bvec r
\end{equation}
$\bvec {\mathcal E}$ 只是 $t$ 的函数, 可以分离
\begin{equation}
\mel{i}{H'(t)}{j} = -q\bvec {\mathcal E} \vdot \mel{i}{\bvec r}{j}
\end{equation}
令
\begin{equation}
\tilde {\bvec {\mathcal E}}(\omega) = \frac{1}{\sqrt{2\pi}} \int \bvec {\mathcal E}(t) \E^{-\I\omega t} \dd{t}
\end{equation}
代入\autoref{HionCr_eq1} 得
\begin{equation}
c_i(t) = \I q \mel{i}{\bvec r}{j} \int_0^t \bvec {\mathcal E(t)} \vdot  \E^{\I\omega_{ij} t} \dd{t} = \I \sqrt{2\pi} q \mel{i}{\bvec r}{j} \tilde {\bvec {\mathcal E}}(-\omega_{ij})
\end{equation}
跃迁概率为
\begin{equation}\label{HionCr_eq2}
P_{i\to j} = \abs{c_i(t)}^2 = -2\pi q^2 \abs{\mel{i}{\bvec r}{j}}^2 \abs{\tilde {\bvec {\mathcal E}}(\omega_{ij})}^2
\end{equation}
结合\autoref{WpEng_eq3}~\upref{WpEng}
\begin{equation}
s(\omega) = 2c\epsilon_0 \abs{\tilde {\bvec {\mathcal E}}(\omega)}^2
\end{equation}
得
\begin{equation}
P_{i\to j} = -\frac{\pi q^2}{c\epsilon_0} \abs{\mel{i}{\bvec r}{j}}^2 s(\omega)
\end{equation}

\subsection{速度规范下的微扰跃迁理论}
\footnote{参考\cite{Merzbacher} 含时微扰相关章节.}速度规范\upref{LVgaug}中,
\begin{equation}
H'(t) = -\frac{q}{m}\bvec A \vdot \bvec p = \frac{\I q}{m}\bvec A \vdot \grad
\end{equation}
$\bvec A$ 只是 $t$ 的函数, 可以分离
\begin{equation}
\mel{i}{H'(t)}{j} = -\frac{q}{m}\bvec A(t) \vdot \mel{i}{\bvec p}{j}
\end{equation}
令
\begin{equation}
\tilde {\bvec A}(\omega) = \frac{1}{\sqrt{2\pi}} \int \bvec A(t) \E^{-\I\omega t} \dd{t}
\end{equation}
代入\autoref{HionCr_eq1} 得
\begin{equation}\label{HionCr_eq4}
c_i(t) = \frac{\I q}{m} \mel{i}{\bvec p}{j} \int_0^t  \bvec A(t) \E^{\I\omega_{ij} t} \dd{t} = \I\sqrt{2\pi}\frac{q}{m} \mel{i}{\bvec p}{j} \tilde {\bvec A}(-\omega_{ij})
\end{equation}
跃迁概率为
\begin{equation}\label{HionCr_eq3}
P_{i\to j} = \abs{c_i(t)}^2 = -\frac{2\pi q^2}{m^2} \abs{\mel{i}{\bvec p}{j}}^2 \abs{\tilde {\bvec A}(\omega_{ij})}^2
\end{equation}
结合波包的频谱公式(\autoref{WpEng_eq5}~\upref{WpEng})
\begin{equation}
s(\omega) = 2c\epsilon_0 \omega^2 \abs{\tilde {\bvec A}(\omega_{ij})}^2
\end{equation}
\begin{equation}
P_{i\to j} = -\frac{\pi q^2}{c\epsilon_0 m^2 \omega^2} \abs{\mel{i}{\bvec p}{j}}^2 s(\omega_{ij})
\end{equation}

\subsubsection{两种规范比较}
把\autoref{LVgaug_eq6}~\upref{LVgaug} 和\autoref{WpEng_eq4}~\upref{WpEng} 带入\autoref{HionCr_eq4} 可以证明, 如果 $\braket{i}{j} = 0$, \autoref{HionCr_eq2} 和\autoref{HionCr_eq3} 是相等的, 但是如果 $\ket{i}$ 是平面波, 而 $\ket{j}$ 是氢原子束缚态, 则内积不为零, 不同规范结果不同.

\subsection{电离截面}
截面可以想象成是电磁波传播方向垂直放置的一块面积为 $\sigma$ 的面元, 使得原子从电磁波中吸收的功率恰好等于电磁波经过该面元的功率.
\begin{equation}
\dv{E}{\omega} = \sigma(\omega) s(\omega)
\end{equation}
