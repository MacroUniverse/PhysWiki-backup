% 群同态

\pentry{映射\upref{map}, 正规子群\upref{Group1}}

\subsection{同构}

让我们来观察两个群 $(\mathbb{Z}, +)$和$(2\mathbb{Z},+)$.如果我们把$2\mathbb{Z}$中的$2$都看成$1$,$4$都看成$2$,以此类推,将$2k$都看成$k$,那么两个群的运算规则是一模一样的.比如说,$2\mathbb{Z}$中有$2+4=6$,对应的是$\mathbb{Z}$中$1+2=3$的等式.

我们研究集合和群的时候,元素叫什么名字并不重要,重要的是元素之间是否相同以及运算规则是怎样的.那么,如果我们真的将$2\mathbb{Z}$中的元素$2k$都重命名为$k$,它就和$\mathbb{Z}$没什么区别了.所以在群的意义上,如果不考虑子群关系,单独把$\mathbb{Z}$和$2\mathbb{Z}$拿出来的时候,我们就认为它们是不可区分的,完全相同的两个群.

如果我们建立一个映射$f:\mathbb{Z}\rightarrow2\mathbb{Z}$,定义为$f(k)=2k$,那么这个$f$就是一个双射,它在两个群的元素之间一一对应地建立了联系.这样,对于任意整数$m, n$,有$f(m)+f(n)=f(m+n)$,也就是说“先运算再映射”和“先映射再运算”结果是相同的.

类似地,对于任意的两个群$G$和$K$,如果存在一个\textbf{双射} $f:G\rightarrow K$,使得对于任意的$x, y\in G$都满足$f(x)f(y)=f(xy)$,那么这两个群的运算结构就是一模一样的.这时我们说这两个群是\textbf{同构(isomorphic)}的,而这个使得它们同构的双射就被称为 $G$ 和 $K$ 之间的\textbf{同构映射(isomorphic mapping)},也可以简称\textbf{同构(isomorphism)}这里加粗的两个“同构”,前者是形容词,后者是名词.

由于同构使得两个群各方面表现一模一样,研究同构其实没有太大意义,我们甚至直接把同构的两个群看成同一个群,不管元素具体怎么命名的.有意思的结构,是以下定义的“同态映射”.

\subsection{同态}

同构映射是一个双射.如果把这个要求拿掉,我们就得到同态的概念:

\begin{definition}{同态映射}\label{Group2_def1}

对于两个群$G$和$K$,如果映射\textbf{(不一定是双射)}$f:G\rightarrow K$使得$\forall x, y\in G, f(x)f(y)=f(xy)$,那么称$G$和$K$是\textbf{同态(homomorphic)}的,称$f$是\textbf{同态映射(homomorphic mapping)}或\textbf{同态(homomorphism)}.

\end{definition}

\begin{definition}{像和核}

沿用\autoref{Group2_def1}的设定.$K$中被映射到的元素构成的集合,称为$f$的\textbf{像(image)},记作$f(G)$.$G$中映射到$K$的单位元$e_K$的元素构成的集合,称为$f$的\textbf{核(kernal)},记为$ker(f)$.

\end{definition}

注意,$f(G)\subset K$,$ker(f)\subset G$.

同态的两个群,运算结构很相似但又不完全一样.在以上定义的例子中,$K$的行为就像是一个弱化版的$G$,丢失了一些细节,但保留的方面和$G$是一模一样的.这么说可能不够具体,我们用习题来理解同态的“似而不同”.

\begin{exercise}{群同态定理}\label{Group2_exe1}
设两个群$G$和$K$,$f:G\rightarrow K$是一个同态.求证:
\begin{itemize}
\item $ker(f)$是$G$的一个正规子群.\footnote{这保证了群$G/ker(f)$存在.}
\item 对于$x, y\in G$,如果$x_1$和$y_1$分别和$x, y$同余,或者换句话来说,$x_1^{-1}x\in H$和$y_1^{-1}y\in H$,那么$f(x)=f(x_1)$,$f(y)=f(y_1)$.
\item 由前两条的结论,证明可以用$f$来导出一个映射$f': G/ker(f)\rightarrow K$是一个同构.

\end{itemize}
\end{exercise}

由\autoref{Group2_exe1},同态的实质就是商群$G\ker(f)$和$K$之间的同构.$G\ker(f)$继承了$G$的运算,但是由于把同余的元素全都当作同一个了,也就丢失了一部分细节.因此我们说同态的两个群也是“似而不同”的.
