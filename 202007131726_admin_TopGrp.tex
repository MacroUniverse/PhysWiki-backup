% 拓扑群
\pentry{连续映射\upref{Topo1}, 群\upref{Group}}

拓扑群,顾名思义,是一种同时具有群和拓扑结构的数学对象.但是光有两个结构还不够,拓扑群还要求满足拓扑和群运算之间的一个联系.

\begin{definition}{拓扑群}
给定一个集合$G$,若在$G$上定义了一个拓扑$\mathcal{T}$和一个群运算$\cdot$,且满足拓扑空间之间的映射$f:G\times G\rightarrow G$是一个连续映射,其中$f(g_1, g_2)=g_1\cdot g_2^{-1}$,那么称$G$是一个\textbf{拓扑群(topological group)}.
\end{definition}

要求$f(g_1, g_2)=g_1\cdot g_2$是一个连续映射,也就保证了$f_1(g_1, g_2)=g_1\cdot g_2$和$f_2(g)=g^{-1}$都是连续映射.

\begin{definition}{拓扑群同态}
设$G$和$H$是两个拓扑群,它们之间有映射$f:G\rightarrow H$.如果$f$在拓扑意义上是连续映射,在代数意义上是群同态,那么称$f$是一个拓扑群之间的同态.
\end{definition}

下面我们看两个简单的例子.

\begin{example}{拓扑群的例子}
\begin{itemize}
\item 取实数轴$\mathbb{R}$,定义通常的度量拓扑以及加法群,则$\mathbb{R}$是一个拓扑群.
\item 取复平面上的单位圆$S^1=\{\mathrm{e}^{\theta\I}|\theta\in\mathbb{R}\}$,定义其拓扑为二维欧几里得空间中的子拓扑,群运算为复数乘法,则$S^1$是一个拓扑群.
\item 定义映射$f:\mathbb{R}\rightarrow S^1$为$f(t)=\mathrm{e}^{2\pi t\I}$,则$f$是一个拓扑群同态.
\end{itemize}
\end{example}

