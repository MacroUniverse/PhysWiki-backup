% 子群和正规子群
\pentry{群的概念\upref{Group}}

\subsection{子群}

\begin{definition}{子群}

给定一个群$(G, \cdot)$,如果集合$G$有一个子集$H$,使得$e\in H$且$H$中的元素在运算$\cdot$下仍然封闭,那么显然\footnote{由于已经知道$(G,\cdot)$构成一个群了,群的四条公理中,结合性、单位元存在性以及逆元存在性都被满足了.}$(H,\cdot)$也构成一个群.称$H$是群$G$的\textbf{子群(subgroup)}.

\end{definition}

虽然群和子群的联系很紧密,但是我们通常还是把它们看作完全不同的集合,只不过可以自然地应用已经存在的群运算来定义子群的运算.这样,将已有的运算直接用在子集上,有时被称作在子集上\textbf{导出}或\textbf{诱导(induce)}了一个运算,有时也称子集上的运算是\textbf{限制在子集上的运算}.比如在定义里,群$H$的运算实际上被认为是和$G$的运算不一样,严格来说应该记为$\cdot|_H$,意思是“限制在$H$上的$\cdot$”. 但是不至于引起混淆的时候,也可以简单记为$\cdot$,并认为是同一个运算.

一般来说,当我们说$H$是一个子群时,强调的是$H$和$G$的关系;但如果我们说$H$是一个群,我们关心的是$H$本身作为群的性质,而没有强调它和其它群的关系.

\subsection{陪集}

群的运算有一个很棒的唯一性,算是我们目前遇到的第一个具体的结构性特征,由以下定理描述:

\begin{exercise}{群运算的唯一性}\label{Group1_exe1}
给定一个群$G$和群中的一个元素$x$.如果$\exists a, b$使得$ax=bx$,那么必然有$a=b$;反过来,如果$xa=xb$,也必然有$a=b$.

\begin{itemize}
\item定理证明留作习题.提示:用$x^{-1}$去参与运算试试.
\end{itemize}
\end{exercise}

唯一性是所有群都有的性质,但是等到我们讨论环(ring)的时候,由于环的乘法不要求逆元一定存在,我们没法对环证明这个唯一性.事实上,很多环都没有唯一性.




\begin{definition}{左陪集}

给定一个群$G$和它的一个子群$H$.从$G$中任意挑一个元素$x$出来,用它来\textbf{左乘}$H$中的每一个元素,得到一个集合$\{x, xh_1, xh_2, \cdots\}$,这里的$h_n$要取遍每一个在$H$中的元素.这个集合,也可以写为$\{xh|h\in H\}$,被叫做\textbf{子群$H$关于元素$x$的左陪集},记作$xH$. 

\end{definition}

如果$h\in H$,而$H$是一个\textbf{有限群},那么由于封闭性,$hH=H$.这是因为,用$h$去左乘$H$中的一切元素(包括$h$自己),那么一方面由于封闭性,运算结果还是在$H$内部;另一方面由于群运算的唯一性,每一个左乘运算都不相同.这个论断不能简单地用于无限群.

特别地,考察这个形式的集合:$\{e, h, hh, hhh, hhhh, \cdots\}$,那么同样地由于封闭性和运算唯一性可知,这个集合还是群$H$的一个子集.特别地,$H$的群运算限制在这个集合上能构成一个循环群.只要我们把$n$个$h$相乘的结果记为$h^n$,$n$个$h^{-1}$相乘的结果记为$h^{-n}$,那么如此生成的循环群就可以用指数的加法运算来处理了.

\begin{definition}{指数}
打算
\end{definition}