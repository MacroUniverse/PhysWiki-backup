% 速度的参考系变换

% 主要是说明科里奥利力词条中的

\pentry{几何矢量\upref{GVec}}

\subsection{无相对转动}
若两个坐标系 $S$ 和 $S'$ 之间无相对转动(注意我们不要求坐标轴 $\uvec x, \uvec y, \uvec z$ 和 $\uvec x', \uvec y', \uvec z'$ 同方向), 那么某时刻两坐标系之间的相对速度是唯一确定的. 我们把 $S'$ 系相对于 $S$ 系的速度记为 $\bvec v_{rel}$ 根据高中所学的速度叠加原理, 若某点 $P$ 相对于 $S$ 系的速度瞬时为 $\bvec v_S$, 相对于 $S'$ 的瞬时速度为 $\bvec v_{S'}$ 我们有
\begin{equation}\label{Vtrans_eq1}
\bvec v_{S} = \bvec v_{S'} + \bvec v_{rel}
\end{equation}
其中三个矢量都可以是时间的函数. 注意该式与点 $P$ 的位置无关只和速度有关.

注意\autoref{Vtrans_eq1} 中的矢量都是抽象的矢量, 不能将 $\bvec v_S$ 或 $\bvec v_S'$ 等同于点 $P$ 在 $S$ 系或 $S'$ 系中的三个坐标的求导. 如果要将\autoref{Vtrans_eq1} 写成坐标的形式, 三个矢量必须在同一坐标系中表示. 我们来举例说明.

\begin{example}{}
令 $\uvec y' = \uvec x$, $\uvec z' = \uvec y$, $\uvec x' = \uvec z$, $\bvec v_{rel} = 2\uvec x = 2\uvec y'$, 点 $P$ 在 $S$ 系中的坐标关于时间的导数为 $(1, 2, 3)$. 请将\autoref{Vtrans_eq1} 表示为列矢量的形式.

容易得出, 点 $P$ 在 $S'$ 系中的坐标关于时间的导数为 $(3, -1, 2)$.

我们先来看错误的理解: 将 $\bvec v_S$ 等同于 $(1, 2, 3)$, $\bvec v_{S'}$ 等同于 $(3, -1, 2)$
\end{example}

\subsection{有相对转动}
对于任意两个坐标系(他们之间可以存在任何, 同样有
\begin{equation}
\bvec v_{xyz} = \bvec v_{abc} + \bvec v_{rel}
\end{equation}
然而
