% 数值解常微分方程(入门)

\begin{issues}
\issueDraft
\end{issues}

\footnote{本文经作者同意转载自知乎专栏 \href{https://www.zhihu.com/column/c_1226443594048942080}{《科学计算》}, 格式有少量修改.}前面几节,我们把求解线性方程组的基本数值方法做了详细的介绍和分析.从这一节开始,我们来尝试使用这些解法,处理更复杂也更贴近实际应用的问题.需要复习小伙伴,可以先去看看前面的内容.

这一节我们来讨论一下常微分方程(Ordinary Differential Equation (ODE)).\textbf{注意,我的整个专栏都是在讨论关于科学计算和数值分析的内容.对于常微分方程的分析特性,解析特解和通解等特性,请参考数学分析课程等.}

\subsubsection{举个例子}

比如,  $y'(t)=\cos(yt)$ ,这实际上是一个非线性的常微分方程 ,而且 $y$ 还隐含的是  $t$  的函数.

再比如,  $y'(t)=ay(t)+b$  ,如果  $a$  和  $b$  都是常数的话,这就是一个线性常微分方程.

再再比如,  $y'(t)=at+b$  ,求解这个方程可以轻松的将右边对  $t$  做积分.

当然,我们也可以写出常微分方程的标准形式

 $y'(t)=f(y(t),t),t _0&lt;t&lt;T,\quad  y(t_0)=\hat{y}$  

这里  $y(t_0)=\hat{y}$  是方程的初始条件,因此这样的问题也叫做初值问题(Initial Value Problem (IVP)).我们后面要讨论的数值方法都是围绕着IVP问题的.当然,常微分方程还有一类是边界问题(Boundary Value Problem(BVP)),这类问题我们留到偏微分方程的数值方法时一起讨论.

那么,对于上面这样的一个微分方程,我们可以从两个不同的角度来得到几乎相同的数值方法.一种是将左边的导数进行泰勒展开的近似,另一种则是将整个方程积分.

\subsection{数值方法——泰勒展开观点}

我们考虑一个很小的值  $h$  
\begin{enumerate}
\item 对  $y(t+h)$  做泰勒展开可以得到  $y(t+h)=y(t)+hy'(t)+\frac{h^2}{2!}y''(t)+\cdots$  .我们舍去高次项,仅保留一阶导数,则  $y'(t)\approx \frac{y(t+h)-y(t)}{h}= f(y(t),t)$  .这样就得到了\textbf{Forward Euler}方法,它的离散化误差(Discretization error)是  $e=\mathcal{O}(h)$  .很显然,这个更新过程这是一个显式方法(Explicit method).
\item 同样的,对  $y(t)$  做泰勒展开还可以得到  $y(t)=y(t+h)-hy'(t+h)+\frac{h^2}{2!}y''(t+h)-\cdots$  .我们同样只保留一阶导数,就得到了  $y'(t+h)\approx \frac{y(t+h)-y(t)}{h}= f(y(t+h),t+h)$  .这就是\textbf{Backward Euler}方法,它的离散化误差(Discretization error)同样是  $e=\mathcal{O}(h)$  .但是,由于  $y(t+h)$  同时出现在等式左右两端,因此需要求解这个方程,才能得到  $y(t+h)$  的值,因此这是一个隐式方法(Implicit method).
\item 我们用1)和2)方法的平均值,就得到了一个新的方法:  $ \frac{y(t+h)-y(t)}{h}= \frac{1}{2}\left(f(y(t+h),t+h)+f(y(t),t)\right)$  .这个方法事实上就是所谓的梯形公式(\textbf{Trapezoidal rule}).它仍然是一个隐式方法,但他的离散化误差提高到了是  $e=\mathcal{O}(h^2)$ .有兴趣的小伙伴可以自己用泰勒展开验证一下.
\end{enumerate}


\subsection{数值方法——数值积分观点}

上面所述的三种方法也可以从数值积分中推导出, 我们先将微分方程两边同时积分,得到  $y(t+h)=y(t)+\int_t^{t+h}f \rm{d}s$  .这个积分的值简单理解就是如下图所示,求函数  $f$  与横坐标轴在  $t$  到  $t+h$  之间围成的面积.它们的数值方法有很多形式,例如

\begin{enumerate}
\item 我们可以用下面蓝色的长方形区域来近似这个面积
\begin{figure}[ht]
\centering
\includegraphics[width=5cm]{./figures/NordEq_1.png}
\caption{用长方形近似面积} \label{NordEq_fig1}
\end{figure}
$y(t+h)\approx y(t)+hf(y(t),t)$  .这正好就是 \textbf{Forward Euler }方法.
\item 我们也可以用下面的绿色区域来近似这个面积
\begin{figure}[ht]
\centering
\includegraphics[width=5cm]{./figures/NordEq_2.png}
\caption{用长方形近似面积} \label{NordEq_fig2}
\end{figure}
$y(t+h)\approx y(t)+hf(y(t+h),t+h)$  .这就是\textbf{Backward Euler}方法.
\item 还可以用下面的橙色梯形区域来近似
\begin{figure}[ht]
\centering
\includegraphics[width=5cm]{./figures/NordEq_3.png}
\caption{用梯形近似面积} \label{NordEq_fig3}
\end{figure}
$y(t+h)\approx y(t)+\frac{1}{2}h\left(f(y(t),t)+f(y(t+h),t+h)\right)$  .这正好就\textbf{Trapezoidal rule,}即梯形公式名字的由来.
\end{enumerate}


\subsection{总结一下}

我们把  $y(t_i)$  称作精确解或者解析解,它的数值近似解记做  $y_i$ .
\begin{itemize}
\item Forward Euler:  $y_{n+1}=y_n+hf(y_n,t_n)$  
\item Backward Euler:  $y_{n+1}=y_n+hf(y_{n+1},t_{n+1})$  
\item Trapezoidal Rule:  $y_{n+1}=y_n+\frac{h}{2}\left(f(y_n,t_n)+f(y_{n+1},t_{n+1}) \right)$  
\end{itemize}


\subsection{求解步骤}

我们来简单概括一下常微分方程的求解过程:

\begin{enumerate}
\item 从初始值开始,令  $y_0=y(t_0)=\hat{y}$.
\item 根据所选用的数值方法,计算出  $y_1$  .这里的方法可以是Forward/Backward Euler,也可以是梯形公式,还可以是其他的一些方法.
\item 将第 2 步的操作重复应用到由  $y_i$  计算  $y_{i+1}$  的过程. 直到  $t_i\ge T$  时,停止.
\end{enumerate}

\subsubsection{需要注意}

在上面的第 2 步使用 Backward Euler 或者梯形公式时,由于它们均为隐式方法,每一步都需要求解一个线性或非线性方程.我们考虑简单的常微分方程  $y'=f(y)$  ,以Backward Euler为例,数值方法为 $y_{n+1}=y_{n}+hf(y_{n+1})$  即  $y_{n+1}-hf(y_{n+1})=y_{n}$  

\begin{itemize}
\item 
\end{itemize}

<ul><li>若  $f(y)=\lambda y$  ,则  $y_{n+1}=\frac{1}{(1-h\lambda)}y_n$  </li><li>若  $f(y)=A y$  ,且A为矩阵,f和y为向量时,那么我们需要求解线性方程组  $(I-hA)y_{n+1}=y_n$  来得到新的  $y_{n+1}$  </li><li>若  $f(y)=y^2$  ,则  $y_{n+1}-hy^2_{n+1} = y_n$ 这是一个非线性方程,通常需要迭代法求得数值解.</li></ul>这也就意味着,通常情况下如果采用隐式方法会比显式方法的运算复杂度更高.因此,\textbf{在没有特殊要求的前提下},我们\textbf{更偏向于使用显式方法求解常微分方程}.(当然,我们会在后面的章节中具体讨论,哪些特殊情况下必须或者更偏向于使用隐式方法)



更多方法

也正是因为上面的原因,Karl Heun将梯形公式改进成了显式方法----Heun方法.它的核心思想就是用Forward Euler方法求出  $\tilde{y}_{n+1}=y_n+hf(y_n,t_n)$ ,然后带入到梯形公式的右边,  $y_{n+1}=y_n+\frac{h}{2}\left(f(y_n,t_n)+f(\tilde{y}_{n+1},t_{n+1}) \right)$ .这样整个过程的每一步都是显式方法,从而避免了任何可能出现的解方程的过程.当然,有些资料上面也会这样描述Heun方法:

 $k_1=f(y_n,t_n)$  

 $k_2=f(y_n+hf(y_n,t_n),t_n+h))=f(y_n+hk_1,t_n+h)$  

 $y_{n+1}=y_n+\frac{h}{2}(k_1+k_2)$  

事实上,这种形式是另一种更有名的方法的二阶特例,即龙格-库塔法(Runge-Kutta).例如,我们最常见的RK4的形式为

 $k_1=f(y_n,t_n)$  

 $k_2=f\left(y_n+\frac{h}{2}k_1, t_n+\frac{h}{2}\right)$  

 $k_3=f\left(y_n+\frac{h}{2}k_2, t_n+\frac{h}{2}\right)$  

 $k_4=f\left(y_n+hk_3, t_n+h\right)$  

 $y_{n+1}=y_{n}+\frac{h}{6}(k_1+2k_2+2k_3+k_4)$  

当然,关于不同阶数RK方法的系数推导以及误差分析,已经超出了这个专题的范围,有兴趣的小伙伴可以留言,我会单独开一个专题来讨论.



最后一个例子

我们来尝试着用上面的一些\textbf{显式方法}来求解一个非线性常微分方程:

\begin{equation}
\begin{cases}
 y'(t) = y-\frac{1}{2}e^{\frac{t}{2}}\cdot\sin(5t)+5e^{\frac{t}{2}}\cdot\cos(5t), 0\le t\le \pi\\     y(0)=0
\end{cases}
\end{equation}

我们分别用了 Forward Euler,Heun和RK4方法,其中,Python 的科学计算包 \verb|scipy.integrate.solve_ivp|   提供了例如RK45(即我们上面提到的RK4),BDF(后面我们会专门讨论它)以及其他的一些常用方法.

有兴趣的小伙伴可以尝试着自己调整步长参数和方法,来看看误差的变化.


三种方法的结果如图,另外为了作为参照,我们选用了非常小的时间间隔,用RK45模拟出了一个解析解.

注:这里我自己写了Forward Euler(或者叫Euler Forward)和Heun方法如下:


    
最后一点

我们在这个例子中可以观察到一点,这三种方法的误差排序大体上是 Forward Euler>Heun>RK4. 那么,下一章我们会来具体分析这其中的原因,也就是所谓的局部残差(Local truncation error)和全局误差(Global error).
