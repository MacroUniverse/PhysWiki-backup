% Mathematica 基础

\begin{issues}
\issueDraft
\end{issues}

\pentry{复数\upref{CplxNo}}

\begin{itemize}
\item 快捷键 \verb|Shift + Enter| 执行选中的行.
\item 一行输入多个表达式用分号, 分号也可以用于抑制输出. \verb|a = 3; b = 5; a b|
\item 函数 \verb|N[表达式, 有效位数]| 把表达式的结果变为数值, 四舍五入到指定的有效数字, 第二个参数可省略. 例如 \verb|N[Pi]| 计算圆周率的前 5 位, \verb|N[Pi, 1000]| 计算 1000 位.
\item 用 \verb|(*注释*)| 写注释. 注释不会被执行. 也可以选中后按 \verb|Alt + /|.
\item 快捷键 \verb|Ctrl+L| 可以调用最近的输入或输出.
\item 导数 \verb|D[Sin[x]]|, 偏导数 \verb|D[Sin[x+y],x]|.
\item 不定积分 \verb|Integrate[1/(x^3-1),x]|, 定积分 \verb|Integrate[f,{x, xmin, xmax}]|, 重积分 \verb|Integrate[f,{x, xmin, xmax},{y, ymin, ymax},...]|.
\item 简单的画图 \verb|Plot3D[Sin[y+Sin[3x]], {x,-3,3}, {y,-3,3}]|.
\end{itemize}

常数都以大写字母开头.
\begin{lstlisting}[language=Mathematica]
Pi // 圆周率
E // 自然对数
Degree // 1° 角的弧度
I // 虚数单位
Infinity // 无穷
\end{lstlisting}
例如 \verb|Sin[20 Degree] // N| 计算 $\sin(\pi/9)$ 的数值结果.

\begin{itemize}
\item 输入符号用 \verb|Esc| 键搜索, 例如搜索 \verb|pi| 按回车可以插入 \verb|𝜋|, 相当于 \verb|Pi| 又例如搜索 \verb|inf| 按回车可以插入 \verb|∞|, 相当于 \verb|Infinity|. 如果你知道 TeX, 也可以搜索 tex 命令, 例如 \verb|\varphi| 插入 \verb|φ|.
\item 使用 \verb|Ctrl+_| 输入下标. 相当于 \verb|Subscript[x, y]|.
\item 使用 \verb|Ctrl+/| 输入分式, 相当于 \verb|x / y|.
\end{itemize}


\subsection{算符}
\begin{itemize}
\item \verb|!=|
\item 连接字符串 \verb|"abc" <> "defg" <> "hij"|.
\end{itemize}

\subsection{常用函数}
\begin{itemize}
\item 函数参数中, 下标在上标之前给出, 例如勒让德多项式 $P_n^m(x)$ 为 \verb|LegendreP[n, m, x]|.
\item \verb|a + I b| 表示复数.
\item \verb|Re[z], Im[z]| 计算复数的实部和虚部.
\item \verb|Conjugate[z]| 复共轭
\item \verb|Abs[z]| 复数的绝对值
\item \verb|Arg[z]| 复数的幅角
\item \verb|Factor[x^25-1]| 因式分解
\item \verb|Expand[(a+b)^3]| 多项式展开
\end{itemize}

