% 物理量和单位转换
% 量纲|物理量|单位|转换

\subsection{量纲和单位制}
\footnote{参考 Wikipedia \href{https://en.wikipedia.org/wiki/Dimensional_analysis}{相关页面}.}我们知道物理公式中有不同的物理量, 例如时间, 长度, 电荷, 等. 我们把这些概念称为\textbf{量纲(physical dimension 或 dimension)}\footnote{注意 dimension 也可以表示空间的维度, 二者是完全不同的概念.}. 对公式中不同物理量之间量纲关系进行分析叫做\textbf{量纲分析(dimensional analysis)}.

同一个量纲又可以有不同的\textbf{单位(unit of measurement 或 unit)}. 例如长度量纲的单位可以有 “英尺”, “米”, “公里”, 等.

不同量纲可以分别取指数再相乘, 得到新的量纲. 例如通过公式 $v = x/t$, 就可以知道速度的量纲由长度的 1 次方和时间的 $-1$ 次方相乘而来. 如果一个量没有任何单位, 那么我们就说它具有\textbf{单位量纲}或者\textbf{无量纲}.

\subsection{单位转换}

\pentry{几何矢量\upref{GVec}}

在物理中我们首先要区分两类变量, 一种是只有数值没有单位, 例如“甲的身高是乙的 $1.27$ 倍”. 另一种既有数值也有单位, 例如“甲的身高是 $1.8\Si{m}$”. 我们把前者叫做一个\textbf{数}, 把后者叫做\textbf{物理量}.

为了方便理解, 我们这里把物理量和一维几何矢量\upref{GVec}的部分性质做一个类比: 一个一维几何矢量本身也不能用一个数描述, 而是需要先选取一个矢量基底, 然后用一个数(即坐标)乘以这个基底才能表示这个矢量. 例如对于同一个矢量, 若甲选择的基底是乙的基底的两倍, 那么甲的坐标将会是乙的一半.

用公式表示, 假设某个一维矢量是 $\bvec v$, 两种不同的基底分别是 $\bvec u_1$ 和 $\bvec u_2$, 则
\begin{equation}\label{Units_eq1}
\bvec v = x_1 \bvec u_1 = x_2 \bvec u_2
\end{equation}
其中 $x_1$ 和 $x_2$ 是数学量. 现在, 如果我们已知 $\bvec u_1$ 和 $\bvec u_2$ 的关系, 例如
\begin{equation}\label{Units_eq2}
\bvec u_1 = c \bvec u_2
\end{equation}
其中 $c$ 也是一个数学量. 我们就可以直接将这个关系代入\autoref{Units_eq1} 中的 $\bvec u_1$, 得
\begin{equation}
x_1 (c \bvec u_2) = x_2 \bvec u_2
\end{equation}
现在等式两边都使用同样的基底, 于是他们的坐标也必定相同, 即
\begin{equation}
x_2 = c x_1
\end{equation}

同样, 若基底间的关系是
\begin{equation}
\bvec u_2 = b\bvec u_1
\end{equation}
显然有 $b = 1/c$, 代入\autoref{Units_eq1} 中的 $\bvec u_2$ 得
\begin{equation}
x_1 = b x_2
\end{equation}

\begin{example}{长度}\label{Units_ex1}
我们来考虑一个表示长度的物理量 $L$. 在我们确定单位(类比矢量基底)以前, 它不能使用任何数表示. 现在规定单位(即基底) 为 $l_1 = 1\Si{cm}$, $l_2 = 1\Si{m}$, 且有
\begin{equation}
l_2 = 100 l_1
\end{equation}
若令 $L = 2\Si{m}$, 则
\begin{equation}
L = 2 l_2 = 2 (100 l_1) = 200 l_1 = 200 \Si{cm}
\end{equation}
或者
\begin{equation}
L = 2 \Si{m} = 2 (100 \Si{cm}) = 200 \Si{cm}
\end{equation}
若令 $L = 5\Si{cm}$, 则
\begin{equation}
L = 5 l_1 = 5 \qty(\frac{1}{100} l_2) = 0.05 \Si{m}
\end{equation}
或者
\begin{equation}
L = 5 \Si{cm} = 5 \cdot \frac{\Si{m}}{100} = 0.05 \Si{m}
\end{equation}
\end{example}

与一维矢量不同的是, 两个相同的物理量可以相除并得到一个无量纲的数, 且这个数不依赖于单位制的选取. 这么做的时候, 可以类比为将两个矢量的模长相除. 例如
\begin{equation}
\frac{200\Si{cm}}{1\Si{m}} = \frac{2\Si{m}}{1000\Si{mm}} = \frac{2\Si{m}}{1\Si{m}} = 2
\end{equation}
所以, 另一种单位转换的办法, 是对每种物理量先凑出一个等于 1 的无量纲分数,称为转换常数. 例如 $\beta_x = 1\Si{m}/(1000\Si{mm}) = 1$, 使得分母是原单位, 分子是新单位, 然后将它乘以需要转换的物理量. 例如
\begin{equation}
2000\Si{mm} = 2000\Si{mm} \times \frac{1\Si{m}}{1000\Si{mm}} = 2\Si{m}
\end{equation}
这与\autoref{Units_ex1} 中的方法是完全等效的. 对于不同量纲下的任意物理量 $\Omega$ 和 $\Omega'$, 和转换常数 $\beta_\Omega$ 可以把该过程记为
\begin{equation}
\Omega' = \beta_\Omega \Omega
\end{equation}

例: 厘米—克—秒单位制\upref{CGS}.

\subsection{物理公式的单位制转换}
使用不同的单位制时, 物理公式可能保持不变也可能会增减一些常数.

\begin{example}{}
无论用国际单位制还是厘米—克—秒(CGS)单位制\upref{CGS}, 都有 $x = vt$. 但如果使用一个新的单位制, 令 $x' = \beta_x x$, $v' = \beta_v v$, $t' = \beta_t t$, 代入得
\begin{equation}
x' = \frac{\beta_v\beta_t}{\beta_x} v' t'
\end{equation}
若从国际单位转换成 CGS 单位, 则 $\beta_x = 0.01 \Si{m/cm}$, $\beta_t = 1$, $\beta_v = 0.01 \Si{m/cm}$, 恰好能使公式中的转换常数都消去使其保持不变. 但若时间单位改用小时, 则 $\beta_t = 3600 \Si{s/h}$. 代入上式变为
\begin{equation}
x' = \beta_v v' t'
\end{equation}
这样,公式中就多出了一个常数.
\end{example}
在这个简单的例子中, 所有的转换常数虽然具有不同的数值, 但本质上都还是无量纲的 1. 但在一些单位制中, 同一个物理量的量纲却有可能不同, 所以转换常数也不等于 1.
\begin{example}{}
例如在高斯单位制\upref{GaussU}中, 为了让国际单位制的库伦定律
\begin{equation}\label{Units_eq3}
F = \frac{1}{4\pi\epsilon_0}\frac{q_1 q_2}{r^2}
\end{equation}
变为
\begin{equation}\label{Units_eq4}
F' = \frac{q'_1 q'_2}{r'^2}
\end{equation}
已知高斯单位制中力和长度的单位为 $\Si{g\cdot cm/s^2}$ 和 $\Si{cm}$. 易得 $q'$ 的量纲为 $\Si{\sqrt{g\cdot cm^3}/s}$.

把 $F = \beta_F F'$, $q = \beta_q q'$, $r = \beta_x r'$ 代入\autoref{Units_eq3} 得
\begin{equation}
F' =  \frac{\beta_q^2}{4\pi\epsilon_0\beta_x^2 \beta_F}\frac{q'_1 q'_2}{r'^2}
\end{equation}
对比\autoref{Units_eq4} 得 $\beta_q^2/(4\pi\epsilon_0\beta_x^2 \beta_F) = 1$, 即
\begin{equation}
\beta_q = \sqrt{4\pi\epsilon_0 \beta_F} \beta_x \approx 3.336\times 10^{-10} \Si{C\cdot s/\sqrt{g\cdot cm^3}}
\end{equation}
\end{example}

另见无量纲的物理公式\upref{NoUnit}.

\subsection{不同物理量的运算}
首先要注意, 把两个具有不同量纲的物理量相加减没有任何意义, 把一个无量纲的数与一个物理量相加减也没有意义. 例如把长度和时间相加, 把速度与数字 2 相加都是没有意义的.

然而, 几个相同或不同量纲的物理量相乘或相除却是很常见的. 例如长度乘以长度等于体积, 长度除以时间等于速度. 相乘或相除后所得物理量的单位也可以写成若干个基本单位相乘或者相除, 最终表示成若干个单位的幂相乘的形式(可以用 “$\cdot$” 表示相乘, 也可以省略). 例如
\begin{equation}
2\Si{m} \times 3\Si{m} = 6\Si{m\cdot m} = 6\Si{m}^2
\end{equation}
\begin{equation}
6\Si{m} / 3\Si{s} = 2 \Si{m/s} = 2 \Si{m \cdot s^{-1}}
\end{equation}
有时候我们会给这些复合单位一个新的记号, 例如根据牛顿第二定律(\autoref{New3_eq1}~\upref{New3}), 力的国际单位应该是 $\Si{kg\cdot m\cdot s^{-2}}$, 但我们也可以定义一个等效单位 $\Si{N}$ (牛).


