% 磁通量
% 磁通量|闭合曲面|矢势|环路积分

定义通过某曲面的磁通量为
\begin{equation}
\Phi  = \int \bvec B \vdot \dd{\bvec a}
\end{equation}
利用磁场矢势%未完成:链接
及旋度定理, %未完成:链接
磁通量变为
\begin{equation} \label{BFlux_eq2}
\Phi  = \int \curl \bvec A \vdot \dd{\bvec a}  = \oint \bvec A \vdot \dd{\bvec r}
\end{equation}
另外, 由于磁场的散度为零, 根据高斯定理, 任何闭合曲面的磁通量都是 0. 用另一种方式来理解: 如果选定一个闭合回路, 以该闭合回路为边界的任何曲面的磁通量都相等.

\subsection{闭合线圈的磁通量}

如何计算一个闭合线圈对自己产生的磁通量呢? 利用磁场矢势公式
\begin{equation}
\bvec A \qty(\bvec r) = \frac{\mu_0 I}{4 \pi} \oint \frac{\dd{\bvec r'}}{\abs{\bvec r - \bvec r'}}
\end{equation}
注意在该积分中, $\bvec r$ 视为常量, 积份完后, 积分变量 $\bvec r$ 消失. 现在根据\autoref{BFlux_eq2} 再次将上式对 $\bvec r$ 进行同一环路积分得到磁通量
\begin{equation}
\Phi  = \frac{\mu_0 I}{4\pi} \oint\oint \frac{\dd{\bvec r'} \dd{\bvec r}}{\abs{\bvec r - \bvec r'}}
\end{equation}
