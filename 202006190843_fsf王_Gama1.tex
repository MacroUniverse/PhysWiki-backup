% 余元公式
\subsubsection{余元公式:}
\begin{align}
Γ(x)Γ(1-x)=\frac{\pi}{\sin  \pi x}
\end{align}

\begin{lemma}{伽马函数的三种等价定义:}
\begin{align}
&欧拉定义:\displaystyle Γ(z):=∫^{∞}_{0}e^{-t}t^{z-1}\text dt\\ 
  &高斯定义:\displaystyle Γ(z):=\lim_{n \rightarrow ∞}{\frac{n!n^z}{z(z+1)⋯ (z+n)}}=\prod_{n}^{∞}\frac{\left( 1+\frac1n \right)^z}{\left( 1+\frac zn \right)}\\ 
  &魏尔斯特拉斯定义:\displaystyle \frac1{Γ(x)}:=ze^{γ z}\prod_{n}^{∞}\left[ \left( 1+\frac{z}{n}\right)e^{-\frac{z}{n}}  \right]
\end{align}
\end{lemma}
首先是 \footnote{这个证法来自于《微积分学教程》第二卷 第408目}$(\cos z+i\sin z)^n=\cos\,nz+i\sin nz$,\\
所以有$\sin nz=n(\cos^{n-1}z)\sin z-\frac{n(n-1)(n-2)}{1· 2·3}(\cos^{n-3}z · \sin z)+⋯$\\
仅仅考虑 $n=2k+1$ 的情况,然后考虑到 $\cos^2 z=1-\sin^2z$ ,\\
则必有 $\sin(2k+1)z=\sin\,z· P(\sin^2z) $.\\
其中 $P(\sin^2z)$ 代表以$ \sin^2z$ 为变量的 $\mathbf{k}$ 次多项式.\\
考虑到在 $z_m=\frac{mπ}{2k+1}\,,m=1,2,...,k $时,\\
对于每一个 $z_m$ ,都有$ \sin(2k+1)z=0,\sin\,z\ne 0 $\\
则可知每一个$ z_m $都是$ P(\sin^2z)$ 的根
即得 $P(\sin^2z)=A(1-\frac{\sin^2z}{\sin^2z_1})(1-\frac{\sin^2z}{\sin^2z_2})...(1-\frac{\sin^2z}{\sin^2z_k}) $.
考虑到 $\displaystyle  A=P(0)=\lim_{z\rightarrow0}\frac{\sin(2k+1)z}{\sin z}=2k+1 $\\
则有 $P(\sin^2z)=(2k+1)(1-\frac{\sin^2z}{\sin^2z_1})(1-\frac{\sin^2z}{\sin^2z_2})...(1-\frac{\sin^2z}{\sin^2z_k}) $\\
也即 $\sin(2k+1)z=\sin z\cdot\,(2k+1)(1-\frac{\sin^2z}{\sin^2z_1})...(1-\frac{\sin^2z}{\sin^2z_k}) $\\
令 $z=\frac{x}{2k+1} $,则为 :$\sin\,x=\sin(\frac x{2k+1})\cdot\,(2k+1)(1-\frac{\sin^2\frac x{2k+1}}{\sin^2z_1})...(1-\frac{\sin^2\frac x{2k+1}}{\sin^2z_k})$\\ 
现在把$ \sin x $乘积分成两部分,即截取 $V^a_k=(1-\frac{\sin^2\frac x{2k+1}}{\sin^2z_{a+1}})...(1-\frac{\sin^2\frac x{2k+1}}{\sin^2z_k}) $\\
而将前面的部分记为 :$U^k_a=(2k+1)\sin(\frac x{2k+1})(1-\frac{\sin^2\frac x{2k+1}}{\sin^2z_1})...(1-\frac{\sin^2\frac x{2k+1}}{\sin^2z_{a}}) $\\
其中 $0<a<k $.\\
由  $\displaystyle \lim_{k\rightarrow\infty}(2k+1)\sin\frac{x}{2k+1}=x$\\
$\displaystyle \lim_{k\rightarrow\infty}\left( \frac{\sin\frac{x}{2k+1}}{\sin\frac{m\pi}{2k+1}} \right)^2=\left( \frac{x}{m\pi} \right)^2\,,m=1,2,...,k $\\
可得 $\displaystyle U_a=\lim_{k\rightarrow\infty}U^k_a=x(1-\frac{x^2}{\pi^2})...(1-\frac{x^2}{a^2\pi^2}) $\\
现在考虑 $V^k_a $.\\
考虑到在 $0<\varphi<\frac\pi2 $时, $\frac2\pi\varphi<sin\varphi$ ,
则有:$\sin^2\frac {h\pi}{2k+1}>\frac4\pi\frac{(h\pi)^2}{(2k+1)^2},h=a+1,...k $
而$ \sin^2\frac x{2k+1}<\frac{x^2}{(2k+1)^2} $是易得的.\\
于是 $1>V^k_a>(1-\frac{x^2}{4(a+1)^2})...(1-\frac{x^2}{4k^2}) $
因为最开始并没有限定 a 的具体数值,所以总可以找到 $a=a_0$ ,使得$ 4(a_0+1)>x^2 $
\begin{lemma}{}
对于充分大的n而言,$\displaystyle  \prod_{n=1}^{∞}(1+a_n) $收敛的充要条件是  $\displaystyle  ∑_{n=1}^{∞}{a_n}$ 收敛.
\end{lemma}
当然,不需要一定从 n=1 开始,去掉有限项都不改变收敛或发散的结果.\\
\begin{lemma}{}
若无穷乘积$\displaystyle  \prod_{n=1}^{∞}p_n$ 收敛,那么总有充分大的 m 使得 $\displaystyle  \lim_{m\rightarrow∞}\prod_{n=m+1}^{∞}p_n=1 $.
\end{lemma}
当然,这要求$ p_n\ne0 $,所有这时应该把$ x=0,\pm\pi,\pm 2\pi... $分出来单独考虑.\\
由于 $\displaystyle \sum_{h=a_0+1}^{\infty}{\frac{x^2}{4h^2}} 收敛,所以 \lim_{a_0\rightarrow\infty}\lim_{k\rightarrow\infty}(1-\frac{x^2}{4(a_0+1)^2})...(1-\frac{x^2}{4k^2})=1 $\\
即为 $\displaystyle \sin\,x=\lim_{a\rightarrow\infty}U_a=x\cdot \prod_{n=1}^{\infty}(1-\frac{x^2}{n^2\pi^2}) $\\
这里可以直接看出,这对于 $x=0,\pm\pi,\pm 2\pi... $也是成立的.\\
另外,有 $\displaystyle \sin\pi x=\pi x\cdot \prod_{n=1}^{\infty}(1-\frac{x^2}{n^2}) $.\\
考虑到 $\displaystyle \Gamma(x)=\frac1x\prod_{n=1}^{\infty}\frac{(1+\frac1n)^{x}}{1+\frac xn} $(威尔斯特斯拉定义)以及$ \Gamma(1+x)=x\Gamma(x) $\\
则有 $\displaystyle  \Gamma(1-x)=-x\Gamma(-x)=\prod_{n=1}^{\infty}\frac{(1+\frac1n)^{-x}}{1-\frac xn} $\\
于是 $\displaystyle  \Gamma(x)\Gamma(1-x)=\frac{1}{x}\cdot \prod_{n=1}^{\infty}\frac1{(1-\frac{x^2}{n^2})}=\frac\pi{\sin\pi x}$ .
