% 尝试
我们知道,起初在不加规范限定的情况下,我们的传播子方程是无解的.即方程
\[
\left(-k^2 g_{\mu\nu}+k_\mu k_\nu\right)\tilde D^{\nu\rho}_F
=i\delta^\rho_\mu
\]
中,由于矩阵 $-k^2 g_{\mu\nu}+k_\mu k_\nu$ 行列式为零,所以这个矩阵没有逆.为了弥补这缺陷,我们在拉格朗日密度中加入规范限定项,并将它加入计算费曼传播子的方程中.我们在这个水帖中就来展示这个技术.

我们以 $R_\xi$ 规范开始,加入规范限定的方程为
\[
\left(-k^2 g_{\mu\nu}+\left(1-\frac{1}{\xi}\right)k_\mu k_\nu\right)\tilde D^{\nu\rho}_F
=i\delta^\rho_\mu
\]
这意味着我们要求矩阵 $-k^2 g_{\mu\nu}+\left(1-\frac{1}{\xi}\right)k_\mu k_\nu$ 的逆.由于 $-k^2$ 是标量,所以我们又可将矩阵变形为
\[
g_{\mu\nu}-\left(1-\frac{1}{\xi}\right)\frac{k_\mu k_\nu}{k^2}
\]
但是记得之后要把它恢复过来.根据经验,我们假设逆矩阵具有如下形式(你写不出更多项了!)
\[
g^{\mu\nu}+a\frac{k^\mu k^\nu}{k^2}
\]
$a$ 就是我们的待定系数,将两式相乘,于是有
\[\begin{split}
&\left(g_{\mu\nu}-\left(1-\frac{1}{\xi}\right)\frac{k_\mu k_\nu}{k^2}\right)
\left(g^{\nu\rho}+a\frac{k^\nu k^\rho}{k^2}\right)\\
=&g_{\mu\nu}g^{\nu\rho}+
ag_{\mu\nu}\frac{k^\nu k^\rho}{k^2}-
g^{\nu\rho}\left(1-\frac{1}{\xi}\right)\frac{k_\mu k_\nu}{k^2}
-a\frac{k^\nu k^\rho}{k^2}\left(1-\frac{1}{\xi}\right)\frac{k_\mu k_\nu}{k^2}\\
=&\delta^\rho_\mu+\frac{k_\mu k^\rho}{k^2}
\left(
a-1+\frac{1}{\xi}-a+\frac{a}{\xi}
\right)
\end{split}\]
第二项应该等于零,于是得到
\[
a=\xi-1
\]
因此费曼传播子应该为
\[
g^{\mu\nu}+(\xi-1)\frac{k^\mu k^\nu}{k^2}
\]
补上省去的 $-k^2$ 和 $i$,于是最终有
\[
\frac{-i}{k^2}\left(g^{\mu\nu}+(\xi-1)\frac{k^\mu k^\nu}{k^2}\right)
\]

下面我们将罗列六个常见规范以及它们的传播子(不完整形式)

1、 $R_\xi$ 规范
\[
\mathcal{L}'=-\frac{1}{2\xi}(\partial_\mu A^\mu)^2
\]
\[
g_{\mu\nu}+(\xi-1)\frac{k_\mu k_\nu}{k^2}
\]
2、洛伦兹规范
\[
\partial_\mu A^\mu=0
\]
\[
g_{\mu\nu}-\frac{k_\mu k_\nu}{k^2}
\]
3、库伦规范
\[
\nabla \vec A=0
\]
\[
g_{\mu\nu}-\frac{k_\mu k^T_\nu+k^T_\mu k_\nu}{k^2_T}+\frac{k_\mu k_\nu}{k^2_T}
\]
4、轴向规范
\[
nA=0
\]
\[
g_{\mu\nu}-\frac{k_\mu n_\nu+n_\mu k_\nu}{kn}+\frac{k_\mu k_\nu}{(kn)^2}n^2
\]
5、伪轴向规范
\[
\mathcal{L}'=-\frac{\beta^2}{2}(nA)^2
\]
\[
g_{\mu\nu}-\frac{k_\mu n_\nu+n_\mu k_\nu}{kn}+\frac{k_\mu k_\nu}{(kn)^2}\left(n^2+\frac{k^2}{\beta^2}\right)
\]
6、平面规范
\[
\mathcal{L}'=\frac{1}{2\xi}nA\partial^2nA
\]
\[
g_{\mu\nu}-\frac{k_\mu n_\nu+n_\mu k_\nu}{kn}+\frac{k_\mu k_\nu}{(kn)^2}n^2(1-\xi) 
