% TT规范
在源的外面,我们有$T_{\mu\nu} = 0$,所以我们有如下方程
\begin{equation}\label{TTGaug_eq1}
\Box \bar h_{\mu\nu} = 0 ~.
\end{equation}
其中$\Box = - (1/c^2) \partial_0^2 +\nabla^2$. 这个方程说明了引力波以光速传播。因为\autoref{Geomet_eq3}没有完全确定规范,在没有源的地方我们能够极大地固定规范。

如果$\Box \xi_\mu = 0$,那么变换$x^\mu \rightarrow x^\mu+\xi^\mu$保持$\partial^\nu \bar h_{\mu\nu}$不变。从$\Box \xi_\mu = 0$可以推出
\begin{equation}
\Box \xi_{\mu\nu} = 0~, \quad \xi_{\mu\nu} \equiv \partial_{\mu} \xi_\nu +\partial_\nu \xi_\mu - \eta_{\mu\nu} \partial_\rho\xi^\rho~. 
\end{equation}
我们可以使用TT规范
\begin{equation}
h^{0\mu} = 0~, \quad h^i_i = 0~, \quad \partial^j j_{ij} = 0~.
\end{equation}
加上洛伦兹条件把十个自由度的对称矩阵$h_{\mu\nu}$变为六个自由度。剩余的四个满足$\Box \xi_\mu = 0$的$\xi^\mu$把它降到了两个自由度。我们把TT规范下的度规记作$h_{ij}^{TT}$. 

注意,TT规范在含有源的时候是不能取的,这是因为$\Box \bar h_{\mu\nu} \neq 0$. \autoref{TTGaug_eq1}有如下的平面波解
\begin{equation}
h_{ij}^{TT} (x) = e_{ij} (\mathbf k) e^{ikx} ~.  
\end{equation}
其中$k^\mu = (\omega/c,\mathbf k)$, $\omega/c = |\mathbf k|$。$e_{ij}(\mathbf k)$被称作极化张量。沿着z方向前进的度规张量是
\begin{equation}
h_{ij}^{TT} (t,z) = 
\begin{pmatrix}
h_+ & h_\times & 0 \\
h_\times & - h_+ & 0 \\
0 & 0 & 0
\end{pmatrix} \cos [\omega (t - z/c)]~. 
\end{equation}
更简要地,我们可以写成
\begin{equation}
h_{ab}^{TT} (t,z) = 
\begin{pmatrix}
h_+ & h_\times   \\
h_\times & - h_+   
\end{pmatrix} \cos [\omega (t - z/c)]~. 
\end{equation}
其中$a,b = 1,2$. 代入到度规中,我们可以写成
\begin{equation}
\begin{aligned}
ds^2 & = - c^2 dt^2 + dz^2 + \{ 1+ h_+ \cos [\omega(t-z/c)] \} dx^2 \\
& + \{ 1-h_+ \cos [\omega(t-z/c)] \} dy^2 + 2 h_\times \cos[\omega(t-z/c)] dx dy~.
\end{aligned}
\end{equation}
给定一个在源外面朝着$\hat{\bf{n}}$方向传播的平面波解$h_{\mu\nu}(x)$.假设我们已经加入了洛伦兹规范但还没有加上TT规范,我们可以通过如下步骤找到TT规范下的波。

首先,我们引入如下的张量
\begin{equation}
P_{ij} (\hat{\bf{n}} ) = \delta_{ij} - n_in_j~.
\end{equation}
这个张量是对称,横向的( $ n^iP_{ij}(\hat{\bf{n}}) = 0 $ ),是一个投影 ( $ P_{ik} P_{kj} = P_{ij} $ ),它的迹是$P_{ii} = 2$。使用上面定义的$P_{ij}$, 我们可以构建如下的投影算符
\begin{equation}
\Lambda_{ij,kl} ( \hat{\bf{n}} )  = P_{ik} P_{jl} - \frac{1}{2} P_{ij} P_{kl} ~. 
\end{equation}
这个投影算符仍然是一个投影,因为
\begin{equation}
\Lambda_{ij,kl} \Lambda_{kl,mn} = \Lambda_{ij,mn} ~. 
\end{equation}
并且它对于所有的指标都是横向的$ n^i \Lambda_{ij,kl} = 0 $, $ n^j \Lambda_{ij,kl} = 0 $, 它对于$(i,j)$和$(k,l)$指标是无迹的
\begin{equation}
\Lambda_{ii,kl} = \Lambda_{ij,kk} = 0 ~. 
\end{equation}
给定引力波的传播方向,我们可以写出$\Lambda_{ij,kl} (\hat{ {\bf n} })$的显性表达式
\begin{equation}
\begin{aligned}
\Lambda_{ij,kl} (\hat{ {\bf n} }   ) & = \delta_{ik} \delta_{jl} - \frac{1}{2} \delta_{ij} \delta_{kl} - n_j n_l \delta_{ik} - n_i n_k \delta_{jl} \\
& + \frac{1}{2} n_k n_l \delta_{ij} + \frac{1}{2} n_i n_j \delta_{kl} + \frac{1}{2} n_i n_j n_k n_l ~.
\end{aligned}
\end{equation}
我们把这个张量叫做Lambda张量。给定一个在洛伦兹规范下,但是还不在TT规范下的平面波$h_{\mu\nu}$,我们可以写出TT规范下的平面波如下.
\begin{equation}
h_{ij}^{TT} = \Lambda_{ij,kl} h_{kl} ~.
\end{equation}
从等式右边可以看出,它对于$(i,j)$指标是横向,无迹的。从$h_{\mu\nu}$是真空中波动方程的一个解和它在洛伦兹规范下可以看出$\Box h_{ij}^{TT} = 0$. 

给定任意的对称张量$S_{ij}$,我们定义它的横向无迹部分为
\begin{equation}
S^{TT}_{ij} = \Lambda_{ij,kl} S_{kl} ~.
\end{equation}
注意到$S^{TT}_{ij}$仍然是对称的。

在TT规范下,$h^{TT}_{ij}$可以做如下展开
\begin{equation}
h^{TT}_{ij}(x) = \int \frac{d^3 k}{(2\pi)^3} (\mathcal A_{ij}(\mathbf k) e^{ikx} + \mathcal A_{ij}^* (\mathbf k) e^{-i k x} ) ~. 
\end{equation}
其中$k^\mu = (\omega/c,\mathbf k)$. $|\mathbf k| = \omega/c = (2\pi f)/c$因此$d^3 k = |\mathbf k|^2 d|\mathbf k| d\Omega = (2\pi/c)^3 f^2 df d\Omega $. 另外$d^2\hat{\bf{n}} = d\cos\theta d\phi$,所以上述方程读作。
\begin{equation}
h^{TT}_{ij}(x) = \frac{1}{c^3} \int^\infty_0 df f^2 \int d^2 \hat{ \bf{n} } ( \mathcal A_{ij} (f,\hat{ \bf{n} }) e^{-2\pi i f(t-\hat{ {\bf n} } \cdot {\bf x} /c  )}  + {c.c} ) 
\end{equation}
注意到在这个表达式里面,只有物理的频率$f>0$进入了展开式。TT规范条件给出了$\mathcal A^i_i (\mathbf k) =0 $以及$k^i \mathcal A_{ij} (\mathbf k) =0 $. 需要注意的是,如果波是由往不同方向传播的波叠加组成的,$h_{ij}(x)$就不能约化到$2\times 2$的矩阵。这一点在我们考虑引力波的随机背景的时候会非常重要。然而,当我们在地球上观测到从单个天体辐射出来的引力波的时候,波的传播方向还是很确定的,我们可以写出如下表达式
\begin{equation}
\mathcal A_{ij} (\mathbf k) = A_{ij} (f) \delta^{(2)} (\hat{  \mathbf n } - \hat{ {\bf n} }_0 )~.
\end{equation}



















