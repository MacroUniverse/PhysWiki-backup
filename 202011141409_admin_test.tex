% test
% test|测试|编辑器

参考 \href{https://docs.julialang.org/en/v1/manual/unicode-input/}{Julia 符号表}和 \href{https://oeis.org/wiki/List_of_LaTeX_mathematical_symbols}{LaTeX 符号表}, 以及 \href{http://www.onemathematicalcat.org/MathJaxDocumentation/TeXSyntax.htm#U}{MathJax 符号表}.

\begin{equation}
\cap, \bigcap, \cup, \bigcup, \vee, \wedge, \int, \iint, \iiint, \oint
\end{equation}
\begin{equation}
\diamond, \ominus, \triangleleft, \triangleright, \Longleftarrow, \Longrightarrow, \iff, \leftrightarrow, \updownarrow, \cdots
\end{equation}
\begin{equation}
\ddots, \top, \bot, \measuredangle
\end{equation}

\subsection{好的}

Another approach is to define angular momentum as the conjugate momentum (also called canonical momentum) of the angular coordinate $\phi$ expressed in the Lagrangian of the mechanical system. Consider a mechanical system with a mass $m$ constrained to move in a circle of radius $a$ in the absence of any external force field. The kinetic energy of the system is
$$
T=\frac{1}{2} m a^{2} \omega^{2}=\frac{1}{2} m a^{2} \dot{\phi}^{2}
$$
And the potential energy is
$$
U=0 .
$$
Then the Lagrangian is
$$
\mathcal{L}(\phi, \dot{\phi})=T-U=\frac{1}{2} m a^{2} \dot{\phi}^{2}
$$
The generalized momentum "canonically conjugate to" the coordinate $\phi$ is defined by
$$
p_{\phi}=\frac{\partial \mathcal{L}}{\partial \dot{\phi}}=m a^{2} \dot{\phi}=I \omega=L
$$
