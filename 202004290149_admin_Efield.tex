% 电场
% 电场|库仑力|万有引力|点电荷|电磁学
\pentry{库仑定律\upref{ClbFrc}}

在经典电磁理论中, 电荷与电荷之间的作用力是通过场的作用产生的. 所以点电荷 $q_1$ 对另一个点电荷 $q_2$ 的库仑力可以理解为 $q_1$ 在其周围产生了电场对 $q_2$ 的作用(反之亦然).

注意在点电荷模型中, 我们假设一个点电荷产生的电场对其自身合力为 0. 另外, 我们一般不讨论点电荷所在位置处的电场强度, 我们可以说电场在该点处没有定义.

电场是可以叠加的, 当空间中有 $N$ 个点电荷, 空间中某点(除了这些电荷所在的点)处的电场等于每个点电荷在该点产生的电场之和. 注意这个求和是矢量相加.

在某个时刻, 空间中的电场是位置 $\bvec r$ 的矢量函数, 即任意一个 $\bvec r$ 对应一个唯一矢量 $\bvec E$. 我们把这个函数记为 $\bvec E(\bvec r)$. 当另一个点电荷处于这个电场中, 它就会受到电场力
\begin{equation}\label{Efield_eq1}
\bvec F(\bvec r) = q \bvec E(\bvec r)
\end{equation}
这个力也是位置的函数, 也就是一个力场\upref{V}.

\subsection{点电荷的电场}
现在我们可以用\autoref{Efield_eq1} 推出点电荷电场的表达式. $q_1$ 对 $q_2$ 的库仑力为(\autoref{ClbFrc_eq1}\upref{ClbFrc})
\begin{equation}\label{Efield_eq3}
\bvec F_{12} = q_2 \bvec E_1(\bvec r_2) = \frac{1}{4\pi\epsilon_0}\frac{q_1 q_2}{\abs{\bvec r_2 - \bvec r_1}^2} \uvec r_{12}
\end{equation}
其中 $\bvec E_1(\bvec r)$ 是 $q_1$ 单独产生的电场分布. 两边除以 $q_2$ 得
\begin{equation}
\bvec E_1(\bvec r_2) = \frac{1}{4\pi\epsilon_0}\frac{q_1}{\abs{\bvec r_2 - \bvec r_1}^2} \uvec r_{12}
\end{equation}

所以, 任意点位于 $\bvec r_i$ 处的点电荷 $q_i$ 产生的电场为
\begin{equation}\label{Efield_eq4}
\bvec E(\bvec r) = \frac{1}{4\pi\epsilon_0}\frac{q_i}{\abs{\bvec r - \bvec r_i}^2} \uvec R_i
\end{equation}
其中单位矢量 $\bvec R$ 由 $\bvec r_i$ 指向 $\bvec r$. 由于电场可叠加, 空间中的 $N$ 个电荷产生的电场为
\begin{equation}\label{Efield_eq2}
\bvec E(\bvec r) = \frac{1}{4\pi\epsilon_0} \sum_{i=1}^N \frac{q_i}{\abs{\bvec r - \bvec r_i}^2} \uvec R_i
\end{equation}

\subsection{连续分布电荷的电场}
当电荷时连续分布时, 我们可以用电荷密度 $\rho(\bvec r)$ 表示其分布情况. \autoref{Efield_eq2} 的求和变为三重积分
\begin{equation}
\bvec E(\bvec r) = \frac{1}{4\pi\epsilon_0} \int \frac{\rho(\bvec r')}{\abs{\bvec r - \bvec r'}^2} \uvec R_i \dd[3]{r'}
\end{equation}

\begin{example}{无限长导线的电场}\label{Efield_ex1}
(未完成)
\end{example}


\begin{example}{均匀带平板的电场}\label{Efield_ex2}
电荷面密度为 $\sigma$,求均匀带电球内外的电场分布.

(未完成)
\end{example}
