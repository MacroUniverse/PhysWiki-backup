% 柯西 - 古萨定理

\pentry{复变函数的积分\upref{CauGou}}

在复变函数的积分\upref{CauGou}里的例子可以发现,有的函数的积分只依赖于积分路径的起点与终点,而与积分路径的形状无关,而有的函数,其积分不仅与积分路径的起点与终点有关,而且与积分路径的形状也有关.深入观察后,可知,前一类函数是解析函数.由此,可提出猜想:解析函数的积分只依赖于积分路径的起点与终点,而与积分路径的形状无关.柯西在1825年给出此定理对猜想作了回答.也就是我们现在要介绍的:柯西-古萨定理.


\begin{theorem}{柯西-古萨定理}
设$G $为复平面上的单连通区域,$C $为$G $内的任意一条围线,如图所示,若$f (z)$在$G $内解析,则
\begin{equation}
\oint_{C} f(z) \mathrm{d} z=0
\end{equation}
\begin{figure}[ht]
\centering
\includegraphics[width=5cm]{./figures/CauGou_1.pdf}
\caption{积分回路} \label{CauGou_fig1}
\end{figure}
\end{theorem}
