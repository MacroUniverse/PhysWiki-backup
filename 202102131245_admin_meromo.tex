% 全纯函数、亚纯函数

\begin{issues}
\issueDraft
\end{issues}

\pentry{复可微}

\footnote{参考 Wikipedia 相关页面.}在复平面的一个开集 $D$ 上, 如果函数 $f(z)$ 处处复可微, 那么它就是一个\textbf{全纯函数(holomorphic function)}也叫做\textbf{解析函数(analytical function)}; 如果除了一些孤立点外处处复可微, 就叫\textbf{亚纯函数(meromorphic function)}.


令复变函数为
\begin{equation}
f(z) = u(x,y) + \I v(x,y)
\end{equation}
如果取一块没有极点的区域, 那么解析要求
\begin{equation}
\pdv{v}{x} - \pdv{u}{y} = 0
\end{equation}
而这恰好是 $x,y$ 平面上的旋度公式.
