% 标量场

\pentry{洛伦兹群\upref{qed1}}
\begin{definition}{}
标量场即洛伦兹变换下不变的场.设在两个惯性系中,标量场分别为$\phi(x)$和$\phi'(x')$,那么有
\begin{equation}
\phi^{\prime}\left(x^{\prime}\right)=\phi(x)
\end{equation}
图景是这样的,对于一个固定点P而言,洛伦兹变换前后对应的坐标不同,场函数形式不同,结果是P的数值不变.
\end{definition}
\begin{corollary}{}
做一个无穷小变换
\begin{equation}
x^{\rho} \rightarrow x^{\prime \rho}=x^{\rho}+\delta x^{\rho}
\end{equation}
并且
\begin{equation}\label{qed2_eq1}
\delta x^{\rho}=\omega_{\sigma}^{\rho} x^{\sigma}=-\frac{i}{2} \omega_{\mu \nu}\left(J^{\mu \nu}\right)_{\sigma}^{\rho} x^{\sigma}
\end{equation}
其中,$\left(J^{\mu \nu}\right)_{\sigma}^{\rho}=i\left(\eta^{\mu \rho} \delta_{\sigma}^{\nu}-\eta^{\nu \rho} \delta_{\sigma}^{\mu}\right)$
显然,令$J^{\mu \nu}=0$,该标量表示(0,0,)能满足标量场的定义.这是个平庸的结果,在场表示里,我们能做到更多.
对于固定点x,我们做场的无穷小变换
\begin{equation}
\delta_{0} \phi \equiv \phi^{\prime}(x)-\phi(x)
\end{equation}
为了找到该表示下的生成元,我们做一阶泰勒展开
\begin{equation}
\delta_{0} \phi=\phi^{\prime}\left(x^{\prime}-\delta x\right)-\phi(x)=-\delta x^{\rho} \partial_{\rho} \phi(x)
\end{equation}
代入\autoref{qed2_eq1} 
\end{corollary}
