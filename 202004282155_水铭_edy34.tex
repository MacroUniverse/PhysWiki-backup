% 超导唯象解释——伦敦方程
\pentry{\upref{EMulPo}}
\subsection{伦敦第一方程}
\begin{equation}
\pdv{t}\bvec J_s=\alpha\bvec E \label{edy34_eq1}
\end{equation}
其中$\bvec J_s$代表超导体中的超导电流密度,$\alpha=n\dfrac {n_se^2}m$,该方程理解,在超导电子运动速度远小于光速$c$的情况下,磁力对超导电子的影响忽略,此时由$\bvec F=m \bvec a$写在电场中的形式$m\pdv{v}{t}=-e\bvec E$推出.
实验证明与超导体的相关性质吻合.为第二方程引出给出先觉条件$E=0$,否则电子速度会不断上升.
\subsection{伦敦第二方程}
\begin{equation}
\curl\bvec J_s=-\alpha\bvec B \label{edy34_eq2}
\end{equation}
由\autoref{edy34_eq1}取旋度加之$\curl E=\pdv{B}{t}$推得.
\subsection{伦敦方程解释超导现象}
\begin{theorem}{迈斯纳效应}

\end{theorem}