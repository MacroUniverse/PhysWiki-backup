% test

\documentclass[UTF8,zihao=-4]{oucart}
\usepackage{indentfirst}
\usepackage{caption}
\usepackage{graphicx, subfig}


\title{个人知识库系统的设计和开发}
\entitle{The Title of the Thesis}
\author{蔡徐坤}
\studentid{123456789}
\advisor{唱跳导师}
\department{信息科学与工程学院}{计算机科学与技术2017级}

\cnabstractkeywords{
出道之后,蔡徐坤大部分精力都投身于新歌的创作和专辑的打造.彼时,他需要随着NINE PERCENT在三个月内完成17场大型巡回见面会,因此写歌的时间必须“挤出来”用.洗澡时、做造型时、飞机上、两个行程间或吃饭的空隙,只要有手机、旋律,任何地方都是他的创作场所;偶尔待在录音室里,甚至成为他的喘息时间.去年,新京报记者见到他时正值午饭,化妆室里传来哼鸣声,“采访完的休息时间,我都可以写一段词.我还年轻,我觉得这都OK.”他曾表示.而《1》的发表同样“违背”偶像市场的规律.蔡徐坤本可以每月发一首,制造更多话题.但他认为,一首首发表并不足以让外界更全面地了解他的音乐风格,“当别人都走得很快,我反而要踏踏实实一步步走.”偶尔听到舆论质疑他没有作品,蔡徐坤也曾犹豫,要不要先发一部分出来?但内心却总有个声音说,“你可以再多做几首不同风格的作品,让大家看到最全面、最好的你,而不是急于求成地去展现自己.”
}{
蔡徐坤,篮球,舞台
}
\enabstractkeywords{
After his debut, Cai devoted most of his energy to the creation of new songs and the creation of albums. At that time, he needed to complete 17 large-scale tour meetings with nine percent in three months, so the time for writing songs had to be ``squeezed out". When bathing, modeling, on the plane, between two itineraries or between meals, as long as there is a mobile phone and melody, anywhere is his creation place; occasionally stay in the studio, even become his breathing time. Last year, when the reporter of the Beijing News saw him, it was lunch time, and there was a hum in the dressing room. ``I can write a paragraph during the rest time after the interview. I'm still young. I think it's OK. " He once said.
}{
  Cai Xukun, Basketball, Dance
}

\begin{document}

\makecover
\makesignature
\makeabstract

\thispagestyle{tableofcontents}
\tableofcontents
\newpage

\pagenumbering{arabic}
\setcounter{page}{1}
% 正文内容
% 建议使用 \input{<文件名>} 指令引用其他文件

\input{includes/section_01}
\input{includes/section_02}


\begin{thebibliography}{99}
\bibitem{ref9}
符志哲. 高光谱遥感图像去条带方法研究[D].南京邮电大学,2020.

\bibitem{ref1}
Fukushima K, Miyake S. Neocognitron: A self-organizing neural network model for a mechanism of visual pattern recognition[M]//Competition and cooperation in neural nets. Springer, Berlin, Heidelberg, 1982: 267-285.

\bibitem{ref2}
Y. LeCun, L. Bottou, Y. Bengio, et al. Gradient-based learning applied to document recognition[J]. Proceedings of the IEEE, 1998, 86(11):2278-2324.

\bibitem{ref3}
Krizhevsky A, Sutskever I, Hinton G E. Imagenet classification with deep convolutional neural networks[J]. Advances in neural information processing systems, 2012, 25: 1097-1105.
\bibitem{ref4}
Simonyan K, Zisserman A. Very deep convolutional networks for large-scale image recognition[J]. arXiv preprint arXiv:1409.1556, 2014.

\bibitem{ref5}
Szegedy C, Liu W, Jia Y, et al. Going deeper with convolutions[C]//Proceedings of the IEEE conference on computer vision and pattern recognition. 2015: 1-9.

\bibitem{ref6}
He K, Zhang X, Ren S, et al. Deep residual learning for image recognition[C]//Proceedings of the IEEE conference on computer vision and pattern recognition. 2016: 770-778.

\bibitem{ref7}
张号逵, 李映, 姜晔楠. 深度学习在高光谱图像分类领域的研究现状与展望[J]. 自动化学报, 2018, 44(6): 961-977.

\bibitem{ref8}
赵诚诚. 基于卷积神经网络的图像分类改进算法的研究[D].南京邮电大学,2020.

\end{thebibliography}




\newpage
\begin{center}
\zihao{3} \textbf{致谢} \\
\end{center}

在论文的最后我想向所有帮助支持过我的亲人、朋友、老师致以崇高的敬意和真诚的感谢,感谢你们在我三年研究生的生活中给予的生活和工作的支持.

2017年9月,我开始了研究生生活,时间飞逝,我即将离开学校,走向社会,在此期间,我要特别感谢XX教授,是两位老师带我进入了XXXX的世界;特别感谢实验室的同学,在我碰到问题的时候伸出援手,帮助我解决问题;最后我要特别感谢我的父母,感谢你们对我学习生涯的资助,感谢你们对我未来决定的支持.




\end{document} 