% Christoffel符号
% 克里斯托费尔|克氏符|克氏|Christoffel|测地线|geodesic|广义相对论|relativity

\pentry{联络形式与结构定理\upref{ConFom}}

流形的特点是局部与我们熟悉的欧几里得空间同胚.尽管我们经常讨论的是流形的内禀性质,不涉及具体的图或者嵌入,但是在实际应用的时候,比如计算广义相对论的现象时,我们却要关心特定图中的数值关系.本节引入的是著名的Christoffel符号,它描述了在特定图中联络的性质.

本节中默认$(M, \nabla)$是一个带仿射联络的流形.


\subsection{Christoffel符号}

对于$M$的任意一个图$(U, \varphi)$,由于$\varphi(U)$是一个欧几里得空间,即实数坐标空间,因此它的光滑向量场集合自带一组标准正交基$\{\frac{\partial}{\partial x^i}\}$.为方便计,我们可以将每个$\frac{\partial}{\partial x^i}$简记为$\partial_i$;点$\varphi(p)\in\varphi(U)$处和$\partial_i$相对应的道路,可以取$c()$

\begin{definition}{Christoffel符号}

\end{definition}
















