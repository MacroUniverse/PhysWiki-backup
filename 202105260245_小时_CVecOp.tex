% 正交曲线坐标系中的矢量算符
% 正交曲线坐标系|矢量分析|梯度|散度|旋度

\pentry{正交曲线坐标系\upref{CurCor}, 梯度\upref{Grad}, 旋度\upref{Curl}}

在位置矢量\upref{Disp}的全微分

令\autoref{CurCor_eq8}~\upref{CurCor} 中三个分数的分子为 $f(u,v,w), g(u,v,w), h(u,v,w)$, 则
\begin{equation}\label{CVecOp_eq4}
\begin{aligned}
\dd{\bvec r} &= \pdv{\bvec r}{u}\dd{u} + \pdv{\bvec r}{v}\dd{v} + \pdv{\bvec r}{w}\dd{w}\\
&= f\dd{u}\,\uvec u + g\dd{v}\,\uvec v + h\dd{w}\,\uvec w
\end{aligned}
\end{equation}

令 $s(u, v, w)$ 为标量函数, $\bvec A(u, v, w)$ 为矢量函数, 且
\begin{equation}
\bvec A(u, v, w) = A_x(u, v, w)\uvec u + A_y(u, v, w)\uvec v + A_z(u, v, w)\uvec w
\end{equation}
那么该坐标系中的梯度\upref{Grad}, 散度\upref{Divgnc}, 旋度\upref{Curl}算符分别为
\begin{equation}
\grad s = \frac{1}{f} \pdv{s}{u}\uvec u + \frac{1}{g}\pdv{s}{v} \uvec v + \frac{1}{h} \pdv{s}{w}\uvec w
\end{equation}
\begin{equation}
\div \bvec A = \frac{1}{fgh}\qty[\pdv{u}(ghA_u) + \pdv{v}(fhA_v) + \pdv{w}(fgA_w)]
\end{equation}
\begin{equation}
\begin{aligned}
&\curl \bvec A = \frac{1}{gh}\qty[\pdv{v}(hA_w) - \pdv{w}(gA_v)]\uvec u\\
&\quad + \frac{1}{fh}\qty[\pdv{w}(fA_u) - \pdv{u}(hA_w)]\uvec v
+ \frac{1}{fg}\qty[\pdv{u}(gA_v) - \pdv{v}(fA_u)]\uvec w
\end{aligned}
\end{equation}
\begin{equation}\label{CVecOp_eq6}
\laplacian s = \frac{1}{fgh}\qty[\pdv{u}\qty(\frac{gh}{f}\pdv{s}{u}) + \pdv{v}\qty(\frac{fh}{g}\pdv{s}{v}) + \pdv{w}\qty(\frac{fg}{h}\pdv{s}{w})]
\end{equation}

\subsection{梯度的推导}
由于正交曲线坐标系中任意一点处 $\uvec u, \uvec v, \uvec w$ 都互相垂直, 所以求该点处梯度的方法和直角坐标系类似: 求出三个方向的方向导数\upref{DerDir}, 并把他们作为梯度的三个分量. 和直角坐标系不同的是, 坐标改变 $\dd{u}$, 位矢 $\bvec r$ 并不是沿 $\uvec u$ 移动 $\dd{u}$ 而是移动


\addTODO{推导参考 \cite{GriffE} 附录.}
