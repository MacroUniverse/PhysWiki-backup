% 切空间(欧几里得空间)

\pentry{矢量空间\upref{LSpace},偏导数\upref{ParDer},光滑映射\upref{SmthM}}

切空间是切向量的集合.从几何直观来说,切向量就是和某个平面或者超平面相切于一点的向量.

在线性代数中,我们将原点默认为所有向量的起点,这样只需要终点就可以表示向量了,因此向量被一一对应到点上.这种方式获得的向量,实际上是更广义的切向量中的一类,即从原点“发射”出去的向量.我们也可以空间中的其它点作为起点,来得到发射出去的向量.当然,如果把向量理解成“具有长度和方向的量”以及此概念的推广,那么向量的起点是无关紧要的;而切向量是区分了起点的.

为什么叫做切向量呢?我们将在讨论光滑流形在欧几里得空间中的嵌入时看到原因.%引用相关词条,写注释时尚未开始该词条,目前预计将该词条命名为“切空间”.

从一个点$P$发射出去的向量,被称为点$P$处的\textbf{切向量}.因此,线性代数中所研究的空间可以看成是原点处的切向量构成的空间,称作\textbf{原点处的切空间}.

如果一个欧氏空间被嵌入到更高维的一个空间中,比如说将$\mathbb{R}^2$以抛物面的形状嵌入到$\mathbb{R}^3$中,那么切空间的意义就非常直观了:点$P$处的切空间就是$\mathbb{R}^3$中的一个与该抛物面切于点$P$的平面,取点$P$作为该平面的原点,那么这个平面也可以看成一个二维实线性空间$\mathbb{R}^2$.

\subsection{切空间的定义}

\subsubsection{几何向量定义}

对于一个欧几里得空间$\mathbb{R}^n$,我们可以简单地把某点$x\in\mathbb{R}^n$处的切向量定义为以该点为起点的向量,而所有该点处的切向量构成一个$x$上的切空间.显然,切空间就是一个线性空间.

几何向量的定义在欧几里得空间里非常直观,但是当我们讨论曲面,或者更一般地说,流形上的切向量时,就不是那么方便了,因此我们会采用以下定义方式.

\subsubsection{道路定义}

如果在欧几里得空间$\mathbb{R}^n$中有一个曲面$S$,假设它是一个光滑的曲面,也就是说可以把它看成某个光滑函数的等值面\footnote{对于函数$f$,任意数值$a\in\mathcal{F}$,称$f^{-1}(a)$是$f$的一个等值面.},那么我们可以使用道路来求出曲面某一点处的切向量.




