% 刚体的绕轴转动 转动惯量
% 转动惯量|角动量定理|刚体|绕轴转动|角速度|角加速度

\pentry{刚体\upref{RigBd}, 角动量定理\upref{AMLaw}}

\subsection{刚体的绕轴转动}

若刚体绕固定轴转动, 那么刚体的位置只需一个变量即可完全确定(一个自由度), 我们令该变量为转角 $\theta$. $\theta$ 关于时间 $t$ 的导数就是刚体绕轴旋转的角速度 $\omega$. 我们还可以定义角速度 $\omega$ 关于时间的导数(即 $\theta$ 关于时间的二阶导数)为\textbf{角加速度}, 记为 $\alpha$.

我们可以把刚体的绕轴转动类比质点的直线运动, 把 $\theta$, $\omega$ 和 $\alpha$ 分别类比为直线运动中的位置 $x$, 速度 $v$ 和 加速度 $a$, 因为后三个变量之间的数学关系是完全相同的. 于是我们可以立即得到匀变速转动(即 $\alpha$ 为常数)的一些公式, 如
\begin{gather}
\theta = \theta_0 + \omega t + \frac12 \alpha t^2\\
\omega_1^2 - \omega_0^2 = 2\alpha \theta
\end{gather}

在以上三个标量的基础上, 我们可以定义它们的矢量形式 $\bvec \theta$, $\bvec \omega$ 和 $\bvec \alpha$, 令它们的方向为转轴的方向, 用右手定则\upref{RHRul} 来判断.

要判断刚体上任意一点的速度, 使用\autoref{CMVD_eq5}\upref{CMVD} 即可(见\autoref{RigRot_fig2})
\begin{equation}
\bvec v = \bvec \omega \cross \bvec r
\end{equation}

\begin{figure}[ht]
\centering
\includegraphics[width=4cm]{./figures/RigRot_2.pdf}
\caption{刚体绕轴旋转时任意一点的线速度} \label{RigRot_fig2}
\end{figure}

\subsection{角动量与转动惯量}

设刚体绕固定轴转动, 令轴的方向为 $\uvec z$. 假设轴光滑, 则轴对刚体可施加 $x, y$ 两个方向的力矩,却不能施加 $z$ 方向的力矩. 所以根据角动量定理, 角动量 $\bvec L$ 的 $z$ 分量 $L_z$ 守恒. 我们下面来推导 $L_z$ 与角速度 $\omega$ 的关系. 矢量 $\bvec L$ 与矢量 $\bvec \omega$ 的关系见惯性张量\upref{ITensr}.

对于单个质点,$L_z = ( \bvec r \cross \bvec p ) \vdot \uvec z$. 首先把质点的位矢在水平方向和竖直方向分解, $\bvec r = \bvec r_z + \bvec r_ \bot$. 由于 $\bvec p$ 一直沿水平方向, 根据叉乘的几何定义, $\bvec r_z \cross \bvec p$ 也是沿水平方向, 只有 $\bvec r_ \bot \cross \bvec p$ 沿 $z$ 方向.另外, 在圆周运动中, 半径始终与速度垂直, 所以 $\bvec r_ \bot$ 始终与 $\bvec p$ 垂直.得出结论
\begin{equation}
L_z = \abs{\bvec r_\bot} \abs{\bvec p} = m r_ \bot v = mr_ \bot ^2\omega 
\end{equation}
若把刚体分成无数小块, 每小块的质量分别为 $m_i$, 离轴的距离 $r_{\bot i} = \sqrt{x_i^2 + y_i^2} $, 则刚体的角动量 $z$ 分量为
\begin{equation}
L_z = \omega \sum_i m_i r_{ \bot i}^2
\end{equation}
用积分写成
\begin{equation}
L_z = \omega \int r_ \bot ^2 \dd{m} = \omega \int r_ \bot ^2\rho  \dd{V} 
\end{equation}

定义刚体绕固定轴旋转的\textbf{转动惯量}为
\begin{equation}
I = \int r_ \bot ^2 \dd{m} 
\end{equation}
(注意角动量的大小不仅取决于刚体的质量分布, 还取决于转轴的位置和方向)则刚体沿轴方向的角动量为
\begin{equation}\label{RigRot_eq5}
L_z = I\omega 
\end{equation}

现在来看“角动量定理\upref{AMLaw}” 的\autoref{AMLaw_eq1}, 注意等号两边是矢量, 所以各个分量必须相等, 我们有
\begin{equation}\label{RigRot_eq6}
\dv{L_z}{t} = \tau_z
\end{equation}
将\autoref{RigRot_eq5} 代入\autoref{RigRot_eq6}, 并利用角加速度的定义得
\begin{equation}\label{RigRot_eq7}
I\alpha = \tau_z
\end{equation}
这就是刚体绕轴转动的动力学方程, 其形式可类比牛顿第二定律\upref{New3}.

\begin{example}{刚体摆}\label{RigRot_ex1}
如\autoref{RigRot_fig1}, 已知质量为 $M$ 的薄片绕某点的转动惯量为 $I$, 转轴到刚体质心的长度为 $r_c$, 转轴和质心的连线与竖直方向夹角为 $\theta$, 求刚体的运动方程.
\begin{figure}[ht]
\centering
\includegraphics[width=3.8cm]{./figures/RigRot_1.pdf}
\caption{刚体摆} \label{RigRot_fig1}
\end{figure}

首先我们把刚体看做质点系, 以转轴为原点计算刚体的合力矩为(由于这是一个平面问题, 力矩必然垂直于该平面)
\begin{equation}\ali{
\bvec \tau &= \sum_i \bvec r_i \cross (m_i \bvec g)
= \qty(\sum_i m_i \bvec r_i) \cross \bvec g
= M \bvec r_c \cross \bvec g\\
&= Mg r_c \sin\theta
}\end{equation}
这就说明, 刚体所受力矩相当于质量为 $M$, 长度为 $r_c$ 的单摆所受的力矩. 代入\autoref{RigRot_eq7} 得刚体摆的运动方程为
\begin{equation}
I\ddot \theta = Mg r_c \sin\theta
\end{equation}
可以验证当刚体的质量全部集中在质心时($I = Mr_c^2$)我们就得到了单摆的运动方程\autoref{Pend_eq4}\upref{Pend}.
\end{example}

\begin{exercise}{陀螺进动的角速度}
在“角动量定理\upref{AMLaw}” 的\autoref{AMLaw_ex2} 中, 如果除 $r_0, m, g$ 外, 还知道陀螺的转动惯量为 $I$ 和陀螺的角速度 $\omega$, 试证明陀螺进动的角速度为
\begin{equation}
\Omega = \frac{mgr_0}{I\omega}
\end{equation}
注意进动角速度与陀螺倾角 $\theta$ 无关.
\end{exercise}
