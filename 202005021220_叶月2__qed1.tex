% 洛伦兹群
\pentry{群\upref{Sample}李群\upref{Sample}洛伦兹变换\upref{Sample}算符对易与共同本征函数\upref{Sample}矩阵\upref{Sample}闵可夫斯基空间\upref{Sample}}
\subsection{定义}
所有洛伦兹变换的集合形成一个群,称为洛伦兹群(Lorentz group).
采取度规张量$g_{\mu \nu}=\operatorname{diag}(1,-1,-1,-1)$,引入时空坐标四矢量标记
\begin{equation}
\begin{aligned}
x^{\mu} &=\left(x^{0}, x^{i}\right) \quad(i=1,2,3)\\
&=(t, \bvec{x})
\end{aligned}
\end{equation}
则对于两个惯性系有
\begin{equation}\begin{aligned}
S^{2} &=x^{0} x^{0}-x^{i} x^{i}=x^{\mu} x^{\nu} g_{\mu \nu} \\
&=x^{\prime 0} x^{\prime 0}-x^{\prime i} x^{\prime i}=x^{\prime \mu} x^{\prime \nu} g_{\mu \nu}
\end{aligned}\end{equation}
设联系两个惯性系的洛伦兹变换为$\Lambda_{\nu}^{\mu}$,即
\begin{equation}x^{\prime \mu}=\Lambda_{\nu}^{\mu} x^{\nu}=\Lambda_{0}^{\mu} x^{0}+\Lambda_{i}^{\mu} x^{i}\end{equation}
保持时空距离不变的要求导致
\begin{equation}g_{\rho \sigma}=g_{\mu \nu} \Lambda_{\rho}^{\mu} \Lambda_{\sigma}^{\nu}\end{equation}
可以证明洛伦兹矩阵的行列式为1
\subsection{推导}说
采取矩阵记法,设两个惯性系的坐标矢量分别为$X$和$X'$,度规矩阵为$g$,洛伦兹矩阵为$L$.则可以分别写为
\begin{equation}
S^{2}=X^{\mathrm{T}} g X
\end{equation}
\begin{equation}
X'=LX
\end{equation}
\begin{equation}
L=L^{\mathrm{T}} g L
\end{equation}
\subsection{分类}



\subsection{洛伦兹群的李代数}