% 速度规范
% 长度规范|速度规范|波函数|规范变换|薛定谔方程|麦克斯韦方程组

\pentry{电磁场中的单粒子薛定谔方程\upref{QMEM}, 偶极子近似} % 链接未完成

我们只在使用偶极子近似时讨论速度规范, 因为我们接下来需要 $\bvec A, \varphi$ 与位置无关. 当空间中存在静止的电荷分布时, 我们可以把标量势能分为 $V + \varphi$ 两部分. 前者由静止电荷根据库伦定律计算, 不参与规范变换, 在这里我们甚至可以不把它看成电磁力而只是某种其他势能. 后者可以随时间变化, 但库仑规范下 $\varphi = 0$. 定义不含时哈密顿算符为
\begin{equation}
H_0 = \frac{\bvec p^2}{2m} + qV
\end{equation}
则伦规范下, 电磁场中带电粒子的哈密顿量为(\autoref{QMEM_eq4}~\upref{QMEM})
\begin{equation}\label{LVgaug_eq2}
H = H_0 - \frac{q}{m} \bvec A \vdot \bvec p + \frac{q^2}{2m} \bvec A^2 + q\varphi
\end{equation}
其中 $\bvec p = m\bvec v + q\bvec A = -\I \grad$ 是广义动量算符(\autoref{QMEM_eq6}~\upref{QMEM}). 例如在长度规范\upref{LenGau}下, $\bvec A = 0$, 则 $\bvec p$ 就是通常的动量.

对库仑规范使用规范变换
\begin{equation}\label{LVgaug_eq3}
\Psi(\bvec r, t) = \exp(\I q\chi)\Psi^V(\bvec r, t)
\end{equation}
\begin{equation}\label{LVgaug_eq4}
\chi(t) = \frac{q}{2m} \int_{-\infty}^t \bvec A^2(t') \dd{t'}
\end{equation}
使用规范变换为(\autoref{QMEM_eq5}~\upref{QMEM})
\begin{equation}\label{LVgaug_eq1}
\bvec A = \bvec A' + \grad \chi = \bvec A'
\end{equation}
可见速度规范和库仑规范中的矢势相同, 所以广义动量(\autoref{QMEM_eq6}~\upref{QMEM})也和库伦规范相同
\begin{equation}
\bvec p = m \bvec v + q\bvec A
\end{equation}

再看标势的变换:
\begin{equation}\label{LVgaug_eq5}
\varphi = \varphi' - \pdv{\chi}{t} = - \frac{q}{2m} \bvec A^2
\end{equation}
\autoref{LVgaug_eq1} 和\autoref{LVgaug_eq5} 带入\autoref{LVgaug_eq2} 可以消除 $\bvec A^2$ 项得
\begin{equation}
H^V = H_0 - \frac{q}{m} \bvec A \vdot \bvec p
\end{equation}
薛定谔方程为
\begin{equation}
H^V \Psi^V = \I \pdv{t} \Psi^V
\end{equation}
这种规范叫做\textbf{速度规范(velocity gauge)}.

与长度规范的关系见\autoref{LenGau_eq3}~\upref{LenGau}
\begin{equation}
\Psi^V = \exp[\I q(\chi^L - \chi^V)]\Psi^L = \exp[-\I\frac{q^2}{2m}\int_{-\infty}^t \bvec A^2(t')\dd{t'} + \I q \bvec A\vdot \bvec r] \Psi^L
\end{equation}
