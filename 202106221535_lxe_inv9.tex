% 9

\subsection{股票基本指标}
我们以从CSMAR数据库上下载的各支股票的周回报率作为股票周收益率.

\subsubsection{1、期望收益率}
运用Excel计算十支股票的期望收益率,计算结果如下表所示.

\begin{table}[ht]
\centering
\caption{十支股票期望收益率}\label{inv9_tab1}
\begin{tabular}{|c|c|c|c|c|c|}
\hline
股票代码 &600016 & 600028 & 600050 & 600104 &	600588 \\
\hline
期望收益率 &-0.000802&0.001282&0.001586	&0.002030&0.008136\\
\hline
股票代码 &601138&601166&601288&601319&601668\\
\hline
期望收益率&-0.0018197&0.002843&0.001452&0.004942	&0.002205\\
\hline
\end{tabular}
\end{table}

\subsubsection{2、标准差}
运用Excel计算十支股票的标准差,计算结果如下表所示.

\begin{table}[ht]
\centering
\caption{十支股票标准差}\label{inv9_tab2}
\begin{tabular}{|c|c|c|c|c|c|}
\hline
股票代码 &600016 & 600028 & 600050 & 600104 &	600588 \\
\hline
标准差 &0.021491&0.029004&0.045676&0.040468&0.065377\\
\hline
股票代码 &601138&601166&601288&601319&601668\\
\hline
标准差&0.053231&0.033032&0.021184&0.082794&0.036732\\
\hline
\end{tabular}
\end{table}
\subsubsection{3、协方差矩阵}
运用MATLAB计算十支股票的协方差矩阵,代码片段如下所示.

图1:计算协方差矩阵代码

计算结果如下图所示.

图2:十支股票协方差矩阵

\subsubsection{4、相关系数矩阵}
运用MATLAB计算十支股票的相关系数矩阵,代码片段如下所示.

图3:计算相关系数矩阵代码

计算结果如下图所示.

图4:十支股票相关系数矩阵
\subsubsection{5、股票相关性分析}
以上相关系数矩阵表中:绿色部分相关系数小于0.3,表示股票之间不存在相关关系;黄色部分相关系数在0.3~0.5,表示股票之间存在低度相关关系;橙色部分相关系数在0.5~0.7,表示股票之间存在显著相关关系.

对十支股票的相关系数矩阵进行分析.从总体上看,所选的十支股票中大部分股票之间不存在相关关系或只存在低度相关关系,只有极少数几支股票之间存在较为显著的相关关系.基于以上分析,我们可以得出结论:我们从上证50中选出的十支股票分散化程度较高,组合的方差较低,投资风险也较低.

\subsection{单支股票的单因素模型}
我们选取上证综指的回报率作为市场指数回报率进行计算,得出十支股票的单指数模型.
\subsubsection{1、单指数模型的回归方程}

回归方程是:
$R_{i}(t)=\alpha_{i}+\beta_{i}R_{M}(t)+e_{i}(t)$.
其中,$\alpha$是当上证综指的超额收益率为0时该股票的期望超额收益率,斜率$\beta_{i}$是股票对指数的敏感性,$e_{i}$的均值为0.

运用MATLAB对数据进行回归处理,代码如下图所示.

图5:计算回归方程代码

计算结果如下所示,十支股票的单指数模型回归方程分别为:

股票代码600016:$R_{1}=-0.001159 + 0.995744R_{M}$

股票代码600028:$R_{2}=0.000920 + 1.066996R_{M}$

股票代码600050:$R_{3}=0.001007 + 1.404773R_{M}$

股票代码600104:$R_{4}=0.001679 + 0.913641R_{M}$

股票代码600588:$R_{5}=0.007769 + 1.144919R_{M}$

股票代码601138:$R_{6}=-0.002407 + 1.144003R_{M}$

股票代码601166:$R_{7}=0.002491 + 0.93768R_{M}$

股票代码601288:$R_{8}=0.001094 + 1.018268R_{M}$

股票代码601319:$R_{9}=0.004582+0.638417R_{M}$

股票代码601668:$R_{10}=0.001851+0.957975R_{M}$


\subsubsection{2、$\alpha$分析}
基于以上得到的各支股票的单指数模型回归方程,我们对$\alpha$进行分析.在这十支股票中,有八只股票的$\alpha$是大于0的,仅有两支股票的$\alpha$小于0.虽然xxxx和xxxx这两支股票的$\alpha$小于0,但基于我们上文的选股分析,这两只股票仍然值得投资.

\subsubsection{3、回归检验}





