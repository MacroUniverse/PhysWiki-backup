% 牛顿—莱布尼兹公式(复变函数)

\begin{issues}
\issueDraft
\end{issues}

复变函数的积分为
\begin{equation}
\int_{C} f(z) \mathrm{d} z=\int_{C} u \mathrm{d} x-v \mathrm{d} y+\mathrm{i} \int_{C} v \mathrm{d} x+u \mathrm{d} y
\end{equation}
可以看作两个矢量场 $\bvec f_R(\bvec r), \bvec f_C(\bvec r)$ 在曲线 $C$ 上的线积分
\begin{equation}
\int_{C} f(z) \dd{z} = \int_C \bvec f_R(\bvec r) \vdot \dd{\bvec r} + \I \int_C \bvec f_C(\bvec r) \vdot \dd{\bvec r}
\end{equation}
其中
\begin{equation}
\bvec f_R(\bvec r) = u\uvec x - v\uvec y
\qquad
\bvec f_I(\bvec r) = v\uvec x + u\uvec y
\end{equation}
而柯西—黎曼条件\upref{CauRie}刚好
