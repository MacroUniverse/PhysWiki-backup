% 角动量 角动量定理 角动量守恒(单个质点)
% 角动量|角动量定理|角动量守恒|力矩|牛顿第二定律

\pentry{牛顿第二定律\upref{New3},力矩\upref{Torque}}

\subsection{质点的角动量}
在我们讨论物体的转动时通常会引入\textbf{角动量(angular momentum)}的概念, 这与直线运动中的动量有许多相似之处.

\begin{definition}{角动量}
一个质点的质量为 $m$, 某时刻速度为 $\bvec v$, 则其动量为 $\bvec p = m\bvec v$. 在三维空间中建立坐标系, 原点为 $O$, $O$ 点到质点的位置矢量\upref{Disp}为 $\bvec r$. 定义该质点关于 $O$ 点的角动量为
\begin{equation}\label{AMLaw1_eq1}
\bvec L = \bvec r \cross \bvec p = m \,\bvec r\cross \bvec v
\end{equation}
\end{definition}

由叉乘的几何定义\upref{Cross} 可知,当速度与位矢平行时角动量为 $\bvec 0$,垂直时角动量模长为距离和动量模长的积 $L = rp$.

注意角动量的值取决于参考系的选取, 如果我们选用不同的参考系, 那么\autoref{AMLaw1_eq1} 中的 $\bvec r$ 就会发生改变, 进而改变角动量. 事实上我们也可以取空间中的任意一点 $\bvec r_0$ 作为角动量的参考点, 定义角动量为
\begin{equation}
\bvec L = (\bvec r - \bvec r_0) \cross \bvec p
\end{equation}
这相当于以 $\bvec r_0$ 为原点建立一个新的参考系, 然后在其中计算角动量.

\begin{example}{质点圆周运动的角动量}
\begin{figure}[ht]
\centering
\includegraphics[width=3cm]{./figures/AMLaw1_1.pdf}
\caption{圆周运动的角动量} \label{AMLaw1_fig1}
\end{figure}
考虑一个质量为 $m$ 的质点绕坐标原点做半径为 $r$ 的逆时针匀速圆周运动(\autoref{AMLaw1_fig1} ), 角速度为 $\omega$, 速度为 $v = \omega r$. 对于圆周运动, 位置矢量和速度矢量始终垂直, 所以角动量大小为
\begin{equation}
L = mvr = m\omega r^2
\end{equation}
又根据右手定则\upref{RHRul}, 角动量的方向垂直纸面向上. 所以质点做圆周运动时, 角动量矢量不随时间变化, 即下文将要介绍的角动量守恒.
\end{example}

\begin{example}{质点直线运动的角动量}\label{AMLaw1_ex1}
\begin{figure}[ht]
\centering
\includegraphics[width=6cm]{./figures/AMLaw1_2.pdf}
\caption{直线运动的角动量} \label{AMLaw1_fig2}
\end{figure}
虽然角动量通常用于描述转动, 但也可以描述质点任意运动. 例如当质点做匀速直线运动, 运动轨迹与原点的距离为 $d$, 则角动量大小始终为
\begin{equation}
L = mvr\sin\theta =  mvd
\end{equation}
方向始终垂直纸面向上. 所以做直线运动的质点同样有角动量守恒.
\end{example}

\subsection{角动量定理}
令质点在某时刻受到的力矩\upref{Torque}为 $\bvec \tau$,可以证明
\begin{equation}
\dv{\bvec L}{t} = \bvec \tau
\end{equation} 
这就是(单个质点的)\textbf{角动量定理}.

特殊地,若质点受到的力矩为零,则 $ \dv*{\bvec L}{t} = \bvec 0$,即角动量不随时间变化.这个现象叫做\textbf{角动量守恒(conservation of angular momentum)}.由力矩的定义(\autoref{Torque_eq3}~\upref{Torque}) $\bvec \tau = \bvec r \cross \bvec F$,可见以下两种情况下力矩为零,角动量守恒.
\begin{enumerate}
\item 质点受合力 $\bvec F= \bvec 0$,即质点静止或做匀速直线运动(\autoref{AMLaw1_ex1} ).
\item $\bvec F$ 与 $\bvec r$ 同向,即质点只受关于 $O$ 点的\textbf{有心力}.
\end{enumerate}

\subsubsection{证明}
我们来证明单个质点的角动量定理. 令质点的速度为 $\bvec v = \dv*{\bvec r}{t}$,加速度为 $\bvec a = \dv*{\bvec v}{t}$, 叉乘的求导法则(\autoref{DerV_eq8}~\upref{DerV}) 与标量乘法求导类似, 牛顿第二定律\upref{New3}为 $\bvec F = m\bvec a$,两个同方向矢量叉乘\upref{Cross}为零,
\begin{equation}
\ali{
\dv{\bvec L}{t} &= \dv{( \bvec r \cross \bvec p )}{t} = m\dv{(\bvec r \cross \bvec v)}{t}
= m \qty( \dv{\bvec r}{t} \cross \bvec v + \bvec r \cross \dv{\bvec v}{t} )\\
&= m(\bvec v \cross \bvec v + \bvec r \cross \bvec a) = \bvec r \cross (m\bvec a)\\
&= \bvec r \cross \bvec F = \bvec \tau
} \end{equation}
证毕.
