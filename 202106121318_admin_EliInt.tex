% 椭圆积分
% 定积分|椭圆积分|第一类不完全椭圆积分

\pentry{换元积分法\upref{IntCV}, 曲线的长度\upref{CurLen}}

\subsection{第一类不完全椭圆积分}
\textbf{第一类不完全椭圆积分(incomplete elliptic integral of the first kind)}为
\begin{equation}\label{EliInt_eq1}
F(\phi, k) = \int_0^\phi \frac{\dd{x}}{\sqrt{1 - k^2\sin^2 x}}
\end{equation}
椭圆积分有时候也记为 $F(\phi | k^2) = F(\sin\phi ; k)$.

使用积分换元法, 令 $t = \sin x$, 有
\begin{equation}
F\qty(\phi, k) = \int_0^\phi \frac{\dd{t}}{\sqrt{(1 - t^2)(1 - k^2 t^2)}}
\end{equation}
另一种换元法是令 $\theta = 2x$, $k = \csc(\theta_0/2)$, $F(\phi, k)$ 也可以表示为
\begin{equation}\label{EliInt_eq2}
F\qty(\phi, \csc\frac{\theta_0}{2}) = \frac{1}{\sqrt{2}} \sin\frac{\theta_0}{2} \int_0^{2\phi} \frac{\dd{\theta}}{\sqrt{\cos\theta - \cos\theta_0}}
\end{equation}

\subsubsection{数值计算}
$F(\phi | m)$ 的 Matlab 函数为 \verb|ellipticF(phi, m)|, % 已数值积分验证
Mathematica 函数为 \verb|EllipticF[phi, m]|.

(图未完成)

\subsection{第二类不完整椭圆积分}
\textbf{第二类不完整椭圆积分(incomplete elliptic integral of the first kind)}为
\begin{equation}\label{EliInt_eq3}
E(\varphi, k) = E(\varphi | k^2) = E(\sin\varphi; k) = \int_0^\varphi \sqrt{1 - k^2\sin^2\theta} \dd{\theta}
\end{equation}
令 $x = \sin\varphi$, 则有
\begin{equation}
E(x; k) = \int_0^x \frac{\sqrt{1 - k^2t^2}}{\sqrt{1 - t^2}}\dd{t}
\end{equation}

该函数可用于计算椭圆的弧长.


\subsubsection{第二类完整椭圆积分}

\begin{equation}
E(k) = \int_0^{\pi/2} \sqrt{1 - k^2\sin^2\theta} \dd{\theta} = \int_0^1 \frac{\sqrt{1 - k^2 t^2}}{\sqrt{1 - t^2}} \dd{t}
\end{equation}
椭圆的周长为
\begin{equation}
c = 4aE(e)
\end{equation}
其中 $a$ 是椭圆的长轴, $e$ 是离心率.

也可以用第二类不完整椭圆积分表示为
\begin{equation}
E(k) = E\qty(\frac{\pi}{2}, k) = E(1; k)
\end{equation}
