% 从天球的音乐到玻尔模型

形或古希腊人所说的 “idea”,有多种含义,比如:形状,这是和视觉有关的;比如风格、分类,这可以是和视觉有关的,也可以无关,比如音乐也可以有风格,这就是和声音有关的了.

和视觉有关的“形”是直观的,我们无须论证,纠结于如何用语言表达,仅凭图形——或者是静态的,或者是想象中动态的——直接给出结果.对形的研究会导向几何学,几何本身是视觉的,而视觉是偏好静的,偏好不动的,但一加“学”,几何“学”或“学”几何就动起来了.

我们如何学呢?或者演示,用圆规和直尺,或者像毕达哥拉斯那样拿根木棍面对沙土,世界是一步一步地被展现出来的,一笔一划本身就是个动态的过程.我们努力说:“首先如何,其次如何,然后,又然后……”

所谓动态就是次序,我们首先只关注首先要解决的,其次,带着对刚刚过去的对首先的记忆,探讨紧接着要解决的问题,我们的思想没法分叉.人在专注的状态下,视觉也需要一个焦点.当我们的视觉遭遇挑战,看不清某物的时候,我们凝眼观瞧,把视线使劲聚焦于某物,凝眼就是凝神,不受诱惑地专注于某物,看清楚一点再继续看下一点.

这个结构很像自然数:“0,1,2,3,……”,一步一步地展示给你看,比如“我是如何用直尺和圆规作图的”,这种线性展开的结构就是时间,“学”的过程,对学的人是学,在Challenge,对展示的人来说是在“证”,在说服,这个过程是世界次第展开的过程,是叙事、是 Chronicle.

\subsection{形与声}

“学” 依赖语言,语言是一种声音现象.

据说人能够发出一个八度再加一个四度的声音.

古代世界,天和地很近,音乐和人也很近.孔子闻韶乐“三月不知肉味”,这种沉浸在声音里的境界和我们今天听流行音乐,把音乐当做一种背景噪音,是完全不同的两种声音技术.今天我们听音乐往往是为了抑制我们心中的背景噪音.

古代的音乐都很简单.简单到好比就是敲击单音音叉发出的声音,单音音叉是校音用的,它在古代世界的对应物是中国的黄钟律管或希腊的单弦琴(Monochord).它们发出很纯的音,基本上就是一个频率.孔子一生关心礼,礼与乐相联,乐就是音及音的混杂与排列.

我们用音高,频率,响度,音色等来描述声音.音高就是频率,是描述“音”诸参数中最重要的一个.人天生就是一个感知音高的灵敏动物,高音激越,使人振奋,低音呜咽,让人伤感.简单的音乐庄重使人入静,而复杂多变的音乐也如一场“视觉的盛宴”,它使我们好奇和沉迷.

听觉和视觉一样,是感觉,同时也是思维,我们的眼睛和耳朵接受信息,同时也处理、歪曲信息以为我们所用.古代的政治传统,古代的教育家都注重音乐教育,这其中最重要的就是对音乐体系的保留和传承.

比如唱歌的时候要先定调,调可以定低点,显得庄重,也可以定高点,显得轻快.定好调后,一系列的声音次第展开,它们的相对音高保持一个固定的结构,比如:

“低,低低,高,高高,低,中中,……”

在给定乐谱的前提下.基准音高的选取,或所谓定调是任意的.我们可以定高点,无非大家唱不上去而已.但因为有人唱不上去,这个定调就也不是完全主观任意的了.

古代政治秩序大多由推崇勇猛进取精神的战士集团建立,对战士共同体而言,最重要的是要保持这种勇猛进取的精神,能够保持这种精神的音乐会与特定音高有关,这是人群的共同经验.比如柏拉图在《理想国》中就说,要摒弃悲伤和软绵绵的吕底亚调和伊奥尼亚调,而推崇多利亚调和佛里吉亚调.

保持这种对声音的共同经验在古代政治传统中是非常重要的,其中之一就是确定音调,或基准音的频率,然后在此基础上给出其他音的定义,其他音是相对于基准音而言的,可以更高,也可以更低,构成一个阶梯状的结构.

原子的“idea”是无所不在的,这里由人的听觉经验,我们再次得到了原子的概念,即存在着“音高”的原子,进一步细分不同音高的原子是没有必要的.

保存音乐制度最简单的方法就是造一套标准的乐器,然后后人反复向这些标准的乐器学习,第一套自然是由城邦的缔造者“铸造”的.

考虑到弦乐器与弦绷紧的程度有关,受湿度、温度影响较大,青铜器制造的发音器会是理想的选择,这是为什么“钟”会成为“国家”符号的原因,塔可夫斯基电影《安德烈·卢布廖夫》再现的是俄罗斯帝国创旦的精神基础,在影片的结尾就出现了工匠之子铸钟的奇迹.

\begin{figure}[ht]
\centering
\includegraphics[width=10cm]{./figures/ClBohr_1.png}
\caption{工匠之子铸钟} \label{ClBohr_fig1}
\end{figure}


钟是要发音的,音高是有标准的,“音高”高一些,低一些,很微妙,但人的耳朵,或某些人的耳朵天生就是辨别音高的灵敏仪器.只有能发出特定音高的钟才是可以被接受,一只发音不准的钟在敲响的时候不嘹亮,不能激发人民激越的精神,这对城邦是不利的.

这里有个似是而非但很有趣的讨论,人有时间感,但人的时间感是非常内在的,几乎不存在什么可以相互交流的基础.这是妨碍人产生运动观念,并研究运动的重要原因.音高即频率,频率就是时间的倒数,人没法准确标记时间的流逝,但人却是辨别频率(时间倒数)的精密仪器.同时我们的发音器官,还能娴熟地对不同音高的声音进行模仿,这是我们具有语言和音乐能力的生物学基础.

类似地,我们还可以讨论视觉,讨论视觉对位置和速度的分辨.人天生就能在相当精确的意义下辨别位置,但我们对速度的判断就要差许多.我们说A比B快,其实是通过位置下的判断,即AB同时出发,但A先撞线,所以A更快.这是亚里士多德无法得到“正确”的落体规律的原因,他受人本身的局限,速度是很难直接被看的.

古代实验技术还没有充分发展起来,而实验技术的充分发展与资本主义的生产方式兴盛有关,近代自然科学与资本主义生产方式同步爆发并非巧合.回顾二者,科学史和资本主义发展史,两者讲的是同一个故事,只是叙事的角度生了变化.

由“造钟”故事,我们得到一个新洞见,即:“音与形有关”.对钟来说这是大大地简单化了,因为材质也很重要,但形状确实决定了钟振动的频率.这意味着:“听音可以定形,定形可以定音”.

形既是形状也是模型,还是形式.在毕达哥拉斯和柏拉图的传统里,形是与数紧密相连的.比如钟的形由何而定呢?长、宽、高、是数字,钟的厚度也是数字,但这一堆数字的集合又有什么意义呢?

当我滔滔不绝地罗列一堆数字的时候,这是没有意义的.我们需要给出数字和数字之间的关系,才有意义.而且最好是只给出一个关系(或最少关系),就能让所有的数字各就各位.找到这样的规律自然是对思维的奖励,是可以向众人夸耀的;同时这也是技术,有了技术我们就能铸钟,小孩的父亲是会铸钟的,但他把技术带到坟墓里去了.

《安德烈·卢布廖夫》中的小孩是幸运的,他必须试试,他也只能试试.在拜占庭衰败之后,东正教来到了俄罗斯与当地的土豪、愚民混合,文明在绝望中重新开始,这就是俄罗斯的宿命.卢布廖夫受不会铸造但却造出钟的小孩的激励,重新拿起画笔开始画注定会塑造俄罗斯民族精神的那些很平、很抽象圣像画.

\begin{figure}[ht]
\centering
\includegraphics[width=5.5cm]{./figures/ClBohr_2.png}
\caption{卢布廖夫的《三圣像》} \label{ClBohr_fig2}
\end{figure}

画是形(idea),音是声(logos).形和声都能塑造性格,前提是我们生活在某种生活中,或我们生活在某种历史中.

\subsection{数字与和谐}

“几何学”(Geometry)是对形的规定,而“和声学”(Harmonics)是对音的规定.所谓规定就是数字之间的联系,最简单的数字和数字间的联系是“相等”,稍微高级点的是比例,是合乎比例.

比如人脸,人脸上五官的位置和尺寸是需要合乎比例的,这种合乎比例是我们天生可以判断的,但很难用数字说清楚.近一二十年随着计算机对数据处理能力的提高和神经科学的进步,在这方面有了很多具体技术的进展.

比例或合乎比例会产生美,这是某种审美观念下的模式识别.比如古代东夷部族以扁头为美,甚至不惜把小孩的头骨弄扁以合乎比例.这个习俗在今天还有遗存,不少地方有端正小孩睡姿以把头睡扁的说法.

在音乐中我们很容易发现音高与数字的关系.这是毕达哥拉斯的贡献.音乐的历史一定很古老.在毕达哥拉斯之前人类就有音乐了,不但有音乐还有规定音高的一套体系,即有一套术语来说清楚“不同音高”的音之间的关系.

比如当我发出一个音后,让你发出一个高四度的音,你就能发出这样一个音,并得到我的认同.这套语言游戏能够玩儿的起来.这些当然都是基于感官经验讲的,本来和数字没啥关系.传说毕达哥拉斯在路过铁匠铺时,受到叮叮当当声音的启发,回去研究各种乐器的音高,比如弦乐.

所谓弦乐器就是一根绷紧的弦,两端固定,中间可以快速振动起来,扰动空气发出声音,弦乐的频率自然就是琴弦发出的声音.这是典型的机械振动的问题,弦上会有波动,但因琴弦两端是固定的,所以波传播不出去,它只能被限制在琴弦上振动,并整体具有一个轮廓,琴弦就在这个轮廓内振动,这种振动叫驻波.

琴弦上的振动是波动,当一列波从左向右传播时碰到弦的端点会反射回来,驻波就是两列相向传播的波的叠加:

\begin{equation}
\frac{A}{2} \left( \cos ( kx - \omega t ) + \cos ( kx + \omega t ) \right) = A \cos kx \cos \omega t
\end{equation}

这里A是振动的幅度,振动的轮廓线是:

\begin{equation}
A \cos kx 
\end{equation}

$k = \frac{2 \pi}{\lambda}$是波矢,$\lambda$是波长,因为琴弦的两端已经被限制住了,琴弦的长度$L$可以取半波长,一个波长,一个半波长,……,简单说就是半波长的整数倍:

\begin{equation}
L = \frac{n \lambda}{2}, n = 1, 2, 3, ...
\end{equation}

波长可以表示为:

\begin{equation}
\lambda = \frac{2L}{n}
\end{equation}

这就是合乎比例.

考虑到弦上波速$v$是个常量,频率可以表示为:

\begin{equation}
\nu = \frac{n v}{2 L }
\end{equation}

给定弦长$L$,只有这样的波动,或这样波动的叠加才可以存在.进一步讲,如果我们考虑一个符合两端被限制住的琴弦的一般运动,这个一般运动总是可以被分解为一系列不同$n$取值的,波长为$\lambda_n = \frac{2L}{n}$,频率为$\nu_n = \frac{n v}{2 L }$的振动的叠加.

\begin{equation}
\sum\limits_{n} A_n \cos \frac{n \pi x}{L} \cos \frac{n \pi vt}{L}
\end{equation}

我们管$n = 1$的音叫做基音,这个频率$\frac{v}{2L}$的声音是最主要的,但弦上也会有$n= 2, 3, ...$的成分,这些音叫做泛音.

拨动长度$L$的琴弦,我们听到的是基因和泛音的混合,最主要的是基因,频率为$\nu_1 = \frac{v}{2L} $,其次是第一个泛音,频率为$\nu_2 = 2 \nu_1$,它们之间是1: 2的关系.

假如我们把琴弦的长度减半,其实就是用手在弦长的一半按住琴弦,此时我们会有新的弦长$L/2$,同时新的基因频率$2 \nu_1$,但此时,因为弦长只剩下一半了,我们拨动琴弦发出的声音里就没有$\nu_1$的成分了.

听起来的感觉是这样的,首先$L/2$琴弦发出的音和$L$琴弦发出的音很像,其次$L/2$琴弦发出的音当然要比$L$琴弦发出的音要高,这就好比是一个人沿螺旋形的楼梯升高,每个台阶都对应一个特定音高的音,在螺旋式升高了几个音之后我们又回到了起始位置,只是高了一些,我们还可以继续螺旋升高,每提升一个台阶都会感觉和曾经的某个台阶很像,只是更高了.

在音乐理论里,我们管这个结构叫“八度”,当音高由$\nu_0$提高一倍到$2 \nu_0$的时候,我们就说“升了八度”.类似地,当音高由$nu_0$降一倍到$\nu_0 /2$时,我们就说“降了八度”.我们一般能发出一个八度再加一个四度的音.

\subsubsection{毕达哥拉斯的和声学}

毕达哥拉斯研究了音和形的关系,并发现这个关系可以被数字精确地描述.比如我们刚刚讨论过的,当弦乐器的弦长比是1:2时,频率比是2:1,正好对应音乐理论中的“八度音程”.

八度关系本来就存在于音乐实践中,属于人的日常经验.现在发现“一个八度”可以表示为精确的数字比1:2,这个数字比其实是对形的描述.只是因为这里弦是一维的,我们对形的描述比较简单.

一个日常经验可以对应于一个数字的比例关系是足够让人兴奋的,毕达哥拉斯讲“万物皆数”,其实讲的是“万物皆合乎比例”.只有合乎比例,万物才能存在,只是这些比例有待我们的发现.

合乎比例是个静态的世界观.

除了1:2,毕达哥拉斯还发现当弦长比是2:3时,音的关系是音乐理论中的五度音程.而弦长比是3:4时是音乐理论中的四度音程.毕达哥拉斯只发现了这几个关系.它足够优美,但还不足以解释音乐理论中所有的音.但这已经足够他嘚瑟的了.

更重要的是他开辟了一个用数字、用比例关系去研究音乐的方法,进而是研究整个宇宙万物的方法,可以说今天的理论家都是毕达哥拉斯的信徒.

\begin{figure}[ht]
\centering
\includegraphics[width=5cm]{./figures/ClBohr_3.png}
\caption{蒙德里安的作品是“万物皆数”、“整体和谐”观念在绘画领域的实践.} \label{ClBohr_fig3}
\end{figure}

毕达哥拉斯方案的缺陷是他被简单数字迷住了,1:2,2:3,3:4确实解释了八度音程、五度音程、和四度音程.但再要想把人对声音的感官经验——极其灵敏的感官经验——和简单数字比(m:n)建立关系就很困难了.

根据近代的十二平均律,我们在八度音程里面做12均分,这个均分是合乎比例地分——作为人,我们当然是凭我们的耳朵来分,这里我们必须赞叹人听觉器官的精密——我们要找到某个合适的比例因子$q$,使得:

$1 \nu_0$,$q \nu_0$,$q^2 \nu_0$,……$q^{12} \nu_0 =2 \nu_0$

因此:

\begin{equation}
q = 2^{\frac{1}{12}}
\end{equation}

这里难的是对2开12次方,$\sqrt{2}$就已经是无理数了,即$\sqrt{2}$就已经没办法表示成一个简单数字的比例了!这是毕达哥拉斯方案失败的原因.

我们解出:$q \approx 1.059463 $,并制表:

