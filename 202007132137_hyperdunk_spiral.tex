% 心脏螺旋波
\section{引言}
\subsection{背景知识}
%\lettrine[nindent=0em,lines=3]
目前,非线性科学在实验研究领域主要有如下几个前沿课题:孤立子与孤波,时空混沌,斑图动力学和分形结构(其中该课题其实在第一次Logistic Map中略有涉足).在斑图动力学的研究领域中,螺旋波动力学的研究一直最为非线性科学家关注的课题之一.原因首先是它在自然界是普遍存在的,在诸多非线性实验系统中,都可以看到它的踪迹.例如液晶中的Ising-Bloch相变,反应扩散系统中的化学波,心脏中的心电信号.最近的理论与实验研究表明,螺旋波的动力学行为存在跨系统的普适性规律.研究和掌握这些规律具有很大的潜在应用价值.例如,生理学的实验表明,在心脏病人中观察到的一类叫做再进入性心律过速现象,可能是由于心肌电信号出现螺旋波而引起的,而心颤致死过程与螺旋波的失稳有密切关系.怎么把心脏中的螺旋波电信号消除,是当前心脏病研究的热点之一.本文通过一篇PRL\cite{zhang2005suppress}将向大家展示通过行波形式的电压将心脏中的螺旋波引出心脏.\\
本文通过有限差分法研究反应扩散方程
\begin{equation}
\left\{\begin{array}{c}
\frac{\partial \mathrm{u}}{\partial \mathrm{t}}=f(u, v)+\mathrm{D} \nabla^{2} u \\
\frac{\partial v}{\partial t}=g(u, v)
\end{array}\right.
\end{equation}
Barkley model:
\begin{equation}
\begin{array}{c}
\mathrm{f}(\mathrm{u}, \mathrm{v})=\frac{1}{\varepsilon} u(1-u)\left(u-\frac{v+b}{a}\right) \\
g(u, v)=f(u)-v
\end{array}
\end{equation}
后由尚振华在论文中提出改进,得到

\begin{equation}

\begin{array}{c}
	f(u, v)=\frac{1}{\varepsilon} u(1-u)\left(u-\frac{v+b}{a}\right)\\
	g(u, v)=f(u)-v
\end{array}
\end{equation}

\begin{equation}

f(u)=\left\{\begin{array}{c}
0,0 \leq u<\frac{1}{3} \\
1-6.75 u(u-1)^{2}, \frac{1}{3} \leq u \leq 1 \\
1, u>1
\end{array}\right.
\end{equation}
本文将采用来进行基本的数值模拟计算.其中有两个待定系数并没有给出因为这两个参数会根据具体呈现出来的样子进行合适的调整.
\subsection{基本性质}
螺旋波也称为受激波.在激活媒介中才会存在.激活媒介的共同特征就是具有“可激活性”:当局部区域处于静息状态时,对微扰时稳定的,但对于足够强的扰动将有一个快速又陡峭的激励响应,然后进入对外界刺激抵制的不应期,最后回归到原始的静息状态.二次局部区域的激励确实相邻区域的有限扰动源故相邻区域同样会经历静息一激励一不应一静息的变化过程. 于是此种激励便形成在空间上传布的不衰变的波动现象.这种受激渡与我们熟悉的经典驶有很大的差别\cite{丁达夫1992心脏猝死的杀手}:
\begin{enumerate}
  \item 受激渡在传播过程中所需能量是由媒质提供, 因此它不会像声波邪样随着距离逐渐衰变;
  \item 当受激波与惰性边界相撞时,不会有反射现象,而是自身熄灭;
  \item 当两受激渡相撞时,不会出现保守系统中的孤立子相互穿透现象,而是互相湮灭.
\end{enumerate}
这是因为受激励的波前被相遇物的不应区包围所致,受激励波与经典波的唯一相同点是他们都有衍射性质.\\
激活媒介中的受激波通常有两种,一种的波前图像射击场上看到的靶图,另一种波的波前具有螺旋形状.后者具有特殊的的意义.是本次研究的主题.

