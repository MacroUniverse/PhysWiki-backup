% 动量和能量
% 动量|动能|能量|势能|势能曲线

\pentry{经典力学\upref{CM0}}

\subsection{动量和动能}
动量描述了了物体运动的物体的惯性. 惯性可以简单理解为让一个运动的粒子停下来有多困难. 这个困难程度可以用力乘以时间表示. 例如一个在光滑水平面有动量为 $p = mv$ 的箱子,要使它停下来,人就要沿运动反方向以恒力 $F$ 推箱子, 若一段时间 $t$ 后箱子停了下来,那么有 $p = Ft$.% 未完成: 用牛顿第二定律推导一下
同理,如果要使箱子从静止达到动量 $p$, 就需要以恒力 $F$ 作用在箱子上同样的时间. 所以动量是力在时间上的累加.

相比动量,能量可以看作是力在空间上的累加. 同样是箱子的例子, 如果想让静止的箱子达到动能 $E_k = mv^2/2$, 就需要用恒力 $F$ 推箱子, 在力的方向移动距离 $s$,使 $E_k = Fs$.

理论上, 有了牛顿定律我们就可以求出所有物体的运动情况, 那为什么还需要动量和动能(能量)? 因为动量和能量在封闭系统中是守恒的, 可以只根据初末状态就得到一些结论而不需要知道具体过程. 例如两个箱子碰撞, 我不需要知道碰撞用了多久, 缓冲距离是多少, 材料的性质等等就可以根据碰撞前的状态求出碰撞后的状态. 又例如物体从轨道滚落, 不需要知道过程, 可以直接从两

\subsection{势能曲线}
在更高级的物理理论中我们往往不讨论力,而是势能. 一维运动(直线运动)中如果受力只是关于位置的函数(如弹簧振子),那么这个力(可以称为力场)就叫保守力(高维的情况下有更复杂的定义). 对于保守力,我们可以计算出每个位置的势能(势能函数/势能曲线). 一维直线运动情况下,坐标为 $x$,势能函数可以记为 $V(x)$. 某个位置力的大小就是曲线的斜率,方向就是曲线下降的方向. 粒子沿受力方向运动,动能增加,势能减小,总能量/机械能(动能加势能)不变.

作为一个比喻,想象高低不平的“直”轨道上的小车,我们可以让轨道的高度为 $h(x) = V(x)/(mg)$(高度与势能成正比,$x$ 处小车具有重力势能 $V(x)$),这样具有一定总能量 $E$ 的小车就会在轨道上运动. $V(x) > E$ 的位置小车无法达到,会原路返回. 如果小车开始时在某 $E \leq V(x)$ 的区间运动,那么它将一直在该区间往返运动(忽略摩擦),某点的速度 $v(x) = \sqrt{2[E - V(x)]/m}$(推导一下). 注意这只是一种比喻,我们讨论的是沿直线运动的粒子(没有重力,只有势能 $V(x)$,无论它是什么力). $h(x)$ 越小,$g$ 越大,二者就越接近(运动更接近直线). 如何从图中看出粒子在各个可能位置的运动情况?先画代表总能量的横线,横线与势能曲线的交点就是拐弯的地方,横线的高度减掉某点势能曲线的高度就是动能 $mv^2/2$.

\subsubsection{简谐振子}
(未完成: 画出势能曲线)

\subsubsection{单摆}
(未完成)
为什么小角的单摆和弹簧振子的势能曲线类似(圆的底部与抛物线很接近)?导致质点类似简谐振动. 为什么需要小角?(上文的 $h(x)$ 越小,二者越接近)

\subsection{量子力学中的常见势能曲线}

% 未完成: 我们可以用穿孔的无限大平行板电容器以及带电粒子来解释势垒, 如果电势差为定值, 而平行板无限靠近, 我们就得到了不连续的势能曲线. 如果保持一定距离, 而电场变得无穷大, 我们就得到了无限深势垒的一侧!

量子力学\footnote{don't panic! 这里只是提了一下 “量子力学” 这个名字, 讲的还是喜闻乐见的经典力学.}中的公式不会出现力, 只会出现势能. 常见的势能(三角势垒见下文,方势垒,方势阱,边界处斜率无穷大怎么办? 类比碰撞时的冲力. 初始动能(总能量), 大于, 小于最大势能的时候分别会沿原方向运动, 反弹. 更理想的情况:无限深势阱(小球在两面墙之间无限反弹),delta 势垒/势阱(受到一个微小扰动,相当于无限窄的方势垒/势阱),对经典粒子运动没有任何影响(物理上不存在,但是作为模型有计算简单的优点).

\subsubsection{有限深势阱}
(未完成)

\subsubsection{无限深势阱}
仍然考虑一维运动, 假设光滑水平面上有两面墙, 粒子来两面墙之间来回反弹, 我们该用什么样的势能函数呢? 如果墙是软的, 例如把两面墙比作弹簧, 得到的势能将如(未完成)
