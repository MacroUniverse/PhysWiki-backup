% 矢量空间的表示
% 列向量|矢量组|线性组合|坐标|表示

\begin{issues}
\issueOther{本词条需要重新创作和整合,融入章节逻辑体系.}
\end{issues}

\pentry{线性相关和线性无关\upref{LinDep},张成空间\upref{VecSpn},线性变换\upref{LTrans}}

由于矢量空间中运算的线性性,可以使用矩阵来表示任何一个矢量空间中的元素和线性变换.对于一个域$\mathbb{F}$上的$n$维线性空间中的矢量,我们惯例上使用一个$n$行$1$列的矩阵来表示,称为\textbf{列向量}.线性变换被表示成一个$n\times n$的矩阵.这些矩阵中的元素都必须取自$\mathbb{F}$.

需要注意的是,这些表示都依赖于该矢量空间的\textbf{基}的选取.


\subsection{用基向量来表示向量和线性变换}

给定域$\mathbb{F}$上的$n$维线性空间$V$和它的一个基$\{\bvec{e}_i\}_{i=1}^{n}$.由于$V$中的每一个向量都可以唯一地表示成基向量的线性组合,因此我们可以用线性组合的系数来构成一个列向量,作为这个向量在基$\{\bvec{e}_i\}_{i=1}^{n}$下的\textbf{坐标}.比如,向量$a_1\bvec{e}_1+\cdots+a_n\bvec{e}_n$在这个基下的坐标就是
\begin{equation}
\pmat{a_1\\ \vdots\\ a_n}
\end{equation}
基的选择不同,同一个向量的坐标也就不一样.

在研究线性变换的时候,我们只需要关注线性变换对基向量的变换,就可以据此计算出任意向量的线性变换.如果某一个线性变换$T$把基向量$\bvec{e}_i$变换成$a_{i1}\bvec{e}_1+\cdots+a_{in}\bvec{e}_n$,那么我们可以在这个基下把$T$表示成一个矩阵:
\begin{equation}
M=\pmat{a_{11},a_{12},\cdots,a_{1n}\\ a_{21},a_{22},\cdots,a_{2n}\\ \vdots\ddots\vdots\\ a_{n1},a_{n2},\cdots,a_{nn}}
\end{equation}

这样,如果把$\bvec{v}$的坐标是列向量$\bvec{c}$,那么$M\bvec{c}$就是$T\bvec{v}$的坐标.

同样地,线性变换的矩阵表示,也依赖于基的选取.