% SLISC 容器的切割

\pentry{SLISC 的矢量和矩阵\upref{SliMat}}

如果我们想把矩阵的一行输入到某个函数中怎么办呢? 如果我们不使用任何容器而是用指针, 函数的输入输出也使用指针, 那么可以通过 \verb|step| 参数实现:
\begin{lstlisting}[language=cpp]
Doub sum(const Doub *x, Long_I N, Long_I step)
{
    Doub s = 0;
    for (Long i = 0; i < N; ++i)
        s += x[step*i];
    return s;
}
\end{lstlisting}

\begin{lstlisting}[language=cpp]
Long N0 = 10, N1 = 10;row = 3;
Doub *a;
new Doub a[N0*N1]; // 列主序
for (Long i = 0; i < N0*N1; ++i)
    a[i] = i;
// 对第 3 行求和
cout << "sum row 3 = " << sum(a+3, N1, N0) << endl;
// 对第 3 列求和
cout << "sum column 3 = " << sum(a+3*N0, N0, 1) << endl;
// 对第 3 列的前 3 个元求和
cout << "sum column 3, 0-2 = " << sum(a+3*N0, 3, 1) << endl;
delete a[];
\end{lstlisting}
再看一个例子, 如果我们需要对子矩阵求和呢? 我们需要一个参数叫做 leading dimension, 该参数表示列标增加 1 时单索引增加的数量.
\begin{lstlisting}[language=cpp]
Doub sum(const Doub *a, Long_I N0, Long_I N1, Long_I lda)
{
    Doub s = 0;
    for (Long j = 0; i < N1; ++i) {
        for (Long i = 0; i < N0; ++i)
            s += a[i];
        a += lda;
    }
    return s;
}
\end{lstlisting}


可见直接使用裸指针给函数传递数组是非常灵活的, 在 C 语言中这也是很常见的做法, 见 BLAS 和 LAPACK. 然而这么做的缺点是容易出错.

在 SLISC 库中, 函数输入或输出除了各种容器外, 也可以是 view 类型的一种. 例如上面的 \verb|sum| 函数也可以定义为
\begin{lstlisting}[language=cpp]
Doub sum(DvecDoub_I x)
{
    Doub s = x[0];
    for (Long i = 1; i < x.size(); ++i)
        s += x[x.step()*i];
    return s;
}
\end{lstlisting}
其中 \verb|DvecDoub| 是 view 类型的一种, 它包含一个指针 \verb|Doub *m_p|, 一个长度 \verb|Long *m_N| 以及一个步长 \verb|Long *m_step|. 它生成和毁灭时不会管理内存, 只用于表示另一个容器的一部分数据, 所以叫做 view, 也叫 slicing.

SLISC 提供一系列开头为 \verb|cut| 的函数, 可以从容器中截取一部分然后返回 view 类型的一种. 例如以上 \verb|sum| 函数可以这样使用
\begin{lstlisting}[language=cpp]
CmatDoub a(10, 10);
// 对第 3 行求和
cout << "sum row 3 = " << sum(cut2(a, 3)) << endl;
\end{lstlisting}
其中 \verb|cut1(a, 3)| 从矩阵 \verb|a| 中切下一行(\verb|1| 表示第 1 个维度, 即 “行”), 然后返回一个临时的 \verb|const DvecDoub &| 类型传给 \verb|sum| 函数.
