% SLISC 的文件读写

\verb|Bool file_exist(Str_I fname, Bool_I case_sens = true)| 判断指定文件是否存在

\verb|Bool dir_exist(Str_I path)| 判断路径是否存在

\verb|Long file_size(Str_I fname)| 获取文件大小(字节数), \verb|fname| 是文件名.

\verb|Str path2dir(Str_I fname)| 从 “路径+文件” 字符串中提取 “路径” 部分

\verb|void mkdir(Str_I path)| 创建路径

\verb|void rmdir(Str_I path)| 删除路径

\verb|void ensure_dir(Str_I dir_or_file)| 如果路径不存在, 创建该路径(可以任意多层)

\verb|void file_list(vecStr_O fnames, Str_I path, Bool_I append = false)| 列出指定路径中所有文件, \verb|append = true| 则 \verb|fnames| 原有的元素不被清空.

\verb|void file_list_full(vecStr_O fnames, Str_I path, Bool_I append = false)| 和 \verb|file_list()| 相同, 只是文件名包含路径.

\verb|void folder_list(vecStr_O folders, Str_I path, Bool_I append = false)| 列出指定路径中所有文件夹

\verb|void folder_list_full(vecStr_O folders, Str_I path, Bool_I append = false)| 和 \verb|folder_list()| 相同, 但包含路径.

\verb|void file_list_r(vecStr_O fnames, Str_I path, Bool_I append = false)| 列出指定路径中所有文件, 包括所有子文件夹中的.

\verb|void file_ext(vecStr_O fnames_ext, vecStr_I fnames, Str_I ext, Bool_I keep_ext = true, Bool_I append = false)| 在文件列表中选出所有具有某个拓展名的文件

\verb|void file_list_ext(vecStr_O fnames, Str_I path, Str_I ext, Bool_I keep_ext = true, Bool_I append = false)| 在某个路径中寻找所有具有某个拓展名的文件

\verb|void file_copy(Str_I fname_out, Str_I fname_in, Bool_I replace = false)| 复制文件 \verb|replace| 用于替换目标的同名文件, 否则会一直提示, 需要手动删除程序才能继续.

\verb|void file_copy(Str_I fname_out, Str_I fname_in, Str_IO buffer, Bool_I replace = false)| 用户提供缓存 \verb|buffer| 的 \verb|file_copy|, 提高速度.

\verb|void file_move(Str_I fname_out, Str_I fname_in, Bool_I replace = false)| 移动文件.

\verb|void file_move(Str_I fname_out, Str_I fname_in, Str_IO buffer, Bool_I replace = false)| 带用户缓存的版本.


