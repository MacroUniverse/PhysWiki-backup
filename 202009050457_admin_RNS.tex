% RNS 超弦

我们使用Ramond-Neveu-Schwarz(RNS)形式来修改玻色弦理论以引入费米子.这个方法在世界面上具有超对称.随后我们会使用具有时空超对称的Green-Schwarz形式.当时空维度是10的时候,这两个方案是等价的.

首先我们考虑共形规范下的Polyakov作用量.
\begin{equation}
S = - \frac{T}{2} \int d^2 \sigma \partial_\alpha X^\mu \partial^\alpha X_\mu~.
\end{equation}
加入自由费米子$\psi$之后,作用量如下
\begin{equation}\label{RNS_eq2}
S = - \frac{T}{2} \int d^2\sigma (\partial_\alpha X^\mu \partial^\alpha X_\mu - i \bar\psi^\mu \rho^\alpha \partial_\alpha \psi_\mu)~.
\end{equation}
$\rho^\alpha$是世界面上的狄拉克矩阵.因为世界面是1+1维的,所以$\rho^\alpha$也是1+1维的狄拉克矩阵.有两个这样的矩阵
\begin{equation}
\rho^0 = \begin{pmatrix}
0 & -i \\
i & 0
\end{pmatrix}~, \quad \rho^1 = \begin{pmatrix}
0 & i \\
i & 0
\end{pmatrix}~.
\end{equation}

\subsubsection{Majorana旋量}
$\psi^\mu = \psi^\mu(\sigma,\tau)$是两分量Majorana旋量.我们把它记作
\begin{equation}
\psi = \begin{pmatrix}
\psi_- \\
\psi_+
\end{pmatrix}~.
\end{equation}
在洛伦兹变换下,这些场按照矢量的规则变换.我们可以定义$\bar\psi^\mu$
\begin{equation}
\bar\psi^\mu = (\psi^\dagger)^\mu \rho^0~.
\end{equation}
我们再定义一个$\rho^3$矩阵如下
\begin{equation}
\rho^3 = \rho^0 \rho^1 = \begin{pmatrix}
1 & 0 \\
0 & -1 
\end{pmatrix}~.
\end{equation}
我们现在进行如下的定义
\begin{equation}
\begin{aligned}
\sigma^{\pm} & = \tau \pm \sigma~, \\
\partial_{\pm} & = \frac{1}{2} (\partial_\tau \pm \partial_\sigma)~, \\
\partial_\tau & = \partial_+ + \partial_-~, \quad \partial_\sigma = \partial_+ - \partial_- ~.
\end{aligned}
\end{equation}
经过计算,我们可以得出
\begin{equation}
\begin{aligned}
S_{\rm F} & = - \frac{T}{2} \int d^2 \sigma (- i \bar \psi^\mu \rho^\alpha \partial_\alpha \psi_\mu ) \\
& = - \frac{T}{2} \int d^2\sigma (-2i) (\psi_-\cdot\partial_+\psi_- + \psi_+\cdot \partial_- \psi_+) \\
& = i T\int d^2\sigma (\psi_-\cdot \partial_+ \psi_- +\psi_+\cdot \partial_- \psi_+ )~.
\end{aligned}
\end{equation}
从上面的作用量我们可以看出如下的运动方程
\begin{equation}
\partial_+\psi^\mu_- = \partial_-\psi^\mu_+ = 0~.
\end{equation}

\subsubsection{世界面上的超对称变换}
我们现在引入超对称变换的参数$\epsilon$.这个参数也是一个Majorana旋量
\begin{equation}
\epsilon = \begin{pmatrix}
\epsilon_- \\
\epsilon_+
\end{pmatrix}~.
\end{equation}
因为$\epsilon$的组成部分被取成常数,这代表了世界面上的全局坐标.超对称变换有如下形式
\begin{equation}\label{RNS_eq1}
\begin{aligned}
\delta X^\mu & = \bar \epsilon \psi^\mu ~, \\
\delta \psi^\mu & = - i \rho^\alpha \partial_\alpha X^\mu \epsilon ~.   
\end{aligned}
\end{equation}
我们的作用量在上面的超对称变换下保持不变.这个变换让自由的玻色子变成费米子,也让自由的费米子变成玻色子.按照分量来写的话,\autoref{RNS_eq1} 的第一个式子可以化简成如下形式
\begin{equation}
\delta X^\mu = \epsilon_- \psi_-^\mu + \epsilon_+ \psi_+^\mu~.
\end{equation}
\autoref{RNS_eq1} 的第二个式子可以化简成如下形式
\begin{equation}
\begin{aligned}
\delta \psi_-^\mu & = -2\partial_-X^\mu \epsilon_+ ~, \\
\delta \psi_+^\mu & = 2 \partial_+ X^\mu \epsilon_-~.
\end{aligned}
\end{equation}
在超对称变换下,我们的拉式量\autoref{RNS_eq2} 按照下面进行变换
\begin{equation}
\delta L = -T[ \partial_\alpha (\bar\epsilon\psi^\mu\partial^\alpha X_\mu) - \partial_\alpha \bar\epsilon (\rho^\beta\rho^\alpha\psi^\mu\partial_\beta X_\mu) ] ~.
\end{equation}
因为第一项是全导数项,我们忽略不计,我们可得守恒流是
\begin{equation}
J^\mu_\alpha = \frac{1}{2} \rho^\beta \rho_\alpha \psi^\mu \partial_\beta X_\mu ~.
\end{equation}

\subsubsection{能量动量张量}
考虑世界面坐标的平移变换
\begin{equation}
\sigma^\alpha \rightarrow \sigma^\alpha + \epsilon^\alpha~.
\end{equation}
玻色场$X^\mu$的变换如下
\begin{equation}
X^\mu \rightarrow X^\mu + \epsilon^\alpha \partial_\alpha X^\mu ~.
\end{equation}
费米子场的变换如下
\begin{equation}
\psi^\mu \rightarrow \psi^\mu + \epsilon^\alpha \partial_\alpha \psi^\mu ~.
\end{equation}
首先我们看拉式量里面的费米子部分,我们有
\begin{equation}
L_{\rm F} = - \frac{i}{2} \bar\psi^\mu \rho^\alpha \partial_\alpha \psi_\mu ~.
\end{equation}
我们对这一项进行变分
\begin{equation}
\begin{aligned}
\delta L_{\rm F} & = - \frac{i}{2} (\delta \bar\psi^\mu) \rho^\alpha \partial_\alpha \psi_\mu - \frac{i}{2} \bar\psi^\mu \rho^\alpha \partial_\alpha (\delta \psi_\mu) \\
& = -\frac{i}{2} (\epsilon^\beta\partial_\beta\bar\psi^\mu) \rho^\alpha \partial_\alpha \psi_\mu - \frac{i}{2} \bar\psi^\mu \rho^\alpha \partial_\alpha (\epsilon^\beta\partial_\beta \psi_\mu) ~.
\end{aligned}
\end{equation}
经过一番计算,我们得到
\begin{equation}
\delta L_{\rm F} = \partial_\alpha \epsilon^\beta \bigg( - \frac{i}{2} \bar\psi^\mu \rho^\alpha \partial_\beta \psi_\mu \bigg)~.
\end{equation}
我们需要进行对称化
\begin{equation}
\delta L_{\rm F} = \partial_\alpha \epsilon^\beta \bigg(    -\frac{i}{4} \bar\psi^\mu \rho^\alpha \partial_\beta \psi_\mu - \frac{i}{4} \bar \psi^\mu \rho^\beta\partial_\alpha \psi_\mu  \bigg)~.
\end{equation}
加上玻色部分,我们可以推出如下的能量动量张量
\begin{equation}
T_{\alpha\beta} = \partial_\alpha X^\mu \partial_\beta X_\mu + \frac{i}{4} \bar\psi^\mu \rho_\alpha \partial_\beta \psi_\mu + \frac{i}{4} \bar \psi^\mu \rho_\beta\partial_\alpha\psi_\mu - {\rm (Trace)}
\end{equation}
减去Trace那部分是为了保证能量动量张量保持无迹.这是标度不变性的要求.使用光锥坐标,我们可以把能量动量张量的非零部分计算如下
\begin{equation}
T_{++} = \partial_+X_\mu\partial_+X^\mu + \frac{i}{2} \psi^\mu_+ \partial_+ \psi_{+\mu}~.
\end{equation}
超对称的流是
\begin{equation}
J_+ = \psi^\mu_+ \partial_+ X_\mu~, \quad J_- = \psi^\mu_- \partial_- X_\mu ~.
\end{equation}
费米子的运动方程是
\begin{equation}
\partial_+\psi^\mu_- = \partial_- \psi^\mu_+ = 0~.
\end{equation}
玻色子的运动方程是
\begin{equation}
\partial_+\partial_- X^\mu = 0~.
\end{equation}
我们可以得到能量动量张量的守恒定律
\begin{equation}
\partial_- T_{++} = \partial_+ T_{--} = 0~.
\end{equation}

\subsubsection{模式展开和边界条件}
我们回到费米子的作用量
\begin{equation}
S_{\rm F} = \int d^2\sigma (\psi_-\partial_+\psi_-+\psi_+\partial_-\psi_+)~.
\end{equation}
以第二项为例,我们对其进行变分,我们得到
\begin{equation}
\delta \int d^2\sigma\psi_+ \partial_- \psi_+ = \int d^2\sigma [\delta\psi_+\partial_-\psi_+ + \psi_+ \partial_- (\delta\psi_+)]~.
\end{equation}
使用分部积分,我们有
\begin{equation}
\int d^2\sigma \psi_+ \partial_- (\delta\psi_+) = \int_{-\infty}^{\infty} d \tau \psi_+ \delta\psi_+ \bigg|_{\sigma = 0}^{\sigma = \pi} - \int d^2\sigma \partial_-\psi_+\delta\psi_+~.
\end{equation}
对另外一些项也进行类似处理,我们可以得到如下的边界项
\begin{equation}
\delta S_{\rm F} = \int_{-\infty}^{\infty} d\tau \{ (\psi_+\delta \psi_+ - \psi_-\delta \psi_-) \bigg|_{\sigma = \pi} - (\psi_+\delta\psi_+ - \psi_-\delta\psi_-) \bigg|_{\sigma = 0} \}~.
\end{equation}

\subsubsection{开弦的边界条件}
洛伦兹不变要求边界项必须为零.于是我们可以得到
\begin{equation}
\psi_+ \delta \psi_+ - \psi_- \delta \psi_- = 0~.
\end{equation}
在$\sigma = 0$处我们可以取
\begin{equation}\label{RNS_eq3}
\psi^+_\mu(0,\tau) = \psi^-_{\mu} (0,\tau) ~.
\end{equation}
更一般地来说$\psi_+ = \pm \psi_-$都可以让边界条件消失.我们一般选择\autoref{RNS_eq3} 作为$\sigma = 0$处的边界条件.这让$\sigma = \pi$处的边界条件ambiguous.我们可以选取两种不同的边界条件
\begin{itemize}
\item $\psi^+_\mu(\pi,\tau) = \psi^-_\mu(\pi,\tau)$ (Ramond)
给出的弦的态是时空费米子
\item $\psi^+_\mu(\pi,\tau) = - \psi^-_\mu(\pi,\tau)$ (Neveau-Schwarz) 给出的弦的态是时空玻色子
\end{itemize}

\subsubsection{开弦的模展开}
首先我们来考虑R部分.模式展开如下
\begin{equation}
\begin{aligned}
\psi^\mu_-(\sigma,\tau) = \frac{1}{\sqrt{2}} \sum_n d^\mu_n e^{- i n (\tau-\sigma)}~, \\
\psi^\mu_+(\sigma, \tau ) = \frac{1}{\sqrt{2}} \sum_n d^\mu_n e^{- i n (\tau+\sigma)}~.
\end{aligned}
\end{equation}
Majorana条件是费米子是实的.这令我们取
\begin{equation}
d^\mu_{-n} = (d^\mu_n)^\dagger~.
\end{equation}
$n$的取值是$n=0,\pm 1,\pm 2,\ldots$.

NS部分具有不同的模式展开.展开如下
\begin{equation}
\begin{aligned}
\psi^\mu_-(\sigma,\tau) = \frac{1}{\sqrt{2}} \sum_r b^\mu_r e^{- i r(\tau-\sigma)}~, \\
\psi^\mu_+(\sigma,\tau) = \frac{1}{\sqrt{2}} \sum_e b^\mu_r e^{- i r (\tau+\sigma)}~.
\end{aligned}
\end{equation}
这里的$r$取值是
\begin{equation}
r = \pm \frac{1}{2}, \pm \frac{3}{2}, \pm \frac{5}{2}~, \ldots.
\end{equation}

\subsubsection{闭弦的边界条件}
闭弦有两种边界条件
\begin{itemize}
\item $\psi_{\pm}(\sigma,\tau) = \psi_{\pm}(\sigma+\pi,\tau)$ (周期性边界条件) 
\item $\psi_{\pm}(\sigma,\tau) = -\psi_{\pm}(\sigma+\pi,\tau)$ (反周期性边界条件)
\end{itemize}

\subsubsection{闭弦的模式展开}
上面的两种边界条件可以分别应用到左行波和右行波上面.模式展开是
\begin{equation}
\begin{aligned}
\psi^\mu_+(\sigma,\tau) = \sum_r \tilde d^\mu_r e^{- 2 i r (\tau+\sigma)}~, \\
\psi^\mu_-(\sigma,\tau) = \sum_r \tilde d^\mu_r e^{- 2 i r (\tau-\sigma)}~.
\end{aligned}
\end{equation}
如果我们选择R部分,那么
\begin{equation}
r = 0, \pm 1, \pm 2, \ldots .
\end{equation}















































