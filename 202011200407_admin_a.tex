% 啊啊、
% a

\item \textbf{微波材料的隐身特性与材料的涂覆厚度有关,试计算本实验测得的微波材料涂覆在金属平板上,微波垂直入射时反射系数随材料厚度以及频率的变化,考察材料厚度对隐身性能的影响.}\\
从教材$^{\cite{jiaocai}}$实验11.3可知,对于涂覆在金属平板(假定其为理想导体)表面的单层吸波材料,空气与涂层界面处的输入阻抗为:
\begin{equation}
Z = Z_0\sqrt{\cfrac{\mu_r}{\varepsilon_r}}\tanh(\gamma d)
\end{equation}
其中$Z_0 = \sqrt{\frac{\mu_0}{\varepsilon_0}} = 377\Omega$是自由空间波阻抗,$\gamma$为电磁波在涂层中的传播常数,$d$为吸波涂层厚度,$\mu_r$,$\varepsilon_r$分别为涂层的相对介电常数和磁导率.
\begin{figure}[H]
\centering
\includegraphics[width=12cm]{fig/skt.pdf}\\
\caption{反射系数随涂层厚度、频率的变化}\label{skt}
\end{figure}
当电磁波由空气向涂层垂直入射时,在界面上的反射系数为:
\begin{equation}
\Gamma = \cfrac{Z-Z_0}{Z+Z_0}
\end{equation}
因此我们使用python对1mm-3mm厚度的图层、26.5MHz-40MHz频率的电磁波对应的反射系数$\Gamma$计算并作图,结果如图(\ref{skt}):
\end{enumerate}
