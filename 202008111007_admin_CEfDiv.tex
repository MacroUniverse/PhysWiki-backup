% 电场的高斯定理证明

\pentry{电场\upref{Efield}, 球坐标系中的梯度散度\upref{SphNab}, 狄拉克 delta 函数\upref{Delta}}

我们先看一个位于原点, 电荷为 $q$ 的点电荷产生的电场(\autoref{Efield_eq4}~\upref{Efield})
\begin{equation}\label{CEfDiv_eq1}
\bvec E(\bvec r) = \frac{q}{4\pi\epsilon_0}\frac{\uvec r}{r^2}
\end{equation}
要计算某点的散度, 最方便的做法是使用球坐标公式(\autoref{SphNab_eq2}~\upref{SphNab})得
\begin{equation}
\div \frac{\uvec r}{r^2} = \frac{1}{r^2} \pdv{r} \qty(r^2 \frac{1}{r^2}) = 0
\end{equation}
注意由于\autoref{CEfDiv_eq1} 在原点处无定义, 也不存在偏导数, 该结论不适用于 $\bvec r = \bvec 0$.

由于散度算符是线性% 未完成
的, 即使空间中有许多点电荷, 第 $i$ 个产生的电场为 $\bvec E_i(\bvec r)$, 空间中任何点(除了点电荷的位置)的电场散度都为零.
\begin{equation}\label{CEfDiv_eq3}
\div \qty(\sum_i \bvec E_i) = \sum_i \div \bvec E_i = 0
\end{equation}
我们还是假设只有一个点电荷位 $q$ 于坐标原点, 现在我们以原点为球心做一个半径为任意 $R > 0$ 的球面 $\mathcal S$, 并计算电场在球面上的通量为
\begin{equation}\label{CEfDiv_eq2}
\Phi = \oint_{\mathcal S} \bvec E \vdot \dd{\bvec s} = \frac{q}{4\pi\epsilon_0 R^2} \cdot 4\pi R^2 = \frac{q}{\epsilon_0}
\end{equation}
然而根据散度定理\autoref{Divgnc_eq1}~\upref{Divgnc}, 如果 $\div \bvec E$ 在球内处处为零, 应该有 $\Phi = 0$ 才对. 所以问题应该出在原点, 由于散度定理要求矢量场处处可偏导, 所以\autoref{CEfDiv_eq2} 的结果并不严格适用于散度定理\footnote{但有办法可以弥补, 见 “点电荷电场的散度”}. % 链接未完成

然而我们可以以另一种方式使用散度定理. 结合\autoref{CEfDiv_eq3} 和\autoref{CEfDiv_eq2}, 散度定理告诉我们: 如果任意另取一个包含球面 $\mathcal S$ 的闭合曲面 $\mathcal S'$ (正方向也向外), 那么电场在该曲面上的通量应该也是相同的. 这是因为, 我们可以把这两个曲面共同看成是它们之间的那块体积 $\bar{\mathcal V}$ 的边界(但 $\mathcal S$ 的正方向需要改变). 而 $\bar{\mathcal V}$ 中散度处处有定义且等于零, 所以由散度定理得
\begin{equation}
\Phi = -\oint_{\mathcal S} \bvec E(\bvec r) \vdot \dd{\bvec s} + \oint_{\mathcal S'} \bvec E(\bvec r)\vdot \dd{\bvec s} = \int_{\bar{\mathcal V}} \div \bvec E(\bvec r) \dd{V} = 0
\end{equation}
移项可得两个面积分相等.

进而可以得到, 对于任意两个包含一个点电荷 $q$ 的曲面(正方向都向外), 电场在它们上的通量都等于 $q/\epsilon_0$, 而 $q$ 也不一定需要在原点. 根据电场的叠加原理和点乘的分配律(\autoref{CrossP_eq1}~\upref{CrossP}), 如果曲面内有多个点电荷 $q_1, \dots, q_N$, 那么它们在曲面上的通量等于每个点电荷产生的通量之和, 即
\begin{equation}
\oint_{\mathcal S} \bvec E(\bvec r) \vdot \dd{\bvec s} = \frac{1}{\epsilon_0} \sum_i q_i
\end{equation}
这就是电场的高斯定理.


相同. 同样, 这个结论也包含.

可以推出原点处有无穷大的散度, 使得积分后恰好等于 $q/\epsilon_0$. 这让我们马上想到(三维)狄拉克 $\delta$ 函数\upref{Delta}, 在原点处为无穷大, 但体积分为有限值
\begin{equation}
\int \delta(\bvec r)\dd{V} = 1
\end{equation}

所以如果我们假设电场在原点处的散度为
\begin{equation}
\div \bvec E = \frac{q}{\epsilon_0}\delta(\bvec r)
\end{equation}
那么恰好可使电场满足散度定理
\begin{equation}
\int \div \bvec E \dd{V} = \frac{q}{\epsilon_0}
\end{equation}
