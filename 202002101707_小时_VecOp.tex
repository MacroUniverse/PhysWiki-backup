% 矢量算符

\pentry{矢量内积\upref{Dot}, 叉乘\upref{Cross}, 偏微分算符\upref{ParOp}}

\subsection{标量函数与矢量函数}
我们先区分两种函数, 第一种是普通的多元函数 $f(x, y, z)$, 也叫\textbf{标量函数}, 即自变量 $x, y, z$ 是实数, 因变量也是实数\footnote{一些情况下也可以是复数}. 另一种是\textbf{矢量函数}, 一般用粗体加以区分(手写的时候在上方加箭头), 如 $\bvec f(x, y, z)$, 即因变量是一个 3 维矢量\footnote{一些情况下也可以是 $N = 1, 2, \dots$ 维}. 矢量函数也可以记为三个分量的形式
\begin{equation}
\bvec f(x, y, z) = f_x(x, y, z) \uvec x + f_y(x, y, z) \uvec y + f_z(x, y, z) \uvec z
\end{equation}

\subsection{矢量算符}
定义三维的\textbf{矢量算符}为 (也叫 nabla 或 del 算符)
\begin{equation}
\Nabla = \uvec x \pdv{x} + \uvec y \pdv{y} + \uvec z \pdv{z}
\end{equation}
$\Nabla$ 作用在标量函数 $f(x, y, z)$ 上的结果称为函数的\textbf{梯度}, 是一个矢量函数
\begin{equation}
\Nabla f(x, y, z) = \qty(\uvec x \pdv{x} + \uvec y \pdv{y} + \uvec z \pdv{z}) f = \pdv{f}{x} \uvec x + \pdv{f}{y} \uvec y + \pdv{f}{z} \uvec z
\end{equation}
这可以类比矢量与标量的乘法.

$\Nabla$ 与矢量函数 $\bvec f(x, y, z)$ 的作用通常有两种定义, 第一是 “点乘”, 结果称为函数的\textbf{散度(divergence)}, 是一个标量函数
\begin{equation}
\begin{aligned}
\div \bvec f(x, y, z) &= \qty(\uvec x \pdv{x} + \uvec y \pdv{y} + \uvec z \pdv{z}) \vdot \qty(\uvec x f_x + \uvec y f_y + \uvec z f_z)\\
&= \pdv{f}{x} + \pdv{f}{y} + \pdv{f}{z}
\end{aligned}
\end{equation}
可以类比两个矢量的点乘(内积)\upref{Dot}.

另一种情况是 “叉乘”, 结果称为函数的\textbf{旋度(curl)}, 是一个矢量函数
\begin{equation}
\begin{aligned}
\curl \bvec f(x, y, z) &= \qty(\uvec x \pdv{x} + \uvec y \pdv{y} + \uvec z \pdv{z}) \cross \qty(\uvec x f_x + \uvec y f_y + \uvec z f_z)\\
&= \vmat{\uvec x&\uvec y&\uvec z\\ \pdv*{x}&\pdv*{y}&\pdv*{z}\\f_x & f_y & f_z}\\
&= \qty(\pdv{F_z}{y} - \pdv{F_y}{z})\uvec x + \qty(\pdv{F_x}{z} - \pdv{F_z}{x})\uvec y + \qty(\pdv{F_y}{x} - \pdv{F_x}{y})\uvec z
\end{aligned}
\end{equation}
可以类比两个矢量的叉乘\upref{Dot}.

另见拉\textbf{普拉斯算符}\upref{Laplac}.
