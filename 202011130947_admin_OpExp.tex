% 算符的指数函数 波函数传播子
% 算符|量子力学|指数函数|波函数|传播子

\pentry{一阶线性常微分方程组\upref{ODEsys}}

\subsection{不含时, 有限维的情况}
我们可以证明对任意厄米矩阵 $\mat H$, $\mat U(t) = \exp(-\I \mat H t)$ 都是酉矩阵, 成为\textbf{传播子(propagator)}. 要证明一个矩阵是酉矩阵, 只需要证明
\begin{equation}
\mat{U}\Her \mat U = \mat I
\end{equation}
由 $\exp(-\I \mat H t)$ 的级数定义以及厄米算符的性质可得 $\exp(-\I \mat H t)\Her = \exp(\I \mat H t)$, 所以
\begin{equation}
\mat U(t)\Her \mat U(t) = \exp(\I \mat H t) \exp(-\I \mat H t) = \exp(\mat 0) = \mat I
\end{equation}
注意只有 $[\mat A, \mat B] = 0$ 时才有
\begin{equation}
\exp(\mat A)\exp(\mat B) = \exp(\mat A + \mat B)
\end{equation}

\subsection{不含时, 无穷维的情况}

算符和有限维矩阵的性质往往有相同之处, 然而当拓展到无穷维的情况时往往就需要高级得多的数学(泛函分析), 我们暂不详细介绍这些数学, 而是直接通过类比给出结论.

将 “一阶线性常微分方程组\upref{ODEsys}” 中\autoref{ODEsys_eq1} 拓展成偏微分方程\footnote{形象理解: 将矢量 $\bvec v$ 看作有无穷多个元, 且取值连续, 就成了 $f(x, t)$}, 令 $A$ 为算符.
\begin{equation}
\pdv{t} f(x, y, \dots, t) = A f(x, y, \dots, t)
\end{equation}
那么当 $A$ 不含 $t$ 时, 有
\begin{equation}
f(x, y, \dots, t) = \exp(A t) f(x, y, \dots, 0)
\end{equation}
若 $A$ 含有 $t$, 形式解\autoref{ODEsys_eq3}~\upref{ODEsys} 变为
\begin{equation}
f(x, y, \dots, t) = \Q {\mathcal T} \exp\qty[\int_0^ t A(t')\dd t'] f(x, y, \dots, t)
\end{equation}

\subsection{含时薛定谔方程的解}

我们把哈密顿算符 $H$ 看作是无穷维矩阵, 薛定谔方程可记为与一阶线性常微分方程组(\autoref{ODEsys_eq1}~\upref{ODEsys})相同的形式
\begin{equation}
\dv{t} \ket{\psi(t)} = -\I H \ket{\psi(t)}
\end{equation}
当哈密顿算符 $H$ 不含时, 解为(根据\autoref{ODEsys_eq2}~\upref{ODEsys})
\begin{equation}
\ket{\psi(t)} = \exp(-\I H t) \ket{\psi(0)}
\end{equation}
当哈密顿算符含时, 形式上可以把解记为
\begin{equation}
\ket{\psi(t)} = \Q {\mathcal T} \exp(\int_0^ t H(t')\dd t') \ket{\psi(0)}
\end{equation}
我们把以上的 $\exp$ 就是波函数的传播子, 其定义依然是使用指数函数的级数展开.
