% 霍奇星算子
% Hodge算子|Hodge 算子|Hodge star operator|星算子|Hodge 星算子|外代数|Grassmann 代数|Exterior algebra|麦克斯韦方程组|对偶

\pentry{外导数\upref{ExtDer}}

\addTODO{未完成:需添加更多讨论以及麦克斯韦方程组的表示.}

\subsection{星算子的定义}

考虑 $n$ 维线性空间 $V$ 上的外代数 $\bigwedge V$,我们注意到各阶的外积空间具有明显的对称性:$\opn{dim}\bigwedge^k V=C^k_n=C^{n-k}_n=\opn{dim}\bigwedge^{n-k} V$.这意味着这样的一对空间之间存在线性同构,我们使用星算子 $\star$ 来描述这一同构.

星算子是一个映射,把一个 $\bigwedge^k V$ 中的元素 $\omega$ 映射为一个 $\bigwedge^{n-k} V$ 中的元素 $\star\omega$.为了方便描述星算子的定义,我们先引入一些新的表示方法.

选定 $V$ 的基 $\{\bvec{e}_i\}$,那么任意 $\omega\in\bigwedge^k V$ 都可以表示为各 $\bvec{e}_{i_1}\wedge\bvec{e}_{i_2}\wedge\cdots\wedge\bvec{e}_{i_k}$ 的线性组合,因此我们只需要描述 $\star\bvec{e}_{i_1}\wedge\bvec{e}_{i_2}\wedge\cdots\wedge\bvec{e}_{i_k}$ 即可定义星算子.

为了方便,我们只考虑 $\bvec{e}_{i_1}\wedge\bvec{e}_{i_2}\wedge\cdots\wedge\bvec{e}_{i_k}$ 中各 $i_{r+1}>i_r$ 的情况,也就是下标顺序排列的情况\footnote{乱序排列的情况无非两种,奇排列和偶排列,根据外代数的定义,前者加上负号即可,后者和顺序排列是相等的.}.规定了只考虑下标顺序排列的情况后,就可以暂时不管顺序的问题,把 $\bvec{e}_{i_1}\wedge\bvec{e}_{i_2}\wedge\cdots\wedge\bvec{e}_{i_k}$ 表示为集合 $\{i_1, i_2, \cdots, i_k\}$.利用这个表达,我们就可以定义 $\star\{i_1, i_2, \cdots i_k\}=\{1, 2, \cdots, n\}-\{i_1, i_2, \cdots, i_k\}$.这样,我们就定义出了 $\bigwedge V$ 中各基向量的星算子了.再加上一条“星算子是线性的”,即 $\star(\sum a_i\omega_i)=\sum a_i\star\omega_i$,就得到星算子的完整定义了:

\begin{definition}{霍奇星算子}
定义域 $\mathbb{F}$ 上 $n$ 维线性空间 $V$ 上的外代数 $\bigwedge V$ 的自同构映射 $\star:\bigwedge V\to \bigwedge V$,满足:
\begin{enumerate}
\item 线性性:对于任意 $a_i\in\mathbb{F}$,$\omega_i\in\bigwedge V$,有 $\star(\sum a_i\omega_i)=\sum a_i\star\omega_i$;
\item 如果把下标顺序排列的 $\bvec{e}_{i_1}\wedge\bvec{e}_{i_2}\wedge\cdots\wedge\bvec{e}_{i_k}$ 表示为集合 $\{i_1, i_2, \cdots, i_k\}$,那么 $\star\{i_1, i_2, \cdots i_k\}=\{1, 2, \cdots, n\}-\{i_1, i_2, \cdots, i_k\}$.
\end{enumerate}
称这一同构为\textbf{霍奇星算子(Hodge star operator)}.
\end{definition}

\subsection{麦克斯韦方程组的外微分形式:后两个方程\cite{KnotsVol4}}

星算子描述的是外积空间中一个非常显眼的对偶,即 $\bigwedge^k V$ 和 $\bigwedge^{n-k} V$ 之间的同构.




