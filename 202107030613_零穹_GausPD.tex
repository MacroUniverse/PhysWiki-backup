% 高斯分布(正态分布)
% 统计|随机变量|微积分|定积分|分布函数|平均值|标准差|方差|高斯分布|Gaussian Distribution|正态分布|Normal Distribution|偶函数|对称性

\begin{issues}
\issueTODO
\end{issues}

\pentry{分布函数\upref{RandF}}
高斯分布函数为
\begin{equation}\label{GausPD_eq1}
f(x) = \frac{1}{\sigma \sqrt{2\pi }} \exp\qty[-\frac{(x - \mu )^2}{2\sigma ^2}]
\end{equation}
其中 $\mu$ 是分布的的平均值, $\sigma$ 是标准差. 满足归一化条件
\begin{equation}
\int_{-\infty}^{\infty} f(x) \dd{x} = 1
\end{equation}
\addTODO{图, 例题, 平均值, 方差}

\subsection{推导}
\textbf{高斯分布(Gaussian Distribution)}又叫\textbf{正态分布(Normal Distribution)}, 具有如下形式
\begin{equation}\label{GausPD_eq2}
f(x) = A\exp\qty[-\lambda (x - x_0)^2]
\end{equation}
可见其主要特征就是指数函数中含有 $\Delta x^2$ 项.由对称性,分布函数是关于 $x =x_0$ 的偶函数,所以平均值显然为 $\mu = x_0$.首先我们补充两个积分, 由换元积分法\upref{IntCV}($x=\sqrt{t}$)以及 $\Gamma$ 函数\upref{Gamma} 的性质得
\begin{equation}\label{GausPD_eq3}
\int_{-\infty }^{+\infty } \exp (-x^2)\dd{x}  = \int_0^{+\infty} t^{-1/2}\E^{ - t} \dd{t}  = \qty(-\frac12)! = \sqrt \pi 
\end{equation}
\begin{equation}\label{GausPD_eq4}
\int_{-\infty}^{+\infty} x^2\exp(-x^2)\dd{x}  = \int_0^{+\infty} t^{1/2}\E^{-t} \dd{t}  = \frac12 ! = \frac12 \qty(-\frac12)! = \frac{\sqrt\pi}{2}
\end{equation}

根据分布函数的归一化条件,结合\autoref{GausPD_eq3} 得
\begin{equation}\label{GausPD_eq5}
1 = \int_{-\infty}^{+\infty} f(x) \dd{x}  = A\int_{-\infty}^{+\infty} \exp\qty[-\lambda (x - x_0)^2] \dd{x}  = A\sqrt{\frac{\pi}{\lambda}}
\end{equation}
即 $A = \sqrt{\lambda/\pi}$. 再来计算高斯分布的方差,结合\autoref{GausPD_eq4} 得
\begin{equation}\label{GausPD_eq6}
\sigma ^2 = \int_{-\infty}^{+\infty} (x - x_0)^2 A\exp\qty[-\lambda (x - x_0)^2] \dd{x}  = \frac{1}{2\lambda}
\end{equation}
用\autoref{GausPD_eq5} 和\autoref{GausPD_eq6} 解得 $\lambda = 1/(2\sigma^2)$ 和 $A = 1/(\sigma\sqrt{2\pi})$,代入\autoref{GausPD_eq2} 可得高斯分布\autoref{GausPD_eq1}.