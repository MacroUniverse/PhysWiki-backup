% 理想气体的熵:纯微观分析

设某理想气体由$N $个原子组成,体积为$V$,能量为$U$,现在,我们利用玻尔兹曼熵公式计算它的熵(可以加上一个常量).假设$N $为定值,但是$U $和$V $可以变化.我们要求的量是$S(U,V)$.

理想气体的能量全部为动能,与粒子的位置无关.我们必须求与$U$和$V$相对应的状态数$\Omega(U,V)$的对数.

我们已经知道,
\begin{equation}
\Omega \left( U,V \right) =\left( \frac{V}{a^3} \right) ^N\times \Omega _p\left( U \right) 
\end{equation}
式中$V/a^3$是每个原子可占据的位置的数量;$\Omega_p(U)$是内能为$U$的气体中,动量微观分布的数目.(计算自由膨胀的\textbf{熵变}时,膨胀前后内能不变,因此,我们忽略了$\Omega_p(U)$.现在$U$可以变化,所以需要计算$\Omega_p(U)$,这使得我们的工作会更艰苦一些).
内能为(对容器内每个原子的所有可能组态)
\begin{equation}
U=\sum_{i=1}^N{\frac{1}{2}m\left| \boldsymbol{v}_i \right|^2}=\sum_{i=1}^N{\frac{\left| \boldsymbol{p}_i \right|^2}{2m}}=\sum_{i=1}^N{\frac{p_{ix}^{2}+p_{iy}^{2}+p_{iz}^{2}}{2m}}
\end{equation}
式中$\boldsymbol p = m\boldsymbol v$是动量.

现在我们构造一个$3N $维的矢量$\mathbf P$:
\begin{equation}
\mathbf{P}=\left( p_{1x},p_{1y},p_{1z},p_{2x},\cdots ,p_{Nz} \right) 
\end{equation}
它不过是$N$个动量矢量$\boldsymbol p_i$的$3 $个分量的集合.如果我们将$P $的分量重新编号为$j=1, \cdots , 3N$,则
\begin{equation}
\mathbf{P}=\left( p_{1x},p_{1y},p_{1z},p_{2x},\cdots ,p_{Nz} \right) 
\end{equation}
这就是说
\begin{equation}
P_1=p_{1x},P_2=p_{1y},P_3=p_{1z},P_4=p_{2x},\cdots ,P_{3N}=p_{Nz}
\end{equation}
内能可以写成
\begin{equation}
U=\sum_{j=1}^{3N}{\frac{P_{j}^{2}}{2m}}
\end{equation}