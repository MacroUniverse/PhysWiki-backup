% SLISC 的稀疏矩阵

\begin{issues}
\issueDraft
\end{issues}

\pentry{SLISC 的密矩阵类\upref{SliMat}}

当一个矩阵对象中有大量 0 元的时候, 用密矩阵数据结构会浪费内存和计算量. COO 是一种常用的稀疏矩阵, 

\begin{lstlisting}[language=cpp]
class McooChar : public VbaseChar
{
private:
    typedef VbaseChar Base;
    using Base::m_p;
    using Base::m_N;
    Long m_N0, m_N1, m_Nnz;
    VecLong m_row, m_col;
public:
    using Base::p;
    McooChar(): m_N0(0), m_N1(0), m_Nnz(0) {};
    McooChar(Long_I N0, Long_I N1);
    McooChar(Long_I N0, Long_I N1, Long_I Ncap);
    Long *row_p();
    const Long *row_p() const;
    Long *col_p();
    const Long *col_p() const;
    // inline void operator<<(McooChar &rhs);
    void push(Char_I s, Long_I i, Long_I j);
    void set(Char_I s, Long_I i, Long_I j);
    Long n0() const;
    Long n1() const;
    Long size() const;
    Long nnz() const;
    Long capacity() const;
    Long find(Long_I i, Long_I j) const;
    Char &ref(Long_I i, Long_I j);
    Char operator()(Long_I i, Long_I j) const;
    Char &operator[](Long_I ind);
    Char operator[](Long_I ind) const;
    Long row(Long_I ind) const;
    Long col(Long_I ind) const;
    void trim(Long_I Nnz);
    void resize(Long_I N);
    void reserve(Long_I N);
    void reshape(Long_I N0, Long_I N1);
};
\end{lstlisting}
