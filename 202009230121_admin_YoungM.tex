% 杨氏模量

\begin{issues}
\issueDraft
\end{issues}

\footnote{本文参考\href{https://en.wikipedia.org/wiki/Young's_modulus}{维基百科}相关页面.}\textbf{杨氏模量(Young's modulus)}是衡量材料弹性的力学性质. 其定义为
\begin{equation}
E = \frac{FL_0}{A\Delta L}
\end{equation}
其中 $F$ 是受力, $L_0$ 是原长, $A$ 是横截面积, $\Delta L$ 是伸长或压缩的长度. 通常我们假设 $\Delta L \ll L_0$, 因为和弹簧一样, 过大的形变会导致非线性效应.

\begin{example}{}
一根横截面为 $2\Si{mm}$ 的钢丝, 松弛长度为 $1\Si{m}$, 要将其拉长 $1\Si{mm}$ 需要在两端施加多大张力?
\end{example}
