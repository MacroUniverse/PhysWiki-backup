% 子空间
% 矢量空间|线性空间|子空间|基底|矢量

\pentry{线性相关和线性无关\upref{LinInd}}

如果一个 $N_S$ 维的矢量空间 $\mathcal S$ 中的所有矢量都属于另一个 $N$ 维的矢量空间 $\mathcal V$, 且量空间中加法和数乘运算的定义相同. 那么前者就是后者的子空间. 严格来说所有矢量空间都是它本身的子空间(即 $N_S = N$ 的情况), 但我们一般是指 $N_S < N$ 的情况.

注意若我们在 $\mathcal V$ 中任意找一组基底, 不一定能恰好从中选出 $N_S$ 个使其作为 $\mathcal S$ 空间的基底. 若想让 $\mathcal V$ 空间的一组基底包含 $\mathcal S$ 空间的基底, 我们可以先在 $\mathcal S$ 空间中选 $N_S$ 个基底, 再在 $\mathcal S$ 空间外选取 $N - N_S$ 个基底即可\footnote{这样的基底一定是存在的, 因为 $N$ 维空间中任意给出 $N_S$ 个线性无关的矢量, 就必定能再找到另外 $N - N_S$ 个线性无关的矢量}.

\begin{example}{三维空间中的平面}
在三维的几何矢量空间 $\mathcal V$ 中, 若建立直角坐标系, 可以选 $\uvec x, \uvec y, \uvec z$ 作为一组正交归一的基底. 现在来看过原点且以矢量 $\uvec x + \uvec y + \uvec z$ 为法向量的平面, 平面方程为\footnote{参考高中数学教材中的立体几何章节}
\begin{equation}\label{SubSpc_eq1}
x + y + z = 0
\end{equation}
所有与该平面重合的矢量可以构成这个三维空间中的一个二维子空间 $\mathcal S$. 证明: 平面上的两个矢量相加仍然落在平面上, 数乘也同样落在平面上, 详细过程略. 

但是, 由于 $\uvec x, \uvec y, \uvec z$ 中任意一个都落在该子空间外面, 所以不可能选出两个作为子空间的基底. 如果需要是选一组 $\mathcal V$ 的基底且包含 $\mathcal S$ 空间的基底, 可以现在 $\mathcal S$ 空间中选两个基底(坐标满足\autoref{SubSpc_eq1}), 例如坐标为 $(1, -2, 1)/\sqrt{6}$ 和 $(1, 0, -1)/\sqrt{2}$ 的两个矢量. 再在空间外取一个基底, 如 $(1, 1, 1)/\sqrt{3}$. 注意这里给出的三个矢量是一组正交归一基底, 但原则上只需要线性无关即可. 在该情况下, 线性无关意味着三个几何矢量两两不共线且不共面(\autoref{LinInd_ex2}\upref{LinInd}).
\end{example}

\subsection{矢量张成的空间}
如果在 $N$ 维矢量空间 $V$ 中给定 $M$ 个矢量(不一定要求 $M \leqslant N$, 也不一定都线性无关) $\ket{v_1}, \dots, \ket{v_M}$, 那么容易证明
\begin{equation}
\ket{v} = \sum_{i=1}^M c_i \ket{v_i}
\end{equation}
的集合就是 $V$ 的一个子空间(包括 $V$ 本身). 我们把这个空间叫做 $\ket{v_1}, \dots, \ket{v_M}$ \textbf{张成}的空间. 当 $\ket{v_1}, \dots, \ket{v_M}$ 线性无关时, 该空间维 $M$ 维, 否则就小于 $M$ 维.

要获得张成空间的一组基底, 我们可以在 $M$ 个矢量中, 将所有可以表示为其他矢量线性组合的矢量剔除, 直到所有剩下的矢量线性无关, 就得到了一组基底.
