% 科里奥利力
% 惯性系|惯性力|非惯性系|旋转参考系|离心力|科里奥利力

\pentry{离心力\upref{Centri},平面旋转矩阵\upref{Rot2D}}%未完成

\textbf{科里奥利力(Coriolis Force)}是匀速旋转的参考系中由质点运动产生的惯性力.
\begin{equation}
\bvec F_c = 2m \bvec v_{abc} \cross \bvec \omega
\end{equation}
其中 $\bvec v_{abc}$ 是质点相对于旋转参考系 $abc$ 的瞬时速度, $\bvec\omega$ 是旋转系相对于某惯性系 $xyz$ 转动的角速度矢量.%未完成: 考虑使用脚注或链接
式中的乘法是叉乘\upref{Cross}.
在匀速转动参考系(属于非惯性系)中,若质点保持相对静止,则惯性力只有离心力.然而当质点与转动参考系有相对速度时,惯性力中还会增加一个与速度垂直的力,这就是科里奥利力.地理中的地转偏向力就是科里奥利力,可用上式计算(见“地球表面的科里奥利力\upref{ErthCf}”).


\subsection{推导}
\pentry{连续叉乘的化简\upref{TriCro}}

这里首先给出一个较符合直觉的结论(暂时不证). 若 $abc$ 系相对 $xyz$ 系以角速度 $\bvec\omega$ 旋转, 对任意一个随时间变化的矢量(假设一阶导数存在), 我们把它在 $xyz$ 和 $abc$ 系中的时间导数分别记为 $(\dv*{\bvec A}{t})_{xyz}$ 和 $(\dv*{\bvec A}{t})_{abc}$, 则有
% 未完成: 确定之前提过 \bvec v = \bvec \omega \cross \bvec r
\begin{equation}\label{Corio_eq4}
\qty(\dv{\bvec A}{t})_{xyz} = \qty(\dv{\bvec A}{t})_{abc} + \bvec\omega\cross\bvec A
\end{equation}

\begin{example}{}
令 $abc$ 系 $t = 0$ 时与 $xyz$ 系重合并绕 $z$ 轴逆时针匀速转动, 又令 $\bvec A(t) = \alpha t \uvec a$, 验证 $\bvec A(t)$ 满足\autoref{Corio_eq4}.

首先将 $\bvec A(t)$ 用 $\uvec x, \uvec y$ 基底表示为 $\bvec A(t) = \alpha t (\cos\omega t\, \uvec x + \sin\omega t\, \uvec y)$, 对其求导得
\begin{equation}
\qty(\dv{\bvec A}{t})_{xyz} = \alpha (\cos\omega t \,\uvec x + \sin\omega t \,\uvec y)
+ \alpha\omega t (-\sin\omega t \,\uvec x + \cos\omega t \,\uvec y)
\end{equation}
而在 $abc$ 系中求导为
\begin{equation}
\qty(\dv{\bvec A}{t})_{abc} = \alpha \uvec a = \alpha (\cos\omega t\, \uvec x + \sin\omega t \,\uvec y)
\end{equation}
最后,
\begin{equation}
\bvec\omega \cross \bvec A = (\omega \uvec c) \cross (\alpha t \uvec a) = \alpha\omega t \uvec c\cross\uvec a = \alpha\omega t \uvec b = \alpha\omega t(-\sin \omega t \,\uvec x + \cos\omega t \,\uvec y)
\end{equation}
将以上三式代入\autoref{Corio_eq4} 可验证\autoref{Corio_eq4} 成立. 注意以上我们将所有的矢量用 $\uvec x, \uvec y$ 基底表示, 类似地, 我们也可以将所有矢量用 $\uvec a, \uvec b$ 表示, 等式同样成立.
\end{example}

我们先令 $\bvec A$ 为质点的位矢 $\bvec r$, 得参考系中质点的速度关系为
\begin{equation}\label{Corio_eq5}
\bvec v_{xyz} = \bvec v_{abc} + \bvec\omega\cross\bvec r
\end{equation}
两边在 $xyz$ 系中对时间求导得
\begin{equation}\label{Corio_eq6}
\bvec a_{xyz} = \qty(\dv{\bvec v_{abc}}{t})_{xyz} + \bvec\omega\cross\bvec v_{xyz}
\end{equation}
注意 $abc$ 系中的加速度 $\bvec a_{abc}$ 并不是上式右边第一项, 而是 $(\dv*{\bvec v_{abc}}{t})_{abc}$. 令\autoref{Corio_eq4} 中的 $\bvec A = \bvec v_{abc}$, 得
\begin{equation}\label{Corio_eq7}
\qty(\dv{\bvec v_{abc}}{t})_{xyz} = \bvec a_{abc} + \bvec\omega\cross\bvec v_{abc}
\end{equation}
将\autoref{Corio_eq5} 和\autoref{Corio_eq7} 代入\autoref{Corio_eq6}, 得
\begin{equation}
\bvec a_{xyz} = \bvec a_{abc} + 2\bvec\omega\cross\bvec v_{abc} + \bvec\omega\cross(\bvec\omega\cross\bvec r)
\end{equation}
所以旋转参考系中的总惯性力(\autoref{Iner_eq1}\upref{Iner})为
\begin{equation}
\bvec f = m(\bvec a_{abc} - \bvec a_{xyz}) = -2m\bvec\omega\cross\bvec v_{abc} - m\bvec\omega\cross(\bvec\omega\cross\bvec r)
\end{equation}
其中第二项为离心力(\autoref{Centri_eq5}\upref{Centri}), 而第一项被称为科里奥利力.
