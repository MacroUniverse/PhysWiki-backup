% 电磁场张量
% 场张量|相对论|电磁场|张量变换|麦克斯韦张量|法拉第张量|麦克斯韦双矢量


\pentry{张量的分类\upref{CatTns}, 电动力学}

一个参考系中的电磁场需要用六个实函数数来刻画,三个用来刻画电场,三个用来刻画磁场.六个实数太过复杂,我们希望寻求一种简单的方式来简化表达.把六个实数合成一个对象的方法,最直接的当然是使用一个六维向量——不过这样并不能带来实质上的简化.实践中我们使用的其实是一个反对称张量场,用它来表示电磁场.

在狭义相对论里,时空是一个线性空间,事件的时空坐标随着基的不同而不同,而不同的基就代表不同的观察者,事件的坐标分量就是观察者的测量值.

和向量一样,任何张量只有给定了空间的基,才有“坐标分量”的概念.换句话说,只有有了观察者,才有观察者的测量值.张量本身不随基的选择而改变,改变的只是坐标分量.对于电磁场张量来说,其坐标分量,或称观察者的测量值,就是电场强度和磁场强度的空间分量,一共六个实函数.

%电磁场张量的推导暂缺.建议用常规推导,同时考虑编写外导数推导的相关词条后引用.
\subsection{电磁场张量的定义}

我们继续使用自然单位制,令$c=1$来简化表达.定义描述了电磁场张量的坐标分量与坐标系对应的电磁场分量观测值的联系.

\begin{definition}{电磁场张量}
一个伪黎曼流形上的电磁场是一个二阶反对称张量$F^{\mu\nu}$\footnote{“反对称”意味着$F^{\mu\nu}=-F^{\nu\mu}$.}.若在某基下其分量为$F^{01}=E_x, F^{02}=E_y, F^{03}=E_z, F^{23}=B_x, F^{31}=B_y, F^{12}=B_z$,那么在这个基对应的观察者所观察到的电场就是$\pmat{E_x, E_y, E_z}\Tr$,磁场就是$\pmat{B_x, B_y, B_z}\Tr$.
\end{definition}

按照本书的规范,电磁场张量作为一个双上标张量,应该表示为“列矩阵的列矩阵”这一嵌套样式——不过直接写列矩阵套列矩阵会很占篇幅,所以借助转置符号来简化表达:
\begin{equation}
F^{\mu\nu}=\pmat{
\pmat{F^{00}, F^{10}, F^{20}, F^{30}}\Tr\\
\pmat{F^{01}, F^{11}, F^{21}, F^{31}}\Tr\\
\pmat{F^{02}, F^{12}, F^{22}, F^{32}}\Tr\\
\pmat{F^{03}, F^{13}, F^{23}, F^{33}}\Tr
}
=
\pmat{
\pmat{0, -E_x, -E_y, -E_z}\Tr\\
\pmat{E_x, 0, -B_z, B_y}\Tr\\
\pmat{E_y, B_z, 0, -B_x}\Tr\\
\pmat{E_z, -B_y, B_x, 0}\Tr
}
\end{equation}

注意,这种表示方式是因为我们把$F^{\mu\nu}$看成是$F^{\mu}$这一类分量纵向排列的结果.这和一些课本中的方式看起来可能会不一样,比如Griffiths的《电动力学导论》中将$F^{\mu\nu}$写成一个二阶矩阵(而非列矩阵套列矩阵),看起来就和我们的规范表达不太一样:
\begin{equation}
F^{\mu\nu}=\pmat{
0, E_x, E_y, E_z\\
-E_x, 0, B_z, -B_y\\
-E_y, -B_z, 0, B_x\\
-E_z, B_y, -B_x, 0
}
\end{equation}









