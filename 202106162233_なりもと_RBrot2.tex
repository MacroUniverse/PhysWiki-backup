% 刚体定轴转动 2
% 力矩|角动量|刚体|惯性张量

\begin{issues}
\issueDraft
\end{issues}

\pentry{惯性张量\upref{ITensr}}

\subsection{转轴提供的力矩}
之前我们在讨论刚体绕固定轴转动时只涉及了角动量和力矩延轴方向的分量 $L_z$, $\tau_z$. 在学习了惯性张量以后我们就可以开始讨论垂直轴方向的角动量和力矩了.

我们假设刚体与转轴之间无摩擦, 那么转轴就不可能给杆提供延轴方向的力矩, 但是却可以提供垂直轴方向的力矩, 这里记为 $\bvec\tau_a$. 我们把所有外力对刚体的力矩记为 $\bvec\tau$, 于是根据角动量定理有
\begin{equation}
\dv{\bvec L}{t} = \bvec\tau + \bvec\tau_a
\end{equation}

之前我们看到, 刚体的运动完全是由外力矩的 $z$ 分量(延轴方向分量) $\tau_z$ 决定的, 所以我们可以断定 $\bvec\tau$ 的任何垂直分量都被 $\bvec\tau_a$ 中的 “反作用力矩” 抵消了\footnote{做一个类比, 假设质点只能延光滑的轨道运动(轨道可以是弯曲的), 那么只有外力延轨道的方向的分量能影响质点的运动, 垂直轨道方向的分量被轨道对质点的反作用力抵消了. 注意轨道给质点的力除了这个反作用力外, 还有一部分用于提供质点在垂直轨道方向的加速度分量.}. 所以不失一般性, 我们接下来假设 $\bvec\tau$ 与转轴共线.

再来看刚体的角动量(注意不仅是 $z$ 分量). 在学惯性张量时我们看到, 一般情况下角动量 $\bvec L$ 和角速度 $\bvec \omega$ 是不共线的. 我们来举一个例子.

=============== 未完成 ==================

例如把一个质量均匀分布的长杆倾斜固定在转轴上匀速旋转,% 图未完成
它绕轴旋转时的角动量始终垂直于长杆且随之一起转动. 这时运用角动量定理就会发现转轴必须给杆提供一个垂直于转轴的力矩.
