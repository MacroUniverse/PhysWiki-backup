% 闭合轨道的条件
% 比耐公式|稳定解|闭合轨道条件|Bertrand 定理

\pentry{比耐公式\upref{Binet}}

比耐公式可记为
\begin{equation}\label{ClsOrb_eq1}
\dv[2]{u}{\theta} + u = -\frac{m}{l^2 u^2} f\qty(\frac 1u)
\end{equation}
其中 $u = 1/r$, 有心力 $f$ 向外为正, $l$ 为轨道角动量. 首先易得圆形轨道满足
\begin{equation}
l = - m r^3 f
\end{equation}
任何 $f < 0$ 的力场都支持圆形轨道, 然而根据等效一维势能, $V' = V + l^2/(2mr^2)$ 必须在最低点的二阶导数大于零轨道才能稳定. 在稳定情况下, 令半径为 $r_0$, 如果给天体一个微扰, $r$ 会呈周期性波动, 我们不妨假设这个波动很小, 使
\begin{equation}
f(r) = f(r_0) + f'(r_0)(r-r_0)
\end{equation}
则可以证明\autoref{ClsOrb_eq1} 的解为
\begin{equation}
u = u_0 + a\cos\beta\theta
\end{equation}
其中
\begin{equation}\label{ClsOrb_eq5}
\beta^2 = 3 + \eval{\frac rf \dv{f}{r}}_{r = r_0}
\end{equation}
这里 $\beta$ 的意义是天体每转过一周关于 $r_0$ 振动的次数, 对于平方反比力的椭圆轨道, 显然有 $\beta = 1$. 所以, 当 $\beta$ 为有理数(即 $\beta = n_1/n_2$)时, 轨道是闭合的.

为了要求在任意距离的半径上轨道都闭合,  $\beta$ 必须不能随 $r$ 变化, 所以可以把\autoref{ClsOrb_eq5} 看成 $f(r)$ 的微分方程, 通解为
\begin{equation}
f(r) = - \frac{k}{r^{3-\beta^2}}
\end{equation}

\subsection{Bertrand 定理}
我们以上只考虑了一阶微扰的情况, 如果轨道与圆形轨道偏离较大, 如何找到闭合条件呢? 我们可以将 $f$ 由上面的微分近似变为高阶泰勒展开, 再来寻找比耐公式的解. J. Bertrand 在 1873 年证明, 只有当 $\beta^2 = 1$ 或 $\beta^2 = 4$ 的时候才能满足所有可能的轨道都闭合, 而这两种情况分别对应平方反比力, 以及胡克定律. 这个定理被称为 \textbf{Bertrand 定理}.