% 刚体的主轴
% 转动惯量|刚体|惯性张量|主轴|本征矢

\begin{issues}
\issueDraft
\end{issues}

\pentry{惯性张量\upref{ITensr}, 对称矩阵的本征问题\upref{SymEig}}

刚体转动惯量和角速度的关系为
\begin{equation}
\bvec L = \mat I \bvec \omega
\end{equation}
我们下面来讨论什么情况下 $\bvec L$ 和 $\bvec \omega$ 会共线.

我们知道惯性张量是一个 $3\times 3$ 的对称矩阵, 即 $I_{ij} = I_{ji}$. 而对称矩阵必定具有三个相互垂直的本征矢量 $\bvec \omega_i$, 满足
\begin{equation}
\bvec I \bvec \omega_i = I_i \bvec \omega_i \qquad (i = 1,2,3)
\end{equation}
注意把 $\bvec \omega_i$ 乘以任意常数仍然满足上式. 也就是说, 如果刚体绕着三个主轴旋转, 那么它的转动惯量与角速度共线.
