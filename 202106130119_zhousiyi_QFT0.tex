% 引言
% 量子场论|引言

\begin{issues}
\issueTODO
\end{issues}



\pentry{量子力学}

QED可能是我们有的最好的基本理论了.这个理论由一系列简单的方程(麦克斯韦方程和狄拉克方程)组成.方程的形式可以由相对论不变性定出来.这些方程的解决定着宏观和微观的物理.

费曼图给了这个理论同样优雅的计算步骤.假如我们想计算某个过程的发生概率,我们可以遵照如下步骤:

\begin{enumerate}
\item 画出这个过程对应的费曼图
\item 根据图写出散射振幅的表达式
\item 化简表达式
\end{enumerate}

\subsection{最简单的情形}
大多数粒子物理实验跟散射有关.所以,量子场论中最常见的就是散射振幅的计算了.那么我们现在就来做一个QED里面最简单的计算,也就是正负电子湮灭产生一对新的正负轻子(比如说$\mu$子)的过程.相关过程的示意图如下:

\begin{figure}[ht]
\centering
\includegraphics[width=10cm]{./figures/QFT0_1.png}
\caption{$e^+e^-\rightarrow\mu^+\mu^-$过程的费曼图} \label{QFT0_fig1}
\end{figure}

实验上,如果我们把一束电子束射到一束正电子束上,就能够实现这个过程.可观测量是这个过程的\textbf{散射截面}.散射截面是这个过程的值信息能力以及入射电子与出射$\mu$子之间的夹角.

为了简单起见,我们使用质心系.则这四个动量之间有如下关系:
\begin{equation}
\mathbf p' = - \mathbf p~, \quad \mathbf k'=-\mathbf k
\end{equation}
另外,我们需要假设质心系能量远大于电子以及$\mu$子的质量.于是我们有
\begin{equation}
|\mathbf p| = |\mathbf p'| = |\mathbf k| = |\mathbf k'| = E = E_{\rm cm}/2
\end{equation}
我们约定粗体代表三动量,斜体代表四动量.

因为电子和$\mu$子的自旋均为$1/2$,我们必须指定自旋的方向.那么比较方便的做法是取这个例子运动的方向作为坐标轴.那么每个粒子就会有平行于或者反平行于这个坐标轴的自旋.

但是实际上来说,电子和正电子束一般都是没有极化的.$\mu$子探测器也通常探测不到$\mu$子的自旋.所以一般的做法是对电子和正电子的自旋求平均,而对$\mu$子的自旋求和.

\textbf{微分散射截面}是一个很重要的物理量.它的定义如下
\begin{definition}{微分散射截面}
\begin{equation}\label{QFT0_eq3}
\frac{d\sigma}{d\Omega} = \frac{1}{64\pi^2E_{\rm cm}^2}\cdot |\mathcal M|^2
\end{equation}
\end{definition}
$E_{\rm cm}^{-2}$给出了微分散射截面的正确的量纲.散射振幅$\mathcal M$是无量纲的.$16\pi^2$只是约定.散射振幅的物理意义是一个过程发生的量子力学概率振幅.那么知道如何计算散射振幅是一个非常重要的技术.这个式子只是用于质心系并且末态是两个无质量例子的情况.更为一般的式子我们将在后面讲到.

需要注意的是,即使是对于如此简单的过程,$\mathcal M$的精确表达式也是未知的.但幸运的是,我们可以通过微扰展开的方法来对$\mathcal M$进行近似计算.这个计算方法是费曼发明的.他发明了一套“费曼图”技术,让$\mathcal M$的计算变得简单而且易于操作.
费曼图形象地画出了基本粒子的运动.

\begin{figure}[ht]
\centering
\includegraphics[width=5cm]{./figures/QFT0_2.png}
\caption{$e^+e^-\rightarrow\mu^+\mu^-$过程的费曼图} \label{QFT0_fig2}
\end{figure}

这个图时间是自下而上流动的.左右代表空间.所以这个图的意思是正负电子湮灭,产生了光子,最后生成了一对正负$\mu$子.图中的波浪线就代表光子.光子的四动量$q$可以用动量守恒算出来
\begin{equation}
q=p+p'=q+q'
\end{equation}
我们可以根据费曼规则直接写下$\mathcal M$.那么因为我们现在还没有学这个规则,我们来简略说一下它的思路. 首先,在量子力学的微扰理论中,我们可以用如下公式计算振幅
\begin{equation} \label{QFT0_eq1}
\langle \text{末态} |H_I| \text{初态} \rangle
\end{equation}
这里$H_I$代表相互作用哈密顿量.那么在我们这个例子里,因为初态是两个电子,所以
\begin{equation}
| \text{初态} \rangle = | e^+ e^- \rangle
\end{equation}
同样的,末态是
\begin{equation}
\langle \text{末态} | = \langle \mu^+ \mu^- |
\end{equation}
电子和$\mu$子的相互作用只能通过交换光子来实现.所以\autoref{QFT0_eq1} 没有一阶贡献.最低阶贡献是二阶的.贡献如下
\begin{equation}\label{QFT0_eq2}
\mathcal M = \langle \mu^+ \mu^- | H_I | \gamma \rangle^\mu \langle \gamma | H_I | e^+ e^- \rangle_\mu 
\end{equation}
这是一个比较粗糙的说法.电子的外腿相当于因子$| e^+ e^- \rangle $,$\mu$子外腿相当于因子$\langle \mu^+ \mu^- |$.顶点相当于是$H_I$.光子内线相当于因子$|\gamma\rangle\langle\gamma|$.这里$\mu$的意思是光子是个矢量粒子.那么实际上散射振幅$\mathcal M$可以按照\autoref{QFT0_eq2} 拆解为两个四矢量的点积.$\mathcal M$本身是一个洛仑兹不变的标量.

因为电子和光子的耦合强度是$e$,$\langle \gamma | H_I | e^+ e^- \rangle_\mu$应正比于$e$. 我们接下来考虑一种可能的自旋方向,如\autoref{QFT0_fig3} 所示
\begin{figure}[ht]
\centering
\includegraphics[width=10cm]{./figures/QFT0_4.png}
\caption{$e^+e^-\rightarrow\mu^+\mu^-$过程的极化示意图} \label{QFT0_fig3}
\end{figure}
电子和$\mu$子的自旋都和他们的运动方向平行,也就是说它们是右手的.反电子和反$\mu$子都是左手的.电子和正电子的自旋加起来相当于是$z$轴方向一个单位的角动量.因为$H_I$是保持角动量守恒的,光子必须有正确的极化矢量,才能保证角动量守恒.我们设光子的极化矢量为$\epsilon^\mu =(0,1,i,0)$,则
\begin{equation}
\langle \gamma | H_I | e^+ e^- \rangle^\mu \propto e (0,1,i ,0)
\end{equation}
$\mu$子的矩阵元应该具有跟$\mu$子的运动方向相对应的极化矢量.也就是说我们应该把\autoref{QFT0_eq3} 在$xz$平面旋转$\theta$角
\begin{equation}
\langle \gamma| H_I \mu^+\mu^-\rangle^\mu \propto e (0,\cos\theta, i, - \sin\theta) 
\end{equation}
那么计算散射振幅的话只需要把\autoref{QFT0_eq3} 点乘\autoref{QFT0_eq3} 就可以了,结果如下
\begin{equation}
\mathcal M(RL\rightarrow RL) = - e^2 (1+\cos\theta)
\end{equation}
当然这里只是一个粗糙的估计,并不能确定总体的系数.但因为我们预先知道了答案,采取了\autoref{QFT0_eq3} 这样的约定,我们这个结果的总体的系数是正确的.当$\theta = \pi$的时候,散射振幅为零.这是因为一个角动量是沿着$+z$方向的态,和一个角动量是沿着$-z$方向的态之间,是没有散射振幅的.

现在我们来考虑电子和正电子都是右手的情况.因为它们的总的自旋角动量是零,但光子必须有角动量.所以$\mathcal M (RR \rightarrow RL)$的振幅为零.另外几个非零的振幅列举如下
\begin{equation}
\mathcal M(RL\rightarrow LR) = - e^2(1-\cos\theta)  
\end{equation}
\begin{equation}
\mathcal M(LR\rightarrow RL) = - e^2 (1-\cos\theta)
\end{equation}
\begin{equation}
\mathcal M(LR \rightarrow LR) = - e^2(1+\cos\theta)
\end{equation}
把这些结果代入\autoref{QFT0_eq4} 可得
\begin{equation}\label{QFT0_eq15}
\frac{d\sigma}{d\Omega}  = \frac{\alpha^2}{4E_{\rm cm}^2}(1+\cos^2\theta) 
\end{equation} 
其中$\alpha = e^2/4\pi=1/137$. 我们把上面的微分截面对$\theta$和$\phi$进行积分可以得到总散射截面,结果如下
\begin{equation}\label{QFT0_eq16}
\sigma_{\rm total} = \frac{4\pi \alpha^2}{3E_{\rm cm}^2}
\end{equation}
\autoref{QFT0_eq15} 和 \autoref{QFT0_eq16} 跟实验的结果相比相差10\%.这是因为我们只考虑了最低阶的树图的贡献的原因.我们将在后面学习如何进行更精确的计算.



























%\subsection{}

