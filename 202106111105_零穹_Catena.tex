% 悬链线
% 微分方程|受力分析|绳子|悬链线|力学|微积分

\pentry{双曲函数\upref{TrigH}, 常微分方程\upref{ODE}}

\footnote{参考 Wikipedia \href{https://en.wikipedia.org/wiki/Catenary}{相关页面}.}\textbf{悬链线(catenary)}在物理上是指一条粗细不计的、 质量均匀分布的柔软绳子两端悬挂在相同高度的两个点后(\autoref{Catena_fig1} )当绳子在重力作用下达到平衡后形成的曲线.

\begin{figure}[ht]
\centering
\includegraphics[width=7cm]{./figures/Catena_1.pdf}
\caption{悬链线} \label{Catena_fig1}
\end{figure}

以绳子的最低点作为原点建立坐标系, 水平向右为 $x$ 轴, 竖直向上为 $y$ 轴. 那么悬链线可以用函数 $y(x)$ 来表示为
\begin{equation}\label{Catena_eq3}
y(x) = \frac{1}{k}[\cosh(kx)-1]
\end{equation}
其中 $\cosh$ 是双曲余弦函数\upref{TrigH}, 常数 $k$ 满足
\begin{equation}
\frac{2}{k}\sinh(\frac{ka}{2}) = L
\end{equation}
其中 $a$ 是两个悬挂点之间的距离, $L$ 是绳子的总长度.

\subsubsection{高度不相等的情况}
\begin{figure}[ht]
\centering
\includegraphics[width=7cm]{./figures/Catena_3.pdf}
\caption{悬挂点左低右高的悬链线} \label{Catena_fig3}
\end{figure}
如\autoref{Catena_fig3}, 假设悬挂点左低右高, 水平距离为 $a'$, 高度差为 $h' > 0$, 绳长为 $L'$. 悬链线仍然具有\autoref{Catena_eq3} 的形式, 我们希望根据 $a', h', L'$ 求出 $k$. 若假设如果把左边的悬链线补全到和右边一样的高度后, 绳两端相距为 $a$ (在下文推导中会使用), 那么 $k$ 满足
\begin{equation}\label{Catena_eq6}
\cosh(ka') - 1 = \frac{k^2}{2}(L'^2 - h'^2)
\end{equation}
注意左边的悬挂点未必需要在最低点的左边. 另外根据\autoref{Catena_eq6} 可以验证当 $L' = h'$ 时, $a' = 0$.

\subsection{推导}
\pentry{曲线的长度\upref{CurLen}, 常微分方程\upref{ODE}}

\addTODO{是否可以用欧拉—拉格朗日方程推出?}
\subsubsection{欧拉-拉格朗日方程法}
在该问题中,悬挂点是固定的两个点,绳子可看成经过这两个点的总长为 $L$ 的曲线,由能量最低原理,我们知道绳子对应曲线的形状将使得绳子的质心最低,曲线可用一个函数表示,那么问题就变成求这样一个函数,使得绳子质心最低,这显然是一个极值问题,更明确的说,是一个求泛函极值的问题.求解这样的问题的普遍方法为变分法.

我们以地面水平方向为 $x$ 轴, 两悬挂点水平方向中垂线为 $y$ 轴,设悬挂点坐标分别为 $(-\frac{a}{2},h)$ 和 $(\frac{a}{2},h)$,绳子对应的曲线函数为 $y(x)$,绳长 $L$,质量为 $m$,则其质心的 $y$ 坐标为
\begin{equation}
y_C=\frac{\int_{0}^{L}y(\frac{m}{L}\dd s)}{m}=\int_{-\frac{a}{2}}^{\frac{a}{2}} \frac{y\sqrt{1+y'^2}}{L}\dd x
\end{equation}
现在,问题就转化为求 $y(x)$,使得 $y_C$最小.这个问题可用欧拉-拉格朗日方程求解.

注意,上式的被积函数是 $y,y'$ 的函数,而与 $x$ 无关.令 
\begin{equation}\label{Catena_eq05}
F(y,y')=\frac{y\sqrt{1+y'^2}}{L}
\end{equation}
对不依赖与 $x$ 的被积函数 $F(y,y')$ 的情形,下面的方程成立:(插入链接)
\begin{equation}
F(y,y')-y'F_{y'}(y,y')=C
\end{equation}
其中 $C$为常数.
代入\autoref{Catena_eq05},得
\begin{equation}
\frac{y\sqrt{1+y'^2}}{L}-\frac{yy'^2}{L\sqrt{1+y'^2}}=C
\end{equation}
整理得
\begin{equation}
y'=\frac{\sqrt{y^2-(CL)^2}}{CL}
\end{equation}
即
\begin{equation}
\frac{\dd y}{\sqrt{y^2-(CL)^2}}=\frac{\dd x}{CL}
\end{equation}
两边积分,右边为
\begin{equation}
\frac{x}{CL}+C'
\end{equation}
其中 $C'$为常数.
左边,令 $y=CL\cosh t $,易得左边为
\begin{equation}
t+C''
\end{equation}
其中,$C''$为常数.于是
\begin{equation}
t+C''=\frac{x}{CL}+C'\Rightarrow t=\frac{x}{CL}+C_0
\end{equation}
其中,$C_0=C'-C''$为常数.

由于 $t=f^{-1}(y)$ 是$y=f(t)=CL\cosh t$的反函数,代入上式
\begin{equation}\label{Catena_eq13}
y=f(\frac{x}{CL}+C_0)=CL\cosh \frac{x}{CL}+C_0
\end{equation}
代入 $x=0$,得到最低点的 $y$ 值
\begin{equation}
y(0)=CL+C_0
\end{equation}
若以最低点为原点,坐标轴$x,y$方向不变,那么 \autoref{Catena_eq13} 变为
\begin{equation}
y=CL(\cosh \frac{x}{CL}-1)
\end{equation}
令$k=\frac{1}{CL}$,则
\begin{equation}
y=\frac{1}{k}[\cosh (kx)-1]
\end{equation}
\subsubsection{受力分析法}
\begin{figure}[ht]
\centering
\includegraphics[width=9cm]{./figures/Catena_2.pdf}
\caption{受力分析} \label{Catena_fig2}
\end{figure}
假设原点处的张力为 $T$, 绳的线密度为 $\lambda$, 那么区间 $[0, x]$ 的曲线长度为(\autoref{CurLen_eq1}~\upref{CurLen})
\begin{equation}\label{Catena_eq1}
L(x) = \int_0^x \sqrt{1 + \dot y(x')^2} \dd{x'}
\end{equation}
其中 $\dot y$ 表示函数 $y(x)$ 的导函数, 下文中 $\ddot y$ 则表示二阶导函数). 区间 $[0, x]$ 所受重力为 $G = \lambda L g$. 根据受力分析, $x$ 点的斜率为 $G/T$, 这样就得到了悬链线 $y(x)$ 的微分—积分方程
\begin{equation}
\dot y = \frac{g\lambda}{T} \int_0^x \sqrt{1 + \dot y^2} \dd{x'}
\end{equation}
两边对 $x$ 再次求导得二阶微分方程
\addTODO{如何对积分上限求导?放链接}
\begin{equation}
\ddot y = k \sqrt{1 + \dot y^2}
\end{equation}
其中令 $k = g\lambda/T$. 注意这是一个非线性二阶常微分方程, 我们还没学过如何求解. 但可以证明它的通解为
\begin{equation}\label{Catena_eq2}
y(x) = \frac{1}{k}\cosh(kx+C_0)+C_1
\end{equation}
其中,$C_0,C_1$为常数.
考虑到整个系统的对称性, $y(x)$ 是偶函数,这意味着 $C_0=0$,又 $y(0)=0$,所以 $C_1=-\frac{1}{k}$,我们有

那么, 如果已知两个悬挂点之间距离为 $a$, 绳子总长度为 $L$, 以及 $\lambda, g$. 如何求出 $k$ 或拉力 $T$ 呢? 根据\autoref{Catena_eq1} 有限制条件
\begin{equation}
\frac{L}{2} = \int_0^{a/2} \sqrt{1 + \dot y^2} \dd{x'}
\end{equation}
把\autoref{Catena_eq2} 代入有
\begin{equation}
\frac{2}{k}\sinh(\frac{ka}{2}) = L
\end{equation}
这样就可以解出 $k$, 进而求出 $T$.

\subsubsection{高度不相等的情况}
根据高度差 $h'$ 的定义可得
\begin{equation}\label{Catena_eq4}
\frac{1}{k}\cosh(k\frac{a}{2}) - \frac{1}{k}\cosh\qty[k\qty(\frac{a}{2}-a')] = h'
\end{equation}
由绳长的约束得
\begin{equation}
L' = \int_{a/2-a'}^{a/2} \sqrt{1 + \dot y(x')^2} \dd{x'}
\end{equation}
把\autoref{Catena_eq3} 代入该式后化简得
\begin{equation}\label{Catena_eq5}
\frac{1}{k}\sinh(k\frac{a}{2}) - \frac{1}{k}\sinh\qty[k\qty(\frac{a}{2}-a')] = L'
\end{equation}
分别把 \autoref{Catena_eq4} 和\autoref{Catena_eq5} 相加和相减得
\begin{equation}
a = \frac{2}{k} \ln\frac{k(h'+L')}{1 - \E^{-ka'}}
\qquad
a = -\frac{2}{k} \ln\frac{k(h'-L')}{1 - \E^{ka'}}
\end{equation}
令两式右边相等得\autoref{Catena_eq6}.
