% 解线性方程组

\begin{issues}
\issueDraft
\end{issues}

求解线性方程组(System of Linear equations)是科学计算中最普遍也是最为常见的问题,几乎所有与科学计算有关的问题都直接或间接与它有关.不论是常微分方程,偏微分方程,非线性方程,最优化,甚至是图像和信号处理,机器学习等等问题,最终都会转化成求解线性方程组.因此,线性方程组的解法也是科学计算领域里研究最广泛的问题之一.

线性方程的数值解法按照求解过程可以分为:	extbf{直接法}(Direct method)和	extbf{迭代法}(Iterative method).其中,直接法顾名思义就是直接求得方程组的解,这个解在很多情况下就是方程组的解析解.一般常用直接法为高斯消元法(Gauss Elimination)或者是LU分解(LU decomposition).

而相对应的,迭代法则是通过有限次的迭代,将数值解不断逼近解析解的过程.因此,迭代法通常都会引入一定的误差.这些误差可以通过增加迭代次数和改进方法逐渐逼近于机器精度.目前常见的迭代法包含了:雅可比法(Jacobi method),高斯-赛德尔迭代(Gauss-Seidel method),Krylov子空间法(Krylov subspace methods)等.由于迭代法对于数值代数的要求较高,这里就不做过多展开了.有兴趣的同学可以在下面留言,我会单独开一个子专栏进行讨论.

高斯消元法和LU分解

事实上,高斯消元法的过程就是构造LU分解的上下三角矩阵的过程.关于这个高斯消元法的基本算法我并不想过多的赘述,详细的过程可以在任意一本科学计算的书中或者网上找到,例如:



这里我想从更宏观的角度来分析一下高斯消元法和LU分解.这个方法的主要思路包含三步,以求解

图

 为例,我们接下来逐一解释.注意,这里的

图

  是一个

图

 的矩阵,

图

 都是

图

 的向量.

<ol><li>	extbf{基本的消元运算}</li></ol>通过高斯消元或者LU分解,得到

图

 , 其中

图

 和

图

 分别是

图

 的上,下三角矩阵.我们将

图

 中第

图

 行, 第

图

 列的元素记做

图

 ,那么消元的算法(伪代码)如下

经过消元得到新的

图

 矩阵实际上就是LU分解中的上三角

图

 矩阵.而在消元过程中用于临时存储系数的

图

 矩阵,加上一个单位矩阵就可以得到LU分解的下三角矩阵.	extbf{事实上,细心的同学可以发现,这样的LU分解可以直接在}

图

 	extbf{的存储空间上进行,无需额外的内存}.

经过简单的计算,这样的消元过程总共需要进行

图

 次浮点运算.

	extbf{2. 前向替换(Forward substitution)}

把

图

 看做一个整体

图

 ,将求解<img src="https://www.zhihu.com/equation?tex=Ax%3Db" alt="Ax=b" eeimg="1"/> 转化为求解

图

 .由于

图

 是下三角矩阵,它的第一行只有一个非0的元素

图

 ,因此这个求解过程可以简单的从第一行开始,逐行替换.

那么,整个的替换过程需要

图

 次浮点运算.

3. 后向替换(Backward substitution)

最后,我们利用从 2 中得出的

图

 , 求解

图

 ,从而得到原本的未知数

图

 .这个过程正好和求解下三角矩阵相反,需要从最后一行开始,依次向上.

同样的,它也需要

图

 次浮点运算.

小结

这样看起来,我们把求解一个线性方程组的问题转化成了一个LU分解和求解两个线性方程组,但是由于

图

 和

图

 都是三角矩阵,它们的求解过程非常简单,因此整个过程的总体运算复杂度始终是由LU分解所主导,即为

图

 .

例子1

让我们来用下面的代码直接测试一下高斯消元法的运算复杂度.

运行后可以得到如下输出

从结果的拟合曲线可以看出,求解线性方程组的总体运算时间基本符合

图

 的三次方函数.
 
 有兴趣的同学也可以试试用二次曲线拟合,看看是否符合.

例子2: 假设我们的计算机每秒可以处理

图

 次浮点运算,即1 giga FLOPS (Floating point operations per second),这其实比现在一般的笔记本电脑都要慢得多.下面这个表分别给出的是对于不同尺寸的问题,进行高斯消元运算和只进行向后替换的理论耗时.其中,

图

 是一次浮点运算所需的时间,即

图

 
即使是当今世界上最快的超级计算机,它们的运算速率可以达到

图

 FLOPS.如果用高斯消元法求解一个

图

 矩阵,也需要至少200年.

然而,很多实际应用问所需要求解的线性方程组的尺度经常会大于

图

 ,例如一些三维或者更高维度物理过程的模拟仿真,天气预报,等等.那么它们是如何被求解的呢?显然高斯消元只适用于中小尺寸的问题,对于大尺寸的线性方程组,我们需要其他运算复杂度更低的方法进行求解.这个我会在接下来的几期里陆续给大家介绍.
