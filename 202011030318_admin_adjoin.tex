% 伴随算符
% 伴随算符|算符|算子|厄米共轭

\begin{issues}
\issueDraft
\end{issues}

\pentry{厄米共轭\uprefupref      {HerMat}}

% 参考泛函分析笔显然可以记. 这里讲的是伴随, 不是自伴

若一个线性算符对应的矩阵是厄米矩阵, 那么它就是一个

有限维的线性算符可以表示为

无限维的情况下, 和对称算符有什么区别? 注意是定义域必须是希尔伯特空间的稠密子空间.
