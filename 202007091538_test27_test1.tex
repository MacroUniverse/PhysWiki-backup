% 信息熵公式推导
让$H(\frac{1}{n},\frac{1}{n},...,\frac{1}{n})=A(n)$,从上文关于$H$的第三个性质条件,
我们可以将$S^m$中进行一次等概率选择,分解为在$S$中进行$m$次等概率选择,即
\begin{equation}
A(s^m)=mA(s)
\end{equation}
类似的有,
\begin{equation}
A(t^n)=nA(t)
\end{equation}
我们令$n$为任意大小,并找到一个$m$满足:
\begin{equation}
s^m\leq t^n\leq s^{m+1}
\end{equation}
化为对数形式并除以$nlogs$
\begin{equation}
\frac{m}{n}\leq \frac{logt}{logs}\leq \frac{m}{n}+\frac{1}{n} or \abs{\frac{m}{n}-\frac{logt}{logs}}< \epsilon
\end{equation}
$\epsilon$为任意小,根据上文提到的第二条性质的单调性可得:
\begin{equation}
A(s^m)\leq A(t^n) \leq A(s^{m+1}), mA(s)\leq nA(t) \leq (m+1)A(s)
\end{equation}
因此,再除以nA(s)
\begin{equation}
\frac{m}{n}\leq \frac{A(t)}{A(s)}\leq \frac{m}{n}+\frac{1}{n} or \abs{\frac{m}{n} -\frac{A(t)}{A(s)} <\epsilon}
\end{equation}
将$(4)$和$(6)$进行合并得到
\begin{equation}
\abs{\frac{A(t)}{A(s)} - \frac{logt}{logs}} <2\epsilon ,A(t)=Klogt
\end{equation}
假定从n种可能选项中选择一个,可测量的概率为$p_i=\frac{n_i}{\sum n_i}$,$n_i$是整数.我们可以把$\sum{n_i}$中进行一次选择,分解为在概率为$p_i,...,p_n$的$n$种可能性中进行一次选择,紧接着假如选定了第$i$个,则以等概率从$n_i$中选择.根据上文性质三,初始$H$等于各个子事件的权加和:
\begin{equation}
klog\sum n_i=H(p_1,...,p_n)+K\sum {p_ilogn_i}=-K\sum {p_ilog\frac{n_i}{\sum n_i}}=-K\sum {p_ilogp_i}
\end{equation}
