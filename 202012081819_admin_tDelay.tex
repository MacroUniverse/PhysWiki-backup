% 量子散射的延迟

\pentry{散射问题}
一个一维波包用傅里叶变换表示为
\begin{equation}
\psi(x) = \int_{-\infty}^{+\infty} A(k) \exp(\I kx - \omega t) \dd{k}
\end{equation}
自由粒子情况下 $\omega = k^2/(2m)$. 如果经过一个势阱, 不同平面波透射后发生相移 $\phi(k)$, 经过后, 波包为
\begin{equation}
\psi(x, t) = \int_{-\infty}^{+\infty} A(k) \exp[\I kx - \omega t + \phi(k)] \dd{k}
\end{equation}
想象一个特殊情况: 经过势阱后 $A(k)$ 不变, $\phi(k) = \Delta t \omega$, 那么波函数变为
\begin{equation}
\psi(x, t) = \int_{-\infty}^{+\infty} A(k) \exp[\I kx - \omega (t - \Delta t) ] \dd{k}
= \psi(x, t - \Delta t)
\end{equation}
这样波包就具有了精确的延迟 $\Delta t$. 近似来说, 如果波包带宽较窄, 频率中心为 $\omega_0$, 那么在带宽以内可以把 $\phi(\omega)$ 近似看成是线性的, 那么延迟近似为
\begin{equation}
\Delta t = \eval{\dv{\phi}{\omega}}_{\omega = \omega_0}
\end{equation}
如果要取一个与波包形状无关的延迟的定义, 那么这个定义是最佳选择. 注意这样定义的延迟与频率有关. 这个延迟被称为 Wigner 延迟.
