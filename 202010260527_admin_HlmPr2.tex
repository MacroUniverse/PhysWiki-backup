% 亥姆霍兹定理证明 2
% 散度|旋度

\begin{theorem}{}
令 $\bvec F$ 为无散场, 即
\begin{equation}
\div \bvec F = 0
\end{equation}
则 $\bvec F(\bvec r)$ 总能表示为另一个矢量场 $\bvec G(\bvec r)$ 的旋度, 即
\begin{equation}
\bvec F = \curl \bvec G
\end{equation}
且 $\bvec G(\bvec r)$ 可以通过以下公式计算:
\begin{equation}\label{HlmPr2_eq1}
\bvec G = \int \bvec F \cross \frac{\bvec R}{R^3} \dd[3]{r'} + \bvec H(\bvec r)
\end{equation}
其中 $\bvec H(\bvec r)$ 是一个任意的无旋场($\curl \bvec H = \bvec 0$), $\bvec R = \bvec r' - \bvec r$, $R = \abs{\bvec R}$.
\end{theorem}

证明: 我们只需要证明\autoref{HlmPr2_eq1} 右边第一项求旋度等于 $\bvec F$:
\begin{equation}
\begin{aligned}
\curl \bvec G &= \int \curl \qty(\bvec F \cross \frac{\bvec R}{R^3}) \dd{V'}\\
&= \int\bvec F \qty(\div \frac{\bvec R}{R^3}) \dd{V'} -  \int(\bvec F \vdot \grad) \frac{\bvec R}{R^3} \dd{V'}
\end{aligned}
\end{equation}
其中第一个等号是因为 “对一个变量积分” 再 “对另一个变量求导” 这两个操作可以交换. % 链接未完成
先来证明第二项为零:
\begin{equation}
123
\end{equation}

考虑到
\begin{equation}
\div \frac{\bvec R}{R^3} = 4\pi \delta^3(\bvec R)
\end{equation}
所以
\begin{equation}
\curl \bvec G = \int \bvec F(\bvec r') 4\pi \delta^3(\bvec r - \bvec r') \dd{V'} = \mu_0 \bvec F(\bvec r)
\end{equation}
