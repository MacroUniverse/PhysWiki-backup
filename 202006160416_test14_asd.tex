% asd
Definition of FEV \\

$$Q(\bar{h})=c_k(\bar{h}-a_k)^{b_k}, k = 1,...,m$$ \\

$Q$ is flow rate and is always given as a function of $\bar{h}$ where $\bar{h}$ is the river level. $a_k$, $b_k$, and $c_k$ are coefficients of rating curve given by Environmental Agency (EA). \\

Flow Rate Equation\\

$$V \approx \sum_{i=1}^{n}(Q(\bar{h_i}))\Delta t$$

where $n$ is number of values for $Q(\bar{h})$ over a certain time period. \\

Excess Flow Rate Equations \\

$$V_e \approx V_{e1} = \sum_{i=1}^{n}(Q(\bar{h_i})-Q_T)\Delta t$$

where $Q_T$ is threshold flow rate i.e. $Q(h_T)$ and $n$ is number of $Q(\bar{h})$ values above threshold. $T_f$, the duration of the flood, is $n \Delta t$.\\

$$V_e \approx V_{e2} = T_f(Q_m-Q_T)$$

$Q_m$ is mean flow rate above the threshold flow rate i.e. $Q(h_m)$ and can found as follows:

$$Q_m = \frac{V_{e1}}{T_f}+Q_T$$ \\

$$V_e \approx V_{e3} = (h_m - h_T)T_f\frac{Q_{max}}{h_{max}}$$

where $Q_{max}$ is max flow rate and $h_{max}$ is the max height. Mainly used when there is no rating curve and $h_m$ is found by

$$Q_m = \frac{Q_{max}}{h_{max}}h_m$$


To calculate the FEV error $\sigma_{V_e}$ , we employ standard error-propagation techniques for multivariate nonlinear functions $y = f (x1, \ldots, xn)$. Given uncorrelated errors $\sigma_{x_j} , j = 1, \ldots, n,$ the square of the error $\sigma_y$ is:
\[{\sigma _{y}}^2=\left ( {\frac{\partial f}{\partial x_{1}}} \right )^2\sigma _{x_{1}}^2+\cdots+\left ( {\frac{\partial f}{\partial x_{n}}} \right )^2\sigma _{x_{n}}^2\]

which is an approximation (using truncated series expansions) of the true (unknown) error and valid only for small $\sigma_{x_j}$ . It is useful to recall and to rework the summed approximation :
\[V_{e}\approx \widehat{V}_{e}\equiv \sum_{k=1}^{N_{m}}\left(Q\left (\bar{h}_{k} \right )-Q\left (h_{T} \right ) \right )\Delta t=\Delta t\sum_{k=1}^{N_{m}} Q\left (\bar{h}_{k} \right )-N_{m}\Delta tQ\left ( h_{T} \right )= \Delta t\sum_{k=1}^{N_{m}}\bar{Q}_{k}-T_{f}Q_{T}\]

the square of the error $\sigma_{V_e}$ is:
\[{\sigma_{V_e}^2} = (\frac{\partial V_e}{\partial\bar Q_k})^2{\sigma_{\bar Q_k}^2} + (\frac{\partial V_e}{\partial Q_T})^2{\sigma_{Q_T}^2}+(\frac{\partial V_e}{\partial T_f})^2(\frac{\partial T_f}{\partial Q_T})^2{\sigma_{Q_T}^2}\]

the partial derivatives of $V_e$ can be readily derived as:
\[\frac{\partial V_e}{\partial\bar Q_k} = \Delta t;\quad  \frac{\partial V_e}{\partial Q_T} = -T_f;\quad  \frac{\partial V_e}{\partial T_f} = \frac{V_e}{T_f}\]


\section*{Verification of Code}
- We are going to be testing the FEV code on existing analyis such as the Stratford-upon-Avon flood in 2012, Hadfields flood in 2007, Calder flood in 2015, and Aire flood in 2015 (Reference Abbey and Onno). \\
- Then, we are going to adapt the code to add error bars. \\
- Also create code in Excel for a consistency check. \\

\section*{Return Period}
- Determining the return period of the river Aire. \\

\section*{River Aire 2020 Flood}
- Use multiple stations to find where best to implement flood mitigation plans. \\
- Using Armley, Crown Point, Birkin Holme, and Snaith Ings Stations.

\section*{Statistical Analysis}
- Linear Regression Model \\
- Extreme Events Statistics (Need Data from a Large Time Period)

\section*{Summary}
