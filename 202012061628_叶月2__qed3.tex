% 洛伦兹群覆盖群SL(2,C)的不可约表示
\pentry{洛伦兹群\upref{qed1},不可约表示\upref{qed1}}
\subsection{李代数的重新推导}
洛伦兹群的李代数是
\begin{equation}
\begin{aligned}
\left[J_{i}, J_{j}\right] &=i \epsilon_{i j k} J_{k} \\
\left[J_{i}, K_{j}\right] &=i \epsilon_{i j k} K_{k} \\
\left[K_{i}, K_{j}\right] &=-i \epsilon_{i j k} J_{k}
\end{aligned}
\end{equation}
洛伦兹群元表达为
\begin{equation}\label{qed3_eq1}
\Lambda=e^{-i \mathbf{J} \cdot \theta+i \mathbf{K} \cdot \phi}
\end{equation}
引入新算符
\begin{equation}
\mathbf{J}^{\pm}=\frac{\mathbf{J} \pm i \mathbf{K}}{2}
\end{equation}
经过计算,新的生成元的对易关系如下所示
\begin{equation}
\begin{aligned}
\left[J^{+, i}, J^{+, j}\right] &=i \epsilon^{i j k} J^{+, k} \\
\left[J^{-, i}, J^{-, j}\right] &=i \epsilon^{i j k} J^{-, k} \\
\left[J^{+, i}, J^{-, j}\right] &=0
\end{aligned}
\end{equation}
显然,这与$SU(2)$群的李代数一致,也就是说洛伦兹群包含了两份SU(2)的李代数.然而洛伦兹群不是单连通群 118 ,李群理论告诉我们,对于非单连通群,不存在李代数的不可约表示和群的表示之间的一一映射 .通过推
导洛伦兹群李代数的不可约表示,可以导出洛伦兹群的覆盖群的表示.可以证明,其覆盖群为$SL(2,C)$.所以$SL(2,C)=SU(2)\oplus SU(2)$.$SU(2)$的每一不可约表示都可以用$SU(2)$的Casimir元对应的标量\textbf{j}来标记.(回忆一下,Casimir算符是群元中与群的所有生成元都对易的算符.本征值在群元所作变换下不变所以可以拿来标记群表示.)设这两份$SU(2)$的$j$为$j_-,j_+$,则SL(2,C)可以以($j_-,j_+$)作为群表示的标记,维数为$j=(2j_-+1)(2j_++1)$
\subsection{SL(2,C)与SO(3,1)关系的简单证明}
令坐标矢量为$(x_0,x_1,x_2,x_3)$,$SO(3,1)$的群元为$\Lambda$.$x'^{\nu}=\Lambda_{v}^{\mu} x^{\nu}$,洛伦兹变换使得该时空距离不变,即$x^\nu x^\nu=x'^\nu x'^\nu$.
\\一般的厄米矩阵可以表示为
\begin{equation}
\chi=\left[\begin{array}{cc}
x_0+x_3 & x_1-\mathrm{i} x_2 \\
x+\mathrm{i}x_2 & x_0-x_3
\end{array}\right]
\end{equation}
显然,该矩阵行列式即为时空距离.对于$\chi'=M \chi M^{*}$,该线性变换同样不会改变时空距离,即矩阵的行列式.由于$\pm M$都对应同一个$\Lambda$,所以$SL(2,C)$是$SO(3,1)$的双覆盖.
\subsection{SL(2,C)的几个表示}
\subsubsection{($0,0$)表示}
该表示维数为1.$\mathbf{J}^{\pm}=0$,则$\mathbf{J},\mathbf{K}$都是0.所以是作用在标量上的简单表示,用来描述在洛伦兹变换下不变的“东西”.
\subsubsection{( $\frac{1}{2},0$)表示与($0,\frac{1}{2}$)表示}
这两个表示都是二维表示,且自旋为$\frac{1}{2}$,所以它们都是作用在旋量上的表示.对于( $\frac{1}{2},0$)作用的旋量,我们标记为$\left(\psi_{L}\right)_{\alpha},\alpha=1,2$,这就是左手的Wely旋量.($0,\frac{1}{2}$)作用的旋量,我们标记为$\left(\psi_{R}\right)_{\alpha},\alpha=1,2$.这就是右手的Wely旋量.
\begin{equation}
\text { Weyl spinors: } \quad \psi_{L} \in\left(\frac{1}{2}, 0\right), \quad \psi_{R} \in\left(0, \frac{1}{2}\right)
\end{equation}
接下来我们来推导这两个表示的具体形式.通过定义,我们知晓,( $\frac{1}{2},0$)表示意味着$\mathbf{J}^{-}$为2×2矩阵.同时,$\mathbf{J}^{+}=0$.符合这一点,也符合李代数的解是$\mathbf{J}^{-}=\boldsymbol{\sigma} / 2$,所以
\begin{equation}
\begin{aligned}
\mathbf{J} &=\mathbf{J}^{+}+\mathbf{J}^{-}=\frac{\sigma}{2} \\
\mathbf{K} &=-i\left(\mathbf{J}^{+}-\mathbf{J}^{-}\right)=i \frac{\sigma}{2}
\end{aligned}
\end{equation}
代入

\subsubsection{( $\frac{1}{2},\frac{1}{2}$)表示}
可以证明,该表示作用在一个二阶旋量上,且与四矢量是等价的.一个 4 向量是二阶旋量,即具有两个分量,根据洛伦兹群的( $\frac{1}{2},\frac{1}{2}$)表示变换的量.因此,洛伦兹变换作用的最基本的对象是旋量.