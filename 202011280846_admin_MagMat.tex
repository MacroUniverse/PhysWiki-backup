% 磁介质
% 磁介质|抗磁质|铁磁质|顺磁质

处于磁场中的实物物质,都会呈现不同程度的磁性,我们把物质具有磁性的过程称为\textbf{磁化(magnetization)}.一切能够被磁化的实物称为\textbf{磁介质(magnetic material)}.设某一电流分布在真空中激发的磁感应强度为$\mathbf B_0$,磁场中放进了某种磁介质后,磁化了的磁介质激发附加磁感应强度$\mathbf B'$,这时磁场中任一点的磁感应强度$\mathbf B $等于$\mathbf B_0$和$\mathbf B'$的矢量和,即
\begin{equation}
\mathbf B=\mathbf B_0+\mathbf B'
\end{equation}
由于磁介质的磁化特性不同,有一些磁介质磁化后使磁介质中的磁感应强度$B$稍大于$B_0$,\textbf{这类磁介质称为顺磁质(paramagnetic material)}, 例如锰、铬、铂、氮等都属于顺磁性物质;另一些磁介质磁化后使磁介质中的磁感应强度$B$稍小于$B_0$,即这类磁介质称为\textbf{抗磁质(diamagnetic material)},例如水银、铜、铋、硫、氯、氢、银、金、锌、铅等都属于抗磁性物质.一切抗磁质以及大多数顺磁质都有一个共同点,那就是它们所激发的附加磁场极其微弱,$B$和$B_0$相差很小.此外还有另一类磁介质,它们磁化后所激发的附加磁感应强度$B'$远大于$B_0$,这类能显著地增强磁场的物质,称为\textbf{铁磁质(ferromagnetic material)},例如铁、镍、钴、钆以及这些金属的合金,还有铁氧体等物质都是铁磁质.