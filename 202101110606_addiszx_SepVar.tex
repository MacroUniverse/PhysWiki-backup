% 分离变量法
% 偏微分方程|分离变量
\pentry{多元函数的傅里叶级数\upref{NdFuri}} % 未完成

在求解各种物理问题中的偏微分方程时, 我们经常使用\textbf{分离变量法}. 在分离变量法中, 我们假设微分方程的解可以表示为
\begin{equation}
f(x_1, \dots , x_N) = \sum_{i_1, \dots, i_N} c_{i_1, \dots, i_N} f_{i_1}(x_1) f_{i_2}(x_2) \dots f_{i_N}(x_N)
\end{equation}
即每个变量都具有一组一元函数, 这些一元函数的乘积的线性组合可以表示方程的解. 若将该式代入偏微分方程, 可以分别得到关于每个变量 $x_i$ 的常微分方程, 我们就说这个偏微分方程式\textbf{可分离变量}的.

% 参考 docx 版的百科
