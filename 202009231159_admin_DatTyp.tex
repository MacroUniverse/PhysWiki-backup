% 机器学习数据类型
\subsection{机器学习}
机器学习是一门多领域交叉学科,涉及概率论、统计学、逼近论、凸分析、算法复杂度理论等多门学科.专门研究计算机怎样模拟或实现人类的学习行为,以获取新的知识或技能,重新组织已有的知识结构使之不断改善自身的性能.
其中统计机器学习,是机器学习的一个重要的研究方向.统计机器学习是基于对数据的初步认识以及学习目的的分析,选择合适的数学模型,拟定超参数,并输入样本数据,依据一定的策略,运用合适的学习算法对模型进行训练,最后运用训练好的模型对数据进行分析预测.通常我们所说的传统机器学习多指统计机器学习.因此本篇中关于机器学习数据类型均为统计学相关定义.

\subsection{基本数据类型}
按照采用的计量尺度不同,可以将统计数据分为分类数据、顺序数据和数值型数据.
\subsubsection{分类数据}
分类数据是只能归于某一类别的非数字型数据,他是对事物进行分类的结果,数据表现为类别,是用文字来表述的.例如,人口按照性别分为男、女两类;企业按行业属性分为医药行业、家电行业、食品行业等,这些均属于分类数据,通常为了方便处理,对于分类数据可以用数字代码来表示各个类别,比如用1代表男性,0代表女性;用1表示医药行业,2表示家电行业,3表示食品行业,等等.
\subsubsection{顺序数据}
顺序数据是只能归于某一有序类别的非数字型数据.顺序数据虽然也是类别,但是这些类别是有序的.比如将产品分为一等品、二等品、三等品、次品等;考试成绩可以分为优、良、可、差等.同样顺序数据也可以用数字代码来表示.
\subsubsection{数值型数据}
数值型数据是按照数字尺度测量的观测值,其结果表现为具体的数值.现实中大部分数据处理的都是数值型数据.

分类数据和顺序数据说明的是事物的品质特征,通常是文字来表述,其结果表现为类别,因而可以统称为定性数据;数值型数据说明的是现象的数量特征,通常是数值来表现的,因此也称为定量数据.
\subsection{变量}
变量是说明现象某种特征的概念,其特点是从一次观察到下一次观察结果会呈现差别或者变化.变量可以分为以下几种类型:
\subsubsection{定类变量(nominal)}
变量的不同取值仅仅代表了不同类的事物,这样的变量叫定类变量.问卷的人口特征中最常使用的问题,而调查被访对象的“性别”,就是定类变量.
\subsubsection{定序变量(ordinal)}
变量的值不仅能够代表事物的分类,还能代表事物按某种特性的排序,这样的变量叫定序变量.问卷的人口特征中最常使用的问题教育程度,以及态度量表题目等都是定序变量,定序变量的值之间可以比较大小,或者有强弱顺序.
\subsubsection{定距变量(interval)}
变量的值之间可以比较大小,两个值的差值有实际意义,这样的变量叫定距变量.有0点,但是0点不代表没有,而是变量的值为0,例如:温度.
\subsubsection{定比变量(ratio variable)}
有绝对0点,如质量,高度,0点代表什么都没有.定比变量与定距变量的差别在于,定距变量取值为0时,不表示没有,仅仅是取值为0.
\subsection{四种变量的比较}
\begin{table}[ht]
\centering
\caption{四种变量的比较}\label{DatTyp_tab1}
\begin{tabular}{|c|c|c|c|c|}
\hline
支持计算 & 定类变量 & 定序变量 & 定距变量 & 定比变量 \\
\hline
计数、分布 & 是 & 是 & 是 & 是 \\
\hline
最大、最小 &   & 是 & 是 & 是 \\
\hline
范围 &   & 是 & 是 & 是 \\
\hline
百分比 &   & 是 & 是 & 是 \\
\hline
方差、标准差 &   &   & 是 & 是 \\
\hline
众数 & 是 & 是 & 是 & 是 \\
\hline
中位数 &   & 是 & 是 & 是 \\
\hline
平均数 &   &   & 是 & 是 \\
\hline
可计数 & 是 & 是 & 是 & 是 \\
\hline
可定义顺序 &   & 是 & 是 & 是 \\
\hline
可定义差异(加减计算) &   &   & 是 & 是 \\
\hline
绝对0点 &   &   &   & 是 \\
\hline
\end{tabular}
\end{table}
