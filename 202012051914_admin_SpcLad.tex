% 太空电梯

首先假设地球静止(事实上由于离心力, 张力还会更小)
\begin{equation}\label{SpcLad_eq1}
F(r) = \int_{r_0}^{r} \frac{GM\lambda(r')}{r'^2} \dd{r'}
\end{equation}
$\lambda(r)$ 是绳的密度, $r_0$ 是地球半径, $r$ 是绳索上某点与地心的距离. 我们需要 $r$ 达到地球同步轨道的高度.

绳截面可承受的最大压强为 $p$, 那么截面为
\begin{equation}
A(r) = F(r)/p
\end{equation}
绳的密度为常数 $\rho$, 那么线密度为
\begin{equation}
\lambda(r) = \rho A(r) = \rho F(r)/p
\end{equation}
带回\autoref{SpcLad_eq1} 有积分方程
\begin{equation}
F(r) = \frac{GM\rho}{p} \int_{r_0}^{r} \frac{F(r')}{r'^2} \dd{r'}
\end{equation}
令积分前面的常数为 $\alpha$, 两边求导
\begin{equation}
\dv{F}{r} = \alpha \frac{F(r)}{r^2}
\end{equation}
分离变量, 解得
\begin{equation}
F = C\E^{-\alpha/r}
\end{equation}
然而初始条件是 $F(0) = F_0$, $F_0$ 是载重. 代入得到方程的解为
\begin{equation}
F(r) = F_0 \exp[\frac{GM\rho}{p} \qty(\frac{1}{r_0} - \frac{1}{r})]
\end{equation}

目前最强的材料碳纳米管有 $p = 6.2\times 10^{10} \Si{Pa}$, 密度为 $\rho = 0.037$-$1.34\times 10^{3} \Si{kg/m^3}$. 同步轨道半径为 $4.2164\times 10^7 \Si{m}$. 地球质量 $M = 5.972 \times 10^{24} \Si{kg}$.
