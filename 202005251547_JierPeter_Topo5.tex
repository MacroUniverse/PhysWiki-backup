% 分离性
\pentry{紧致性\upref{Topo2}}

\subsection{分离性的种类一览}

分离性是描述一个拓扑空间里,任意的点、子集等彼此之间能被不相交的开集分开的程度.我会在这里先列出常见的分离性和它们的简单解释,但你不需要掌握所有分离性,只有其中两个是很重要的.

\begin{definition}{分离性的种类}
\begin{itemize}

\item $T_0$分离性,是指取空间中不同的两点$x,y$,总存在一个开集$U$,使得$U$包含其中一点而不包含另一点.
\item $T_1$分离性,是指取空间中不同的两点$x,y$,总存在两个开集分别含有其中一个点,但各自不含有另一个点.
\item $T_2$分离性,是指取空间中不同的两点$x,y$,总存在两个开集,分别含有其中一个点,并且这两个开集不相交.
\item $T_3$分离性,是指取空间中不同的两点$x,y$或者将$y$替换为一个不含$x$的闭集,那么总存在两个开集,分别含有其中一个点或闭集,并且这两个开集不相交.
\item 正规分离性,是指取空间中不相交的两个闭集$A, B$,总存在两个开集,分别含有其中一个闭集,并且这两个开集不相交.
\item $T_4$分离性,既正规分离又$T_2$分离的性质.


\end{itemize}
\end{definition}

这些分离性之间的区别很细微,看起来很绕,对不对?数学家们将分离性的分类做得比这个要详细得多,除了列表里的,他们还研究了诸如$T_{2.5}$分离性,$T_{3.5}$分离性,$R_1$分离性,完全正规分离性,正则分离性,正则Hausdorff分离性等非常多的分离性.但常用的重要分离性只有其中两个,\textbf{$T_2$分离性}和\textbf{正规分离性}.