% 李代数
% 李代数
\pentry{域上的代数\upref{AlgFie}}


李代数是对域上的代数\upref{AlgFie}进行的一种推广.域上的代数是指域上的线性空间配合了一个矢量乘法,使得这个线性空间在矢量乘法下也能构成一个环.李代数也是域上线性空间配合了一个矢量乘法,这个矢量乘法和构成环的乘法几乎一样,只有一点不同:将环乘法的结合律替代为Jacobi结合性.

\begin{definition}{李代数}
给定域$\mathbb{F}$上的一个线性空间$V$.在$V$上定义一个“乘法”运算:对于任意$\bvec{v}_1, \bvec{v}_2\in V$,将它们的运算结果记为$[\bvec{v}_1, \bvec{v}_2]$.称这个结构$(V, [*, *])$为一个李代数,如果它满足以下性质:
\begin{itemize}
\item \textbf{封闭性} 对于任意$\bvec{v}_1, \bvec{v}_2\in V$,$[\bvec{v}_1, \bvec{v}_2]\in V$.
\item \textbf{双线性性} 对于任意$k_i, c_i\in \mathbb{F}$和任意$\bvec{v}_i, \bvec{u}_i\in V$,都有$[k_1\bvec{v}_1+k_2\bvec{v}_2, c_1\bvec{u}_1, c_2\bvec{u}_2]=k_1c_1[\bvec{v}_1, \bvec{u}_1]+k_1c_2[\bvec{v}_1, \bvec{u}_2]+k_2c_1[\bvec{v}_2, \bvec{u}_1]+k_2c_2[\bvec{v}_2, \bvec{u}_2]$.
\item \textbf{反对称性} 对于任意$\bvec{v}, \bvec{u}\in V$,有$[\bvec{v}, \bvec{u}]=-[\bvec{u}, \bvec{v}]$.
\item \textbf{Jacobi结合性} 对于任意$\bvec{x}, \bvec{y}, \bvec{z}\in V$,有$[\bvec{x}, ([\bvec{y}, \bvec{z}])]+[\bvec{z}, ([\bvec{x}, \bvec{y}])]+[\bvec{y}, ([\bvec{z}, \bvec{x}])]=0$.
\end{itemize}
\end{definition}




