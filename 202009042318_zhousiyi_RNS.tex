% RNS 超弦

我们使用Ramond-Neveu-Schwarz(RNS)形式来修改玻色弦理论以引入费米子.这个方法在世界面上具有超对称.随后我们会使用具有时空超对称的Green-Schwarz形式.当时空维度是10的时候,这两个方案是等价的.

首先我们考虑共形规范下的Polyakov作用量.
\begin{equation}
S = - \frac{T}{2} \int d^2 \sigma \partial_\alpha X^\mu \partial^\alpha X_\mu~.
\end{equation}
加入自由费米子$\psi$之后,作用量如下
\begin{equation}\label{RNS_eq2}
S = - \frac{T}{2} \int d^2\sigma (\partial_\alpha X^\mu \partial^\alpha X_\mu - i \bar\psi^\mu \rho^\alpha \partial_\alpha \psi_\mu)~.
\end{equation}
$\rho^\alpha$是世界面上的狄拉克矩阵.因为世界面是1+1维的,所以$\rho^\alpha$也是1+1维的狄拉克矩阵.有两个这样的矩阵
\begin{equation}
\rho^0 = \begin{pmatrix}
0 & -i \\
i & 0
\end{pmatrix}~, \quad \rho^1 = \begin{pmatrix}
0 & i \\
i & 0
\end{pmatrix}~.
\end{equation}

\subsubsection{Majorana旋量}
$\psi^\mu = \psi^\mu(\sigma,\tau)$是两分量Majorana旋量.我们把它记作
\begin{equation}
\psi = \begin{pmatrix}
\psi_- \\
\psi_+
\end{pmatrix}~.
\end{equation}
在洛伦兹变换下,这些场按照矢量的规则变换.我们可以定义$\bar\psi^\mu$
\begin{equation}
\bar\psi^\mu = (\psi^\dagger)^\mu \rho^0~.
\end{equation}
我们再定义一个$\rho^3$矩阵如下
\begin{equation}
\rho^3 = \rho^0 \rho^1 = \begin{pmatrix}
1 & 0 \\
0 & -1 
\end{pmatrix}~.
\end{equation}
我们现在进行如下的定义
\begin{equation}
\begin{aligned}
\sigma^{\pm} & = \tau \pm \sigma~, \\
\partial_{\pm} & = \frac{1}{2} (\partial_\tau \pm \partial_\sigma)~, \\
\partial_\tau & = \partial_+ + \partial_-~, \quad \partial_\sigma = \partial_+ - \partial_- ~.
\end{aligned}
\end{equation}
经过计算,我们可以得出
\begin{equation}
\begin{aligned}
S_{\rm F} & = - \frac{T}{2} \int d^2 \sigma (- i \bar \psi^\mu \rho^\alpha \partial_\alpha \psi_\mu ) \\
& = - \frac{T}{2} \int d^2\sigma (-2i) (\psi_-\cdot\partial_+\psi_- + \psi_+\cdot \partial_- \psi_+) \\
& = i T\int d^2\sigma (\psi_-\cdot \partial_+ \psi_- +\psi_+\cdot \partial_- \psi_+ )~.
\end{aligned}
\end{equation}
从上面的作用量我们可以看出如下的运动方程
\begin{equation}
\partial_+\psi^\mu_- = \partial_-\psi^\mu_+ = 0~.
\end{equation}

\subsubsection{世界面上的超对称变换}
我们现在引入超对称变换的参数$\epsilon$.这个参数也是一个Majorana旋量
\begin{equation}
\epsilon = \begin{pmatrix}
\epsilon_- \\
\epsilon_+
\end{pmatrix}~.
\end{equation}
因为$\epsilon$的组成部分被取成常数,这代表了世界面上的全局坐标.超对称变换有如下形式
\begin{equation}\label{RNS_eq1}
\begin{aligned}
\delta X^\mu & = \bar \epsilon \psi^\mu ~, \\
\delta \psi^\mu & = - i \rho^\alpha \partial_\alpha X^\mu \epsilon ~.   
\end{aligned}
\end{equation}
我们的作用量在上面的超对称变换下保持不变.这个变换让自由的玻色子变成费米子,也让自由的费米子变成玻色子.按照分量来写的话,\autoref{RNS_eq1} 的第一个式子可以化简成如下形式
\begin{equation}
\delta X^\mu = \epsilon_- \psi_-^\mu + \epsilon_+ \psi_+^\mu~.
\end{equation}
\autoref{RNS_eq1} 的第二个式子可以化简成如下形式
\begin{equation}
\begin{aligned}
\delta \psi_-^\mu & = -2\partial_-X^\mu \epsilon_+ ~, \\
\delta \psi_+^\mu & = 2 \partial_+ X^\mu \epsilon_-~.
\end{aligned}
\end{equation}
在超对称变换下,我们的拉式量\autoref{RNS_eq2} 按照下面进行变换
\begin{equation}
\delta L = -T[ \partial_\alpha (\bar\epsilon\psi^\mu\partial^\alpha X_\mu) - \partial_\alpha \bar\epsilon (\rho^\beta\rho^\alpha\psi^\mu\partial_\beta X_\mu) ] ~.
\end{equation}
因为第一项是全导数项,我们忽略不计,我们可得守恒流是
\begin{equation}
J^\mu_\alpha = \frac{1}{2} \rho^\beta \rho_\alpha \psi^\mu \partial_\beta X_\mu ~.
\end{equation}

\subsubsection{能量动量张量}
考虑世界面坐标的平移变换
\begin{equation}
\sigma^\alpha \rightarrow \sigma^\alpha + \epsilon^\alpha~.
\end{equation}


























