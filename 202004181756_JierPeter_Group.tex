% 群论
\pentry{集合论}

\subsection{基本概念}

\begin{definition}{二元运算}
给定一个非空集合$A$,取A中任意两个元素$a, b$(a和b可能是同一个元素).如果有一条规则使得两个元素可以组合,指向另一个元素$c$,则称这条规则为一个\textbf{二元运算}(operation).运算的符号可以任意决定,如果使用
\end{definition}

运算的符号可以任意决定,如果使用$\cdot$作为运算符,那么定义中的情况就可以简单写为$a\cdot b=c$.注意,这里的$\cdot$只是表示某一个运算,不一定是我们通常的乘法或点乘运算.

和二元运算类似,我们也可以称更多元素间相组合的规则为多元运算.

\begin{definition}{群}
一个群$(G, \cdot)$是在集合$G$上赋予了一个运算$\cdot$的结构,该运算满足以下要求:\\


1.封闭性:$\forall x, y\in G, x\cdot y\in G$,即:任意G中元素x,y满足$x\cdot y$仍是G中元素;\\

2.结合性:$\forall x, y, z\in G, x\cdot(y\cdot z)=(x\cdot y)\cdot z$;\\

3.单位元存在性:$\exists e\in G, \forall x\in G, e\cdot x=x\cdot e=x$;\\

4.逆元存在性:$\forall x\in G, \exists y\in G, x\cdot y=e$.通常我们会把这样的$y$称作$x$的逆元,并记为$x^{-1}$. 

\end{definition}