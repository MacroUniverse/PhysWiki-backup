% 长度规范

\begin{issues}
\issueTODO
\end{issues}

\pentry{电磁场中的单粒子薛定谔方程\upref{QMEM}, 偶极子近似} % 链接未完成

我们只在使用偶极子近似时讨论长度规范, 因为我们接下来需要 $\bvec A$ 与位置无关. 当空间中存在静止的电荷分布时, 我们可以把标量势能分为 $V + \varphi$ 两部分. 前者由静止电荷根据库伦定律计算, 不参与规范变换, 在这里我们甚至可以不把它看成电磁力而只是某种其他势能. 令不含时哈密顿算符为
\begin{equation}
H_0 = \frac{\bvec p^2}{2m} + qV
\end{equation}
其中 $\bvec p$ 是所选规范下的广义动量算符(\autoref{QMEM_eq6}~\upref{QMEM})
\begin{equation}
\bvec p = m \bvec v + q\bvec A = -\I \grad
\end{equation}

我们令 $\bvec A_C, \varphi_C, \Psi_C$ 代表库仑规范, $\bvec A_L, \varphi_L, \Psi_L$ 代表长度规范. 将后者代入与规范无关的哈密顿量(\autoref{QMEM_eq2}~\upref{QMEM})得
\begin{equation}\label{LenGau_eq2}
H_L = H_0 - \frac{q}{2m} (\bvec A_L \vdot \bvec p + \bvec p \vdot \bvec A_L)
+ \frac{q^2}{2m} \bvec A_L^2 + q \varphi_L
\end{equation}


长度规范的思路是: 如果使用某种 $\bvec A' = 0$ 的规范, 就可以简化该式. 用不带撇的变量表示库仑规范, 我们令
\begin{equation}\label{LenGau_eq1}
\Psi_C(\bvec r, t) = \exp(\I q\chi)\Psi_L(\bvec r, t)
\end{equation}
\begin{equation}
\chi(\bvec r, t) = \bvec A(t) \vdot \bvec r
\end{equation}
利用规范变换为(\autoref{QMEM_eq5}~\upref{QMEM}), 以及 $-\pdv*{\bvec A}{t} = \bvec {\mathcal E}$, ($\bvec {\mathcal E}(t)$ 是除静电场以外的含时电场)(库伦规范+偶极子近似,引用未完成)以及 $\varphi = 0$
\begin{equation}
\bvec A' = \bvec A - \grad \chi = \bvec 0
\end{equation}
\begin{equation}
\varphi'(t) = \varphi + \pdv*{\chi}{t} = -\bvec {\mathcal E}(t) \vdot \bvec r
\end{equation}
由于形式不变, 把以上两式代入\autoref{LenGau_eq2} 得
\begin{equation}
H^L = H_0 - q\bvec{\mathcal{E}} \vdot \bvec r
\end{equation}
薛定谔方程为
\begin{equation}
H^L \Psi^L = \I \pdv{t} \Psi^L
\end{equation}
这种规范叫做\textbf{长度规范(length gauge)}.

\subsection{与速度规范的关系}
\addTODO{移动到速度规范}
对比\autoref{LenGau_eq1} 和\autoref{LVgaug_eq3}~\upref{LVgaug}, 有
\begin{equation}\label{LenGau_eq3}
\Psi^L = \exp[\I q(\chi^V - \chi^L)]\Psi^V = \exp[\I\frac{q^2}{2m}\int_{-\infty}^t \bvec A^2(t')\dd{t'} - \I q \bvec A\vdot \bvec r] \Psi^V
\end{equation}
