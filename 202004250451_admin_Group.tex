% 群论
\pentry{逻辑量词, 整数\upref{intger}}

\subsection{基本概念}

\begin{definition}{二元运算}
给定一个非空集合$A$,取$A$中任意两个元素$a, b$(可能是同一个元素). 如果有一条规则使得两个元素可以组合,指向另一个元素 $c$(不一定属于 $A$), 则称这条规则为一个\textbf{二元运算(binary operation)}. 
\end{definition}

运算的符号可以任意决定,如果使用$\cdot$作为运算符,那么定义中的情况就可以简单写为$a· b=c$. 注意,这里的 $\cdot$ 只是表示某一个运算,不一定是我们通常的乘法或点乘运算.

运算的符号可以任意决定,

和二元运算类似,我们也可以称更多元素间相组合的规则为多元运算.

\begin{definition}{群}
一个群$(G, \cdot)$是在集合$G$上赋予了一个运算$\cdot$的结构,该运算满足以下要求:\\

\begin{enumerate}
\item 封闭性:$∀x, y∈ G, x· y∈ G$,即:任意$G$中元素$x$,$y$满足$x· y$仍是G中元素
\item 结合性:$∀ x, y, z\in G, x·(y· z)=(x· y)· z$
\item 单位元存在性:$∃ e\in G, ∀ x∈ G, e· x=x· e=x$
\item 逆元存在性:$∀ x∈ G, ∃ y∈ G, x· y=y· x=e$.通常我们会把这样的$y$称作$x$的逆元,并记为$x^{-1}$\\


\end{enumerate}
严格来说,这样的一个群应该表示为$(G,\cdot)$,而$G$表示的是没有赋予运算的集合.但是为了方便讨论,我们通常也会直接用定义群的这个集合来称呼这个群,比如简单地把上述定义的群叫做\textbf{群$G$.}
\end{definition}

实际上,我们可以用更为弱化的公理系统来定义群,比如只要求存在左逆元,即第4条中只要求$∀ x∈ G, ∃ y∈ G, y· x=e$. 在这种情况下我们仍然可以证明左逆元都是右逆元.但是为了方便理解,我们用了以上对称的公理系统.

\subsection{群的例子}


\begin{example}{二元群}\label{Group_ex1}
定义一个只含有两个元素的集合,记为$\{0, 1\}$.在这个集合上定义运算 “$\cdot$”,由于只有四种运算方式,所以可以通过列举出每一个运算的结果来定义这个运算:
\begin{enumerate}
\item $$0\cdot0=0$$
\item $$0\cdot1=1$$
\item $$1\cdot0=1$$
\item $$1\cdot1=0$$
\end{enumerate}



容易验证,这个运算满足四个群运算的定义,因此这个二元集合配上运算$\cdot$构成一个群.

\end{example}

在以上例子中,尽管$0\cdot1=1\cdot0$,我依然把它们分别单独写了出来,这是因为群的运算没有要求可交换,也就是说,群运算允许$x\cdot y\neq y\cdot x$的存在.群元素选为$0$和$1$没有特殊原因,只是代表这是群里两个不同的元素而已,任何由两个元素构成的群我们都看作同一个.运算满足交换律的群,被称为\textbf{阿贝尔群(abelian group)}, 否则称为\textbf{非阿贝尔群(non-abelian group)}. 习惯上,我们把阿贝尔群的运算叫做加法,记为“$+$”,而把非阿贝尔群的运算叫做乘法,记为“$\cdot$”,甚至简化为没有符号,比如$ab\not= ba$.

一般地,由于在朴素集合论中我们最多只讨论了集合的基数问题,集合中的元素具体如何命名是没有约束的,因此在集合论意义下元素数目相同的集合都看作同一个.比如说,我们认为$\{\text{猪},3, Kaiser \}$和$\{1,2,3\}$是同一个集合.而现在在集合上定义了一个群运算以后所得到的群,即使构成它们的集合相同,群也可能由群运算的不同而产生不同的结构,从而被看作是不同的群.

\begin{example}{n元循环群}\label{Group_ex2}

取一个由$n$个元素组成的集合$G$,由于集合元素命名的任意性,不妨把$G$记为$\{0, 1, \cdots n-1\}$,定义运算为模$n$的加法,即在一个有$n$个整点的钟表上的加法(见词条\textbf{整数}\upref{intger}).那么这个运算构成$G$上的一个群运算,所构成的群$G$称为$n$元循环群.




\end{example}

\begin{example}{n个元素的置换群}\label{Group_ex3}

首先给定一个$n$元集合,记作$K=\{1,2, \cdots, n\}$,并将$K$中的元素按现有的顺序编号. 把$K$看作是$n$个桶中分别装了1个写着编号的球,初始状态下球的编号和桶的编号一致.我们可以把球从桶里面拿出来并进行任意的置换,保持每个桶里还是只有一个球,但是球的编号不一定和桶的编号一致了.每一个置换可以详细描述为“把1号桶的球和2号桶的球交换”,“把1号桶的球放入3号桶,3号桶的放入4号桶,4号桶的放入1号桶”等等.

我们用全体“置换”动作来作为元素,构成一个集合,称作$n$个元素的置换集合($n$元置换集),记为$S_n$.\textbf{从原始状态}进行任意置换,所得到的结果状态和置换是一一对应的,所以我们也可以用“从原始状态进行置换$f$所得的结果”来表示置换$f$本身.

(未完待续)


\end{example}