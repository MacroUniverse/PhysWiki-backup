% 球坐标系中的角动量算符

\pentry{角动量(量子)\upref{QOrbAM}}

本文使用原子单位制\upref{AU}.先看角动量平方算符 $L^2$ 在球坐标中的形式. 一种简单的方法是, 经典力学中, 球坐标系中哈密顿量可以记为(\autoref{HamCan_eq3}~\upref{HamCan})
\begin{equation}
H = \frac{p_r^2}{2m} + \frac{L^2}{2mr^2} + V
\end{equation}
其中 $p_r = m\dot r$, $L = mr^2\dot\theta$.

而球坐标中, 哈密顿算符为
\begin{equation}
H = -\frac{1}{2m}\laplacian + V = -\frac{1}{2m}\laplacian_r -\frac{\laplacian_\Omega}{2mr^2} + V
\end{equation}
这让我们很容易猜出 $p_r^2$ 和 $L^2$ 对应的算符分别 $-\laplacian_r$ 和 $-\laplacian_\Omega$, 事实也的确如此. 要想严格证明, 我们必须把\autoref{QOrbAM_eq2}~\upref{QOrbAM}通过链式法则\upref{PChain}用球坐标表示(留作习题).
\begin{equation}
L_x = \I \qty(\sin\phi\pdv{\theta} + \cot\theta\cos\phi\pdv{\phi})
\end{equation}

\begin{equation}
L_z = -\I\pdv{\phi}
\end{equation}

