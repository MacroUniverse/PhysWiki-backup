% 二元关系
\pentry{朴素集合论\upref{NSet}}
\subsection{关系}

在朴素集合论\upref{NSet}中我们只关心了集合的基数,即集合中元素的数目.在这种语境下,任何两个元素数量相同的集合都可以看作是同一个集合.但是仅仅讨论集合的基数未免太过单调,缺少了很多有意思的理论,于是我们希望在集合的元素之间建立一些结构,来进行更细致的划分和研究.

\textbf{关系(relation)}是集合上最基础的一种结构.给定一个关系,我们就可以讨论一些元素之间\textbf{是否满足}这个关系.比如说,如果取一家三口构成一个集合,$\sim$代表的是“年龄大于”,那么我们可以说“爸爸对于孩子具有这个关系”,但是反过来“孩子对于爸爸不具有这个关系”.从这个例子可以看出,关系的表达方式很灵活,而且可以是有方向性的.讨论关系时,我们唯一关心的是给定元素之间是否具有这样的关系.

关系可以用在两个元素之间,也可以用在三个元素之间,甚至可以用在不特定的元素之间.

\subsection{二元关系}

在绝大多数数学和物理领域,我们只关心集合上的\textbf{二元关系(binary relation)}.如果$\sim$是在集合$A$上定义的一个二元关系,那么任意给定两个元素,我们都可以讨论它们之间是否具有这种关系,但如果给定三个元素,讨论就没有意义了.比如,如果$\sim$的定义是“年龄大于”,那么把三个人的年龄都拿过来比较就没有意义;不过,如果$\sim$的定义是“比后面两个人的年龄都大”,那么$\sim$就可以用在三个人身上.

对于集合$A$上的二元关系$\sim$,如果$x, y\in A$满足这个关系,我们可以把这句话表述为$x\sim y$.如果不满足,则可以表述为$x\not\sim y$.用这样简洁的表示方法,我们可以把以上“年龄大于”的关系表述如下:

$$\text{爸爸}\sim\text{孩子}$$ $$\text{妈妈}\sim\text{孩子}$$ $$\text{孩子}\not\sim\text{爸爸}$$ $$\text{孩子}\not\sim\text{妈妈}$$

爸爸和妈妈之间,在没有进一步信息的情况下,无法判断是否满足这个关系,但这个关系是存在的.



\subsection{等价关系}

最基础的一类关系,是\textbf{等价关系(equivalence)}. 在集合$A$上定义的关系$\sim$是一个等价关系,要求满足条件:
\begin{enumerate}
\item 自反性:$\forall x\in A, x\sim x$
\item 对称性:$\forall x, y\in A, x\sim y \Rightarrow y\sim x$
\item 传递性:$\forall x, y, z\in A, x\sim y, y\sim z\Rightarrow x\sim z$
\end{enumerate}
如果把“具有关系$\sim$”看成是两个元素间相互连接(没有方向性,因为对称性),那么传递性保证了当多个元素相连时,这些元素也两两互联;自反性甚至保证了每个元素必然和自身相连.由此一来,我们可以根据等价关系把集合$A$划分成\textbf{等价类(equivalence class)},每个等价类都是$A$的一个子集,所包含的元素彼此相连.

等价类的划分是数学中非常重要的思维方法,它可以利用等价关系来大大简化一个集合的复杂程度,将每一个等价类中的元素都看成是无差别的.有了等价关系后,我们可以用相应的等价类来组成一个新的集合,这样的集合被称作\textbf{商集(quotient set)}.和原来集合中的元素不一样,商集中的元素是原集合中的子集.

作为例子,我们取全体非负整数的集合$\mathbb{Z}$,并在上面定义关系$\sim$为“两数的差是3的倍数”,那么容易验证,$1\sim4, 7\sim304, 0\not\sim 2, 2\not\sim 1$. 利用这个等价关系,我们可以把非负整数集合划分成三个等价类,分别是$\{0, 3, 6, 9, 12\cdots \}$,$\{1, 4, 7, 10, 13\cdots\}$和$\{2, 5, 8, 11, 14\cdots\}$. 这样,我们可以得到一个含有三个元素的商集,用数学专业术语来说,叫做集合$A$模去关系$\sim$所得的商集.这里的\textbf{模(mod)}的意思来自于“除法”,在群论中会看到为什么用除法来命名商集.


