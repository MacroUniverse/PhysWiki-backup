% 编辑器事项

\subsection{Editor BUG}
\subsubsection{itemize 环境中的美元符号补全}
\begin{itemize}
\item 【完成】打第一个美元符号不会出现一对美元符号 $abc^2$
\item 【完成】手动打第二个美元符号时会出现两个 $1 + x (x \in X)$
\end{itemize}
其他环境中(如 enumerate, table)是否也存在同样问题?

\subsubsection{表格以后的正文高亮}
【完成】 “原子单位\upref{AU}” 的表格中和表格结束以后所有字体都变成了橙色

\subsection{Editor TODO}

\begin{itemize}
\item 增加一个脚注按钮, 插入 \lstinline|\footnote{脚注}|, 并自动选中 “脚注”

\item 增加一个代码按钮, 图标为 \lstinline|</>|, 按下以后弹出输入框 “请指定语言(不指定则没有高亮)” \lstinline|"\\begin{lstlisting}[language=语言]\n${1}\n\\end{lstlisting}"|

\item 【完成】“插入词条引用按钮” 同时也插入词条的中文名, 且自动选中中文名以便修改. 例如 “词条示例\upref{Sample}”

\item 【完成】保存缓慢时区分是因为网络缓慢还是 PhysWikiScan 无响应. 如果是前者, 就一直尝试连接直到手动关闭提示框.

\item 【完成】上传的图片保存文件名的格式为 "词条名_序号.后缀名"

\item iOS 的 Safari 中拖动文字导致拖动整个屏幕

\item iOS 的 Safari 中选中文字光标位置错误

\item iOS 的 Safari 中键盘有时候无法弹出(即使外接键盘)

\item 【完成】选中一段文字后点击链接按钮, 插入 \lstinline|\href{http://www.example.com}{被选中的文字}|, 自动选中网址

\item 【如果麻烦就算了系列】在菜单上增加某种 GUI 公式编辑器, 例如\href{http://latex.codecogs.com/eqneditor/editor.php}{这个}. 把插入公式按钮也做成像图片按钮那样, 可以选择两种不同模式.

\item 上一条中的那个编辑器貌似价格不菲, 但是\href{https://www.codecogs.com/latex/eqneditor.php?lang=zh-cn}{试用版}是免费的, 所以也可以点击按钮直接跳到这个页面, 编辑完后再把代码复制回来(不知道国内加载速度怎么样)
\end{itemize}

\subsection{PhysWikiScan BUG}

\subsubsection{表格标签多次定义}
删除某个表格再重新插入一个具有同样标签的表格就会出现 “标签多次定义” 的错误.

\subsubsection{表格标题}
第二个表格会具有第一个表格的标题
\begin{table}[ht]
\centering
\caption{第一个标题}\label{edTODO_tab2}
\begin{tabular}{|c|c|}
\hline
* & * \\
\hline
* & * \\
\hline
\end{tabular}
\end{table}

\begin{table}[ht]
\centering
\caption{第二个标题}\label{edTODO_tab3}
\begin{tabular}{|c|c|}
\hline
* & * \\
\hline
* & * \\
\hline
\end{tabular}
\end{table}

\subsection{PhysWikiScan TODO}

\subsubsection{【完成】改用 MathJax2}
MathJax3 在 iOS 的 Safari 上显示有时候公式上半部分消失. 注意当前的 MathJax3 文件夹复制一个备份

\subsection{不要使用 MathJax 的 newcommand}
而是直接进行命令替换, 从而增加公式代码在其他网站的兼容性

\subsubsection{脚注加上返回链接}
参考维基百科和知乎

