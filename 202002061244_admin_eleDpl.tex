% 电偶极子
% 偶极子|电场|电荷

% 这是一级词条, 先介绍两个点电荷

\pentry{电场\upref{Efield}}

\subsection{偶极子的电场}

令空间中两个位置不同的点电荷具有等量的异号电荷, 则他们构成一对\textbf{电偶极子(electric dipole)}. 令他们的电荷量分别为 $q1$ 和 $q2$ ($q_1 + q_2 = 0$), 位置矢量分别为 $\bvec r_1$ 和 $\bvec r_2$, 则它们的总电场为两个电荷各自电场的矢量和(见\autoref{Efield_eq2}\upref{Efield})

(公式未完成)

总电势同样是两个点电荷的电势之和% 链接未完成: N 个点电荷的电势


(公式未完成)

可以拓展到多个电荷的情况或者连续分布的情况


我们常常讨论的是偶极子远处的电势和电场分布, 即 $\abs{\bvec r} \gg \abs{\bvec r_2 - \bvec r_1}$ 的情况. 定义\textbf{电偶极矩为}
\begin{equation}
\bvec p = q_1 \bvec r_1 + q_2 \bvec r_2 = q_2 (\bvec r_2 - \bvec r_1)
\end{equation}
则电势分布为
\begin{equation}
V(\bvec r) = \frac{1}{4πϵ_0} \frac{\bvec p\vdot \bvec r}{r^3} = \frac{1}{4πϵ_0} \frac{\bvec p \vdot \uvec r}{r^2}
\end{equation}
由此可以求出远处的电场分布
\begin{equation}
\bvec E(\bvec r) = 
\end{equation}
