% 正交曲线坐标系中的矢量算符
% 正交曲线坐标系|矢量分析|梯度|散度|旋度

\pentry{正交曲线坐标系\upref{CurCor}, 旋度\upref{Curl}, 拉普拉斯算符\upref{Laplac}}

\footnote{参考 \cite{GriffE} 附录.}在位置矢量\upref{Disp}的全微分
\begin{equation}
\dd{\bvec r} = \pdv{\bvec r}{u}\dd{u} + \pdv{\bvec r}{v}\dd{v} + \pdv{\bvec r}{w}\dd{w}
\end{equation}
中, 令三个偏微分为 $f(u,v,w), g(u,v,w), h(u,v,w)$, 则
\begin{equation}\label{CVecOp_eq4}
\dd{\bvec r} = f\dd{u}\,\uvec u + g\dd{v}\,\uvec v + h\dd{w}\,\uvec w
\end{equation}

令 $s(u, v, w)$ 和 $\bvec A(u, v, w)$ 分别为一阶可微的标量函数和矢量函数, 且
\begin{equation}
\bvec A(u, v, w) = A_x(u, v, w)\uvec u + A_y(u, v, w)\uvec v + A_z(u, v, w)\uvec w
\end{equation}
那么该坐标系中的梯度\upref{Grad}, 散度\upref{Divgnc}, 旋度\upref{Curl}和拉普拉斯算符\upref{Laplac} 分别为
\begin{equation}\label{CVecOp_eq1}
\grad s = \frac{1}{f} \pdv{s}{u}\uvec u + \frac{1}{g}\pdv{s}{v} \uvec v + \frac{1}{h} \pdv{s}{w}\uvec w
\end{equation}
\begin{equation}\label{CVecOp_eq2}
\div \bvec A = \frac{1}{fgh}\qty[\pdv{u}(ghA_u) + \pdv{v}(fhA_v) + \pdv{w}(fgA_w)]
\end{equation}
\begin{equation}\label{CVecOp_eq3}
\begin{aligned}
&\curl \bvec A = \frac{1}{gh}\qty[\pdv{v}(hA_w) - \pdv{w}(gA_v)]\uvec u\\
&\quad + \frac{1}{fh}\qty[\pdv{w}(fA_u) - \pdv{u}(hA_w)]\uvec v
+ \frac{1}{fg}\qty[\pdv{u}(gA_v) - \pdv{v}(fA_u)]\uvec w
\end{aligned}
\end{equation}
\begin{equation}\label{CVecOp_eq6}
\laplacian s = \frac{1}{fgh}\qty[\pdv{u}\qty(\frac{gh}{f}\pdv{s}{u}) + \pdv{v}\qty(\frac{fh}{g}\pdv{s}{v}) + \pdv{w}\qty(\frac{fg}{h}\pdv{s}{w})]
\end{equation}

\subsection{梯度的推导}
球坐标和柱坐标的具体推导见\autoref{Grad_sub1}~\upref{Grad}. 由于正交曲线坐标系中任意一点处 $\uvec u, \uvec v, \uvec w$ 都互相垂直, 所以求该点处梯度的方法和直角坐标系类似: 求出三个方向的方向导数\upref{DerDir}, 并把他们作为梯度的三个分量. 和直角坐标系不同的是, $u$ 坐标增加 $\dd{u}$ 时, 位矢 $\bvec r$ 并不是沿 $\uvec u$ 移动 $\dd{u}$ 而是移动 $f \dd{u}$, 所以 $\uvec u$ 方向的方向导数是 $(\pdv*{s}{u})/f$.

当然这只是粗略的推导, 我们也可以直接把\autoref{CVecOp_eq4} 乘以\autoref{CVecOp_eq1} 得
\begin{equation}
\grad s \vdot \dd{\bvec r} = \pdv{s}{u}\dd{u} + \pdv{s}{v}\dd{v} + \pdv{s}{w}\dd{w} = \dd{\bvec s}
\end{equation}
这符合梯度的定义(\autoref{Grad_eq5}~\upref{Grad}).

\subsection{散度的推导}
一种幼稚的想法是: 类比上面梯度的推导, 根据直角坐标系中的散度公式(\autoref{Divgnc_eq7}~\upref{Divgnc})得
\begin{equation}
\div \bvec A = \frac{1}{f} \pdv{A_x}{u} + \frac{1}{g}\pdv{A_y}{v} + \frac{1}{h} \pdv{A_z}{w} \qquad (\text{错})
\end{equation}
这是错误的, 因为在直角坐标系的推导过程\upref{Divgnc}中, 我们假设小长方体的相对两个面的表面积完全相等, 而在正交曲线坐标系中这不成立. 例如球坐标中的体积元 $r\dd{r} \cdot r\dd{\theta} \cdot r\sin\theta \dd{\phi}$ 中, 垂直于 $\uvec r$ 的两个面的表面积为 $r\dd{\theta} \cdot r\sin\theta \dd{\phi}$, 显然与 $r$ 有关, 即对 $r$ 求导不为零.

所以我们重新按照直角坐标系的推导过程再推导一次, 只不过这回考虑上表面积的变化: 体积元的体积为 $f\dd{u}\cdot g\dd{v}\cdot h\dd{w}$, 三个方向的表面积分别为 $g\dd{v}\cdot h\dd{w}$, $h\dd{w}\cdot f\dd{u}$, $f\dd{u}\cdot g\dd{v}$.  $A_u$ 在两个与 $\uvec u$ 垂直的表面上的通量为
\begin{equation}
\begin{aligned}
&A_u(u+\dd{u}, v, w)\cdot g(u+\dd{u}, v, w)\dd{v}\cdot h(u+\dd{u}, v, w)\dd{w}\\
&-A_u(u, v, w)\cdot g(u, v, w)\dd{v}\cdot h(u, v, w)\dd{w}
 = \pdv{u} (A_u g h) \dd{u} \dd{v} \dd{w}
\end{aligned}
\end{equation}
同理, 总通量为
\begin{equation}
\Phi = \pdv{u} (A_u g h) \dd{u} \dd{v} \dd{w} + \pdv{v} (A_v fh) \dd{u} \dd{v} \dd{w} + \pdv{w} (A_w fg) \dd{u} \dd{v} \dd{w}
\end{equation}
最后除以体积元的体积, 就得到\autoref{CVecOp_eq2}.

\subsection{旋度和拉普拉斯算子的推导}
旋度的推导同样不能直接把直角坐标系公式(\autoref{Curl_eq9}~\upref{Curl})中的偏微分替换为方向导数. 原因和散度的推导一样: 在正交曲线坐标系中取一个小四边形, 同样无法保证相对的两条边的边长完全相等. 同样地, 在直角坐标系中的推导过程中考虑边长的变化, 就可以得到\autoref{CVecOp_eq3}.

拉普拉斯算符 $\laplacian s$ 可以看作梯度的散度 $\div (\grad s)$. 把\autoref{CVecOp_eq1} 代入\autoref{CVecOp_eq2} 即可.

\begin{exercise}{}
完成\autoref{CVecOp_eq3} 和\autoref{CVecOp_eq6} 的推导.
\end{exercise}
